
% compile with xelatex

\documentclass[11pt]{standalone}
\usepackage{color}
\usepackage[cmyk]{xcolor}
\usepackage{amsmath}
\usepackage{xltxtra}
\usepackage{libertine}

\usepackage{tikz}
\usetikzlibrary{intersections,arrows,calc}
% \tikzstyle{every picture}+=[thick]

\usepackage[active,tightpage]{preview}
\PreviewEnvironment{tikzpicture}
\setlength\PreviewBorder{1pt}%

 \def\Pb      {\ensuremath{\mathrm{b}}}
 \def\Pc      {\ensuremath{\mathrm{c}}}
 \def\Pd      {\ensuremath{\mathrm{d}}}
 \def\Pq      {\ensuremath{\mathrm{q}}}
 \def\Ps      {\ensuremath{\mathrm{s}}}
 \def\Pt      {\ensuremath{\mathrm{t}}}
 \def\Pu      {\ensuremath{\mathrm{u}}}

\def\quark     {\ensuremath{\Pq}}
\def\quarkbar  {\ensuremath{\overline \quark}}
\def\qqbar     {\ensuremath{\quark\quarkbar}}
\def\uquark    {\ensuremath{\Pu}}
\def\uquarkbar {\ensuremath{\overline \uquark}}
\def\uubar     {\ensuremath{\uquark\uquarkbar}}
\def\dquark    {\ensuremath{\Pd}}
\def\dquarkbar {\ensuremath{\overline \dquark}}
\def\ddbar     {\ensuremath{\dquark\dquarkbar}}
\def\squark    {\ensuremath{\Ps}}
\def\squarkbar {\ensuremath{\overline \squark}}
\def\ssbar     {\ensuremath{\squark\squarkbar}}
\def\cquark    {\ensuremath{\Pc}}
\def\cquarkbar {\ensuremath{\overline \cquark}}
\def\ccbar     {\ensuremath{\cquark\cquarkbar}}
\def\bquark    {\ensuremath{\Pb}}
\def\bquarkbar {\ensuremath{\overline \bquark}}
\def\bbbar     {\ensuremath{\bquark\bquarkbar}}
\def\tquark    {\ensuremath{\Pt}}
\def\tquarkbar {\ensuremath{\overline \tquark}}
\def\ttbar     {\ensuremath{\tquark\tquarkbar}}

\def\Vud{\ensuremath{V^{\phantom{*}}_{\uquark\dquark}}}
\def\Vubst{\ensuremath{V^{*}_{\uquark\bquark}}}
\def\Vcd{\ensuremath{V^{\phantom{*}}_{\cquark\dquark}}}
\def\Vcbst{\ensuremath{V^{*}_{\cquark\bquark}}}
\def\Vtd{\ensuremath{V^{\phantom{*}}_{\tquark\dquark}}}
\def\Vtbst{\ensuremath{V^{*}_{\tquark\bquark}}}
\def\rhobar{\ensuremath{\kern 0.07em \overline{\kern -0.07em \rho \kern 0.05em}\kern -0.05em}}
\def\etabar{\ensuremath{\kern 0.07em \overline{\kern -0.07em \eta \kern 0.05em}\kern -0.05em}}



\begin{document}
\sffamily
\begin{tikzpicture}[scale=4]
  \begin{scope}[>=latex]
    \draw [name path=real,black,->] (-180:0.15) -- ( 0:1.15) node[below] {\(\Re\)}; % Real axis
    \draw [name path=imag,black,->] ( -90:0.15) -- (90:0.65) node[left]  {\(\Im\)}; % Imag axis

    %\node [scale=0.75] at (0.6,0.9) {
    %  \(\Vud\Vubst + \Vcd\Vcbst + \Vtd\Vtbst = 0\)
    %};

    % triangle base, side with β, side with γ
    \path (0:0)     coordinate (origin);
    \path (0:1)     coordinate (unity);
    \path (53:0.60) coordinate (apex);

    \fill [black] (origin) circle (0.01); % origin

    \draw [black,thick,->] (origin) -- (unity) -- (apex) -- cycle; % triangle
    \draw [black, thick,->] (origin) -- (unity);
    \draw [black,thick,->] (unity)  -- (apex);
    \draw [black,thick,->] (apex)   -- (origin);

    % vertex labels
    \path (origin) +(-0.1,0) node [below] {\footnotesize{(0,0)}};
    \path (unity)            node [below] {\footnotesize{(1,0)}};
    \path (apex)             node [above] {\footnotesize{(\rhobar,\etabar)}};

    % side labels
    \path (unity)  -- node[above,sloped,midway]   {\(\frac{\Vtd\Vtbst}{\Vcd\Vcbst}\)} (apex);
    \path (origin) -- node[above,sloped,pos=0.68] {\(\frac{\Vud\Vubst}{\Vcd\Vcbst}\)} (apex);

    % angles
    \draw [black,stealth-] (apex)   +(-37:0.09) arc (-37:-127:0.09); % α: 90 degrees
    \draw [black,stealth-] (unity)  +(180:0.17) arc (180:143:0.17);  % β: 37 degrees
    \draw [black,-stealth] (origin) +(0:0.10)   arc (0:53:0.10);     % γ: 53 degrees

    % angle labels
    \path (apex)   +(-83:0.135) node {\(\alpha\)};
    \path (unity)  +(163:0.22)  node {\(\beta\)} ;
    \path (origin) +(26:0.145)  node {\(\gamma\)};
  \end{scope}
\end{tikzpicture}
\end{document}
