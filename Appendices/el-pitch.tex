The natural world is the standard source of inspiration for researchers. Giving in to
our intrinsic curiosity the LHCb collaboration at CERN attempts to understand
the, almost complete, absence of antimatter in the observed universe by pushing the boundaries
of our understanding of the natural world. The goal is to challenge the Standard Model,
which is the established theory that governs the quantum interactions of all fundamental particles,
like the electron for example, in the known universe. This highly complex analysis measures the magnitude of
matter-antimatter asymmetries in the decays of fundamental particles, called B mesons,
at a very high precision; and subsequently searches for deviations from the predictions of the Standard Model.
However, this high precision comparison cannot be performed unless certain sub-leading,
penguin, processes which can fake the presence of these deviations, are controlled
and removed prior to any comparison with the established theory. The current thesis comes
to enable exactly this high precision comparison by assessing the magnitude of these penguin
contributions and thus making sure they are not misinterpreted for deviations from the established theory.
