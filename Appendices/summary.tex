\chapter*{Summary}
\chaptermark{}
\addcontentsline{toc}{chapter}{Summary - Samenvatting}

{\Large\bf
  The Weak Phase \phis and Penguin Topologies.
}
\vspace*{0.05\textwidth}

Within the domain of modern pysics there exists a particular field of research
that attempts to answer a fundamental question about the natural world. This
question follows intuitivelly when observing the natural world and it troubled
philospphers since the ancient times. These philosophers attempted to understand
which are the fundamental blocks that build the world around them. Remarkably
enough after more that 2000 years since Democitus, who is commonly accepted as
the first person that introduced the idea of idivisible blocks of nature named
atoms (or \textgreek{άτομα} in Greek), contemprary scientists still have not found a definite
answer on the fundamental blocks of nature. So far, it seems that the observable
universe consists of a handful of elementary particles, which are clasified in two
distinct categories, namely {\it gauge bosons}, responsible for mediating all the
known fundamental forces of nature (with the exception of gravity) and {\it fermions}
which are the constituents of matter. There are 5 gauge bosons and 12 fermions.
Fermions can be divided further into {\it quarks}, which constitute composite
particles like the proton. A typical {\it lepton} is the electron which orbits
the nucleus of all atoms.

\subsubsection{Particle Physics and The Standard Model}
The state of the art mathematical framework necessary to describe the interactions between the
fermions is called the \textit{Standard Model} of Particle Physics \cite{sm-glashow,sm-weinberg,sm-salam}.
Describing an interaction in this context means being able to predict the probability for a certain
outcome in a (fundamental) particle colision. Given the fact that energy and mass can convert
from one to the other, the type of the initial and final particles is not the same. For example,
two initial quarks can colide, or more precicely annihilate, and produce an electron-positron pair.
This is counter intuitive compared to the clasical, non-quantum, description of particle collisions,
where the atomic structure of particles does not change. Thus, the established predictive power of
the Standard Model is an important achievement of particle physicists. Furthermore, the recently
discovered Higgs boson \cite{higgs-cms,higgs-atlas}, which plays a special role in explaining how
particles acquire mass, makes the Standard Model robust.

However, despite the its sucess in describing particle colisions, there are certain
established phenomena and observations that Standard Model does not account for.
Perhaps the most striking one is the absence of any description about the most familiar,
yet the weakest, force of nature, meaning gravity. Or perhaps the observed matter-antimatter
asymmetry in the universe \cite{more-cpv-huet,more-cpv-gavela_I,more-cpv-gavela_II}.

The above phenomena are a few examples that reveal the incompletness of the model.
Thus the scientific method compels scientists to continue testing Standard Model predictions
and look for ways to improve it. Any significant diviation from these predictions could
hint the presence of {\it New Physics} beyond the established model.

\subsubsection{\CP Violation and the Weak Phase \phis}

\begin{itemize}
\item Matter and antimatter in the uni
\item Non SM cp violation and probe NP
\item Status and high precision era, (show phis delta gammas contour)
\end{itemize}

\subsubsection{The \lhcb Detector}

\begin{itemize}
\item \lhc cern most powerfull accelrator
\item \lhcb dedicated bfys experiment, unique capabilities, high signal purity. maybe mention proximity to beam and pid.
\item future increase of low pt muon efficiency extents NP reach.
\end{itemize}

\subsubsection{Analysing Particle Collisions}

\begin{itemize}
\item Non infinite resolution and experimental effects and background forces statistical approach.
\item The \BsJpsiKst decay as a probe for penguins
\item Angular analysis of the decay products to probe polarization fractions. The amplitude has many
      components depending on the oriantation of the spin vectors.
\end{itemize}

\subsubsection{Impact and conclusion}

\begin{itemize}
\item Accordin to Robert recepie Delta phis is found to be blah.
\item This is summarised in the picture blah.
\item It is intreasting becaus blahhhh...
\end{itemize}
