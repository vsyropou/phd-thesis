
Thinks I have to remember or be sure about:
-Why do you not reweight the muons pt in the iterative procedure.
-How are the erros of the efficiency moments caculated.
-Make sure you understand the models used in the Kpi mas dependence
-Why the oscilation damping parameter kappa in the production assymetry looks the way it does
-Why no systematic in the detection assymetry.
-When is a matrix invertible
-Have a genral idea on the transversity framework
-Understand the convension on the signs of production and detection assymetry.
-Why is the efficiency ratio between jpsiphi and jpsikst around two.
-Make sure you know what the p value and chi2 probability is.
-Make sure you know how the luminosity of the mc is scaled to the one of the data.
-nll scans:
    In deltaS1 and deltaS4 the correspondint fS hits upper and lower boundaries.
    In f0 lo stats spiekes. Not there with Bd.
-toys:
    mkpi bin phase and fraction are fine. the gaussian is choped because fS3 is close to a physical boundary. A gaussian pull is not appropriate.
-Acceptance is at the 2 sigma level the same between bs and bd
- Understan why your ang acc systematic is smaller than in the offcial analysis.
- Make sure you knwo what CLT is about.
- Make sure you understand the systematics table of the Br measurement

- What is a left/right handed spinor. How dows handedness relates to the V-A nature of the weak interaction
- Why Yuakawa is of the type u_L d_R whereas the CC u_L u_L
- Check all the formulas in the theoretical introduction.
- Why not overconstrain CKM elements in the A,lamda plane?? (cpv does not manifest there.)
- Why wolfenstein is an approximate parametreization.


Alltime checks:
-- Make sure systematics table is up to date
-- Uniform table and figure format
-- Comas in equations
-- Full stops in footnotes

To Do (kind of urgent).
-- Polish references layout.
-- Why one group of toys is bad
-- Update Jeroen's thesis reference once it is ready.
-- Gerhahrd comment on luminosity in the peaking background section.
