
Thinks I have to remember or be sure about:
-What is gradient boosting.
-Choice of bdtg input variables. Why those variables
-Why do you not reweight the muons pt in the iterative procedure.
-How are the erros of the efficiency moments caculated.
-Make sure you understand the models used in the Kpi mas dependence
-Why the oscilation damping parameter kappa in the production assymetry looks the way it does
-Why no systematic in the detection assymetry.
-When is a matrix invertible
-Have a genral idea on the transversity framework
-Understand the convension on the signs of production and detection assymetry.
-Why is the efficiency ratio between jpsiphi and jpsikst around two.
-Make sure you know what the p value and chi2 probability is.
-Make sure you know how the luminosity of the mc is scaled to the one of the data.
-nll scans:
    In deltaS1 and deltaS4 the correspondint fS hits upper and lower boundaries.
    In f0 lo stats spiekes. Not there with Bd.
-toys:
    mkpi bin phase and fraction are fine. the gaussian is choped because fS3 is close to a physical boundary. A gaussian pull is not appropriate.
-Acceptance is at the 2 sigma level the same between bs and bd
- Understan why your ang acc systematic is smaller than in the offcial analysis.
- Make sure you knwo what CLT is about.

Alltime checks:
-- Make sure systematics table is up to date
-- Uniform table format
-- Comas in equations
-- Full stops in footnotes

To Do (kind of urgent).
-- Why one group of toys is bad
-- Update Jeroen's thesis reference once it is ready.
-- Polish references layout.
-- Gerhahrd comment on luminosity in the peaking background section.
-- Check which of the two BrFractions are used in the normalization (jpsiPhi or JpsiKst maybe both. averaged).
-- Show acceptance plot before and after reweighting
