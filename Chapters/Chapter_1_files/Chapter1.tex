% Chapter 1

%%%%%%%%%%%%%%%%%%%%%%%%%%%%%%%%%%%%%%%%%%%%%%%%%%%%%%%%%%%%%%%%%%%%%%%%%%%%%%%%%%%%%%%%%
%% Introduction %%
%%%%%%%%%%%%%%%%%%%%%%%%%%%%%%%%%%%%%%%%%%%%%%%%%%%%%%%%%%%%%%%%%%%%%%%%%%%%%%%%%%%%%%%%%
\chapter{Introduction}
\label{Introduction}

The main focus of the current thesis is to provide the required ingredients for a high precision measurement of
the weak phase \phis. The latter is an important parameter related to \CP violation, which, as explained throughout
the current chapter, is related to the matter-antimatter asymmetry in the universe. Measuring \phis enables one to look
for deviations from the established theory of elementary particles. Given the state of the art measurement of \phis \cite{phis-3fb-paper},
it is clear that in order to probe potential deviations, \phis needs to be measured with increased precision.
However, entering this high precision regime one finds out that there are sub-leading contributions that need to be
controlled first. Unless this is achieved, a high precision measurement of \phis can not provide insight on possible
deviations. While measuring \phis is based on the analysis of \BsJpsiPhi decays, controlling these sub-leading contributions
requires a different decay channel, like \BsJpsiKst, which plays the role of a control channel. The reason for this becomes apparent in \secref{jpsiphi_amp_struct}.
Essentially, estimating these contributions is more precise when using \BsJpsiKst instead of \BsJpsiPhi decays,
as their effect is enhanced in the former.

The current chapter places \phis into perspective with respect to the established theory of particle physics.
\chapref{lhcb_detector} introduces the \lhcb detector. The analysis of the \BsJpsiKst decays that are
required to control the above mentioned higher order effects is presented in \chapref{Data_Analysis}.
The interpretation and impact of the analysis can be found in \chapref{Penguins}.
Lastly, \chapref{Muon_id_hlt} is dedicated to a particular part of the \lhcb data acquisition system,
which is responsible for selecting a certain type of elementary particles called {\it muons}, see \tabref{quarksLeptons}.
Note that the latter particles are a necessary ingredient for the measurements of \phis and
the analysis of \BsJpsiKst decays.

Finally, the work described in \chapref{Data_Analysis} resulted in the analysis of the \BsJpsiKst
decays \cite{bsjpsikst-paper}, and contributed to the \phis measurement \cite{phis-3fb-paper}.
The work performed in \secref{mvm_algorrithm} is an essential part of the muon identification
software update \cite{Albrecht:2253050} of the \lhcb detector in preparation for the \runtwo data taking period.

%%%%%%%%%%%%%%%%%%%%%%%%%%%%%%%%%%%%%%%%%%%%%%%%%%%%%%%%%%%%%%%%%%%%%%%%%%%%%%%%%%%%%%%%%
%% Standard Model %%
%%%%%%%%%%%%%%%%%%%%%%%%%%%%%%%%%%%%%%%%%%%%%%%%%%%%%%%%%%%%%%%%%%%%%%%%%%%%%%%%%%%%%%%%%
\section{Standard Model and the Weak Interaction}
\label{The_Standard_Model}
After a few decades of research in the subatomic scale it has been possible to account for a large number of phenomena
in nature, based on the existance of only a handfull of particles. Which is impressive given the vast dimmensions that the
universe spans over. Two distinct categories emerge from the above collection of particles, namelly the {\it gauge
bossons} responsible for mediating three out of the four know fundamental forces of nature, and quarks plus leptons which are the
constituents of matter
(and antimatter)\footnote{Anitmatter is not as exotic as it sounds. It is a state of matter where the
signs of all quantum numbers (like the electric charge) of a particle are fliped.}.
The resently discoverd higgs bosson [refff] is a special category by itself due to each spacial role of explaining how particles acquire mass.
Figure \figref{} illustrates those matter particles and their interactions via the gauge bossons (also known as force cariers).
The mathematical "language" necessary to describe those interactions is spelled out by the \textit{Standard Model} of Particle Physics
which is a quantum field
theory\footnote{Quantum field theories are[ref], roughly speaking, the result of combining Einstein's special theory of
relativity[reference] and quantum mechanics[ref].}.
In this framework particles are described by quantum fields which is a dificult
to grasp but very usefull way to represent a
particle\footnote{For further reading on quantum filed theories [refff]}.

\begin{equation}
\mathscr{L}_{SM} = \underbrace{i \bar\psi D^{\mu} \gamma _{\mu} \psi}_{\mathscr{L}_{\text{Kinetic}}} +
                   % \underbrace{(D^\mu\phi)^{\dagger} (D^\mu\phi) -\mu^2\phi^{\dagger}\phi - \lambda(\phi^{\dagger}\phi)^2}_{\mathscr{L}_{\text{Higgs}}} +
                   \underbrace{(D^\mu\phi)^{\dagger} (D^\mu\phi) + V(\phi)}_{\mathscr{L}_{\text{Higgs}}} +
                   \underbrace{Y_{ij}\bar\psi_{Li}\phi\psi_{Rj} \; + \; h.c.}_{\mathscr{L}_{\text{Yukawa}}},
\label{lagrangian}
\end{equation}

% \footnote{$D^\mu=\partial ^\mu + ig_sG_{\alpha}^\mu L_{\alpha} +  igW_b^\mu\sigma_b + ig^\prime B^\mu Y$ \cite{covariant derivative??}}

The above mentioned mathematical language in its most compact form i shown in \equref{lagrangian}. The last equation is the so called
{\it lagrangian}\footnote{Lgrangians are elegant equations that discribe the dynamics of a system. They are used both in classical physical
systems such as motions of planets and in the quantum world of tiny distances.}
of the Standard Model. The first {\it Kinnetic}, term of \equref{lagrangian} describes the possible interactions between the quarcks or the leptons.
The second, {\it Higgs}, term is the one that includes the masses of the gauge and higgs bossons. Whereas the last, {\it Yukawa}, attributes
masses to the quarks and
leptons\footnote{The {\it Kinnetic} and {\it Higgs} terms appear more naturealy in the lagrangian wheras the last one is added by hand, which is perfectly fine.}
Quarcks or leptons and the higgs fields are represented by $\psi$ and $\phi$ in \equref{lagrangian} respectivelly. The $\gamma_\mu$ is a Dirac matrix
that takes care of the additional structure of the fields due to their intrinsic spin. $V(\phi)$ is the potential term of the
higgs field and it contains its mass. The $L$ and $R$ subscripts denote the left and right handed projections of the quarck or lepton fields.
$Y_{ij}$ are called Yuakawa cuplings. Their meaning and relevance to the current thesis is shown in \secref{}.
The symbol $D_\mu$ introduces the interactions of the gauge bossons with the quarks or leptons quantum fields and it is called covariant derivative[ref????].

\begin{figure}[h]
  \begin{center}
    \includegraphics[width=0.5\textwidth]{Figures/Chapter1/Standard_Model_Particles.png}
    \caption{Standard Model matter particles. Combinations of three and two quarks build hadrons and messons respectivelly.
    Hadrons are for example protons and neutrons wheares a messon could be the \Bs particle which is relevant for the current thesis.
    Leptons can be charged (\electron,\mmu,\mtau) and uncharged (\neue,\neum,\neut). The first of the charged ones is the familiar
    electron that exists in the neucleous of every atom and the rest are heavier "brothers" of it. Netrinos on the other hand
    do not take part in any known nucleous-like formation they are also massless within the Standard Model making them extremelly (and notoriously)
    difficult to detect. They live in the least well known "neighborhoud"of the Standard Model. Their presence is indirectly
    implied in the radioactive decay of atoms and. The gauge bossons \g, g, \Wpm \Z  are responsible for mediating the electromagnetic,
    strong, and weak interactions between quarcks or leptons repsectivelly. These possible interactions are indicated by the blue
    lines. Some of the gauge bossons can interact with themselves which is illustrated by the blue closed loop lines.}
    \label{sm_particles}
  \end{center}
\end{figure}

The core element that makes \equref{lagrangian} elegant is the so called {\it local gauge invariance}[ref??].
Briefely speaking this means that \equref{lagrangian} is invariant (or symmetric) under spacetime dependant
phase\footnote{the quark and lepton quantum fields are complex valued quantities. Thus phase here means complex phase.} transformations of the quark or lepton fields.
By construction, the Lagrangian obeys the symmetry
group\footnote{with the mathematical notion of a group implied} $SU(3)_c\otimes SU(2)_L\otimes U(1)_Y$.
This means that there are three distinct transformations of the $\psi$ fields that \equref{lagrangian} is symetric to.
Each one introduces the electromagnetic, weak and strong interactions between the fields. The mathematics
behind the above mentioned transoformations really gives the current paragreaph proper meaning but it would
be completeley out of context here. Further reading on the symmetries of th Standard Model can be found in [?????????]

% In addition the quarks and the gluon quantum fields have an extra degre of freedom called color. Each quark can have three
% any of the three color degrees of freadom. Which implies that as far as the strong interaction is concerned the the quarks are
% actually 9 and not 3. In case of the gluons the color degree of freadom implies that there are eight gluons resulting from
% the symmetry structure of the strong interaction. Lastly, the vast majority of the observed matter is built from \uquark, \dquark
% quarks and \electron which makes the explanation of the triplet like structure of quarcks and leptons very enigmatic and difficult
% to understand from first principles.In addition the quarks and the gluon quantum fields have an extra degre of freedom called color. Each quark can have three
% any of the three color degrees of freadom. Which implies that as far as the strong interaction is concerned the the quarks are
% actually 9 and not 3. In case of the gluons the color degree of freadom implies that there are eight gluons resulting from
% the symmetry structure of the strong interaction.

Within the accuracy of the current experiments the Standard Model has seen its predictions confirmed to a great extend.
The most recent and perhaps one of the most crucial meaning the discovery of the mechanism that particles acquire mass, makes the Standard
Model looks quite robust. However there are phenomena and observations that it can not account for. Perhaps the most stricking one is the
absence of any discription about the most familiar and strong force of nature, meaning gravity\footnote{Gravity is not renormalisable[\cite{}]
theory and thus cannot be described as a quantum field theory}. or the peculiar value of the cosmological constant leading to dark energy
interpretations[\cite{}].But even at the heart of the Standard Model there obscure aspects for example the well established fact that neutrinos
have non zero mass[\cite{}] or the unexplained amount of the observed matter-antimatter assymetry in the universe [\cite{}]. For all of the
above reasons plus our curiosity driven nature scientists are compled to continue testing the Standard Model and look for deviations of
its predictions.


%%%%%%%%%%%%%%%%%%%%%%%%%%%%%%%%%%%%%%%%%%%%%%%%%%%%%%%%%%%%%%%%%%%%%%%%%%%%%%%%%%%%%%%%%
%% Flavor Phisics %%
%%%%%%%%%%%%%%%%%%%%%%%%%%%%%%%%%%%%%%%%%%%%%%%%%%%%%%%%%%%%%%%%%%%%%%%%%%%%%%%%%%%%%%%%%
\section{Flavor Physics}
\label{Flavor_Physics}
As it was mentioned in \secref{The_Standard_Model}, quarks and leptons acquire mass through the Yukawa term
of the Standard Model Lagrangian. An important aspect of the weak interaction emerges from that term,
namely the {\it quark flavor mixing}. A brief description of the latter and its relevance to \CP violation
is given in what follows. The section concludes with introducing the  {\it CKM-mixing matrix},
which is the core of {\it flavor physics} in the quark sector of the Standard Model.

\subsubsection{Quark flavor mixing}
By applying the Higgs mechanism \cite{PhysRevLett.13.321,PhysRevLett.13.508} to the Standard Model Lagrangian
the Higgs field obtains a {\it vacuum expectation value}, which corresponds to the lowest energy value of the Higgs
potential. After this step the Yukawa term for the quark fields, ignoring quark-Higgs field interaction terms, is:

\begin{subequations}
\label{yukawa_flavor}
\centering
  \begin{align}
  % -\mathscr{L}_{\text{Yukawa}} = M_{ij}^d \bar{d_{Li}} d_{Rj} + M_{ij}^u \bar{u_{Li}} u_{Rj} + h.c.,
  -\mathscr{L}_{\text{Yukawa}} &= \left[ y_{ij}^d \bar{d}_{Li} d_{Rj} + y_{ij}^u \bar{u}_{Li} u_{Rj} \right] \frac{v}{\sqrt{2}} + h.c. + ...  \\
                               &= \left[ m_{ij}^d \bar{d}_{Li} d_{Rj} + m_{ij}^u \bar{u}_{Li} u_{Rj} \right] + h.c. + ...,  \\
                               \text{with} \;\;\; m^{u,d}_{i,j} = \frac{v}{\sqrt{2}} & y_{ij}^{u,d}, \nonumber
  \end{align}
\end{subequations}

\noindent where $v$ is the Higgs vacuum expectation value and $y_{ij}^{u,d}$ are complex valued numbers called {\it Yukawa couplings}.
The latter are free parameters that represent the coupling strength between Higgs and quark fields.
Note that for equations \ref{yukawa_flavor} to \ref{CClagrangian} it is implied that $u$ and $d$ indicate
any up or a down type quark. The generation of the up(down) type quark is specified by the indices $i,j$,
whereas, indices $L,R$ indicate the left or right handedness of the quark field.
Finally the matrix $m^{u,d}$ expresses the desired quark masses.

The quark fields, $u$ and $d$,  in \equref{yukawa_flavor} have a definite quantum number that labels the generation to which they belong
and also whether they are of up or down type. This quantum number is commonly called {\it flavor} and thus the quark fields
are flavor eigenstates. By construction the mass matrix is not diagonal which means that
a quark with a well defined flavor does not have a well defined mass. Or, in more formal phrasing,
the flavor and mass eigenstates of the quark fields do not coincide. In order to obtain proper quark
masses the matrix $m^{u,d}$ has to be diagonalized, as:

\begin{equation}
  \centering
  m^{d,u}_{\rm diag} = V_L^{d,u} m^{d,u} \left(V_R^{d,u}\right)^{\dagger},
  \label{diagM}
\end{equation}

\noindent where the matrices $V$ are required to be unitary. Since \equref{yukawa_flavor} has to stay intact after $m^{d,u}$ is replaced with
$m^{d,u}_{\rm diag}$, quark fields need to be rotated as well as shown in \equref{quark_rotation}, such that they cancel the additional $V$ matrices
of \equref{diagM}.

\begin{equation}
  \centering
  \left( d_{i}^m \right)_{L,R} = \left( V^d_{ij} d_{j} \right)_{L,R}, \;\;\;\; \left( u_{i}^m \right)_{L,R} = \left( V^u_{ij} u_{j} \right)_{L,R}.
  \label{quark_rotation}
\end{equation}

\noindent At this point the quark fields in the rest of the Standard Model Lagrangian are still flavor eigenstates.
The field rotations of \equref{quark_rotation} need to be applied in these terms as well. Specifically, it must be
applied to the kinetic term involving quark interactions with the charged weak bosons \Wpm, also known as {\it charged current}
interaction, shown in \equref{CClagrangian}. The charged current interaction in the same equation is expressed in two ways.
One with quark fields expressed as flavor eigenstates, \equref{CClagrangianInt} and two as mass eigenstates, \equref{CClagrangianMass}.

\begin{subequations}
  \centering
  \begin{align}
    \mathscr{L}_{\text{Kinetic}}^{CC} & \propto \bar{u}_{Li} \gamma_\mu {\Wm}^\mu d_{Ri} + \bar{d}_{Li} \gamma_\mu {\Wp}^\mu u_{Ri}  \label{CClagrangianInt} \\
                                      & \propto \bar{u}_{Li}^m  {V_{\text{CKM}}} \gamma_\mu {\Wm}^\mu d_{Ri}^m + \bar{d}_{Li}^m V_{\text{CKM}} {\Wp}^\mu \gamma_\mu u_{Ri}^m, \label{CClagrangianMass}
  \end{align}
  \label{CClagrangian}
\end{subequations}

\noindent with $V_{\text{CKM}} \equiv V^u_LV^{d\dagger}_L$ and \Wpm are the charged weak boson fields, whereas $\gamma_\mu$ are Dirac matrices.
The resulting matrix is the  {\it CKM mixing matrix}:

\begin{equation}
  \centering
  \begin{pmatrix} \dquark \\ \squark \\ \bquark  \end{pmatrix} =
  \underbrace{\begin{pmatrix} \Vud & \Vus & \Vub \\ \Vcd & \Vcs & \Vcb \\ \Vtd & \Vts & \Vtb \end{pmatrix}}_{V_{\text{CKM}}}
    \begin{pmatrix} \dquark^m \\ \squark^m \\ \bquark^m  \end{pmatrix}.
      \label{quark_field_rotation}
  \end{equation}

\begin{figure}[t]
  \centering
  {\sffamily 

\hspace*{0.05\textwidth}
\begin{fmffile}{Figures/Chapter1/qqMixing}
  \fmfframe(8,16)(8,16){
    \begin{fmfgraph*}(60,35)
      \fmfstraight
      \fmfleft{u}
      \fmfright{d,W}
      \fmf{fermion}{u,V,d}
      \fmf{boson}{V,W}
      \fmflabel{$u_i$}{u}
      \fmflabel{$d_j$}{d}
      \fmflabel{$\Wp$}{W}
      \fmflabel{\hspace{0.2cm}$V_{ij}$}{V}
    \end{fmfgraph*}
  }
\end{fmffile}

%
% \hspace*{0.05\textwidth}
% \begin{fmffile}{Figures/Chapter1/udMixing}
%   \fmfframe(8,16)(8,16){
%     \begin{fmfgraph*}(60,35)
%       \fmfstraight
%       \fmfleft{u}
%       \fmfright{d,W}
%       \fmf{fermion}{u,V,d}
%       \fmf{boson}{V,W}
%       \fmflabel{$u_i$}{u}
%       \fmflabel{$d^m_j$}{d}
%       \fmflabel{$\Wp$}{W}
%       \fmflabel{\hspace{0.2cm}$V_{id}$}{V}
%     \end{fmfgraph*}
%   }
% \end{fmffile}
% \hspace*{0.05\textwidth}
% \begin{fmffile}{Figures/Chapter1/usMixing}
%   \fmfframe(8,16)(8,16){
%     \begin{fmfgraph*}(60,35)
%       \fmfstraight
%       \fmfleft{u}
%       \fmfright{d,W}
%       \fmf{fermion}{u,V,d}
%       \fmf{boson}{V,W}
%       \fmflabel{$u_i$}{u}
%       \fmflabel{$s^m_j$}{d}
%       \fmflabel{$\Wp$}{W}
%       \fmflabel{\hspace{0.2cm}$V_{is}$}{V}
%     \end{fmfgraph*}
%   }
% \end{fmffile}
% \hspace*{0.05\textwidth}
% \begin{fmffile}{Figures/Chapter1/ubMixing}
%   \fmfframe(8,16)(8,16){
%     \begin{fmfgraph*}(60,35)
%       \fmfstraight
%       \fmfleft{u}
%       \fmfright{d,W}
%       \fmf{fermion}{u,V,d}
%       \fmf{boson}{V,W}
%       \fmflabel{$u_i$}{u}
%       \fmflabel{$b^m_j$}{d}
%       \fmflabel{$\Wp$}{W}
%       \fmflabel{\hspace{0.2cm}$V_{ib}$}{V}
%     \end{fmfgraph*}
%   }
% \end{fmffile}
}
  \caption{Feynman diagram where an up-type quark couples to any of the three, (\dquark,\squark,\bquark), down type quarks,
           via a \Wp boson. Time flows from left to right.}
  \label{QuarkMixing}
\end{figure}

The charged current term, \equref{CClagrangianMass}, incorporates an important aspect of the weak interaction,
namely the {\it quark flavor mixing}. The latter is due to the fact that quark mass eigenstates are superpositions
of the flavor eigenstates. By definition the $V_{\text{CKM}}$ rotates only the down type quarks, implying that
the mass eigenstates of the up-type quarks are identical to the flavor eigenstates. Thus an up type quark
can change its flavor to any of the down type quarks, see \figref{QuarkMixing}, with a certain probability.
The probability of such a transition is given by the corresponding element of the CKM mixing matrix.

\subsubsection{CKM mixing matrix}
The CKM matrix, or simply CKM matrix, is a unitary matrix which, as previously mentioned, describes the strength of quark couplings, or in other words,
the probability of a certain quark flavor transitions. The elements of the CKM matrix have been measured,
see \eg Chapter 12 of \cite{PDG}, showing the following structure for their magnitudes:

\begin{equation}
  |V_{\text{CKM}}|
                   = \begin{pmatrix} \VudMag & \VusMag & \VubMag \\ \VcdMag & \VcsMag & \VcbMag \\ \VtdMag & \VtsMag & \VtbMag \end{pmatrix}
              \simeq \begin{pmatrix} 1 & 0.2 & 0.008 \\ 0.2 & 1 & 0.04 \\ 0.008 & 0.04 & 1 \end{pmatrix}.
      \label{CKMmatrix}
  \end{equation}

\noindent Essentially the structure implies that transitions between generations
are suppressed with respect to transitions within the same generation in a hierarchical way. The most suppressed transitions are between
the first and third generations followed by the ones between the second and third and the least suppressed are between first and second.

The unitarity of the CKM matrix is at the center of flavor physics. In order to achieve such tests a
parametrization of the CKM matrix is useful. After exploiting the unitarity of $V_{\rm CKM}$ and all the
redundant quark field phases, the CKM matrix has, by construction, 3 real parameters and one complex phase.
The choice of the CKM matrix parametrization is \aprior arbitrary. However due to the observed structure the
{\it Wolfenstein} \cite{Wolfenstein:1983yz,Buras-wolfenstein} parametrization is a standard choice.
The Wolfenstein parametrization utilizes three real parameters, $\lambda,A,\rho$ and an imaginary one $i\eta$, as follows:

\begin{equation}
\centering
  |V_{\text{CKM}}|
                   = \begin{pmatrix} \VudWolf & \VusWolf & \VubWolf \\
                                     \VcdWolf & \VcsWolf & \VcbWolf \\
                                     \VtdWolf & \VtsWolf & \VtbWolf \end{pmatrix} + \mathcal{O}(\lambda^4).
      \label{CKMwolfenstein}
\end{equation}

\noindent As previously mentioned, the CKM matrix is unitary, meaning that $V_{\text{CKM}} V_{\text{CKM}}^\dagger = I_{3\times3}$.
This leads to the orthogonality relations. The latter are sums of complex numbers that are equal to zero,
and thus can be represented as {\it unitarity triangles} in the complex plane. There are six orthogonality relations, two of which, shown
in \equref{unitConstraints}, are relevant for this thesis since the CKM elements present in these relations govern the
dynamics in the \Bs and \Bd meson systems.

\begin{subequations}
  \centering
  \label{unitConstraints}
  \begin{align}
    \Bd : & \quad \Vud\Vub^* + \Vcd\Vcb^* + \Vtd\Vtb^* = 0,
    \label{unitConstraints_Bd} \\
    \Bs : & \quad \Vus\Vub^* + \Vcs\Vcb^* + \Vts\Vtb^* = 0.
    \label{unitConstraints_Bs}
  \end{align}
\end{subequations}

\noindent These two relations, after dividing by $\Vcd\Vcb^*$, $\Vcs\Vcb^*$ respectively for \Bd and \Bs,
are illustrated in \figref{unitTriangles}. Note that the CKM element $\Vts$ has a complex part at higher
order in $\lambda$, see section 13.3 of \cite{PDG}. By inspecting the triangles one can define some of the angles
as follows:

\begin{figure}[t]
  \centering
  \begin{subfigure}{0.475\textwidth}
    \raggedright
    \includegraphics[width=\textwidth]{Figures/Chapter1/b-d-triangle}
    \caption{}
    \label{unitTriangles_bd}
  \end{subfigure}%
  \hfill%
  \begin{subfigure}{0.525\textwidth}
    \raggedleft
    \includegraphics[width=\textwidth]{Figures/Chapter1/b-s-triangle}
    \caption{}
    \label{unitTriangles_bs}
  \end{subfigure}
  \caption{CKM-unitarity triangles. \Bd triangle (left), corresponding to \equref{unitConstraints_Bd}. \Bs triangle (right),
           corresponding to \equref{unitConstraints_Bs}. Triangle sides have been normalized, see text.
           This way one of the sides is real with unit length. Note that triangles are not drawn to scale. Figures from \cite{jeroenThesis}. }
  \label{unitTriangles}
\end{figure}

\begin{align}
  \centering
  \alpha \equiv \arg\left( -\frac{\Vtd\Vtb^*}{\Vud\Vub^*} \right),
  \;\;
  \beta  \equiv & \arg\left( -\frac{\Vcd\Vcb^*}{\Vtd\Vtb^*} \right),
  \;\;
  \gamma \equiv \arg\left( -\frac{\Vud\Vub^*}{\Vcd\Vcb^*} \right), \nonumber \\
  \betas \equiv & \arg\left( -\frac{\Vts\Vtb^*}{\Vcs\Vcb^*} \right).
  \label{ckm_angles_def}
\end{align}

\noindent Note that the definitions of \equref{ckm_angles_def} are independent of
the quark field phases. Thus the above angles are useful observables, regardless of the chosen
CKM matrix parametrization.

There is no fundamental reason known in the Standard model for the observed hierarchy of the CKM elements.
One of the main goals of flavor physics is to verify the consistency of the CKM picture.
Overlaying measurements of flavor physics observables in the complex $\bar{\uprho}-\bar{\upeta}$ plane
of \figref{unitarity_triangle} should show a compatible picture regarding the position of the
apex of the unitarity triangle. The apexes of the \Bd and \Bs triangles are defined in a
convention independent way as:

\begin{equation}
  \bar{\uprho} + i \bar{\upeta} = -\frac{\Vud\Vub^*}{\Vcd\Vcb^*}, \quad \quad
  \bar{\uprho}_{\rm s} + i \bar{\upeta}_{\rm s} = -\frac{\Vus\Vub^*}{\Vcs\Vcb^*}.
  \label{apexes}
\end{equation}

\begin{figure}[!t]
  \centering
    \includegraphics[trim=0cm 0.5cm 0cm 11cm, clip=true, width=\textwidth]{Figures/Chapter1/rhoeta_large}
    \caption{Global fit of the \Bd triangle \cite{ckm-fitter-phis-pred}.
             Inputs from measurements of flavor physics observables are indicated by colored bands.}
    \label{unitarity_triangle}
\end{figure}

\noindent For completeness it is interesting to mention that in the
lepton sector a similar matrix is present. The observed hierarchy there is completely
different which is yet another intriguing feature of nature.

\subsubsection{\CP violation and fermion masses}
Having introduced both the fermion masses and \CP violation it is interesting to point out their common origin in
the Standard Model \cite{KM-mechanism}. Elaborating more, from one hand the CKM matrix was introduced to couple
the Higgs to the fermions and thus provide them with a mass. On the other hand the fact that $V_{\rm CKM}$ is a
complex valued matrix allows for \CP violation in the Standard Model via the weak interaction. This is because
the charged current interaction of \equref{CClagrangianMass} is invariant under \CP operations only
if $V_{\rm CKM}=V_{\rm CKM}^*$, which is evidently not the case.


%%%%%%%%%%%%%%%%%%%%%%%%%%%%%%%%%%%%%%%%%%%%%%%%%%%%%%%%%%%%%%%%%%%%%%%%%%%%%%%%%%%%%%%%%
%% The BBbar system%%
%%%%%%%%%%%%%%%%%%%%%%%%%%%%%%%%%%%%%%%%%%%%%%%%%%%%%%%%%%%%%%%%%%%%%%%%%%%%%%%%%%%%%%%%%
\section{The \Bs Meson and \CP violation}
\label{Phenomenology}


The \Bs meson\footnote{Mesons are bound by the strong interaction states of two quarks.} is an electrically neutral
particle that consists of two quarks, particularly (\bquarkbar\squark). An important feature of
the \Bs meson, and all neutral mesons, is that it can spontaneously change into its antiparticle,
the \Bsb meson and vice versa. This feauture is called {\it meson oscilations} and it is posible
in the Standard Model via the so called {\it box diagram} of \figref{bs_box}.
Meson oscilations play a central role in flavor physics. Particularly because they are sensitive to the
existance of new particles as it will be explained in \secref{probe_new_phys}. CP-Violating effects can
apper in the \BBbarSyst oscilations and in the subsequent decay. The current section as well
as \secref{WeakPhase} will qualitatively address the way that these effects manifest themselves.

\begin{figure}[h]
  \centering
  \begin{subfigure}{0.5\textwidth}
    \centering
    {\sffamily \input{Figures/Chapter1/box1}}
    \caption{}
    \label{bs_box_1}
  \end{subfigure}%
  \begin{subfigure}{0.5\textwidth}
    \centering
    {\sffamily %%BoundingBox: -8 0 123 75
%%HiResBoundingBox: -8 0 122.57008 74.71962

\begin{fmffile}{Figures/Chapter1/box2}
  \fmfframe(19,3)(17,3){
    \begin{fmfgraph*}(115,75)
      \fmfbottom{i1,d1,o1}%dummy vertex
      \fmftop{i2,d2,o2}
      \fmf{fermion,label=s,l.side=left}{i1,v1}
      \fmf{fermion,label=b,l.side=left}{v2,o1}
      \fmf{fermion,label=b,l.side=left}{v3,i2}
      \fmf{fermion,label=s,l.side=left}{o2,v4}
      \fmf{boson,label=W$^-$,l.side=left}{v1,v2}
      \fmf{boson,label=W$^+$,l.side=left}{v4,v3}
      \fmffreeze
      \fmf{fermion,label={t,,c,,u},l.side=left}{v4,v2}
      \fmf{fermion,label={t,,c,,u},l.side=left}{v1,v3}
      \fmf{plain,left=0.2}{o1,o2}
      \fmf{plain,left=0.2,label=$\Bsb$}{o2,o1}
      \fmf{plain,left=0.2}{i2,i1}
      \fmf{plain,left=0.2,label=$\Bs$}{i1,i2}
    \end{fmfgraph*}
  }
\end{fmffile}
}
    \caption{}
    \label{bs_box_2}
  \end{subfigure}
  \caption{Leading order diagrams for \BBbarSyst oscilations. Figures from~\cite{jeroenThesis}.}
  \label{bs_box}
\end{figure}

\subsubsection{The \BBbarSyst System}

The oscilating behaviour of the \Bs and \Bsb mesons invites a phenomenological approach
where the two mesons are treated as a coupled quantum mechanical system. This approach
is based on~\cite{Weisskopf:1930au,Weisskopf:1930ps} approach and uses an {\it effective hamiltonian}~\cite{eff-hamiltonian-bs-syst,DeBruyn-thesis}
to describe the time evolution of the \Bs or \Bsb meson. The wavefunctions of the last mesons
are defined in \equref{bs_wavefunctions}

\begin{equation}
\ket{\Bs}  \equiv  \ket{\bquarkbar\squark}, \;\;\; \ket{\Bsb} \equiv  \ket{\bquark\squarkbar}
\label{bs_wavefunctions}
\end{equation}

\noindent impling that the states \Bs and \Bsb have a definite quark content or in other words
thei are flavour eigenstates. The effective hamiltonian, $H_{\rm eff}$ is a 2x2 non diagonal non hermitian matrix.
The non diagonal feature is due to the oscilating feature of the \BBbarSyst system whreas the non
hermirianity takes into acount the probability that the system will eventually decay to some final state.
By diagonalising the effective on can obtain the mass eigenstates, \ket{\Bs{}_{,H}}, \ket{\Bs{}_{,L}}, of the system, shown in \equref{bs_mass_eigen}.
More details on the exact calcualtions can be found in section 13.1 of~\cite{PDG} and in ~\cite{jeroenThesis,DeBruyn-thesis}.

\begin{align}
\ket{\Bs{}_{,H}} &= p \ket{\Bs} + q \ket{\Bsb}, \nonumber \\
\ket{\Bs{}_{,L}} &= p \ket{\Bs} - q \ket{\Bsb}
\label{bs_mass_eigen}
\end{align}

The amount of mixing between the mass and flavour eigenstates is governed by the complex parameters $p,q$.
The ratio $|\qoverp|$ is compatible with one. This is supported both from theoretical calculations~\cite{Lenz:2011ti}
and from experimental measurement~\cite{asl-paper} measurements\footnote{Some tensions between theory and experiment will be resolved in the future. The tensions are
mainly fue to the \dzero measuremnt~\cite{Abazov:2013uma}}. In addition the difference in mass and lifetime between the two
mass eigenstates are important observables of the \BBbarSyst and are denoted as $\Delta m_s$ and $\Delta\Gamma_s$ respectively.

\subsubsection{Types of CP-Violation}
The ratio $|\qoverp|$ is connected a type of CP-Violation. Specifically the CP-Violation {\it in the mixing} (of the \BBbarSyst system).
The connection is shown in the asymmetry of \equref{acp_mixing}.

\begin{equation}
\Acp{\text{mix}}      = \frac{\bra{\Bsb}\ket{\Bs} - \bra{\Bs}\ket{\Bsb}} {\bra{\Bsb}\ket{\Bs} + \bra{\Bs}\ket{\Bsb}}
                \propto \left|\frac{q}{p}\right|^2 \parenthesis{ 1 - \left|\frac{p}{q}\right|^4},
\label{acp_mixing}
\end{equation}

\noindent where it can be seen that $\Acp{\text{mix}}$ vanshes if $|\qoverp|= 1$.

Another type of CP-Violation is called {\it in the decay} and its based on the amplitude, $A_f$, of a certain meson decay to some final state $f$.
Using the CP conjugated $\bar{A}_{\bar{f}}$ amplitude the asymmetry of \equref{cpv_decay} can be constructed.
The last asymmetry vanishes in case $|\nicefrac{\bar{A}_{\bar{f}}}{A_f}| = 1$  and it is also relevant to other charged mesons as well.

\begin{equation}
\Acp{\text{dir}} = \frac{\Gamma(\Bs \rightarrow f) - \Gamma(\Bsb \rightarrow \bar{f})} {\Gamma(\Bs \rightarrow f) + \Gamma(\Bsb \rightarrow \bar{f})}
                = \frac{ |\nicefrac{\bar{A}_{\bar{f}}}{A_f}|^2 - 1}{|\nicefrac{\bar{A}_{\bar{f}}}{A_f}|^2 + 1}
\label{cpv_decay}
\end{equation}

\noindent where $\Gamma$ is the decay rate of the $\Bs \rightarrow f$ process and $f$ is some final state.
Both calsifications are based on section 13.1.4 of~\cite{PDG}.

There is one more type of CP Violation and that is in the {\it interference} between
a neutral meson decying directlly to a final state $f$ or first oscialting to its antiparticle and then decaying
to the same final state $f$. Due to its relevance to the current thesis, this type of CP-Violation is explained
in the next section.


%%%%%%%%%%%%%%%%%%%%%%%%%%%%%%%%%%%%%%%%%%%%%%%%%%%%%%%%%%%%%%%%%%%%%%%%%%%%%%%%%%%%%%%%%
%% The BBbar system%%
%%%%%%%%%%%%%%%%%%%%%%%%%%%%%%%%%%%%%%%%%%%%%%%%%%%%%%%%%%%%%%%%%%%%%%%%%%%%%%%%%%%%%%%%%
\section{The Weak Phase $\boldsymbol{\phis}$}
\label{WeakPhase}


{\color{red}In this section we desctibe blah blah and blah}

\subsubsection{The \phis parameter in $b \rightarrow ssl$ transitions}
\phis is a parameter that arises in the b2ssl transitions, which are CP eigenstates.
See diagram,
It is an overal phase difference of two diagrams. The decay and the mixing.

The assymetry can be computed by inspectign the diagrams using the pahse definitions in blah.
Give references.
SHow Acp formula with phis.


\begin{itemize}
  \item What is it
  \item diagrams
  \item NP in \phis
  \item \phis status
\end{itemize}

\subsubsection{Measuring \phis}
New particles in the box


\begin{itemize}
  \item Show how do you read phis from the decay time distribution.SHow terms in pdf
\end{itemize}


%%%%%%%%%%%%%%%%%%%%%%%%%%%%%%%%%%%%%%%%%%%%%%%%%%%%%%%%%%%%%%%%%%%%%%%%%%%%%%%%%%%%%%%%%
%% The BBbar system%%
%%%%%%%%%%%%%%%%%%%%%%%%%%%%%%%%%%%%%%%%%%%%%%%%%%%%%%%%%%%%%%%%%%%%%%%%%%%%%%%%%%%%%%%%%
