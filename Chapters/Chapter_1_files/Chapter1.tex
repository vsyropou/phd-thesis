% Chapter 1

%%%%%%%%%%%%%%%%%%%%%%%%%%%%%%%%%%%%%%%%%%%%%%%%%%%%%%%%%%%%%%%%%%%%%%%%%%%%%%%%%%%%%%%%%
%% Introduction %%
%%%%%%%%%%%%%%%%%%%%%%%%%%%%%%%%%%%%%%%%%%%%%%%%%%%%%%%%%%%%%%%%%%%%%%%%%%%%%%%%%%%%%%%%%
\chapter{Introduction}
\label{Introduction}

{\color{red} Write a nice introduction after the chapter is complete.}

%
% High energy physics is the field of physics that studies the interactions between elementary particles. The theoretical framework relevant to
% that field of physics is called \textit{Standard Model} of Particle Physics. So far this model has been very successful in describing the experimental
% observations of the high energy physics experiments. However, due to several hints for Physics beyond the \textit{Standard Model}, further testing of
% it is necesarry. This is the reason that lead scientists to build more powerful accelerators, with the most recent one being the \textit{Large Hadron Collider}
% (LHC) at CERN. More powerful machines enables testing of the model at a new energy scale where the model makes certain predictions and it is up to the experiment
%  to confirm or falsify those predictions. LHC is an outstanding combination of accelerator and detector machines designed to explore the experimentally unknown energy
%  region of up to $\sqrt{s}=14 \;\text {TeV}$. Out of the four major detectors at CERN, LHCb is the one designed to study CP-Violation, a puzzling phenomenon
%  responsible for the difference between matter and antimatter. LHCb's strategy to accomplish that is based on the detection of $B$ mesons and their final decay
%  particles, from a primary \textit{proton-proton} collision.
%

%%%%%%%%%%%%%%%%%%%%%%%%%%%%%%%%%%%%%%%%%%%%%%%%%%%%%%%%%%%%%%%%%%%%%%%%%%%%%%%%%%%%%%%%%
%% Standard Model %%
%%%%%%%%%%%%%%%%%%%%%%%%%%%%%%%%%%%%%%%%%%%%%%%%%%%%%%%%%%%%%%%%%%%%%%%%%%%%%%%%%%%%%%%%%
\section{The Standard Model of Particle Physics}

After a few decades of research in the subatomic scale it has been possible to account for a large number of phenomena
in nature, based on the existance of only a handfull of particles. Which is impressive given the vast dimmensions that the
universe spans over. Two distinct categories emerge from the above collection of particles, namelly the {\it gauge
bossons} responsible for mediating three out of the four know fundamental forces of nature, and quarks plus leptons which are the
constituents of matter (and antimatter)\footnote{Anitmatter is not as exotic as it sounds. It is a state of matter where the
signs of all quantum numbers of a particle are fliped.}.The resently discoverd higgs bosson [refff] is a special category by
itself due to each spacial role of explaining why particles have mass.
Figure \figref{} illustrates those matter particles and their interactions via the gauge bossons (also known as force cariers).
The mathematical "language necessary" to describe the expermental data is provided by the \textit{Standard Model} of Particle Physics
which is a quantum field theory\footnote{Quantum field theories are[ref], roughly speaking, the result of combining Einstein's special theory of
relativity[reference] and quantum mechanics[ref].}. In this framework particles are described by quantum fields which is a dificult
to grasp but very usefull way to represent a particle\footnote{For further reading on quantum filed theories [refff]}.

\begin{figure}[h]
\begin{center}
  \includegraphics[width=0.5\textwidth]{Figures/Chapter1/Standard_Model_Particles.png}
  \caption{Standard Model matter particles. Combinations of three and two quarks build hadrons and messons respectivelly.
           Hadrons are for example protons and neutrons wheares a messon could be the \Bs particle which is relevant for the current thesis.
           Leptons can be charged (\electron,\mmu,\mtau) and uncharged (\neue,\neum,\neut). The first of the charged ones is the familiar
           electron that exists in the neucleous of every atom and the rest are heavier "brothers" of it. Netrinos on the other hand
           do not take part in any known nucleous-like formation they are also massless within the Standard Model making them extremelly (and notoriously)
           difficult to detect. They live in the least well known "neighborhoud"of the Standard Model. Their presence is indirectly
           implied in the radioactive decay of atoms and. The gauge bossons \g, g, \Wpm \Z  are responsible for mediating the electromagnetic,
           strong, and weak interactions between quarcks or leptons repsectivelly. These possible interactions are indicated by the blue
           lines. Some of the gauge bossons can interact with themselves which is illustrated by the blue closed loop lines.
           In addition the quarks and the gluon quantum fields have an extra degre of freedom called color. Each quark can have three
           any of the three color degrees of freadom. Which implies that as far as the strong interaction is concerned the the quarks are
           actually 9 and not 3. In case of the gluons the color degree of freadom implies that there are eight gluons resulting from
           the symmetry structure of the strong interaction. Lastly, the vast majority of the observed matter is built from \uquark, \dquark
           quarks and \electron which makes the explanation of the triplet like structure of quarcks and leptons very enigmatic and difficult
           to understand from first principles.}
  \label{sm_particles}
\end{center}
\end{figure}

The above mentioned mathematical language is shown its most compact form in \equref{lagrangian}. The last equation is so called lagrangian
of the Standard Model. Lgrangians are elegant equations that discribe the dynamics of a system. They are used both in classical physical
systems such as motions of planets and in the quantum world of tiny distances. The first {\it kinnetic}, term of \equref{lagrangian}
describes the possible interactions between the quarcks or the leptons. The second, {\it Higgs}, term is the one explainng the masses
of the gauge and higgs bosson. Whereas the last, {\it Yukawa}, attributes masses to the quarks and leptons\footnote{The first two terms
appear more naturealy in the lagrangian wheras the last one is added by hand, which is perfectly fine.}
By construction, the Lagrangian obeys the symmetry group $SU(3)_c\otimes SU(2)_L\otimes U(1)_Y$, which depicts the symmetry structure
of the Strong, Weak and Electromagnetic fields respectively [reference??].

% describe the interactions and couplings between the gauge () and the Higgs
%  ($\phi$) fields. Those couplings emerge from the symmetry requirements of the Lagrangian and are generated by the covariant derivative $D^\mu$.
%  The Yukawa part describes the coupling of the Higgs to the fermion fields and it is added by hand,
%  that describes the interactions of elementary particles [{\color{red} REF??}]. Three out of the four known fundamental
% interactions are described by it, Electromagnetic, Strong and Weak. The symmetries that particles obey as well as their allowed interactions, are described by an elegant mathematical object,
% namely the Lagrangian, shown in \equref{lagrangian}.

\begin{equation}
\mathscr{L}_{SM} = \underbrace{i \bar\psi D^{\mu} \gamma _{\mu} \psi}_{\mathscr{L}_{\text{Kinetic}}} +
                   % \underbrace{(D^\mu\phi)^{\dagger} (D^\mu\phi) -\mu^2\phi^{\dagger}\phi - \lambda(\phi^{\dagger}\phi)^2}_{\mathscr{L}_{\text{Higgs}}} +
                   \underbrace{(D^\mu\phi)^{\dagger} (D^\mu\phi) + V(\phi)}_{\mathscr{L}_{\text{Higgs}}} +
                   \underbrace{Y_{ij}\bar\psi_{Li}\phi\psi_{Rj} \; + \; h.c.}_{\mathscr{L}_{\text{Yukawa}}},
\label{lagrangian}
\end{equation}

In \equref{lagrangian} $\psi$ and $\phi$ are the quark or lepton and higgs quantum fields respectivelly. The $\gamma_mu$ is a Dirac matrix
that takes care of the additional structure of the fields due to their intrinsic spin quantum number. $V(\phi)$ is the potential term of the
higgs field and it contains its mass. The $L$ and $R$ subscripts denote the left and right handed projections of the quarck or lepton fields.
$Y_{ij}$ are called Yuakawa cuplings and they are very important for what follows in \secref{}.
The symbol $D_{mu}$\footnote{$D^\mu=\partial ^\mu + ig_sG_{\alpha}^\mu L_{\alpha} +  igW_b^\mu\sigma_b + ig^\prime B^\mu Y$ \cite{covariant derivative??}}
introduces the interactions of the gauge bossons quantum fields with the quarks or leptons and it is called covariant derivative.

Within the accuracy of the current experiments the Standard Model has seen its predictions confirmed to a great extend.
The most recent and perhaps one of the most crucial meaning the discovery of the mechanism that particles acquire mass, makes the Standard
Model looks quite robust. However there are phenomena and observations that it can not account for. Perhaps the most stricking one is the
absence of any discription about the most familiar and strong force of nature, meaning gravity\footnote{Gravity is not renormalisable[\cite{}]
theory and thus cannot be described as a quantum field theory}. or the peculiar value of the cosmological constant leading to dark energy
interpretations[\cite{}].But even at the heart of the Standard Model there obscure aspects for example the well established fact that neutrinos
have non zero mass[\cite{}] or the unexplained amount of the observed matter-antimatter assymetry in the universe [\cite{}]. For all of the
above reasons plus our curiosity driven nature scientists are compled to continue testing the Standard Model and look for deviations of
its predictions.



%%%%%%%%%%%%%%%%%%%%%%%%%%%%%%%%%%%%%%%%%%%%%%%%%%%%%%%%%%%%%%%%%%%%%%%%%%%%%%%%%%%%%%%%%
%% Flavour Phisics %%
%%%%%%%%%%%%%%%%%%%%%%%%%%%%%%%%%%%%%%%%%%%%%%%%%%%%%%%%%%%%%%%%%%%%%%%%%%%%%%%%%%%%%%%%%
\section{Flavour Physics}
\begin{itemize}
  \item Quark mixing
  \item CKM picture
  \item three generatios anthopic principle.
  \item The falvour physics puzzle, hits for Np, We need everything.
\end{itemize}

%%%%%%%%%%%%%%%%%%%%%%%%%%%%%%%%%%%%%%%%%%%%%%%%%%%%%%%%%%%%%%%%%%%%%%%%%%%%%%%%%%%%%%%%%
%% The BBbar system%%
%%%%%%%%%%%%%%%%%%%%%%%%%%%%%%%%%%%%%%%%%%%%%%%%%%%%%%%%%%%%%%%%%%%%%%%%%%%%%%%%%%%%%%%%%
\section{The $\BBbarSyst$ system and the \phis parameter}
\label{Phenomenology}

\begin{itemize}
  \item Effective hamiltonian for $\BBbarSyst$ (mass and cp eigenstates)
  \item feynman diagrams and qft and lagrangian.s (maybe in the next sectyion)
  \item \phis parameterization
  \item \phis status
  \item NP in \phis
\end{itemize}

%%%%%%%%%%%%%%%%%%%%%%%%%%%%%%%%%%%%%%%%%%%%%%%%%%%%%%%%%%%%%%%%%%%%%%%%%%%%%%%%%%%%%%%%%
%% Higher order effects in phis%%
%%%%%%%%%%%%%%%%%%%%%%%%%%%%%%%%%%%%%%%%%%%%%%%%%%%%%%%%%%%%%%%%%%%%%%%%%%%%%%%%%%%%%%%%%
\section{Higher order effects in \phis and the \BsJpsiKst decay}
\label{TheBsJpsiKstDecay}

\begin{itemize}
\item Decay topologies
\item Feynman diagrams
\item \BsJpsiKst as a control chanell for \phis
\item Estimate of the expected yield
\item Define parameters of interest
\item Flavour specific final state (no oscilations, the sign tags the flavour at production)
\item Event display
\end{itemize}



{\small
\begin{eqnarray}
A^{CP} (B^0_{(s)} \to f_{(s)})  = &  \nonumber \\
\frac{ \displaystyle \int_0^\infty \Big[ \Gamma(B^0_{(s)} \to \bar{f_{(s)}}) + \Gamma(\bar{B^0_{(s)}} \to \bar{f_{(s)}}) \Big]
{\rm d}t \;-\; \int_0^\infty \Big[ \Gamma(B^0_{(s)} \to f_{(s)}) + \Gamma(\bar{B^0_{(s)}} \to f_{(s)}) \Big]
{\rm d}t}{ \displaystyle \int_0^\infty \Big[ \Gamma(B^0_{(s)} \to \bar{f_{(s)}}) + \Gamma(\bar{B^0_{(s)}} \to \bar{f_{(s)}}) \Big] {\rm d}t
 \;+\; \int_0^\infty \Big[ \Gamma(B^0_{(s)} \to f_{(s)}) + \Gamma(\bar{B^0_{(s)}} \to f_{(s)}) \Big] {\rm d}t}  &
\label{decayrates}
\end{eqnarray}
 }

\begin{equation}
%A_i^{CP} = \frac{ |\cAbarfbar(0)|^2 - |\cAf(0)|^2 }{ |\cAbarfbar(0)|^2 + |\cAf(0)|^2 }
A^{CP} = \frac{\Gamma(\bar{B^0_{(s)}} \to \bar{f_{(s)}})-\Gamma(B^0_{(s)} \to f_{(s)})}{\Gamma(\bar{B^0_{(s)}} \to \bar{f_{(s)}})+ \Gamma(B^0_{(s)} \to f_{(s)})}
\end{equation}

{\color{red} Assuming negligible cpv in mixing, Mention this.}

\begin{equation}
A_{CP}^{\rm raw}(B^0_{(s)} \to f_{(s)}) = \frac{N^{\rm obs}(\bar{f_{(s)}}) -N^{\rm obs}(f_{(s)}) }{
N^{\rm obs}(\bar{f_{(s)}}) + N^{\rm obs}(f_{(s)})}
\label{acp_mes}
\end{equation}
