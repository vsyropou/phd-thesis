% Chapter 1

%%%%%%%%%%%%%%%%%%%%%%%%%%%%%%%%%%%%%%%%%%%%%%%%%%%%%%%%%%%%%%%%%%%%%%%%%%%%%%%%%%%%%%%%%
%% Introduction %%
%%%%%%%%%%%%%%%%%%%%%%%%%%%%%%%%%%%%%%%%%%%%%%%%%%%%%%%%%%%%%%%%%%%%%%%%%%%%%%%%%%%%%%%%%
\chapter{Introduction}
\label{Introduction}

{\color{red}
\begin{itemize}
  \item Write a short (half a page) general introduction on what this thesis is about, focusing on the search for new physics.
  \item Maybe cite some textbooks for further reading on standard model, and on cp violation.
\end{itemize}
}

%
% High energy physics is the field of physics that studies the interactions between elementary particles. The theoretical framework relevant to
% that field of physics is called \textit{Standard Model} of Particle Physics. So far this model has been very successful in describing the experimental
% observations of the high energy physics experiments. However, due to several hints for Physics beyond the \textit{Standard Model}, further testing of
% it is necesarry. This is the reason that lead scientists to build more powerful accelerators, with the most recent one being the \textit{Large Hadron Collider}
% (LHC) at CERN. More powerful machines enables testing of the model at a new energy scale where the model makes certain predictions and it is up to the experiment
%  to confirm or falsify those predictions. LHC is an outstanding combination of accelerator and detector machines designed to explore the experimentally unknown energy
%  region of up to $\sqrt{s}=14 \;\text {TeV}$. Out of the four major detectors at CERN, LHCb is the one designed to study CP-Violation, a puzzling phenomenon
%  responsible for the difference between matter and antimatter. LHCb's strategy to accomplish that is based on the detection of $B$ mesons and their final decay
%  particles, from a primary \textit{proton-proton} collision.
%

%%%%%%%%%%%%%%%%%%%%%%%%%%%%%%%%%%%%%%%%%%%%%%%%%%%%%%%%%%%%%%%%%%%%%%%%%%%%%%%%%%%%%%%%%
%% Standard Model %%
%%%%%%%%%%%%%%%%%%%%%%%%%%%%%%%%%%%%%%%%%%%%%%%%%%%%%%%%%%%%%%%%%%%%%%%%%%%%%%%%%%%%%%%%%
\section{The Standard Model and the Weak Interaction}
\label{The_Standard_Model}
After a few decades of research in the subatomic scale it has been possible to account for a large number of phenomena
in nature, based on the existance of only a handfull of particles. Which is impressive given the vast dimmensions that the
universe spans over. Two distinct categories emerge from the above collection of particles, namelly the {\it gauge
bossons} responsible for mediating three out of the four know fundamental forces of nature, and quarks plus leptons which are the
constituents of matter
(and antimatter)\footnote{Anitmatter is not as exotic as it sounds. It is a state of matter where the
signs of all quantum numbers (like the electric charge) of a particle are fliped.}.
The resently discoverd higgs bosson [refff] is a special category by itself due to each spacial role of explaining how particles acquire mass.
Figure \figref{} illustrates those matter particles and their interactions via the gauge bossons (also known as force cariers).
The mathematical "language" necessary to describe those interactions is spelled out by the \textit{Standard Model} of Particle Physics
which is a quantum field
theory\footnote{Quantum field theories are[ref], roughly speaking, the result of combining Einstein's special theory of
relativity[reference] and quantum mechanics[ref].}.
In this framework particles are described by quantum fields which is a dificult
to grasp but very usefull way to represent a
particle\footnote{For further reading on quantum filed theories [refff]}.

\begin{equation}
\mathscr{L}_{SM} = \underbrace{i \bar\psi D^{\mu} \gamma _{\mu} \psi}_{\mathscr{L}_{\text{Kinetic}}} +
                   % \underbrace{(D^\mu\phi)^{\dagger} (D^\mu\phi) -\mu^2\phi^{\dagger}\phi - \lambda(\phi^{\dagger}\phi)^2}_{\mathscr{L}_{\text{Higgs}}} +
                   \underbrace{(D^\mu\phi)^{\dagger} (D^\mu\phi) + V(\phi)}_{\mathscr{L}_{\text{Higgs}}} +
                   \underbrace{Y_{ij}\bar\psi_{Li}\phi\psi_{Rj} \; + \; h.c.}_{\mathscr{L}_{\text{Yukawa}}},
\label{lagrangian}
\end{equation}

% \footnote{$D^\mu=\partial ^\mu + ig_sG_{\alpha}^\mu L_{\alpha} +  igW_b^\mu\sigma_b + ig^\prime B^\mu Y$ \cite{covariant derivative??}}

The above mentioned mathematical language in its most compact form i shown in \equref{lagrangian}. The last equation is the so called
{\it lagrangian}\footnote{Lgrangians are elegant equations that discribe the dynamics of a system. They are used both in classical physical
systems such as motions of planets and in the quantum world of tiny distances.}
of the Standard Model. The first {\it Kinnetic}, term of \equref{lagrangian} describes the possible interactions between the quarcks or the leptons.
The second, {\it Higgs}, term is the one that includes the masses of the gauge and higgs bossons. Whereas the last, {\it Yukawa}, attributes
masses to the quarks and
leptons\footnote{The {\it Kinnetic} and {\it Higgs} terms appear more naturealy in the lagrangian wheras the last one is added by hand, which is perfectly fine.}
Quarcks or leptons and the higgs fields are represented by $\psi$ and $\phi$ in \equref{lagrangian} respectivelly. The $\gamma_\mu$ is a Dirac matrix
that takes care of the additional structure of the fields due to their intrinsic spin. $V(\phi)$ is the potential term of the
higgs field and it contains its mass. The $L$ and $R$ subscripts denote the left and right handed projections of the quarck or lepton fields.
$Y_{ij}$ are called Yuakawa cuplings. Their meaning and relevance to the current thesis is shown in \secref{}.
The symbol $D_\mu$ introduces the interactions of the gauge bossons with the quarks or leptons quantum fields and it is called covariant derivative[ref????].

\begin{figure}[h]
  \begin{center}
    \includegraphics[width=0.5\textwidth]{Figures/Chapter1/Standard_Model_Particles.png}
    \caption{Standard Model matter particles. Combinations of three and two quarks build hadrons and messons respectivelly.
    Hadrons are for example protons and neutrons wheares a messon could be the \Bs particle which is relevant for the current thesis.
    Leptons can be charged (\electron,\mmu,\mtau) and uncharged (\neue,\neum,\neut). The first of the charged ones is the familiar
    electron that exists in the neucleous of every atom and the rest are heavier "brothers" of it. Netrinos on the other hand
    do not take part in any known nucleous-like formation they are also massless within the Standard Model making them extremelly (and notoriously)
    difficult to detect. They live in the least well known "neighborhoud"of the Standard Model. Their presence is indirectly
    implied in the radioactive decay of atoms and. The gauge bossons \g, g, \Wpm \Z  are responsible for mediating the electromagnetic,
    strong, and weak interactions between quarcks or leptons repsectivelly. These possible interactions are indicated by the blue
    lines. Some of the gauge bossons can interact with themselves which is illustrated by the blue closed loop lines.}
    \label{sm_particles}
  \end{center}
\end{figure}

The core element that makes \equref{lagrangian} elegant is the so called {\it local gauge invariance}[ref??].
Briefely speaking this means that \equref{lagrangian} is invariant (or symmetric) under spacetime dependant
phase\footnote{the quark and lepton quantum fields are complex valued quantities. Thus phase here means complex phase.} transformations of the quark or lepton fields.
By construction, the Lagrangian obeys the symmetry
group\footnote{with the mathematical notion of a group implied} $SU(3)_c\otimes SU(2)_L\otimes U(1)_Y$.
This means that there are three distinct transformations of the $\psi$ fields that \equref{lagrangian} is symetric to.
Each one introduces the electromagnetic, weak and strong interactions between the fields. The mathematics
behind the above mentioned transoformations really gives the current paragreaph proper meaning but it would
be completeley out of context here. Further reading on the symmetries of th Standard Model can be found in [?????????]

% In addition the quarks and the gluon quantum fields have an extra degre of freedom called color. Each quark can have three
% any of the three color degrees of freadom. Which implies that as far as the strong interaction is concerned the the quarks are
% actually 9 and not 3. In case of the gluons the color degree of freadom implies that there are eight gluons resulting from
% the symmetry structure of the strong interaction. Lastly, the vast majority of the observed matter is built from \uquark, \dquark
% quarks and \electron which makes the explanation of the triplet like structure of quarcks and leptons very enigmatic and difficult
% to understand from first principles.In addition the quarks and the gluon quantum fields have an extra degre of freedom called color. Each quark can have three
% any of the three color degrees of freadom. Which implies that as far as the strong interaction is concerned the the quarks are
% actually 9 and not 3. In case of the gluons the color degree of freadom implies that there are eight gluons resulting from
% the symmetry structure of the strong interaction.

Within the accuracy of the current experiments the Standard Model has seen its predictions confirmed to a great extend.
The most recent and perhaps one of the most crucial meaning the discovery of the mechanism that particles acquire mass, makes the Standard
Model looks quite robust. However there are phenomena and observations that it can not account for. Perhaps the most stricking one is the
absence of any discription about the most familiar and strong force of nature, meaning gravity\footnote{Gravity is not renormalisable[\cite{}]
theory and thus cannot be described as a quantum field theory}. or the peculiar value of the cosmological constant leading to dark energy
interpretations[\cite{}].But even at the heart of the Standard Model there obscure aspects for example the well established fact that neutrinos
have non zero mass[\cite{}] or the unexplained amount of the observed matter-antimatter assymetry in the universe [\cite{}]. For all of the
above reasons plus our curiosity driven nature scientists are compled to continue testing the Standard Model and look for deviations of
its predictions.


%%%%%%%%%%%%%%%%%%%%%%%%%%%%%%%%%%%%%%%%%%%%%%%%%%%%%%%%%%%%%%%%%%%%%%%%%%%%%%%%%%%%%%%%%
%% Flavour Phisics %%
%%%%%%%%%%%%%%%%%%%%%%%%%%%%%%%%%%%%%%%%%%%%%%%%%%%%%%%%%%%%%%%%%%%%%%%%%%%%%%%%%%%%%%%%%
\section{Flavour Physics}
\label{Flavour_Physics}
As it was mentioned in \secref{The_Standard_Model} quarks and leptons acquire mass through the Yukawa term
of the Standard Model lagrangian. An important aspect of the weak interaction, emerges from that term,
namelly the {\it quark flavour mixing}. A brief description of the last and its relevance to CP-Violation is given in what follows.
The section concludes with introducing the so called {\it CKM-mixing matrix}, which is the core of the
{\it flavor physics} in the quark sector of the Standard Model.

\subsubsection{Quark Flavour Mixing}
By applying the higgs mechanism {\color{red} refff} to the Standard Model lagrangian the higgs field obtains a
{\it vacum expectation value}, which is the lowest value of the higgs field potential.
After this step the Yukawa term for the quark fields is shown in \equref{yukawa_flavour},
 ignoring quark-higgs field interaction terms.

\begin{subequations}
\label{yukawa_flavour}
  \begin{align}
  % -\mathscr{L}_{\text{Yukawa}} = M_{ij}^d \bar{d_{Li}} d_{Rj} + M_{ij}^u \bar{u_{Li}} u_{Rj} + h.c.,
  -\mathscr{L}_{\text{Yukawa}} &= \left[ y_{ij}^d \bar{d}_{Li} d_{Rj} + y_{ij}^u \bar{u}_{Li} u_{Rj} \right] \frac{v}{\sqrt{2}} + h.c. + ...  \\
                               &= \left[ m_{ij}^d \bar{d}_{Li} d_{Rj} + m_{ij}^u \bar{u}_{Li} u_{Rj} \right] + h.c. + ...,  \\
                               \text{with} \;\;\; m^{u,d}_{i,j} = \frac{v}{\sqrt{2}} y_{ij}^{u,d} & \nonumber
  \end{align}
\end{subequations}

\noindent where $v$ is the higgs vacum expectation value and $y_{ij}^{u,d}$ are complex valued numbers called {\it yukawa couplings}.
The last are free parametres that represent the coupling strength between higgs and quark fields.
Note that for equations \equref{yukawa_flavour} to \equref{CClagrangian} it is implied that $u$ and $d$ indicate
an up or a down type quark. The exact generation of the up(down) type quark is specified by the indices $i,j$.
Whereas, indices $L,R\;$ idicate the left or right handedness of the quark field.
Finally the matrix $\bf m^{u,d}$ expresees the desired quark masses.

The quark fields, $u$ and $d$,  in \equref{yukawa_flavour} have a definite quantum number that labels the generation to which they belong
to and also wheather they are of up or down type. This quantum number is commonly called {\it flavour} and thus the quark fields
are flavour eigenstates. By constraction the mass matrix is not diagonal which means that
a quark with a well defined flavour does not have a well defined mass. Or in more formal pharasing flavour and mass eigenstates of
the quark fields do not coincide. In order to obtain propper quark masses the matrix $\bf m^{u,d}$ has to be diagonalised, as shown in \equref{diagM}.

\begin{equation}
  m^{d,u}_{\text diag} = V_L^{d,u} m^{d,u} \left(V_R^{d,u}\right)^{\dagger},
  \label{diagM}
\end{equation}

\noindent where the matrices $V$ are required to be unitary. Since \equref{yukawa_flavour} has to stay intact after $m^{d,u}$ is replaced with
$m^{d,u}_{\text diag}$, the quark fields need to be rotated, as shown in \equref{quark_rotation} such that they cancell the additional $V$ matrices
of \equref{diagM}.

\begin{equation}
  \left( d_{i}^m \right)_{L,R} = \left( V^d_{ij} d_{j} \right)_{L,R}, \;\;\;\; \left( u_{i}^m \right)_{L,R} = \left( V^u_{ij} u_{j} \right)_{L,R}
  \label{quark_rotation}
\end{equation}

\noindent Since the quark fields are still flavour eigenstetes in the rest of the Standard Model lagrangian,
the field rotations of \equref{quark_rotation} need to be applied there as well. Specifically
to the kinetic term involving quark interactions with the charged weak bosons \Wpm, also known as {\it charged current}
interaction, shown in \equref{CClagrangian}. In the last equation the charged current interaction is exressed in two ways.
Once with quark fields expresed as flavour eigenstates, \equref{CClagrangianInt}, and second with the quarks as mass eigenstates, \equref{CClagrangianMass}.

\begin{subequations}
  \label{CClagrangian}
  \begin{align}
    \mathscr{L}_{\text{Kinetic}}^{CC} & \propto \bar{u}_{Li} \gamma_\mu {\Wm}^\mu d_{Ri} + \bar{d}_{Li} \gamma_\mu {\Wp}^\mu u_{Ri}  \label{CClagrangianInt} \\
                                      & \propto \bar{u}_{Li}^m  {V_{\text{CKM}}} \gamma_\mu {\Wm}^\mu d_{Ri}^m + \bar{d}_{Li}^m V_{\text{CKM}} {\Wp}^\mu \gamma_\mu u_{Ri}^m \label{CClagrangianMass} \\
                                      \text{with} \;\;\; V_{\text{CKM}} \equiv V^u_LV^{d\dagger}_L, & \nonumber
  \end{align}
\end{subequations}

\noindent and \Wpm the charged weak boson fields, whereas $\gamma_\mu$ are Dirac matrices.
The matrix $V_{\text{CKM}}$ is the so called {\it CKM mixing matrix}, shown in \equref{quark_field_rotation}.

\begin{equation}
  \begin{pmatrix} \dquark \\ \squark \\ \bquark  \end{pmatrix} =
  \underbrace{\begin{pmatrix} \Vud & \Vus & \Vub \\ \Vcd & \Vcs & \Vcb \\ \Vtd & \Vts & \Vtb \end{pmatrix}}_{V_{\text{CKM}}}
    \begin{pmatrix} \dquark^m \\ \squark^m \\ \bquark^m  \end{pmatrix}
  \label{quark_field_rotation}
  \end{equation}

The charged current term, \equref{CClagrangian}, incorporates an important aspect of the weak interaction,
namelly the {\it quark flavor mixing}. The last is due to the fact that quark mass eigenstates are superpositions
of the flavour eigenstates.

\begin{figure}[h]
  \centering
  {\sffamily 

\hspace*{0.05\textwidth}
\begin{fmffile}{Figures/Chapter1/qqMixing}
  \fmfframe(8,16)(8,16){
    \begin{fmfgraph*}(60,35)
      \fmfstraight
      \fmfleft{u}
      \fmfright{d,W}
      \fmf{fermion}{u,V,d}
      \fmf{boson}{V,W}
      \fmflabel{$u_i$}{u}
      \fmflabel{$d_j$}{d}
      \fmflabel{$\Wp$}{W}
      \fmflabel{\hspace{0.2cm}$V_{ij}$}{V}
    \end{fmfgraph*}
  }
\end{fmffile}

%
% \hspace*{0.05\textwidth}
% \begin{fmffile}{Figures/Chapter1/udMixing}
%   \fmfframe(8,16)(8,16){
%     \begin{fmfgraph*}(60,35)
%       \fmfstraight
%       \fmfleft{u}
%       \fmfright{d,W}
%       \fmf{fermion}{u,V,d}
%       \fmf{boson}{V,W}
%       \fmflabel{$u_i$}{u}
%       \fmflabel{$d^m_j$}{d}
%       \fmflabel{$\Wp$}{W}
%       \fmflabel{\hspace{0.2cm}$V_{id}$}{V}
%     \end{fmfgraph*}
%   }
% \end{fmffile}
% \hspace*{0.05\textwidth}
% \begin{fmffile}{Figures/Chapter1/usMixing}
%   \fmfframe(8,16)(8,16){
%     \begin{fmfgraph*}(60,35)
%       \fmfstraight
%       \fmfleft{u}
%       \fmfright{d,W}
%       \fmf{fermion}{u,V,d}
%       \fmf{boson}{V,W}
%       \fmflabel{$u_i$}{u}
%       \fmflabel{$s^m_j$}{d}
%       \fmflabel{$\Wp$}{W}
%       \fmflabel{\hspace{0.2cm}$V_{is}$}{V}
%     \end{fmfgraph*}
%   }
% \end{fmffile}
% \hspace*{0.05\textwidth}
% \begin{fmffile}{Figures/Chapter1/ubMixing}
%   \fmfframe(8,16)(8,16){
%     \begin{fmfgraph*}(60,35)
%       \fmfstraight
%       \fmfleft{u}
%       \fmfright{d,W}
%       \fmf{fermion}{u,V,d}
%       \fmf{boson}{V,W}
%       \fmflabel{$u_i$}{u}
%       \fmflabel{$b^m_j$}{d}
%       \fmflabel{$\Wp$}{W}
%       \fmflabel{\hspace{0.2cm}$V_{ib}$}{V}
%     \end{fmfgraph*}
%   }
% \end{fmffile}
}
  \caption{Feynman diagram where an up-type quark couples to any of the three, ($d,s,b$), down type quarks,
           via a \Wp boson. Time flows from left to right.}
  \label{QuarkMixing}
\end{figure}

\noindent By construction the $V_{\text{CKM}}$ rotates only the down type quarks, impling that
the mass eigenstates of the up-type quarks are identical to the flavour eigenstates. Thus an up type quark
changes its flavour to any of the down type quarks, see \figref{QuarkMixing}, with a certain probabilty.
The probability of such a transitions is given by the elements of the CKM mixing matrix, or simply CKM matrix.

\subsubsection{CKM mixing matrix}
The CKM matrix, is a complex matrix which as already mentioned describes the streanght of quark couplings,
or in other words the probability of a certain quark flavour transition. The elements of the CKM matrix have been measured, see Chapter 12 of~\cite{PDG},
showing an intreasting structure, see \equref{CKMmatrix}. Essentially the structure implies that transitions between generations
are supresed with respect to transitions within the same generation, in a symmetric way. The most suppressed transitions are between
the first and third generations followed by the ones between the second and third and the least suppressed are between first and second.

\begin{equation}
  |V_{\text{CKM}}|
                   = \begin{pmatrix} \VudMag & \VusMag & \VubMag \\ \VcdMag & \VcsMag & \VcbMag \\ \VtdMag & \VtsMag & \VtbMag \end{pmatrix}
              \simeq \begin{pmatrix} 1 & 0.2 & 0.008 \\ 0.2 & 1 & 0.04 \\ 0.008 & 0.04 & 1 \end{pmatrix}
      \label{CKMmatrix}
  \end{equation}

Testing the consistency of the CKM elements measurments is a central goal of flavour physics.
In order to achieve such tests it is usefull to define a paramterization of the CKM matrix.
By construction the CKM matrix has 3 real parameters and one compelx phase\footnote{After exploiting the unitarity of $V_{\rm CKM}$ and all the redundant quark field phases.}
The choise of the CKM matrix parametrazation is arbitrary. However due to the observed structure the so called {\it Wolfenstein}~\cite{Wolfenstein:1983yz,Buras-wolfenstein}
parametrization shown in \equref{CKMwolfenstein} is a standard parametrization. Where the three real parameters, $\lambda,A,\rho$
and the complex one $\eta$ can be seen.

\begin{equation}
  |V_{\text{CKM}}|
                   = \begin{pmatrix} \VudWolf & \VusWolf & \VubWolf \\
                                     \VcdWolf & \VcsWolf & \VcbWolf \\
                                     \VtdWolf & \VtsWolf & \VtbWolf \end{pmatrix} + \mathcal{O}(\lambda^4)
      \label{CKMwolfenstein}
\end{equation}

\noindent As it was previously mentioned the CKM matrix is a unitary matrix, meaning that $V_{\text{CKM}} V_{\text{CKM}}^\dagger = I_{3x3}$.
This leads to the so called unitarity and orthogonality relations. The last are sums of complex numbers that are equal to zero, thus can be
represented by triangles in the complex plane. There are six orthogonality releations two of which are relevant for the current thesis since
the CKM elements present in those relations govern the dynamics in the \Bs and \Bd meson systems. These two relations are shown in
\equref{unitConstraints}. The last equations are devidied by $\Vcd\Vcb^*$ for \Bd and $\Vcs\Vcb^*$ for \Bs and then illustrated in \figref{unitTriangles}.
Note that the CKM element $\Vts$ has a complex part at higher order in $\lambda$, see section 13.3 of ~\cite{PDG}.

\begin{subequations}
  \label{unitConstraints}
  \begin{align}
    \Bd : & \quad \Vud\Vub^* + \Vcd\Vcb^* + \Vtd\Vtb^* = 0
    \label{unitConstraints_Bd} \\
    \Bs : & \quad \Vus\Vub^* + \Vcs\Vcb^* + \Vts\Vtb^* = 0
    \label{unitConstraints_Bs}
  \end{align}
\end{subequations}

\begin{figure}[h]
  \centering
  \begin{subfigure}{0.475\textwidth}
    \raggedright
    \includegraphics[width=\textwidth]{Figures/Chapter1/b-d-triangle}
    \caption{}
    \label{unitTriangles_bd}
  \end{subfigure}%
  \begin{subfigure}{0.525\textwidth}
    \raggedleft
    \includegraphics[width=\textwidth]{Figures/Chapter1/b-s-triangle}
    \caption{}
    \label{unitTriangles_bs}
  \end{subfigure}
  \caption{CKM-unitarity triangles. (A) \Bd triangle, corresponding to \equref{unitConstraints_Bd}. (B) \Bs triangle,
           corresponding to \equref{unitConstraints_Bs}. Triangle sides have been normalised with respect to to one of them.
           This way one of the sides is real with unit leangth. Note that triangles are not drawn to scale. Figures from~\cite{jeroenThesis}. }
  \label{unitTriangles}
\end{figure}

The coordinates of the apex in the two triangles are defined as $(\bar{\rho},\bar{\eta})$ and $(\bar{\rho}_s,\bar{\eta}_s)$ respectivelly for \figref{unitTriangles_bd} and
\figref{unitTriangles_bs}. Whreas by inspecting the triangles one can define the angles as shown in \equref{bdAnglesDef} and \equref{bsAnglesDef}.

\begin{equation}
  \label{bdAnglesDef}
  \alpha \equiv \arg\left( -\frac{\Vtd\Vtb^*}{\Vud\Vub^*} \right)
  \quad
  \beta  \equiv \arg\left( -\frac{\Vcd\Vcb^*}{\Vtd\Vtb^*} \right)
  \quad
  \gamma \equiv \arg\left( -\frac{\Vud\Vub^*}{\Vcd\Vcb^*} \right)
\end{equation}

\begin{equation}
  \label{bsAnglesDef}
  %\alpha_s \equiv \arg\left( -\frac{\Vus\Vub^*}{\Vts\Vtb^*} \right)
  %\quad
  \beta_s \equiv \arg\left( -\frac{\Vts\Vtb^*}{\Vcs\Vcb^*} \right)
  %\quad
  %\gamma_s \equiv \arg\left( -\frac{\Vcs\Vcb^*}{\Vus\Vub^*} \right)
\end{equation}

\begin{figure}[h]
  \begin{center}
    \includegraphics[trim=0cm 0cm 0cm 0cm, clip=true, width=\textwidth]{Figures/Chapter1/rhoeta_large.png}
    \caption{Global fit of the apex $(\bar{\rho},\bar{\eta)}$ of the \Bd triangle by the CKM fitter group~\cite{ckm-fitter-phis-pred}.
             Inputs from measurements of flavor physics observables are indicated by colored bands.}
    \label{unitarity_triangle}
  \end{center}
\end{figure}

There is no fundamental reason known in the Standard model for the observed hierarchy of the CKM elemetns.
One of the main goals of flavour physics is to check the consistency of the CKM picture.
Overlaying many measurements of flavour physics observables in the complex plane should show a consistent
picture, see \figref{unitarity_triangle}. For completnes it is intreasting to mention that in the lepton
sector a similar mixing matrix is active. The hierarchy there is completelly different which is yet another
intriguing detail of the Standard Model.


\subsubsection{CP-Violation and fermion masses}
It is intreasting to point out the common origin of CP-Violation and fermion masses in the Standard Model~\cite{KM-mechanism}.
From the one hand the CKM matrix which was introduced to couple the higgs to the fermions and thus give them mass.
On the other hand the fact that $V_{\rm CKM}$ is a complex valued matrix allows for CP-Violation in the
Standard Model by the weak intraction. This is because the charged current interaction of \equref{CClagrangianMass}
will be invariant under the CP operation only if $V_{\rm CKM}=V_{\rm CKM}^*$, which is not true.


%%%%%%%%%%%%%%%%%%%%%%%%%%%%%%%%%%%%%%%%%%%%%%%%%%%%%%%%%%%%%%%%%%%%%%%%%%%%%%%%%%%%%%%%%
%% The BBbar system%%
%%%%%%%%%%%%%%%%%%%%%%%%%%%%%%%%%%%%%%%%%%%%%%%%%%%%%%%%%%%%%%%%%%%%%%%%%%%%%%%%%%%%%%%%%
\section{The \Bs Meson and CP Violation}
\label{Phenomenology}


The \Bs meson\footnote{Mesons are bound by the strong interaction states of two quarks.} is an electrically neutral
particle that consists of two quarks, particularly (\bquarkbar\squark). An important feature of
the \Bs meson, and all neutral mesons, is that it can spontaneously change into its antiparticle,
the \Bsb meson and vice versa. This feauture is called {\it meson oscilations} and it is posible
in the Standard Model via the so called {\it box diagram} of \figref{bs_box}.
Meson oscilations play a central role in flavor physics. Particularly because they are sensitive to the
existance of new particles as it will be explained in \secref{probe_new_phys}. CP-Violating effects can
apper in the \BBbarSyst oscilations and in the subsequent decay. The current section as well
as \secref{WeakPhase} will qualitatively address the way that these effects manifest themselves.

\begin{figure}[h]
  \centering
  \begin{subfigure}{0.5\textwidth}
    \centering
    {\sffamily \input{Figures/Chapter1/box1}}
    \caption{}
    \label{bs_box_1}
  \end{subfigure}%
  \begin{subfigure}{0.5\textwidth}
    \centering
    {\sffamily %%BoundingBox: -8 0 123 75
%%HiResBoundingBox: -8 0 122.57008 74.71962

\begin{fmffile}{Figures/Chapter1/box2}
  \fmfframe(19,3)(17,3){
    \begin{fmfgraph*}(115,75)
      \fmfbottom{i1,d1,o1}%dummy vertex
      \fmftop{i2,d2,o2}
      \fmf{fermion,label=s,l.side=left}{i1,v1}
      \fmf{fermion,label=b,l.side=left}{v2,o1}
      \fmf{fermion,label=b,l.side=left}{v3,i2}
      \fmf{fermion,label=s,l.side=left}{o2,v4}
      \fmf{boson,label=W$^-$,l.side=left}{v1,v2}
      \fmf{boson,label=W$^+$,l.side=left}{v4,v3}
      \fmffreeze
      \fmf{fermion,label={t,,c,,u},l.side=left}{v4,v2}
      \fmf{fermion,label={t,,c,,u},l.side=left}{v1,v3}
      \fmf{plain,left=0.2}{o1,o2}
      \fmf{plain,left=0.2,label=$\Bsb$}{o2,o1}
      \fmf{plain,left=0.2}{i2,i1}
      \fmf{plain,left=0.2,label=$\Bs$}{i1,i2}
    \end{fmfgraph*}
  }
\end{fmffile}
}
    \caption{}
    \label{bs_box_2}
  \end{subfigure}
  \caption{Leading order diagrams for \BBbarSyst oscilations. Figures from~\cite{jeroenThesis}.}
  \label{bs_box}
\end{figure}

\subsubsection{The \BBbarSyst System}

The oscilating behaviour of the \Bs and \Bsb mesons invites a phenomenological approach
where the two mesons are treated as a coupled quantum mechanical system. This approach
is based on~\cite{Weisskopf:1930au,Weisskopf:1930ps} approach and uses an {\it effective hamiltonian}~\cite{eff-hamiltonian-bs-syst,DeBruyn-thesis}
to describe the time evolution of the \Bs or \Bsb meson. The wavefunctions of the last mesons
are defined in \equref{bs_wavefunctions}

\begin{equation}
\ket{\Bs}  \equiv  \ket{\bquarkbar\squark}, \;\;\; \ket{\Bsb} \equiv  \ket{\bquark\squarkbar}
\label{bs_wavefunctions}
\end{equation}

\noindent impling that the states \Bs and \Bsb have a definite quark content or in other words
thei are flavour eigenstates. The effective hamiltonian, $H_{\rm eff}$ is a 2x2 non diagonal non hermitian matrix.
The non diagonal feature is due to the oscilating feature of the \BBbarSyst system whreas the non
hermirianity takes into acount the probability that the system will eventually decay to some final state.
By diagonalising the effective on can obtain the mass eigenstates, \ket{\Bs{}_{,H}}, \ket{\Bs{}_{,L}}, of the system, shown in \equref{bs_mass_eigen}.
More details on the exact calcualtions can be found in section 13.1 of~\cite{PDG} and in ~\cite{jeroenThesis,DeBruyn-thesis}.

\begin{align}
\ket{\Bs{}_{,H}} &= p \ket{\Bs} + q \ket{\Bsb}, \nonumber \\
\ket{\Bs{}_{,L}} &= p \ket{\Bs} - q \ket{\Bsb}
\label{bs_mass_eigen}
\end{align}

The amount of mixing between the mass and flavour eigenstates is governed by the complex parameters $p,q$.
The ratio $|\qoverp|$ is compatible with one. This is supported both from theoretical calculations~\cite{Lenz:2011ti}
and from experimental measurement~\cite{asl-paper} measurements\footnote{Some tensions between theory and experiment will be resolved in the future. The tensions are
mainly fue to the \dzero measuremnt~\cite{Abazov:2013uma}}. In addition the difference in mass and lifetime between the two
mass eigenstates are important observables of the \BBbarSyst and are denoted as $\Delta m_s$ and $\Delta\Gamma_s$ respectively.

\subsubsection{Types of CP-Violation}
The ratio $|\qoverp|$ is connected a type of CP-Violation. Specifically the CP-Violation {\it in the mixing} (of the \BBbarSyst system).
The connection is shown in the asymmetry of \equref{acp_mixing}.

\begin{equation}
\Acp{\text{mix}}      = \frac{\bra{\Bsb}\ket{\Bs} - \bra{\Bs}\ket{\Bsb}} {\bra{\Bsb}\ket{\Bs} + \bra{\Bs}\ket{\Bsb}}
                \propto \left|\frac{q}{p}\right|^2 \parenthesis{ 1 - \left|\frac{p}{q}\right|^4},
\label{acp_mixing}
\end{equation}

\noindent where it can be seen that $\Acp{\text{mix}}$ vanshes if $|\qoverp|= 1$.

Another type of CP-Violation is called {\it in the decay} and its based on the amplitude, $A_f$, of a certain meson decay to some final state $f$.
Using the CP conjugated $\bar{A}_{\bar{f}}$ amplitude the asymmetry of \equref{cpv_decay} can be constructed.
The last asymmetry vanishes in case $|\nicefrac{\bar{A}_{\bar{f}}}{A_f}| = 1$  and it is also relevant to other charged mesons as well.

\begin{equation}
\Acp{\text{dir}} = \frac{\Gamma(\Bs \rightarrow f) - \Gamma(\Bsb \rightarrow \bar{f})} {\Gamma(\Bs \rightarrow f) + \Gamma(\Bsb \rightarrow \bar{f})}
                = \frac{ |\nicefrac{\bar{A}_{\bar{f}}}{A_f}|^2 - 1}{|\nicefrac{\bar{A}_{\bar{f}}}{A_f}|^2 + 1}
\label{cpv_decay}
\end{equation}

\noindent where $\Gamma$ is the decay rate of the $\Bs \rightarrow f$ process and $f$ is some final state.
Both calsifications are based on section 13.1.4 of~\cite{PDG}.

There is one more type of CP Violation and that is in the {\it interference} between
a neutral meson decying directlly to a final state $f$ or first oscialting to its antiparticle and then decaying
to the same final state $f$. Due to its relevance to the current thesis, this type of CP-Violation is explained
in the next section.


%%%%%%%%%%%%%%%%%%%%%%%%%%%%%%%%%%%%%%%%%%%%%%%%%%%%%%%%%%%%%%%%%%%%%%%%%%%%%%%%%%%%%%%%%
%% The BBbar system%%
%%%%%%%%%%%%%%%%%%%%%%%%%%%%%%%%%%%%%%%%%%%%%%%%%%%%%%%%%%%%%%%%%%%%%%%%%%%%%%%%%%%%%%%%%
\section{The Weak Phase \phis}
\label{WeakPhase}


{\color{red}In this section we desctibe blah blah and blah}

\subsubsection{The \phis parameter in $b \rightarrow ssl$ transitions}
\phis is a parameter that arises in the b2ssl transitions, which are CP eigenstates.
See diagram,
It is an overal phase difference of two diagrams. The decay and the mixing.

The assymetry can be computed by inspectign the diagrams using the pahse definitions in blah.
Give references.
SHow Acp formula with phis.


\begin{itemize}
  \item What is it
  \item diagrams
  \item NP in \phis
  \item \phis status
\end{itemize}

\subsubsection{Measuring \phis}
New particles in the box


\begin{itemize}
  \item Show how do you read phis from the decay time distribution.SHow terms in pdf
\end{itemize}


%%%%%%%%%%%%%%%%%%%%%%%%%%%%%%%%%%%%%%%%%%%%%%%%%%%%%%%%%%%%%%%%%%%%%%%%%%%%%%%%%%%%%%%%%
%% The BBbar system%%
%%%%%%%%%%%%%%%%%%%%%%%%%%%%%%%%%%%%%%%%%%%%%%%%%%%%%%%%%%%%%%%%%%%%%%%%%%%%%%%%%%%%%%%%%
\section{The \lhcb Detector}
\label{lhcb_detector}


\begin{itemize}
  \item Describe the relevance and performance of the subdetectors relevant ot my analys
  \item Empliseize a bit more to the trigger.
  \item Show an event dispaly
\end{itemize}


\subsection{Event reconstruction.}
\label{reconstruction}

\subsection{The \BsJpsiKst decay at \lhcb}
\label{BspsiKst_at_lhcb}

\begin{itemize}
  \item Production of \Bs mesons
  \item Event display
\end{itemize}

\subsection{Online event selection}

