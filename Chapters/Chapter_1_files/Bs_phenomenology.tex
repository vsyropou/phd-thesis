The \Bs meson is an electrically neutral particle that consists of a quark anti-quark pair, specifically (\bquarkbar\squark).
The two constituents of the mesons are bound by the strong interaction. An important feature of
the \Bs meson, and all heavy neutral mesons, is that it can change into its antiparticle,
the \Bsb meson and \viceversa. This feature is called {\it meson oscillations} and it is possible
in the Standard Model via  {\it box diagram} of \figref{bs_box}.
Meson oscillations play a central role in flavor physics. Particularly because they are sensitive to the
existence of new particles as will be explained in \secref{probe_new_phys}. \CP violating effects can
appear in \BBbarSyst oscillations, in the subsequent decay and in the interference between oscillation
and decay. The current section as well as \secref{WeakPhase} qualitatively addresses the way that
these effects manifest themselves.

\begin{figure}[!h]
  \centering
  \begin{subfigure}{0.5\textwidth}
    \centering
    \raggedright
    \scalebox{0.9}{\sffamily \input{Figures/Chapter1/box1}}
    \caption{}
    \label{bs_box_1}
  \end{subfigure}%
  \hfill%
  \begin{subfigure}{0.5\textwidth}
    \centering
    \raggedleft
    \scalebox{0.9}{\sffamily %%BoundingBox: -8 0 123 75
%%HiResBoundingBox: -8 0 122.57008 74.71962

\begin{fmffile}{Figures/Chapter1/box2}
  \fmfframe(19,3)(17,3){
    \begin{fmfgraph*}(115,75)
      \fmfbottom{i1,d1,o1}%dummy vertex
      \fmftop{i2,d2,o2}
      \fmf{fermion,label=s,l.side=left}{i1,v1}
      \fmf{fermion,label=b,l.side=left}{v2,o1}
      \fmf{fermion,label=b,l.side=left}{v3,i2}
      \fmf{fermion,label=s,l.side=left}{o2,v4}
      \fmf{boson,label=W$^-$,l.side=left}{v1,v2}
      \fmf{boson,label=W$^+$,l.side=left}{v4,v3}
      \fmffreeze
      \fmf{fermion,label={t,,c,,u},l.side=left}{v4,v2}
      \fmf{fermion,label={t,,c,,u},l.side=left}{v1,v3}
      \fmf{plain,left=0.2}{o1,o2}
      \fmf{plain,left=0.2,label=$\Bsb$}{o2,o1}
      \fmf{plain,left=0.2}{i2,i1}
      \fmf{plain,left=0.2,label=$\Bs$}{i1,i2}
    \end{fmfgraph*}
  }
\end{fmffile}
}
    \caption{}
    \label{bs_box_2}
  \end{subfigure}
  \caption{Leading order diagrams for \BBbarSyst oscillations. Figures from \cite{jeroenThesis}.}
  \label{bs_box}
\end{figure}

\subsubsection{The \BBbarSyst System}
\label{the_bbar_system}

The behavior of the \BBbarSyst is descibed phenomenologically, where the two mesons are treated as a
coupled quantum mechanical system. This approach
is based on \cite{Weisskopf:1930au,Weisskopf:1930ps} and uses an {\it effective Hamiltonian}
\cite{eff-hamiltonian-bs-syst,DeBruyn-thesis} to describe the time evolution of the system.
The wave-functions of the mesons are:

\begin{equation}
  \centering
  \ket{\Bs}  \equiv  \ket{\bquarkbar\squark}, \;\;\; \ket{\Bsb} \equiv  \ket{\bquark\squarkbar},
  \label{bs_wavefunctions}
\end{equation}

\noindent implying that the states \Bs and \Bsb have a definite quark content, or in other words,
they are flavor eigenstates. The effective Hamiltonian, \Heff, is a 2x2 non-diagonal non-Hermitian matrix.
The non-diagonal feature is due to the presence of \BBbarSyst transition amplitudes,
whereas the non Hermitian description takes into account the probability that the system will
eventually decay to some final state. By diagonalizing the effective Hamiltonian one can obtain
the mass eigenstates, \ket{\Bs{}_{,H}}, \ket{\Bs{}_{,L}}, of the system:

\begin{align}
  \centering
  \ket{\Bs{}_{,H}} &= p \ket{\Bs} + q \ket{\Bsb}, \nonumber \\
  \ket{\Bs{}_{,L}} &= p \ket{\Bs} - q \ket{\Bsb}.
  \label{bs_mass_eigen}
\end{align}

\noindent More details on the exact calculations can be found both in section 13.1
of \cite{PDG} and in \cite{jeroenThesis,DeBruyn-thesis}. The amount of mixing between the mass
and flavor eigenstates is governed by the complex parameters $p,q$.
Theoretical calculations \cite{Lenz:2011ti} and experimental measurement \cite{asl-paper} point to the ratio
\qoverp being compatible with one. In addition the difference in mass and lifetime between the two mass
eigenstates are important observables of the \BBbarSyst system and are denoted as $\Delta m_s$ and $\Delta\Gamma_s$ respectively.

\subsubsection{Types of \CP violation}
The ratio $|\qoverp|$ is related to a type of \CP violation. Specifically the \CP violation {\it in the mixing} (of the \BBbarSyst system).
The relation appears in the  {\it semileptonic asymmetry} $a_{\rm sl}^{s}$:

\begin{equation}
  \centering
  a_{\rm sl}^{s}  \equiv \frac{\bra{\Bsb}\Heff\ket{\Bs} - \bra{\Bs}\Heff\ket{\Bsb}} {\bra{\Bsb}\Heff\ket{\Bs} + \bra{\Bs}\Heff\ket{\Bsb}}
                       = \frac{1 - \left|\nicefrac{q}{p}\right|^4}{1 + \left|\nicefrac{p}{q}\right|^4},
  \label{acp_mixing}
\end{equation}

\noindent where it can be seen that the asymmetry vanishes if $|\qoverp|= 1$. The subscript "sl" stands for a
semileptonic final state, that is typically used to measure this type of asymmetries. A semileptonic final state
consists of mesons and leptons. From the lepton charge it can be inferred whether the neutral meson oscillated or not.

Another type of \CP violation, the  {\it in decay} or {\it direct} \CP violation, is based on the
amplitude, $A_f$, of a certain meson decay to some final state $f$.
Using the \CP conjugated $\bar{A}_{\bar{f}}$ amplitude the asymmetry of \equref{cpv_decay} can be constructed.
This asymmetry vanishes in case $|\nicefrac{\bar{A}_{\bar{f}}}{A_f}| = 1$.

\begin{equation}
  \centering
  \Acp{\text{dir}} = \frac{\Gamma(\Bs \rightarrow f) - \Gamma(\Bsb \rightarrow \bar{f})} {\Gamma(\Bs \rightarrow f) + \Gamma(\Bsb \rightarrow \bar{f})}
                   = \frac{ |\nicefrac{\bar{A}_{\bar{f}}}{A_f}|^2 - 1}{|\nicefrac{\bar{A}_{\bar{f}}}{A_f}|^2 + 1}.
\label{cpv_decay}
\end{equation}

Finally there is one more type of \CP violation and that originates in the {\it interference} between
a neutral meson decaying directly to a final state $f$ or oscillating to its antiparticle and then decaying
to the same final state $f$. Due to its relevance to the current thesis, this type of \CP violation is explained
in more detail in the next section. All the above classifications are based on section 13.1.4 of \cite{PDG}.
