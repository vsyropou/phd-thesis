

{\color{red} In this section I describe blah blah and blah.
CP-Violation in the Standard Model manifests only when the weak interaction is involved.}

\subsubsection{The \BBbarSyst system}
The \Bs meson\footnote{Mesons are bound states of two quarks.} is an electrically neutral
particle that consists of two quarks, a \bquarkbar and an \squark. An important feature of
the \Bs meson, and all neutral mesons, is that it can spontaneously change into its antiparticle,
the \Bsb meson and vice versa. This oscilating feauture is posible in the Standard Model via the
so called {\it box diagram}, see \figref{bs_box}. {\it Neutral Meson oscilations}, as they commonly
quoted, play a central role in understanding and constraining the CKM matrix. Furthermore they are
usefull probes of new particles as it will be explained soon.

The \Bs and \Bsb mesons in \figref{bs_box} are flavour eigenstates. Since the weak intreraction
is active in those oscilations there is no reason to expect that the \Bs (and \Bsb) flavour eigenstates
coincide with the mass eigenstates. For all practical purposes, this situation invites a phenomenological
aproach when describing the time evolution of a \Bs meson[{\color{red} pdg oscilations ref}]. Specifically
an {\it effective hamiltonian} [{\color{red} ref rob or something} ] is used to describe the \Bs and \Bsb
as a quantum mechanical system. The effective hamiltonian for the \BBbarSyst is a $2x2$\footnote{non hermitian}
matrix, with the off-diagonal elements acounting for the \Bs to \Bsb oscilations.
The mass eigenstates of the $\BBbarSyst$ are obtained after diagonalizing the effective hamiltonian, yielding \equref{bs_mass_eigen}.

\begin{align}
\ket{\Bs{}_{,H}} &= p \ket{\Bs} + q \ket{\Bsb}, \nonumber \\
\ket{\Bs{}_{,L}} &= p \ket{\Bs} - q \ket{\Bsb}
\label{bs_mass_eigen}
\end{align}

\noindent Where \ket{\Bs}, \ket{\Bsb} and \ket{\Bs{}_{,H}}, \ket{\Bs{}_{,L}} are the flavour and mass
eigenstates respectively. The amount of mixing between the two types of eigenstates is governed by
the parameters $p,q$. The last are releated to the elements of the effective hamiltonian and can be
copmputed theoretically{\color{red} give ref}. Experimentally and theoretically it has been found{\color{red} ref for no cpv in mixing.}
that their ratio for the \BBbarSyst is one within the uncertainties.

\subsubsection{Types of CP-Violation}
An important\footnote{and usefull for the next section} message releated to the ratio $\nicefrac{q}{p}$ is that
the last characterizes a type of CP-Violation. In more detail the so called CP-Violation {\it in the mixing} (of neutral mesons),
$Acp{\text{mix}}$, vanshes if $\nicefrac{q}{p} = 1$, see \equref{acp_mixing}.

\begin{equation}
\Acp{\text{mix}} \propto \brackets{\bra{\Bsb}\ket{\Bs} - \bra{\Bs}\ket{\Bsb}} \propto \left|\frac{q}{p}\right|^2 \parenthesis{ 1 - \left|\frac{p}{q}\right|^4}
\label{cpv_mixing}
\end{equation}

\noindent In addition it is usefull to define another type of CP-Violation, since it is relevant to what follows.
This type of CP violation is not restricted to neutral mesons only and it is simply the assymetry emerging from
the decay rate comparision of a process and its CP conjugated one, as shown in \equref{cpv_decay}. This type is
called CPV {\it in the decay}.

\begin{equation}
\Acp{\text{decay}} \propto \Gamma(\Bs \rightarrow f) - \Gamma(\Bsb \rightarrow \bar{f})
\label{cpv_decay}
\end{equation}

\noindent Where $\Gamma$ is the decay rate of the $\Bs \rightarrow f$ process and $f$ is some final state,
like $\jpsi\phi$. There is one more type of CP Violation and that is in the {\it interference} between
a neutral meson decying directlly to a final state $f$ or first oscialting to its antiparticle and then decaying
to the same final state $f$. Due to its central role, this type of CP-Violation is explained within
the context of a the $\jpsi\phi$ final state in the next section.
