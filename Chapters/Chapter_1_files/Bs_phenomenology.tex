The \Bs meson is an electrically neutral particle that consists of a quark anti-quark pair, specifically (\bquarkbar\squark).
The two constituents of the mesons are bound by the strong interaction. An important feature of
the \Bs meson, and all heavy neutral mesons, is that it can change into its antiparticle,
the \Bsb meson and \viceversa. This feature is called {\it meson-antimeson oscillations} and it is possible
in the Standard Model via  {\it box diagram} of \figref{bs_box}.
Meson oscillations play a central role in flavor physics. Particularly because they are sensitive to the
existence of new particles as will be explained in \secref{probe_new_phys}. \CP violating effects can
appear in \BBbarSyst oscillations, in the subsequent decay as well as in the interference between oscillation
and decay. The current section as well as \secref{WeakPhase} qualitatively addresses the way that
these effects manifest themselves.

\begin{figure}[!h]
  \centering
  \begin{subfigure}{0.5\textwidth}
    \centering
    \raggedright
    \scalebox{0.9}{\sffamily \input{Figures/Chapter1/box1}}
    \caption{}
    \label{bs_box_1}
  \end{subfigure}%
  \hfill%
  \begin{subfigure}{0.5\textwidth}
    \centering
    \raggedleft
    \scalebox{0.9}{\sffamily %%BoundingBox: -8 0 123 75
%%HiResBoundingBox: -8 0 122.57008 74.71962

\begin{fmffile}{Figures/Chapter1/box2}
  \fmfframe(19,3)(17,3){
    \begin{fmfgraph*}(115,75)
      \fmfbottom{i1,d1,o1}%dummy vertex
      \fmftop{i2,d2,o2}
      \fmf{fermion,label=s,l.side=left}{i1,v1}
      \fmf{fermion,label=b,l.side=left}{v2,o1}
      \fmf{fermion,label=b,l.side=left}{v3,i2}
      \fmf{fermion,label=s,l.side=left}{o2,v4}
      \fmf{boson,label=W$^-$,l.side=left}{v1,v2}
      \fmf{boson,label=W$^+$,l.side=left}{v4,v3}
      \fmffreeze
      \fmf{fermion,label={t,,c,,u},l.side=left}{v4,v2}
      \fmf{fermion,label={t,,c,,u},l.side=left}{v1,v3}
      \fmf{plain,left=0.2}{o1,o2}
      \fmf{plain,left=0.2,label=$\Bsb$}{o2,o1}
      \fmf{plain,left=0.2}{i2,i1}
      \fmf{plain,left=0.2,label=$\Bs$}{i1,i2}
    \end{fmfgraph*}
  }
\end{fmffile}
}
    \caption{}
    \label{bs_box_2}
  \end{subfigure}
  \caption{Leading order diagrams for \BBbarSyst oscillations. Figures from \cite{jeroenThesis}.}
  \label{bs_box}
\end{figure}

\subsubsection{The \BBbarSyst System}
\label{the_bbar_system}

The behavior of the \BBbarSyst is described phenomenologically, where the two mesons are treated as a
coupled quantum mechanical system. This approach
is based on \cite{Weisskopf:1930au,Weisskopf:1930ps} and uses an {\it effective Hamiltonian}
\cite{eff-hamiltonian-bs-syst,DeBruyn-thesis} to describe the time evolution of the system.
The wave-functions of the mesons are:

\begin{equation}
  \centering
  \ket{\Bs}  \equiv  \ket{\bquarkbar\squark}, \;\;\; \ket{\Bsb} \equiv  \ket{\bquark\squarkbar},
  \label{bs_wavefunctions}
\end{equation}

\noindent implying that the states \Bs and \Bsb have a definite quark content, or in other words,
they are flavor eigenstates. The effective Hamiltonian, \Heff, is a 2x2 non-diagonal non-Hermitian matrix.
The non-diagonal feature is due to the presence of \BBbarSyst transition amplitudes,
whereas the non Hermitian description takes into account the probability that the system will
eventually decay. By diagonalizing the effective Hamiltonian one can obtain
the mass eigenstates of the system:

\begin{align}
  \centering
  \ket{\BsH} &= p \ket{\Bs} + q \ket{\Bsb}, \nonumber \\
  \ket{\BsL} &= p \ket{\Bs} - q \ket{\Bsb}.
  \label{bs_mass_eigen}
\end{align}

\noindent Note the difference between the masses, \mass{\BsH} and \mass{\BsL}, of the previous mass
eigenstates and the average mass, \mass{\Bs}, of the \BBbarSyst system, which is used in \secref{Event_Selection}.
Similarly the lifetimes \tauH and \tauL, refer to the lifetime of the heavy and light mass eigenstates, whereas
the \tauBs is the average lifetime of the previous system. More details on the exact calculations can be found both in section 13.1
of \cite{PDG} and in \cite{jeroenThesis,DeBruyn-thesis}. The amount of mixing between the mass
and flavor eigenstates is governed by the complex parameters $p,q$.
Theoretical calculations \cite{Lenz:2011ti} and experimental measurement \cite{asl-paper} point to the ratio
\qoverp being compatible with one. In addition the difference in mass and decay width between the two mass
eigenstates are important observables of the \BBbarSyst system and are denoted as $\Delta m_s$ and $\Delta\Gamma_s$
respectively. Note that decay widths, $\Gamma_x$ are the inverse of the corresponding lifetime $\tau_x$, where $x:\{\Bs,{\rm H},{\rm L}\}$.

%\subsubsection{Types of \CP violation}
\subsubsection{Classification of \CP Violation effects}
The ratio $|\qoverp|$ is related to a type of \CP violation. Specifically,  \CP violation {\it in the mixing}
(of the \BBbarSyst system). This type of asymmetry vanishes if $|\qoverp|= 1$ which implies that the mass
eigenstates of \equref{bs_mass_eigen} consist of equal amounts of the flavor eigenstates in \equref{bs_wavefunctions}.

Another type of \CP violation, {\it in the decay} or {\it direct} \CP violation, is based on the
amplitude, $A_f$, of a certain meson decay to some final state $f$. Given the \CP conjugated $\bar{A}_{\bar{f}}$
amplitude, this type of asymmetry vanishes when $|\nicefrac{\bar{A}_{\bar{f}}}{A_f}| = 1$,
which implies that the probabilities for the $\B \to f$ and $\Bbar \to \bar{f}$ processes are equal.
Note that unlike \CP violation in mixing which is restricted to neutral mesons,
\CP violation in the decay can manifest itself in the decays of charged mesons.

There is one more type of \CP violation and that originates from the {\it interference} between
a neutral meson decaying directly to a final state $f$ or oscillating to its antiparticle and then decaying
to the same final state $f$. Due to its relevance to the current thesis, this type of \CP violation is explained
in more detail in the next section. All the above classifications are based on section 13.1.4 of \cite{PDG}.

Finally, it is useful at that stage to introduce the decay rate equations of the \Bs (and \Bsb) meson.
The decay rate equations are quoted in \equref{decay_rate_master_eqs} (for a full derivation see \cite{PDG,DeBruyn-thesis,jeroenThesis}).
The letter $t$ in the same equation represents time and $\lambda_f$ is a parameter commonly used to parameterize
\CP violation in meson decays which is defined as:

\begin{equation}
  \centering
  \lambda_{f} = \frac{q}{p} \frac{\bar{A}_f}{A_f}. % \equiv \left|\lambda_f\right| e^{i\phis}.
\label{lambda_cpv}
\end{equation}

\newcommand{\ampSq}{\ensuremath{|A_f|^2}\xspace}
\newcommand{\ampBSq}{\ensuremath{|\bar{A}_f|^2}\xspace}
\newcommand{\lambSq}{\ensuremath{|\lambda_f|^2}\xspace}
\newcommand{\eGammast}{\ensuremath{e^{-\Gamma_s t}}\xspace}
\newcommand{\qopSq}{\ensuremath{|\qoverp|^2}\xspace}
\newcommand{\DeltaGammat}{\ensuremath{\frac{\Delta\Gamma_s t}{2}}\xspace}
\newcommand{\DeltaMt}{\ensuremath{\frac{\Delta m_s t}{2}}\xspace}

\begin{subequations}
 \label{decay_rate_master_eqs}
 \begin{align}
 \centering
    \Gamma(\Bsb \to f) =          \qopSq & \ampSq (1+\lambSq) \frac{\eGammast}{2} \Big[ \cosh\DeltaGammat + \Acp{\Delta\Gamma}\sinh\DeltaGammat \nonumber \\
                                         &  - \Acp{\rm dir}\cos\DeltaMt + \Acp{\rm mix}\sin\DeltaMt \Big]  \\
    \Gamma(\Bs \to f) = \phantom{\qopSq} & \ampSq (1+\lambSq) \frac{\eGammast}{2} \Big[ \cosh\DeltaGammat + \Acp{\Delta\Gamma}\sinh\DeltaGammat \nonumber  \\
                                         & + \Acp{\rm dir}\cos\DeltaMt - \Acp{\rm mix}\sin\DeltaMt \Big].
 \end{align}
\end{subequations}

\noindent
\noindent The parameter $\lambda_f$ above is not to be confused with the $\lambda$ of the
Wolfenstein parametrization. Note also the coefficients of the trigonometric functions in
\equref{decay_rate_master_eqs}. These parameters become relevant in \chapref{Data_Analysis}
and \chapref{Penguins} and are thus defined in the following equation:

\begin{equation}
  \centering
  \Acp{\rm dir}      \equiv \frac{ 1-|\lambda_f|^2} {1 + |\lambda_f|^2},\;\;\;\;\;
  \Acp{\rm mix}      \equiv \frac{ 2 \Im\lambda_f} {1 + |\lambda_f|^2},\;\;\;\;\;
  \Acp{\Delta\Gamma} \equiv \frac{ 2 \Re\lambda_f} {1 + |\lambda_f|^2}.
\label{cp_asym_lambda}
\end{equation}

\noindent The first asymmetry, \Acp{\rm dir}, quantifies \CP violation in the decay while the
second asymmetry, \Acp{\rm mix}, expresses the mixing induced \CP asymmetry. The last one,
\Acp{\Delta\Gamma} is associated with the difference in the decay rate of the mass eigenstates of \equref{bs_mass_eigen}.
