

The \Bs meson\footnote{Mesons are bound by the strong interaction states of two quarks.} is an electrically neutral
particle that consists of two quarks, particularly (\bquarkbar\squark). An important feature of
the \Bs meson, and all neutral mesons, is that it can spontaneously change into its antiparticle,
the \Bsb meson and vice versa. This feature is called {\it meson oscillations} and it is possible
in the Standard Model via the so called {\it box diagram} of \figref{bs_box}.
Meson oscillations play a central role in flavor physics. Particularly because they are sensitive to the
existence of new particles as it will be explained in \secref{probe_new_phys}. CP-Violating effects can
appear in the \BBbarSyst oscillations and in the subsequent decay. The current section as well
as \secref{WeakPhase} will qualitatively address the way that these effects manifest themselves.

\begin{figure}[h]
  \centering
  \begin{subfigure}{0.5\textwidth}
    \centering
    {\sffamily \input{Figures/Chapter1/box1}}
    \caption{}
    \label{bs_box_1}
  \end{subfigure}%
  \begin{subfigure}{0.5\textwidth}
    \centering
    {\sffamily %%BoundingBox: -8 0 123 75
%%HiResBoundingBox: -8 0 122.57008 74.71962

\begin{fmffile}{Figures/Chapter1/box2}
  \fmfframe(19,3)(17,3){
    \begin{fmfgraph*}(115,75)
      \fmfbottom{i1,d1,o1}%dummy vertex
      \fmftop{i2,d2,o2}
      \fmf{fermion,label=s,l.side=left}{i1,v1}
      \fmf{fermion,label=b,l.side=left}{v2,o1}
      \fmf{fermion,label=b,l.side=left}{v3,i2}
      \fmf{fermion,label=s,l.side=left}{o2,v4}
      \fmf{boson,label=W$^-$,l.side=left}{v1,v2}
      \fmf{boson,label=W$^+$,l.side=left}{v4,v3}
      \fmffreeze
      \fmf{fermion,label={t,,c,,u},l.side=left}{v4,v2}
      \fmf{fermion,label={t,,c,,u},l.side=left}{v1,v3}
      \fmf{plain,left=0.2}{o1,o2}
      \fmf{plain,left=0.2,label=$\Bsb$}{o2,o1}
      \fmf{plain,left=0.2}{i2,i1}
      \fmf{plain,left=0.2,label=$\Bs$}{i1,i2}
    \end{fmfgraph*}
  }
\end{fmffile}
}
    \caption{}
    \label{bs_box_2}
  \end{subfigure}
  \caption{Leading order diagrams for \BBbarSyst oscillations. Figures from~\cite{jeroenThesis}.}
  \label{bs_box}
\end{figure}

\subsubsection{The \BBbarSyst System}
\label{the_bbar_system}

The oscillating behaviour of the \Bs and \Bsb mesons invites a phenomenological approach
where the two mesons are treated as a coupled quantum mechanical system. This approach
is based on~\cite{Weisskopf:1930au,Weisskopf:1930ps} approach and uses an {\it effective hamiltonian}~\cite{eff-hamiltonian-bs-syst,DeBruyn-thesis}
to describe the time evolution of the \Bs or \Bsb meson. The wave-functions of the last mesons
are defined in \equref{bs_wavefunctions}

\begin{equation}
\ket{\Bs}  \equiv  \ket{\bquarkbar\squark}, \;\;\; \ket{\Bsb} \equiv  \ket{\bquark\squarkbar},
\label{bs_wavefunctions}
\end{equation}

\noindent implying that the states \Bs and \Bsb have a definite quark content or in other words
they are flavour eigenstates. The effective hamiltonian, $H_{\rm eff}$ is a 2x2 non diagonal non hermitian matrix.
The non diagonal feature is due to the oscillating feature of the \BBbarSyst system whereas the non
hermitianity takes into account the probability that the system will eventually decay to some final state.
By diagonalizing the effective on can obtain the mass eigenstates, \ket{\Bs{}_{,H}}, \ket{\Bs{}_{,L}}, of the system, shown in \equref{bs_mass_eigen}.

\begin{align}
 \ket{\Bs{}_{,H}} &= p \ket{\Bs} + q \ket{\Bsb}, \nonumber \\
 \ket{\Bs{}_{,L}} &= p \ket{\Bs} - q \ket{\Bsb}.
\label{bs_mass_eigen}
\end{align}

More details on the exact calculations can be found in section 13.1 of~\cite{PDG} and in ~\cite{jeroenThesis,DeBruyn-thesis}.

The amount of mixing between the mass and flavour eigenstates is governed by the complex parameters $p,q$.
The ratio $|\qoverp|$ is compatible with one. This is supported both from theoretical calculations~\cite{Lenz:2011ti}
and from experimental measurement~\cite{asl-paper} measurements\footnote{Some tensions between theory and experiment will be resolved in the future. The tensions are
mainly due to the \dzero measurement~\cite{Abazov:2013uma}}. In addition the difference in mass and lifetime between the two
mass eigenstates are important observables of the \BBbarSyst and are denoted as $\Delta m_s$ and $\Delta\Gamma_s$ respectively.

\subsubsection{Types of CP-Violation}
The ratio $|\qoverp|$ is connected a type of CP-Violation. Specifically the CP-Violation {\it in the mixing} (of the \BBbarSyst system).
The connection is shown in the asymmetry of \equref{acp_mixing}.

\begin{equation}
a_{\rm sl}^{s}  \equiv \frac{\bra{\Bsb}\ket{\Bs} - \bra{\Bs}\ket{\Bsb}} {\bra{\Bsb}\ket{\Bs} + \bra{\Bs}\ket{\Bsb}}
                     = \frac{1 - \left|\nicefrac{q}{p}\right|^4}{1 + \left|\nicefrac{p}{q}\right|^4},
\label{acp_mixing}
\end{equation}

\noindent where it can be seen that $a_{\rm sl}^{s}$ vanishes if $|\qoverp|= 1$. The subscript $sl$ stands for a semileptonic
final state that is typically used to measure this type of asymmetries in the neutral mesons. A semileptonic final state
consists of mesons and leptons. From the lepton sign it can be inferred whether the original neutral meson oscillated or not.

Another type of CP-Violation is called {\it in the decay} or {\it direct} and its based on the amplitude, $A_f$, of a certain meson decay to some final state $f$.
Using the CP conjugated $\bar{A}_{\bar{f}}$ amplitude the asymmetry of \equref{cpv_decay} can be constructed.
The last asymmetry vanishes in case $|\nicefrac{\bar{A}_{\bar{f}}}{A_f}| = 1$  and it is also relevant to other charged mesons as well.

\begin{equation}
\Acp{\text{dir}} = \frac{\Gamma(\Bs \rightarrow f) - \Gamma(\Bsb \rightarrow \bar{f})} {\Gamma(\Bs \rightarrow f) + \Gamma(\Bsb \rightarrow \bar{f})}
                 = \frac{ |\nicefrac{\bar{A}_{\bar{f}}}{A_f}|^2 - 1}{|\nicefrac{\bar{A}_{\bar{f}}}{A_f}|^2 + 1}.
\label{cpv_decay}
\end{equation}


There is one more type of CP Violation and that is in the {\it interference} between
a neutral meson decaying directly to a final state $f$ or first oscillating to its antiparticle and then decaying
to the same final state $f$. Due to its relevance to the current thesis, this type of CP-Violation is explained
in the next section. All the above classifications are based on section 13.1.4 of~\cite{PDG}.
