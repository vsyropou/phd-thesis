As it was mentioned in \secref{The_Standard_Model}, quarks and leptons acquire mass through the Yukawa term
of the Standard Model Lagrangian. An important aspect of the weak interaction emerges from that term,
namely the {\it quark flavor mixing}. A brief description of the latter and its relevance to \CP violation
is given in what follows. The section concludes with introducing the  {\it CKM-mixing matrix},
which is the core of {\it flavor physics} in the quark sector of the Standard Model.

\subsubsection{Quark flavor mixing}
By applying the Higgs mechanism \cite{PhysRevLett.13.321,PhysRevLett.13.508} to the Standard Model Lagrangian
the Higgs field obtains a {\it vacuum expectation value}, which corresponds to the lowest energy value of the Higgs
potential. After this step the Yukawa term for the quark fields, ignoring quark-Higgs field interaction terms, is:

\begin{subequations}
\label{yukawa_flavor}
\centering
  \begin{align}
  % -\mathscr{L}_{\text{Yukawa}} = M_{ij}^d \bar{d_{Li}} d_{Rj} + M_{ij}^u \bar{u_{Li}} u_{Rj} + h.c.,
  -\mathscr{L}_{\text{Yukawa}} &= \left[ y_{ij}^d \bar{d}_{Li} d_{Rj} + y_{ij}^u \bar{u}_{Li} u_{Rj} \right] \frac{v}{\sqrt{2}} + h.c. + ...  \\
                               &= \left[ m_{ij}^d \bar{d}_{Li} d_{Rj} + m_{ij}^u \bar{u}_{Li} u_{Rj} \right] + h.c. + ...,  \\
                               \text{with} \;\;\; m^{u,d}_{i,j} = \frac{v}{\sqrt{2}} & y_{ij}^{u,d}, \nonumber
  \end{align}
\end{subequations}

\noindent where $v$ is the Higgs vacuum expectation value and $y_{ij}^{u,d}$ are complex valued numbers called {\it Yukawa couplings}.
The latter are free parameters that represent the coupling strength between Higgs and quark fields.
Note that for equations \ref{yukawa_flavor} to \ref{CClagrangian} it is implied that $u$ and $d$ indicate
any up or a down type quark. The generation of the up(down) type quark is specified by the indices $i,j$,
whereas, indices $L,R$ indicate the left or right handedness of the quark field.
Finally the matrix $m^{u,d}$ expresses the desired quark masses.

The quark fields, $u$ and $d$,  in \equref{yukawa_flavor} have a definite quantum number that labels the generation to which they belong
and also whether they are of up or down type. This quantum number is commonly called {\it flavor} and thus the quark fields
are flavor eigenstates. By construction the mass matrix is not diagonal which means that
a quark with a well defined flavor does not have a well defined mass. Or, in more formal phrasing,
the flavor and mass eigenstates of the quark fields do not coincide. In order to obtain proper quark
masses the matrix $m^{u,d}$ has to be diagonalized, as:

\begin{equation}
  \centering
  m^{d,u}_{\rm diag} = V_L^{d,u} m^{d,u} \left(V_R^{d,u}\right)^{\dagger},
  \label{diagM}
\end{equation}

\noindent where the matrices $V$ are required to be unitary. Since \equref{yukawa_flavor} has to stay intact after $m^{d,u}$ is replaced with
$m^{d,u}_{\rm diag}$, quark fields need to be rotated as well as shown in \equref{quark_rotation}, such that they cancel the additional $V$ matrices
of \equref{diagM}.

\begin{equation}
  \centering
  \left( d_{i}^m \right)_{L,R} = \left( V^d_{ij} d_{j} \right)_{L,R}, \;\;\;\; \left( u_{i}^m \right)_{L,R} = \left( V^u_{ij} u_{j} \right)_{L,R}.
  \label{quark_rotation}
\end{equation}

\noindent At this point the quark fields in the rest of the Standard Model Lagrangian are still flavor eigenstates.
The field rotations of \equref{quark_rotation} need to be applied in these terms as well. Specifically, it must be
applied to the kinetic term involving quark interactions with the charged weak bosons \Wpm, also known as {\it charged current}
interaction, shown in \equref{CClagrangian}. The charged current interaction in the same equation is expressed in two ways.
One with quark fields expressed as flavor eigenstates, \equref{CClagrangianInt} and two as mass eigenstates, \equref{CClagrangianMass}.

\begin{subequations}
  \centering
  \begin{align}
    \mathscr{L}_{\text{Kinetic}}^{CC} & \propto \bar{u}_{Li} \gamma_\mu {\Wm}^\mu d_{Ri} + \bar{d}_{Li} \gamma_\mu {\Wp}^\mu u_{Ri}  \label{CClagrangianInt} \\
                                      & \propto \bar{u}_{Li}^m  {V_{\text{CKM}}} \gamma_\mu {\Wm}^\mu d_{Ri}^m + \bar{d}_{Li}^m V_{\text{CKM}} {\Wp}^\mu \gamma_\mu u_{Ri}^m, \label{CClagrangianMass}
  \end{align}
  \label{CClagrangian}
\end{subequations}

\noindent with $V_{\text{CKM}} \equiv V^u_LV^{d\dagger}_L$ and \Wpm are the charged weak boson fields, whereas $\gamma_\mu$ are Dirac matrices.
The resulting matrix is the  {\it CKM mixing matrix}:

\begin{equation}
  \centering
  \begin{pmatrix} \dquark \\ \squark \\ \bquark  \end{pmatrix} =
  \underbrace{\begin{pmatrix} \Vud & \Vus & \Vub \\ \Vcd & \Vcs & \Vcb \\ \Vtd & \Vts & \Vtb \end{pmatrix}}_{V_{\text{CKM}}}
    \begin{pmatrix} \dquark^m \\ \squark^m \\ \bquark^m  \end{pmatrix}.
      \label{quark_field_rotation}
  \end{equation}

\begin{figure}[t]
  \centering
  {\sffamily 

\hspace*{0.05\textwidth}
\begin{fmffile}{Figures/Chapter1/qqMixing}
  \fmfframe(8,16)(8,16){
    \begin{fmfgraph*}(60,35)
      \fmfstraight
      \fmfleft{u}
      \fmfright{d,W}
      \fmf{fermion}{u,V,d}
      \fmf{boson}{V,W}
      \fmflabel{$u_i$}{u}
      \fmflabel{$d_j$}{d}
      \fmflabel{$\Wp$}{W}
      \fmflabel{\hspace{0.2cm}$V_{ij}$}{V}
    \end{fmfgraph*}
  }
\end{fmffile}

%
% \hspace*{0.05\textwidth}
% \begin{fmffile}{Figures/Chapter1/udMixing}
%   \fmfframe(8,16)(8,16){
%     \begin{fmfgraph*}(60,35)
%       \fmfstraight
%       \fmfleft{u}
%       \fmfright{d,W}
%       \fmf{fermion}{u,V,d}
%       \fmf{boson}{V,W}
%       \fmflabel{$u_i$}{u}
%       \fmflabel{$d^m_j$}{d}
%       \fmflabel{$\Wp$}{W}
%       \fmflabel{\hspace{0.2cm}$V_{id}$}{V}
%     \end{fmfgraph*}
%   }
% \end{fmffile}
% \hspace*{0.05\textwidth}
% \begin{fmffile}{Figures/Chapter1/usMixing}
%   \fmfframe(8,16)(8,16){
%     \begin{fmfgraph*}(60,35)
%       \fmfstraight
%       \fmfleft{u}
%       \fmfright{d,W}
%       \fmf{fermion}{u,V,d}
%       \fmf{boson}{V,W}
%       \fmflabel{$u_i$}{u}
%       \fmflabel{$s^m_j$}{d}
%       \fmflabel{$\Wp$}{W}
%       \fmflabel{\hspace{0.2cm}$V_{is}$}{V}
%     \end{fmfgraph*}
%   }
% \end{fmffile}
% \hspace*{0.05\textwidth}
% \begin{fmffile}{Figures/Chapter1/ubMixing}
%   \fmfframe(8,16)(8,16){
%     \begin{fmfgraph*}(60,35)
%       \fmfstraight
%       \fmfleft{u}
%       \fmfright{d,W}
%       \fmf{fermion}{u,V,d}
%       \fmf{boson}{V,W}
%       \fmflabel{$u_i$}{u}
%       \fmflabel{$b^m_j$}{d}
%       \fmflabel{$\Wp$}{W}
%       \fmflabel{\hspace{0.2cm}$V_{ib}$}{V}
%     \end{fmfgraph*}
%   }
% \end{fmffile}
}
  \caption{Feynman diagram where an up-type quark couples to any of the three, (\dquark,\squark,\bquark), down type quarks,
           via a \Wp boson. Time flows from left to right.}
  \label{QuarkMixing}
\end{figure}

The charged current term, \equref{CClagrangianMass}, incorporates an important aspect of the weak interaction,
namely the {\it quark flavor mixing}. The latter is due to the fact that quark mass eigenstates are superpositions
of the flavor eigenstates. By definition the $V_{\text{CKM}}$ rotates only the down type quarks, implying that
the mass eigenstates of the up-type quarks are identical to the flavor eigenstates. Thus an up type quark
can change its flavor to any of the down type quarks, see \figref{QuarkMixing}, with a certain probability.
The probability of such a transition is given by the corresponding element of the CKM mixing matrix.

\subsubsection{CKM mixing matrix}
The CKM matrix, or simply CKM matrix, is a unitary matrix which, as previously mentioned, describes the strength of quark couplings, or in other words,
the probability of a certain quark flavor transitions. The elements of the CKM matrix have been measured,
see \eg Chapter 12 of \cite{PDG}, showing the following structure for their magnitudes:

\begin{equation}
  |V_{\text{CKM}}|
                   = \begin{pmatrix} \VudMag & \VusMag & \VubMag \\ \VcdMag & \VcsMag & \VcbMag \\ \VtdMag & \VtsMag & \VtbMag \end{pmatrix}
              \simeq \begin{pmatrix} 1 & 0.2 & 0.008 \\ 0.2 & 1 & 0.04 \\ 0.008 & 0.04 & 1 \end{pmatrix}.
      \label{CKMmatrix}
  \end{equation}

\noindent Essentially the structure implies that transitions between generations
are suppressed with respect to transitions within the same generation in a hierarchical way. The most suppressed transitions are between
the first and third generations followed by the ones between the second and third and the least suppressed are between first and second.

The unitarity of the CKM matrix is at the center of flavor physics. In order to achieve such tests a
parametrization of the CKM matrix is useful. After exploiting the unitarity of $V_{\rm CKM}$ and all the
redundant quark field phases, the CKM matrix has, by construction, 3 real parameters and one complex phase.
The choice of the CKM matrix parametrization is \aprior arbitrary. However due to the observed structure the
{\it Wolfenstein} \cite{Wolfenstein:1983yz,Buras-wolfenstein} parametrization is a standard choice.
The Wolfenstein parametrization utilizes three real parameters, $\lambda,A,\rho$ and an imaginary one $i\eta$, as follows:

\begin{equation}
\centering
  |V_{\text{CKM}}|
                   = \begin{pmatrix} \VudWolf & \VusWolf & \VubWolf \\
                                     \VcdWolf & \VcsWolf & \VcbWolf \\
                                     \VtdWolf & \VtsWolf & \VtbWolf \end{pmatrix} + \mathcal{O}(\lambda^4).
      \label{CKMwolfenstein}
\end{equation}

\noindent As previously mentioned, the CKM matrix is unitary, meaning that $V_{\text{CKM}} V_{\text{CKM}}^\dagger = I_{3\times3}$.
This leads to the orthogonality relations. The latter are sums of complex numbers that are equal to zero,
and thus can be represented as {\it unitarity triangles} in the complex plane. There are six orthogonality relations, two of which, shown
in \equref{unitConstraints}, are relevant for this thesis since the CKM elements present in these relations govern the
dynamics in the \Bs and \Bd meson systems.

\begin{subequations}
  \centering
  \label{unitConstraints}
  \begin{align}
    \Bd : & \quad \Vud\Vub^* + \Vcd\Vcb^* + \Vtd\Vtb^* = 0,
    \label{unitConstraints_Bd} \\
    \Bs : & \quad \Vus\Vub^* + \Vcs\Vcb^* + \Vts\Vtb^* = 0.
    \label{unitConstraints_Bs}
  \end{align}
\end{subequations}

\noindent These two relations, after dividing by $\Vcd\Vcb^*$, $\Vcs\Vcb^*$ respectively for \Bd and \Bs,
are illustrated in \figref{unitTriangles}. Note that the CKM element $\Vts$ has a complex part at higher
order in $\lambda$, see section 13.3 of \cite{PDG}. By inspecting the triangles one can define some of the angles
as follows:

\begin{figure}[t]
  \centering
  \begin{subfigure}{0.475\textwidth}
    \raggedright
    \includegraphics[width=\textwidth]{Figures/Chapter1/b-d-triangle}
    \caption{}
    \label{unitTriangles_bd}
  \end{subfigure}%
  \hfill%
  \begin{subfigure}{0.525\textwidth}
    \raggedleft
    \includegraphics[width=\textwidth]{Figures/Chapter1/b-s-triangle}
    \caption{}
    \label{unitTriangles_bs}
  \end{subfigure}
  \caption{CKM-unitarity triangles. \Bd triangle (left), corresponding to \equref{unitConstraints_Bd}. \Bs triangle (right),
           corresponding to \equref{unitConstraints_Bs}. Triangle sides have been normalized, see text.
           This way one of the sides is real with unit length. Note that triangles are not drawn to scale. Figures from \cite{jeroenThesis}. }
  \label{unitTriangles}
\end{figure}

\begin{align}
  \centering
  \alpha \equiv \arg\left( -\frac{\Vtd\Vtb^*}{\Vud\Vub^*} \right),
  \;\;
  \beta  \equiv & \arg\left( -\frac{\Vcd\Vcb^*}{\Vtd\Vtb^*} \right),
  \;\;
  \gamma \equiv \arg\left( -\frac{\Vud\Vub^*}{\Vcd\Vcb^*} \right), \nonumber \\
  \betas \equiv & \arg\left( -\frac{\Vts\Vtb^*}{\Vcs\Vcb^*} \right).
  \label{ckm_angles_def}
\end{align}

\noindent Note that the definitions of \equref{ckm_angles_def} are independent of
the quark field phases. Thus the above angles are useful observables, regardless of the chosen
CKM matrix parametrization.

There is no fundamental reason known in the Standard model for the observed hierarchy of the CKM elements.
One of the main goals of flavor physics is to verify the consistency of the CKM picture.
Overlaying measurements of flavor physics observables in the complex $\bar{\uprho}-\bar{\upeta}$ plane
of \figref{unitarity_triangle} should show a compatible picture regarding the position of the
apex of the unitarity triangle. The apexes of the \Bd and \Bs triangles are defined in a
convention independent way as:

\begin{equation}
  \bar{\uprho} + i \bar{\upeta} = -\frac{\Vud\Vub^*}{\Vcd\Vcb^*}, \quad \quad
  \bar{\uprho}_{\rm s} + i \bar{\upeta}_{\rm s} = -\frac{\Vus\Vub^*}{\Vcs\Vcb^*}.
  \label{apexes}
\end{equation}

\begin{figure}[!t]
  \centering
    \includegraphics[trim=0cm 0.5cm 0cm 11cm, clip=true, width=\textwidth]{Figures/Chapter1/rhoeta_large}
    \caption{Global fit of the \Bd triangle \cite{ckm-fitter-phis-pred}.
             Inputs from measurements of flavor physics observables are indicated by colored bands.}
    \label{unitarity_triangle}
\end{figure}

\noindent For completeness it is interesting to mention that in the
lepton sector a similar matrix is present. The observed hierarchy there is completely
different which is yet another intriguing feature of nature.

\subsubsection{\CP violation and fermion masses}
Having introduced both the fermion masses and \CP violation it is interesting to point out their common origin in
the Standard Model \cite{KM-mechanism}. Elaborating more, from one hand the CKM matrix was introduced to couple
the Higgs to the fermions and thus provide them with a mass. On the other hand the fact that $V_{\rm CKM}$ is a
complex valued matrix allows for \CP violation in the Standard Model via the weak interaction. This is because
the charged current interaction of \equref{CClagrangianMass} is invariant under \CP operations only
if $V_{\rm CKM}=V_{\rm CKM}^*$, which is evidently not the case.
