As it was mentioned in \secref{The_Standard_Model}, quarks and leptons acquire mass through the Yukawa term
of the Standard Model lagrangian. An important aspect of the weak interaction emerges from that term,
namely the {\it quark flavour mixing}. A brief description of the last and its relevance to CP-Violation is given in what follows.
The section concludes with introducing the so called {\it CKM-mixing matrix}, which is the core of the
{\it flavor physics} in the quark sector of the Standard Model.

\subsubsection{Quark Flavour Mixing}
By applying the Higgs mechanism~\cite{PhysRevLett.13.321,PhysRevLett.13.508} to the Standard Model Lagrangian the Higgs field obtains a
{\it vacuum expectation value}, which is the lowest energy value of the Higgs field potential.
After this step the Yukawa term for the quark fields is shown in \equref{yukawa_flavour},
 ignoring quark-higgs field interaction terms.

\begin{subequations}
\label{yukawa_flavour}
  \begin{align}
  % -\mathscr{L}_{\text{Yukawa}} = M_{ij}^d \bar{d_{Li}} d_{Rj} + M_{ij}^u \bar{u_{Li}} u_{Rj} + h.c.,
  -\mathscr{L}_{\text{Yukawa}} &= \left[ y_{ij}^d \bar{d}_{Li} d_{Rj} + y_{ij}^u \bar{u}_{Li} u_{Rj} \right] \frac{v}{\sqrt{2}} + h.c. + ...  \\
                               &= \left[ m_{ij}^d \bar{d}_{Li} d_{Rj} + m_{ij}^u \bar{u}_{Li} u_{Rj} \right] + h.c. + ...,  \\
                               \text{with} \;\;\; m^{u,d}_{i,j} = \frac{v}{\sqrt{2}} y_{ij}^{u,d} & \nonumber
  \end{align}
\end{subequations}

\noindent where $v$ is the Higgs vacuum expectation value and $y_{ij}^{u,d}$ are complex valued numbers called {\it Yukawa couplings}.
The last are free parameters that represent the coupling strength between Higgs and quark fields.
Note that for equations \equref{yukawa_flavour} to \equref{CClagrangian} it is implied that $u$ and $d$ indicate
an up or a down type quark. The generation of the up(down) type quark is specified by the indices $i,j$,
whereas, indices $L,R$ indicate the left or right handedness of the quark field.
Finally the matrix $\bf m^{u,d}$ expresses the desired quark masses.

The quark fields, $u$ and $d$,  in \equref{yukawa_flavour} have a definite quantum number that labels the generation to which they belong
and also whether they are of up or down type. This quantum number is commonly called {\it flavour} and thus the quark fields
are flavour eigenstates. By construction the mass matrix is not diagonal which means that
a quark with a well defined flavour does not have a well defined mass. Or, in more formal phrasing, flavour and mass eigenstates of
the quark fields do not coincide. In order to obtain proper quark masses the matrix $\bf m^{u,d}$ has to be diagonalized, as shown in \equref{diagM}.

\begin{equation}
  m^{d,u}_{\text diag} = V_L^{d,u} m^{d,u} \left(V_R^{d,u}\right)^{\dagger},
  \label{diagM}
\end{equation}

\noindent where the matrices $V$ are required to be unitary. Since \equref{yukawa_flavour} has to stay intact after $m^{d,u}$ is replaced with
$m^{d,u}_{\text diag}$, the quark fields need to be rotated, as shown in \equref{quark_rotation} such that they cancel the additional $V$ matrices
of \equref{diagM}.

\begin{equation}
  \left( d_{i}^m \right)_{L,R} = \left( V^d_{ij} d_{j} \right)_{L,R}, \;\;\;\; \left( u_{i}^m \right)_{L,R} = \left( V^u_{ij} u_{j} \right)_{L,R}.
  \label{quark_rotation}
\end{equation}

\noindent At this point the quark fields in the rest of the Standard Model Lagrangian are still flavour eigenstates.
The field rotations of \equref{quark_rotation} need to be applied in these terms as well. Specifically, it must be
applied to the kinetic term involving quark interactions with the charged weak bosons \Wpm, also known as {\it charged current}
interaction, shown in \equref{CClagrangian}. In the last equation the charged current interaction is expressed in two ways.
Once with quark fields expressed as flavour eigenstates, \equref{CClagrangianInt}, and second with the quarks as mass eigenstates, \equref{CClagrangianMass}.

\begin{subequations}
  \label{CClagrangian}
  \begin{align}
    \mathscr{L}_{\text{Kinetic}}^{CC} & \propto \bar{u}_{Li} \gamma_\mu {\Wm}^\mu d_{Ri} + \bar{d}_{Li} \gamma_\mu {\Wp}^\mu u_{Ri}  \label{CClagrangianInt} \\
                                      & \propto \bar{u}_{Li}^m  {V_{\text{CKM}}} \gamma_\mu {\Wm}^\mu d_{Ri}^m + \bar{d}_{Li}^m V_{\text{CKM}} {\Wp}^\mu \gamma_\mu u_{Ri}^m \label{CClagrangianMass} \\
                                      \text{with} \;\;\; V_{\text{CKM}} \equiv V^u_LV^{d\dagger}_L, & \nonumber
  \end{align}
\end{subequations}

\noindent and \Wpm the charged weak boson fields, whereas $\gamma_\mu$ are Dirac matrices.
The resulting matrix $V_{\text{CKM}}$ is the so called {\it CKM mixing matrix}, shown in \equref{quark_field_rotation}.

\begin{equation}
  \begin{pmatrix} \dquark \\ \squark \\ \bquark  \end{pmatrix} =
  \underbrace{\begin{pmatrix} \Vud & \Vus & \Vub \\ \Vcd & \Vcs & \Vcb \\ \Vtd & \Vts & \Vtb \end{pmatrix}}_{V_{\text{CKM}}}
    \begin{pmatrix} \dquark^m \\ \squark^m \\ \bquark^m  \end{pmatrix}.
  \label{quark_field_rotation}
  \end{equation}

The charged current term, \equref{CClagrangian}, incorporates an important aspect of the weak interaction,
namely the {\it quark flavor mixing}. The last is due to the fact that quark mass eigenstates are super-positions
of the flavour eigenstates.

\begin{figure}[h]
  \centering
  {\sffamily 

\hspace*{0.05\textwidth}
\begin{fmffile}{Figures/Chapter1/qqMixing}
  \fmfframe(8,16)(8,16){
    \begin{fmfgraph*}(60,35)
      \fmfstraight
      \fmfleft{u}
      \fmfright{d,W}
      \fmf{fermion}{u,V,d}
      \fmf{boson}{V,W}
      \fmflabel{$u_i$}{u}
      \fmflabel{$d_j$}{d}
      \fmflabel{$\Wp$}{W}
      \fmflabel{\hspace{0.2cm}$V_{ij}$}{V}
    \end{fmfgraph*}
  }
\end{fmffile}

%
% \hspace*{0.05\textwidth}
% \begin{fmffile}{Figures/Chapter1/udMixing}
%   \fmfframe(8,16)(8,16){
%     \begin{fmfgraph*}(60,35)
%       \fmfstraight
%       \fmfleft{u}
%       \fmfright{d,W}
%       \fmf{fermion}{u,V,d}
%       \fmf{boson}{V,W}
%       \fmflabel{$u_i$}{u}
%       \fmflabel{$d^m_j$}{d}
%       \fmflabel{$\Wp$}{W}
%       \fmflabel{\hspace{0.2cm}$V_{id}$}{V}
%     \end{fmfgraph*}
%   }
% \end{fmffile}
% \hspace*{0.05\textwidth}
% \begin{fmffile}{Figures/Chapter1/usMixing}
%   \fmfframe(8,16)(8,16){
%     \begin{fmfgraph*}(60,35)
%       \fmfstraight
%       \fmfleft{u}
%       \fmfright{d,W}
%       \fmf{fermion}{u,V,d}
%       \fmf{boson}{V,W}
%       \fmflabel{$u_i$}{u}
%       \fmflabel{$s^m_j$}{d}
%       \fmflabel{$\Wp$}{W}
%       \fmflabel{\hspace{0.2cm}$V_{is}$}{V}
%     \end{fmfgraph*}
%   }
% \end{fmffile}
% \hspace*{0.05\textwidth}
% \begin{fmffile}{Figures/Chapter1/ubMixing}
%   \fmfframe(8,16)(8,16){
%     \begin{fmfgraph*}(60,35)
%       \fmfstraight
%       \fmfleft{u}
%       \fmfright{d,W}
%       \fmf{fermion}{u,V,d}
%       \fmf{boson}{V,W}
%       \fmflabel{$u_i$}{u}
%       \fmflabel{$b^m_j$}{d}
%       \fmflabel{$\Wp$}{W}
%       \fmflabel{\hspace{0.2cm}$V_{ib}$}{V}
%     \end{fmfgraph*}
%   }
% \end{fmffile}
}
  \caption{Feynman diagram where an up-type quark couples to any of the three, ($d,s,b$), down type quarks,
           via a \Wp boson. Time flows from left to right.}
  \label{QuarkMixing}
\end{figure}

\noindent By definition the $V_{\text{CKM}}$ rotates only the down type quarks, implying that
the mass eigenstates of the up-type quarks are identical to the flavour eigenstates. Thus an up type quark
can change its flavour to any of the down type quarks, see \figref{QuarkMixing}, with a certain probability.
The probability of such a transitions is given by the elements of the CKM mixing matrix, or simply CKM matrix.

\subsubsection{CKM mixing matrix}
The CKM matrix is a unitary matrix which, as previousy mentioned, describes the strength of quark couplings,
or, in other words, the probability of quark flavour transitions. The elements of the CKM matrix have been measured, see \eg Chapter 12 of~\cite{PDG},
showing an interesting structure, see \equref{CKMmatrix}. Essentially the structure implies that transitions between generations
are suppressed with respect to transitions within the same generation, in a symmetric way. The most suppressed transitions are between
the first and third generations followed by the ones between the second and third and the least suppressed are between first and second.

\begin{equation}
  |V_{\text{CKM}}|
                   = \begin{pmatrix} \VudMag & \VusMag & \VubMag \\ \VcdMag & \VcsMag & \VcbMag \\ \VtdMag & \VtsMag & \VtbMag \end{pmatrix}
              \simeq \begin{pmatrix} 1 & 0.2 & 0.008 \\ 0.2 & 1 & 0.04 \\ 0.008 & 0.04 & 1 \end{pmatrix}.
      \label{CKMmatrix}
  \end{equation}

Testing the unitarity of the CKM matrix is a central goal of flavour physics.
In order to achieve such tests it is useful to define a parametrization of the CKM matrix. By construction the CKM matrix has
3 real parameters and one complex phase\footnote{After exploiting the unitarity of $V_{\rm CKM}$ and all the redundant quark field phases.}
The choice of the CKM matrix parametrization is \aprior arbitrary. However due to the observed structure the so called
{\it Wolfenstein}~\cite{Wolfenstein:1983yz,Buras-wolfenstein} parametrization shown in \equref{CKMwolfenstein} is a
standard parametrization. It utilizes the three real parameters, $\lambda,A,\rho$ and the imaginary one $i\eta$.

\begin{equation}
  |V_{\text{CKM}}|
                   = \begin{pmatrix} \VudWolf & \VusWolf & \VubWolf \\
                                     \VcdWolf & \VcsWolf & \VcbWolf \\
                                     \VtdWolf & \VtsWolf & \VtbWolf \end{pmatrix} + \mathcal{O}(\lambda^4).
      \label{CKMwolfenstein}
\end{equation}

\noindent As previously mentioned, the CKM matrix is a unitary matrix, meaning that $V_{\text{CKM}} V_{\text{CKM}}^\dagger = I_{3x3}$.
This leads to the so called unitarity and orthogonality relations. The last are sums of complex numbers that are equal to zero, thus can be
represented by triangles in the complex plane. There are six orthogonality relations, two of which are relevant for this thesis since
the CKM elements present in those relations govern the dynamics in the \Bs and \Bd meson systems. These two relations, shown in
\equref{unitConstraints}, after dividing by $\Vcd\Vcb^*$, $\Vcs\Vcb^*$ respectivelly for \Bd and \Bs, are illustrated in \figref{unitTriangles}.
Note that the CKM element $\Vts$ has a complex part at higher order in $\lambda$, see section 13.3 of ~\cite{PDG}.

\begin{subequations}
  \label{unitConstraints}
  \begin{align}
    \Bd : & \quad \Vud\Vub^* + \Vcd\Vcb^* + \Vtd\Vtb^* = 0,
    \label{unitConstraints_Bd} \\
    \Bs : & \quad \Vus\Vub^* + \Vcs\Vcb^* + \Vts\Vtb^* = 0.
    \label{unitConstraints_Bs}
  \end{align}
\end{subequations}

\begin{figure}[h]
  \centering
  \begin{subfigure}{0.475\textwidth}
    \raggedright
    \includegraphics[width=\textwidth]{Figures/Chapter1/b-d-triangle}
    \caption{}
    \label{unitTriangles_bd}
  \end{subfigure}%
  \begin{subfigure}{0.525\textwidth}
    \raggedleft
    \includegraphics[width=\textwidth]{Figures/Chapter1/b-s-triangle}
    \caption{}
    \label{unitTriangles_bs}
  \end{subfigure}
  \caption{CKM-unitarity triangles. (A) \Bd triangle, corresponding to \equref{unitConstraints_Bd}. (B) \Bs triangle,
           corresponding to \equref{unitConstraints_Bs}. Triangle sides have been normalized with respect to to one of them.
           This way one of the sides is real with unit length. Note that triangles are not drawn to scale. Figures from~\cite{jeroenThesis}. }
  \label{unitTriangles}
\end{figure}

The coordinates of the apex of the two triangles are defined as $(\bar{\rho},\bar{\eta})$ and $(\bar{\rho}_s,\bar{\eta}_s)$ respectively for \figref{unitTriangles_bd} and
\figref{unitTriangles_bs}. By inspecting the triangles one can define the angles as shown in \equref{bdAnglesDef} and \equref{bsAnglesDef}.

\begin{equation}
  \label{bdAnglesDef}
  \alpha \equiv \arg\left( -\frac{\Vtd\Vtb^*}{\Vud\Vub^*} \right),
  \quad
  \beta  \equiv \arg\left( -\frac{\Vcd\Vcb^*}{\Vtd\Vtb^*} \right),
  \quad
  \gamma \equiv \arg\left( -\frac{\Vud\Vub^*}{\Vcd\Vcb^*} \right),
\end{equation}

\begin{equation}
  \label{bsAnglesDef}
  %\alpha_s \equiv \arg\left( -\frac{\Vus\Vub^*}{\Vts\Vtb^*} \right)
  %\quad
  \betas \equiv \arg\left( -\frac{\Vts\Vtb^*}{\Vcs\Vcb^*} \right).
  %\quad
  %\gamma_s \equiv \arg\left( -\frac{\Vcs\Vcb^*}{\Vus\Vub^*} \right)
\end{equation}

\begin{figure}[h]
  \begin{center}
    \includegraphics[trim=0cm 0cm 0cm 0cm, clip=true, width=\textwidth]{Figures/Chapter1/rhoeta_large.png}
    \caption{Global fit of the apex $(\bar{\rho},\bar{\eta)}$ of the \Bd triangle by the CKM fitter group~\cite{ckm-fitter-phis-pred}.
             Inputs from measurements of flavor physics observables are indicated by colored bands.}
    \label{unitarity_triangle}
  \end{center}
\end{figure}

\noindent Note that the definitions \equref{bdAnglesDef} and \equref{bsAnglesDef} are independant of
the quark field phases. Thus the above angles are usufull observables, regardless of the chossen
CKM matrix pramtetriation.

There is no fundamental reason known in the Standard model for the observed hierarchy of the CKM elements.
One of the main goals of flavour physics is to check the consistency of the CKM picture.
Overlaying measurements of flavour physics observables in the complex plane should show a compatible
picture, see \figref{unitarity_triangle}. For completeness it is interesting to mention that in the lepton
sector a similar mixing matrix is active. The hierarchy there is completely different which is yet another
intriguing feature of the Standard Model.


\subsubsection{CP-Violation and fermion masses}
It is interesting to point out the common origin of CP Violation and fermion masses in the Standard Model~\cite{KM-mechanism}.
From one hand the CKM matrix was introduced to couple the Higgs to the fermions and thus give them mass.
On the other hand the fact that $V_{\rm CKM}$ is a complex valued matrix allows for CP-Violation in the
Standard Model by the weak interaction. This is because the charged current interaction of \equref{CClagrangianMass}
will be invariant under the CP operation only if $V_{\rm CKM}=V_{\rm CKM}^*$, which is evidently not the case.
