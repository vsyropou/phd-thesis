As it was mentioned in \secref{The_Standard_Model} quarks and leptons acquire mass through the Yukawa term
of the Standard Model lagrangian. An important aspect of the weak interaction, emerges from the mass acquisition mechanism,
namelly the quark flavour mixing. A brief description of the last and its relevance to CP-Violation is given in what follows.
The section concludes with a brief status overview of {\it flavour physics}, which as it was implied in \ref{The_Standard_Model}
is the part of Standard Model responsible for up-down quark type transitions.

\subsubsection{Quark Flavour Mixing}
By applying the higgs mechanism to the Standard Model lagrangian the higgs field obtains a
{\it vacum expectation value}\footnote{This is the lowest value of the higgs field potential.} {\color{red} ref ivo,niels}.
After this step a part of the Yukawa term for the quark fields is
shown\footnote{Ignoring quark-higgs field interaction terms} in \equref{yukawa_flavour}.

\begin{align}
  % -\mathscr{L}_{\text{Yukawa}} = M_{ij}^d \bar{d_{Li}} d_{Rj} + M_{ij}^u \bar{u_{Li}} u_{Rj} + h.c.,
  -\mathscr{L}_{\text{Yukawa}} &= \left[ y_{ij}^d \bar{d}_{Li} d_{Rj} + y_{ij}^u \bar{u}_{Li} u_{Rj} \right] \frac{v}{\sqrt{2}} + h.c. + ... \nonumber \\
                               &= \left[ m_{ij}^d \bar{d}_{Li} d_{Rj} + m_{ij}^u \bar{u}_{Li} u_{Rj} \right] + h.c. + ..., \\
                               \text{with} \;\;\; m^{u,d} = \frac{v}{\sqrt{2}} y_{ij}^{u,d} & \nonumber
                               \label{yukawa_flavour}
\end{align}

\noindent where $v$ is the higgs vacum expectation value and $y_{ij}^{u,d}$ are called {\it yukawa couplings}.
The last are free parametres that represent the coupling strength between higgs and quark fields.
Indices $i,j$ label the three generations. Whereas indices $L,R\;$ idicate the Left, Right handed
projection\footnote{The weak interaction couples only to left handed particvles {\color{red} rephrase... }} of the quark field.
Finally the matrix $\bf m^{u,d}$ expresees the desired quark masses.

Quark fields $y_{ij}^{u,d}$ in \equref{yukawa_flavour} have a definite quantum number that labels the generation to which they belong
to and also wheather they are of up or down type. This quantum number is commonly called {\it flavour}. Or in other words quark fields
are flavour eigenstates. By constraction the mass matrix is not diagonal which implies that
a quark with a well defined flavour does not have a well defined mass. Or in more formal pharasing flavour and mass eigenstates of
the quark fields do not coincide. In order to obtain propper quark masses the matrix $\bf m^{u,d}$ has to be diagonalised, as shown in \equref{diagM}.

\begin{equation}
  m^{d,u}_{\text diag} = V_L^{d,u} m^{d,u} \left(V_R^{d,u}\right)^{\dagger},
  \label{diagM}
\end{equation}

\noindent where the matrices $V$ are required to be unitary. Since \equref{yukawa_flavour} has to stay intact after $m^{d,u}$ is replaced with
$m^{d,u}_{\text diag}$, the quark fields need to be rotated, as shown in \equref{quark_rotation} such that they cancell the additional $V$ matrices.

\begin{equation}
  \left( d_{i}^m \right)_{L,R} = \left( V^d_{ij} d_{j} \right)_{L,R}, \;\;\;\; \left( u_{i}^m \right)_{L,R} = \left( V^u_{ij} u_{j} \right)_{L,R}
  \label{quark_rotation}
\end{equation}

\noindent Since the quark fields are still flavour eigenstetes in the rest of the Standard Model lagrangian,
the field rotations of \equref{quark_rotation} need to be applied there as well. Specifically
to the kinetic term involving quark interactions with the charged weak bosons \Wpm, also known as {\it charged current}
interaction. In \equref{CClagrangian} the charged current interaction is exressed in two ways. One where the quarks
are flavour and mass eigenstates respectivelly.

\begin{align}
  \mathscr{L}_{\text{Kinetic}}^{CC} &\propto \bar{u}_{Li} \gamma_\mu {\Wm}^\mu d_{Ri} + \bar{d}_{Li} \gamma_\mu {\Wp}^\mu u_{Ri} + {\color{red} h.c.??}  \\
                               &\propto \bar{u}_{Li}^m \left( V^u_LV^{d\dagger}_L\right)_{ij} \gamma_\mu {\Wm}^\mu d_{Ri}^m + \bar{d}_{Li}^m \left( V^u_LV^{d\dagger}_L\right)_{ij} {\Wp}^\mu \gamma_\mu u_{Ri}^m + {\color{red} h.c.??} \nonumber \\
                               % &\propto \bar{u}_{Li}^m V_{\text{CKM}} \gamma_\mu {\Wm}^\mu d_{Ri}^m + \bar{d}_{Li}^m V_{\text{CKM}} \gamma_\mu {\Wp}^\mu u_{Ri}^m + {\color{red} h.c.??}
  \label{CClagrangian}
\end{align}

\noindent Where, \Wpm are the charged weak boson fields and $\gamma_\mu$ are Dirac matrices.
The matrix $V^u_LV^{d\dagger}_L$ is the so called {\it CKM mixing matrix}.

The last equation summarizes an important aspect of the weak interaction, namelly that
quark flavour mixing allows for flavour transitions between the generations,
despite the fact that the charged weak bosons $\Wpm$ couple only to flavour eigenstates.
The probability of these transitions are described by the elements of the CKM mixing matrix,
or simply CKM matrix. Also, from the same equation it is deduced that CP-Violation in the Standard Model is possible
only if \equref{diagM} is not invariant under complex conjugation. The last turns out to be true
bacause of the fact that CKM matrix is a complex valued matrix. Thus quark masses and CP-Violation
have a common origin in the standard model.

\subsubsection{CKM mixing matrix}
The CKM matrix, is a complex matrix which as already mentioned describes the streanght of quark couplings,
or the probability of a certain quark transition. The elements of the CKM matrix have been measured [{\color{red} ref pdg}]
showing an intreasting structure, see \equref{CKMmatrix}. Essentially the structure implies that transitions between generations
are supresed with respect to transitions in the same generation in a symmetric way. The most suppressed transitions are between
the first and third generations then between second and third and the lesast suppressed are between first and second.

\begin{equation}
  |V_{\text{CKM}}| \equiv |V^u_L V^{d\dagger}_L|
                   = \begin{pmatrix} \VudMag & \VusMag & \VubMag \\ \VcdMag & \VcsMag & \VcbMag \\ \VtdMag & \VtsMag & \VtbMag \end{pmatrix}
              \simeq \begin{pmatrix} 1 & 0.2 & 0.008 \\ 0.2 & 1 & 0.04 \\ 0.008 & 0.04 & 1 \end{pmatrix}
      \label{CKMmatrix}
  \end{equation}

By construction the CKM matrix has 3 real paramters and one compelx phase\footnote{Due to unitary constraints plus quark field rotations.}
The choise of the CKM matrix parametrazation is arbitrary. However, due to the observed structure the so called {\it Wolfenstein} parametrization
is a common choise. see \equref{}.

\begin{figure}[h]
  \centering
  {\sffamily 

\hspace*{0.05\textwidth}
\begin{fmffile}{Figures/Chapter1/qqMixing}
  \fmfframe(8,16)(8,16){
    \begin{fmfgraph*}(60,35)
      \fmfstraight
      \fmfleft{u}
      \fmfright{d,W}
      \fmf{fermion}{u,V,d}
      \fmf{boson}{V,W}
      \fmflabel{$u_i$}{u}
      \fmflabel{$d_j$}{d}
      \fmflabel{$\Wp$}{W}
      \fmflabel{\hspace{0.2cm}$V_{ij}$}{V}
    \end{fmfgraph*}
  }
\end{fmffile}

%
% \hspace*{0.05\textwidth}
% \begin{fmffile}{Figures/Chapter1/udMixing}
%   \fmfframe(8,16)(8,16){
%     \begin{fmfgraph*}(60,35)
%       \fmfstraight
%       \fmfleft{u}
%       \fmfright{d,W}
%       \fmf{fermion}{u,V,d}
%       \fmf{boson}{V,W}
%       \fmflabel{$u_i$}{u}
%       \fmflabel{$d^m_j$}{d}
%       \fmflabel{$\Wp$}{W}
%       \fmflabel{\hspace{0.2cm}$V_{id}$}{V}
%     \end{fmfgraph*}
%   }
% \end{fmffile}
% \hspace*{0.05\textwidth}
% \begin{fmffile}{Figures/Chapter1/usMixing}
%   \fmfframe(8,16)(8,16){
%     \begin{fmfgraph*}(60,35)
%       \fmfstraight
%       \fmfleft{u}
%       \fmfright{d,W}
%       \fmf{fermion}{u,V,d}
%       \fmf{boson}{V,W}
%       \fmflabel{$u_i$}{u}
%       \fmflabel{$s^m_j$}{d}
%       \fmflabel{$\Wp$}{W}
%       \fmflabel{\hspace{0.2cm}$V_{is}$}{V}
%     \end{fmfgraph*}
%   }
% \end{fmffile}
% \hspace*{0.05\textwidth}
% \begin{fmffile}{Figures/Chapter1/ubMixing}
%   \fmfframe(8,16)(8,16){
%     \begin{fmfgraph*}(60,35)
%       \fmfstraight
%       \fmfleft{u}
%       \fmfright{d,W}
%       \fmf{fermion}{u,V,d}
%       \fmf{boson}{V,W}
%       \fmflabel{$u_i$}{u}
%       \fmflabel{$b^m_j$}{d}
%       \fmflabel{$\Wp$}{W}
%       \fmflabel{\hspace{0.2cm}$V_{ib}$}{V}
%     \end{fmfgraph*}
%   }
% \end{fmffile}
}
  \caption{Quark mixing. An up-type quark with specific flavour couples to any of the three, ($d,s,b$), down type quarks.
           The coupling strength coresponds to an element of the CKM mixing matrix. Time flows from left to right.}
  \label{QuarkMixing}
\end{figure}


There is no aprior hierarchy\footnote{neutrinos do not have the same hierarchy.}.
Regardless of that the CKM matrix inside SM has to be consistent.

\begin{itemize}
  \item no aprior hierarchy
  \item neutrinos do not have similar structure
  \item consistency test
  \item The falvour physics observables, puzzle fasi, hits for Np, We need everything.
\end{itemize}
