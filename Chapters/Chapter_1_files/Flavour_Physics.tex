As it was mentioned in \secref{The_Standard_Model} quarks and leptons acquire mass through the Yukawa term
of the Standard Model lagrangian. An important aspect of the weak interaction, emerges from this term,
namelly the quark field mixing. A brief description of quark mixing and its relevance to CP-Violation is given in what follows.
The section concludes with a brief status overview of the {\it flavour physics}, which as it was implied in \secref{The_Standard_Model}
is the part of Standard Model responsible for up-down quark type transitions.

\subsubsection{Quark Mixing}
By applying the higgs mechanism to the Standard Model lagrangian the higgs field obtains a
{\it vacum expectation value}\footnote{This is the lowest value of the higgs field potential.}.
After this step the Yukawa term for the quark fields is
shown\footnote{Ignoring quark-higgs field interaction terms} in \equref{yukawa_flavour}.

\begin{align}
  % -\mathscr{L}_{\text{Yukawa}} = M_{ij}^d \bar{d_{Li}} d_{Rj} + M_{ij}^u \bar{u_{Li}} u_{Rj} + h.c.,
  -\mathscr{L}_{\text{Yukawa}} &= \left[ y_{ij}^d \bar{d}_{Li} d_{Rj} + y_{ij}^u \bar{u}_{Li} u_{Rj} \right] \frac{v}{\sqrt{2}} + h.c. \nonumber \\
                               &= \left[ m_{ij}^d \bar{d}_{Li} d_{Rj} + m_{ij}^u \bar{u}_{Li} u_{Rj} \right] + h.c.,
  \label{yukawa_flavour}
\end{align}

\noindent where $v$ is the higgs vacum expectation value and $y_{ij}^{u,d}$ are free parametres that represent the
coupling strength between higgs and quark fields. The matrix $\bf m^{u,d}$, which combines the last two contains
expresees the quark masses. Indices $i,j$ label the three generations. Whereas indices $L,R$ idicate that the Left, Right handed
projection\footnote{The weak interaction couples only to left handed particvles {\color{red} rephrase... }} of the quark field is used.

Quark fields in \equref{yukawa_flavour} have a definite generation number and up or down type, or definite flavour as it
is commonly termed. However, the mass matrix $\bf m^{u,d}$ is not diagonal which implies that quarks cannot be represented
in single basis with definite flavour and mass. In order to obtain propper quark masses the last matrix has to be diagonalised,
as shown in

\begin{equation}
  m^{d,u}_{\text diag} = V_L^{d,u} m^{d,u} \left(V_R^{d,u}\right)^{\dagger},
  \label{diagM}
\end{equation}

\noindent where the matrices $V$ are required to be unitary. Since \equref{yukawa_flavour} has to stay intact after the last
rotaion, the quark fields need to be rotated, as shown in \equref{quark_rotation} such that they cancell the additional $V$ matrices.

\begin{equation}
  \left( d_{i}^m \right)_{L,R} = \left( V^d_{ij} d_{j} \right)_{L,R}, \;\;\;\; \left( u_{i}^m \right)_{L,R} = \left( V^u_{ij} u_{j} \right)_{L,R}
  \label{quark_rotation}
\end{equation}

The change of base of \equref{quark_rotation} needs to be applied to the rest of the \equref{lagrangian}. Specifically
to the kinetic term involving quark interactions with the charged weak bosons \Wpm, also known as {\it charged current}
interaction. \equref{CClagrangian} shows the last term in both bases.

\begin{align}
  \mathscr{L}_{\text{Kinetic}}^{CC} &\propto \bar{u}_{Li} \gamma_\mu {\Wm}^\mu d_{Ri} + \bar{d}_{Li} \gamma_\mu {\Wp}^\mu u_{Ri} + {\color{red} h.c.??}  \\
                               &\propto \bar{u}_{Li}^m \left( V^u_LV^{d\dagger}_L\right)_{ij} \gamma_\mu {\Wm}^\mu d_{Ri}^m + \bar{d}_{Li}^m \left( V^u_LV^{d\dagger}_L\right)_{ij} {\Wp}^\mu \gamma_\mu u_{Ri}^m + {\color{red} h.c.??} \nonumber \\
                               &\propto \bar{u}_{Li}^m V_{\text{CKM}} \gamma_\mu {\Wm}^\mu d_{Ri}^m + \bar{d}_{Li}^m V_{\text{CKM}} \gamma_\mu {\Wp}^\mu u_{Ri}^m + {\color{red} h.c.??}
  \label{CClagrangian}
\end{align}

\noindent Where, \Wpm are the charged weak boson fields and $\gamma_\mu$ are Dirac matrices.
\equref{CClagrangian} summarizes an important aspect of the weak interaction, namelly that the quark flavour eigenstates
do not coincide with the mass eigenstates and thus quark flavours are mixed when quarks are expressed in the mass eigenstate basis.
{\it Quark flavour mixing} allows for up-down flavour transitions between the generations, see \figref{QuarkMixing}.
The probability of these transitions are described by the so called {\it CKM mixing matrix}, shown in \equref{CKMmatrix}.

\subsubsection{CKM mixing matrix}
The CKM mixing matrix, or simply CKM matrix, is a complex matrix which as already mentioned describes the
strenght of quark couplings.
CKM is the source of CP violation in the SM
\begin{equation}
  \left|V_{\text{CKM}}\right| = \begin{pmatrix} \Vud & \Vus & \Vub \\ \Vcd & \Vcs & \Vcb \\ \Vtd & \Vts & \Vtb \end{pmatrix}
      \label{CKMmatrix}
  \end{equation}

\begin{figure}[h]
  \centering
  {\sffamily 

\hspace*{0.05\textwidth}
\begin{fmffile}{Figures/Chapter1/qqMixing}
  \fmfframe(8,16)(8,16){
    \begin{fmfgraph*}(60,35)
      \fmfstraight
      \fmfleft{u}
      \fmfright{d,W}
      \fmf{fermion}{u,V,d}
      \fmf{boson}{V,W}
      \fmflabel{$u_i$}{u}
      \fmflabel{$d_j$}{d}
      \fmflabel{$\Wp$}{W}
      \fmflabel{\hspace{0.2cm}$V_{ij}$}{V}
    \end{fmfgraph*}
  }
\end{fmffile}

%
% \hspace*{0.05\textwidth}
% \begin{fmffile}{Figures/Chapter1/udMixing}
%   \fmfframe(8,16)(8,16){
%     \begin{fmfgraph*}(60,35)
%       \fmfstraight
%       \fmfleft{u}
%       \fmfright{d,W}
%       \fmf{fermion}{u,V,d}
%       \fmf{boson}{V,W}
%       \fmflabel{$u_i$}{u}
%       \fmflabel{$d^m_j$}{d}
%       \fmflabel{$\Wp$}{W}
%       \fmflabel{\hspace{0.2cm}$V_{id}$}{V}
%     \end{fmfgraph*}
%   }
% \end{fmffile}
% \hspace*{0.05\textwidth}
% \begin{fmffile}{Figures/Chapter1/usMixing}
%   \fmfframe(8,16)(8,16){
%     \begin{fmfgraph*}(60,35)
%       \fmfstraight
%       \fmfleft{u}
%       \fmfright{d,W}
%       \fmf{fermion}{u,V,d}
%       \fmf{boson}{V,W}
%       \fmflabel{$u_i$}{u}
%       \fmflabel{$s^m_j$}{d}
%       \fmflabel{$\Wp$}{W}
%       \fmflabel{\hspace{0.2cm}$V_{is}$}{V}
%     \end{fmfgraph*}
%   }
% \end{fmffile}
% \hspace*{0.05\textwidth}
% \begin{fmffile}{Figures/Chapter1/ubMixing}
%   \fmfframe(8,16)(8,16){
%     \begin{fmfgraph*}(60,35)
%       \fmfstraight
%       \fmfleft{u}
%       \fmfright{d,W}
%       \fmf{fermion}{u,V,d}
%       \fmf{boson}{V,W}
%       \fmflabel{$u_i$}{u}
%       \fmflabel{$b^m_j$}{d}
%       \fmflabel{$\Wp$}{W}
%       \fmflabel{\hspace{0.2cm}$V_{ib}$}{V}
%     \end{fmfgraph*}
%   }
% \end{fmffile}
}
  \caption{Quark mixing. An up-type quark with specific flavour couples to any of the three, ($d,s,b$), down type quarks.
           The coupling strength coresponds to an element of the CKM mixing matrix. Time flows from left to right.}
  \label{QuarkMixing}
\end{figure}




\begin{itemize}
  \item CKM picture
  \item The falvour physics observables, puzzle fasi, hits for Np, We need everything.
\end{itemize}
