As it was mentioned in \secref{The_Standard_Model} quarks and leptons acquire mass through the Yukawa term
of the Standard Model lagrangian. An important aspect of the weak interaction, emerges from that term,
namelly the quark flavour mixing. A brief description of the last and its relevance to CP-Violation is given in what follows.
The section concludes with a brief status overview of {\it flavour physics}, which as it was implied in \ref{The_Standard_Model}
is the part of Standard Model responsible for up-down quark type transitions.

\subsubsection{Quark Flavour Mixing}
By applying the higgs mechanism to the Standard Model lagrangian the higgs field obtains a
{\it vacum expectation value}\footnote{This is the lowest value of the higgs field potential.}.
After this step the Yukawa term for the quark fields is shown\footnote{Ignoring quark-higgs field interaction terms} in \equref{yukawa_flavour}.

\begin{subequations}
\label{yukawa_flavour}
  \begin{align}
  % -\mathscr{L}_{\text{Yukawa}} = M_{ij}^d \bar{d_{Li}} d_{Rj} + M_{ij}^u \bar{u_{Li}} u_{Rj} + h.c.,
  -\mathscr{L}_{\text{Yukawa}} &= \left[ y_{ij}^d \bar{d}_{Li} d_{Rj} + y_{ij}^u \bar{u}_{Li} u_{Rj} \right] \frac{v}{\sqrt{2}} + h.c. + ...  \\
                               &= \left[ m_{ij}^d \bar{d}_{Li} d_{Rj} + m_{ij}^u \bar{u}_{Li} u_{Rj} \right] + h.c. + ...,  \\
                               \text{with} \;\;\; m^{u,d}_{i,j} = \frac{v}{\sqrt{2}} y_{ij}^{u,d} & \nonumber
  \end{align}
\end{subequations}

\noindent where $v$ is the higgs vacum expectation value and $y_{ij}^{u,d}$ are complex valued numbers called {\it yukawa couplings}.
The last are free parametres that represent the coupling strength between higgs and quark fields.
Indices $i,j$ label the three generations. Whereas indices $L,R\;$ idicate the Left, Right handed
projection\footnote{The weak interaction couples only to left handed particvles {\color{red} rephrase... }} of the quark field.
Finally the matrix $\bf m^{u,d}$ expresees the desired quark masses.

The quark fields, $u$ and $d$,  in \equref{yukawa_flavour} have a definite quantum number that labels the generation to which they belong
to and also wheather they are of up or down type. This quantum number is commonly called {\it flavour}. Impling that quark fields
are flavour eigenstates. By constraction the mass matrix is not diagonal which means that
a quark with a well defined flavour does not have a well defined mass. Or in more formal pharasing flavour and mass eigenstates of
the quark fields do not coincide. In order to obtain propper quark masses the matrix $\bf m^{u,d}$ has to be diagonalised, as shown in \equref{diagM}.

\begin{equation}
  m^{d,u}_{\text diag} = V_L^{d,u} m^{d,u} \left(V_R^{d,u}\right)^{\dagger},
  \label{diagM}
\end{equation}

\noindent where the matrices $V$ are required to be unitary. Since \equref{yukawa_flavour} has to stay intact after $m^{d,u}$ is replaced with
$m^{d,u}_{\text diag}$, the quark fields need to be rotated, as shown in \equref{quark_rotation} such that they cancell the additional $V$ matrices
of \equref{diagM}.

\begin{equation}
  \left( d_{i}^m \right)_{L,R} = \left( V^d_{ij} d_{j} \right)_{L,R}, \;\;\;\; \left( u_{i}^m \right)_{L,R} = \left( V^u_{ij} u_{j} \right)_{L,R}
  \label{quark_rotation}
\end{equation}

\noindent Since the quark fields are still flavour eigenstetes in the rest of the Standard Model lagrangian,
the field rotations of \equref{quark_rotation} need to be applied there as well. Specifically
to the kinetic term involving quark interactions with the charged weak bosons \Wpm, also known as {\it charged current}
interaction. In \equref{CClagrangian} the charged current interaction is exressed in two ways.
First the quarks are flavour, \equref{CClagrangianInt}, and second mass, \equref{CClagrangianMass}, eigenstates.

\begin{subequations}
  \label{CClagrangian}
  \begin{align}
    \mathscr{L}_{\text{Kinetic}}^{CC} & \propto \bar{u}_{Li} \gamma_\mu {\Wm}^\mu d_{Ri} + \bar{d}_{Li} \gamma_\mu {\Wp}^\mu u_{Ri}  \label{CClagrangianInt} \\
                                      & \propto \bar{u}_{Li}^m  {V_{\text{CKM}}} \gamma_\mu {\Wm}^\mu d_{Ri}^m + \bar{d}_{Li}^m V_{\text{CKM}} {\Wp}^\mu \gamma_\mu u_{Ri}^m \label{CClagrangianMass}
  \end{align}
\end{subequations}

\noindent Where, \Wpm are the charged weak boson fields and $\gamma_\mu$ are Dirac matrices.
The matrix $V_{\text{CKM}} \equiv V^u_LV^{d\dagger}_L$ is the so called {\it CKM mixing matrix}, shown in \equref{quark_field_rotation}.

\begin{equation}
  \begin{pmatrix} \dquark \\ \squark \\ \bquark  \end{pmatrix} =
  \underbrace{\begin{pmatrix} \Vud & \Vus & \Vub \\ \Vcd & \Vcs & \Vcb \\ \Vtd & \Vts & \Vtb \end{pmatrix}}_{V_{\text{CKM}}}
    \begin{pmatrix} \dquark^m \\ \squark^m \\ \bquark^m  \end{pmatrix}
  \label{quark_field_rotation}
  \end{equation}

The charged current term, \equref{CClagrangian}, incorporates an important aspect of the weak interaction,
namelly that the quark mass eigenstates are superpositions of the flavour eigenstates. This feauture is called
quark flavour mixing.

\begin{figure}[h]
  \centering
  {\sffamily 

\hspace*{0.05\textwidth}
\begin{fmffile}{Figures/Chapter1/qqMixing}
  \fmfframe(8,16)(8,16){
    \begin{fmfgraph*}(60,35)
      \fmfstraight
      \fmfleft{u}
      \fmfright{d,W}
      \fmf{fermion}{u,V,d}
      \fmf{boson}{V,W}
      \fmflabel{$u_i$}{u}
      \fmflabel{$d_j$}{d}
      \fmflabel{$\Wp$}{W}
      \fmflabel{\hspace{0.2cm}$V_{ij}$}{V}
    \end{fmfgraph*}
  }
\end{fmffile}

%
% \hspace*{0.05\textwidth}
% \begin{fmffile}{Figures/Chapter1/udMixing}
%   \fmfframe(8,16)(8,16){
%     \begin{fmfgraph*}(60,35)
%       \fmfstraight
%       \fmfleft{u}
%       \fmfright{d,W}
%       \fmf{fermion}{u,V,d}
%       \fmf{boson}{V,W}
%       \fmflabel{$u_i$}{u}
%       \fmflabel{$d^m_j$}{d}
%       \fmflabel{$\Wp$}{W}
%       \fmflabel{\hspace{0.2cm}$V_{id}$}{V}
%     \end{fmfgraph*}
%   }
% \end{fmffile}
% \hspace*{0.05\textwidth}
% \begin{fmffile}{Figures/Chapter1/usMixing}
%   \fmfframe(8,16)(8,16){
%     \begin{fmfgraph*}(60,35)
%       \fmfstraight
%       \fmfleft{u}
%       \fmfright{d,W}
%       \fmf{fermion}{u,V,d}
%       \fmf{boson}{V,W}
%       \fmflabel{$u_i$}{u}
%       \fmflabel{$s^m_j$}{d}
%       \fmflabel{$\Wp$}{W}
%       \fmflabel{\hspace{0.2cm}$V_{is}$}{V}
%     \end{fmfgraph*}
%   }
% \end{fmffile}
% \hspace*{0.05\textwidth}
% \begin{fmffile}{Figures/Chapter1/ubMixing}
%   \fmfframe(8,16)(8,16){
%     \begin{fmfgraph*}(60,35)
%       \fmfstraight
%       \fmfleft{u}
%       \fmfright{d,W}
%       \fmf{fermion}{u,V,d}
%       \fmf{boson}{V,W}
%       \fmflabel{$u_i$}{u}
%       \fmflabel{$b^m_j$}{d}
%       \fmflabel{$\Wp$}{W}
%       \fmflabel{\hspace{0.2cm}$V_{ib}$}{V}
%     \end{fmfgraph*}
%   }
% \end{fmffile}
}
  \caption{Feynman diagram where an up-type quark couples to any of the three, ($d,s,b$), down type quarks,
           via a \Wp boson. Time flows from left to right.}
  \label{QuarkMixing}
\end{figure}

\noindent By construction the $V_{\text{CKM}}$ rotates only the down type quarks, impling that
the mass eigenstates of the up-type quarks are identical to the flavour eigenstates. Thus an up type quark
changes its flavour to any of the down type quarks, see \figref{QuarkMixing}, with a certain probabilty.
The probability of such a transitions is given by the elements of the CKM mixing matrix, or simply CKM matrix.

\subsubsection{CKM mixing matrix}
The CKM matrix, is a complex matrix which as already mentioned describes the streanght of quark couplings,
or in other words the probability of a certain quark flavour transition. The elements of the CKM matrix have been measured [{\color{red} ref pdg}]
showing an intreasting structure, see \equref{CKMmatrix}. Essentially the structure implies that transitions between generations
are supresed with respect to transitions within the same generation, in a symmetric way. The most suppressed transitions are between
the first and third generations followed by the ones between the second and third and the least suppressed are between first and second.

\begin{equation}
  |V_{\text{CKM}}|
                   = \begin{pmatrix} \VudMag & \VusMag & \VubMag \\ \VcdMag & \VcsMag & \VcbMag \\ \VtdMag & \VtsMag & \VtbMag \end{pmatrix}
              \simeq \begin{pmatrix} 1 & 0.2 & 0.008 \\ 0.2 & 1 & 0.04 \\ 0.008 & 0.04 & 1 \end{pmatrix}
      \label{CKMmatrix}
  \end{equation}

Testing the consistency of the CKM elements measurments is a central goal of flavour physics.
In order to achieve such tests it is usefull to define a paramterization of the CKM matrix.
By construction the CKM matrix has 3 real parameters and one compelx phase\footnote{Due to unitary constraints plus quark field rotations.,{\color{red} clarify}}
The choise of the CKM matrix parametrazation is arbitrary. Due to the observed structure the so called {\it Wolfenstein}[{\color{red} ref pdg}]
parametrization shown in \equref{CKMwolfenstein} is a standard parametrization, where one can see the three real parameters, $\lambda,A,\rho$
and the complex one $\eta$. Note that the CKM element $\Vts$ has a complex part at higher order in $\lambda$.

\begin{equation}
  |V_{\text{CKM}}|
                   = \begin{pmatrix} \VudWolf & \VusWolf & \VubWolf \\
                                     \VcdWolf & \VcsWolf & \VcbWolf \\
                                     \VtdWolf & \VtsWolf & \VtbWolf \end{pmatrix} + \mathcal{O}(\lambda^4)
      \label{CKMwolfenstein}
\end{equation}

\noindent As it was previously mentioned the CKM matrix is a unitary matrix, meaning that $V_{\text{CKM}} V_{\text{CKM}}^\dagger = I_{3x3}$.
This leads to the so called unitarity and orthogonality relations. The last are sums of complex numbers that are equal to zero, thus can be
represented by triangles in the complex plane. There are six orthogonality releations two of which are relevant fo the current thesis since
the CKM elements present in those relations govern the dynamics in the \Bs and \Bd meson systems. These two relations and their corresponding
triangles are show in \equref{unitConstraints}, \figref{unitTriangles} respectivelly.

\begin{subequations}
  \label{unitConstraints}
  \begin{align}
    \Bd : & \quad \Vud\Vub^* + \Vcd\Vcb^* + \Vtd\Vtb^* = 0
    \label{unitConstraints_Bd} \\
    \Bs : & \quad \Vus\Vub^* + \Vcs\Vcb^* + \Vts\Vtb^* = 0
    \label{unitConstraints_Bs}
  \end{align}
\end{subequations}

\begin{figure}[h]
  \centering
  \begin{subfigure}{0.475\textwidth}
    \raggedright
    \includegraphics[width=\textwidth]{Figures/Chapter1/b-d-triangle}
    \caption{}
    \label{unitTriangles_bd}
  \end{subfigure}%
  \begin{subfigure}{0.525\textwidth}
    \raggedleft
    \includegraphics[width=\textwidth]{Figures/Chapter1/b-s-triangle}
    \caption{}
    \label{unitTriangles_bs}
  \end{subfigure}
  \caption{CKM-unitarity triangles. (A) \Bd triangle, corresponding to \equref{unitConstraints_Bd}. (B) \Bs triangle,
           corresponding to \equref{unitConstraints_Bs}. Triangle sides have been normalised with respect to to one of them.
           This way one of the sides is real with unit leangth. Note that triangles are not drawn to scale. Figures from {\color{red} Jeroen}
           rho bar and eta bar is defined as {\color{red} define rho bar eta bar}.  }
  \label{unitTriangles}
\end{figure}

\noindent The angles in \figref{unitTriangles} are defined by inspecting the triangles in the complex plane,
as shown in \equref{bdAnglesDef} and \equref{bdAnglesDef}.

\begin{equation}
  \label{bdAnglesDef}
  \alpha \equiv \arg\left( -\frac{\Vtd\Vtb^*}{\Vud\Vub^*} \right)
  \quad
  \beta  \equiv \arg\left( -\frac{\Vcd\Vcb^*}{\Vtd\Vtb^*} \right)
  \quad
  \gamma \equiv \arg\left( -\frac{\Vud\Vub^*}{\Vcd\Vcb^*} \right)
\end{equation}

\begin{equation}
  \label{bsAnglesDef}
  %\alpha_s \equiv \arg\left( -\frac{\Vus\Vub^*}{\Vts\Vtb^*} \right)
  %\quad
  \beta_s \equiv \arg\left( -\frac{\Vts\Vtb^*}{\Vcs\Vcb^*} \right)
  %\quad
  %\gamma_s \equiv \arg\left( -\frac{\Vcs\Vcb^*}{\Vus\Vub^*} \right)
\end{equation}

\begin{figure}[h]
  \begin{center}
    \includegraphics[trim=0cm 0cm 0cm 0cm, clip=true, width=\textwidth]{Figures/Chapter1/rhoeta_large.png}
    \caption{Show rhobar and etabar wrt rho and eta and lamda.... Mention one side is real because we devided by it.}
    \label{unitarity_triangle}
  \end{center}
\end{figure}

There is no fundamental reason known in the Standard model for the observed hierarchy of the CKM elemetns.
One of the main goals of flavour physics is to check the consistency of all CKM elements measurements by overlaying them on the
complex plane, see \figref{unitarity_triangle}. For complteness it is intreasting to mention that in the lepton sector a similar
mixing matrix is active but with completelly different structure.


\subsubsection{Standard Model CP Violation}
From \equref{CClagrangianMass} it can be deduced[{\color{red} ref niels}] that CP-Violation
in the Standard Model is possible only if \equref{diagM} is not invariant under complex conjugation.
The last turns out to be true bacause of the fact that CKM matrix is a complex valued matrix.
Thus, the CKM matrix which was introduced to asign propper masses to the quarks is also the source of
CP-Violation in the Standard Model [{\color{red} KM paper}], which is is an intreasting coinsidence\footnote{or an apparent coincidence.}
