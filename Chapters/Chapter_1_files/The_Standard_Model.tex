After a few decades of research in the subatomic level it has been possible to account for a large number of phenomena
in nature, based on the existance of only a handfull of particles. Which is impressive given the vast dimmensions that the
universe spans over. Two distinct categories emerge from the above collection of particles, namelly the {\it gauge
bossons} responsible for mediating three out of the four know fundamental forces of
nature\footnote{These are the gravity, electromagnetism, strong, and weak nuclear forces.}, and {\it quarks} plus {\it leptons} which are the
constituents of matter
(and antimatter)\footnote{Anitmatter is not as exotic as it sounds. It is a state of matter where the
signs of all quantum numbers (like the electric charge) of a particle are fliped.}. The are collectivelly called as {\it fermions}.
The resently discoverd higgs bosson [{\color{red}refff}] does not fall in any of the above clasification due to each spacial role of explaining how particles acquire mass.
\figref{sm_particles} illustrates those matter particles and their interactions via the gauge bossons.
The mathematical "language" necessary to describe those interactions is spelled out by the \textit{Standard Model} of Particle Physics
which is a quantum field
theory\footnote{Quantum field theories are[{\color{red}ref}], roughly speaking, theories that combine Einstein's special theory of
relativity and quantum mechanics.}.
In this framework particles are described by quantum fields which is a dificult
to grasp yet the simplest known way to represent a particle.

\begin{equation}
\mathscr{L}_{\text{SM}} =
                  %  \underbrace{i \bar\psi D^{\mu} \gamma _{\mu} \psi}_{\mathscr{L}_{\text{Kinetic}}} +
                  %  % \underbrace{(D^\mu\phi)^{\dagger} (D^\mu\phi) -\mu^2\phi^{\dagger}\phi - \lambda(\phi^{\dagger}\phi)^2}_{\mathscr{L}_{\text{Higgs}}} +
                  %  \underbrace{(D^\mu\phi)^{\dagger} (D^\mu\phi) + V(\phi)}_{\mathscr{L}_{\text{Higgs}}} +
                  %  \underbrace{Y_{ij}\bar\psi_{Li}\phi\psi_{Rj} \; + \; h.c.}_{\mathscr{L}_{\text{Yukawa}}},
\mathscr{L}_{\text{Kinetic}} + \mathscr{L}_{\text{Higgs}} + \mathscr{L}_{\text{Yukawa}}
\label{lagrangian}
\end{equation}

% \footnote{$D^\mu=\partial ^\mu + ig_sG_{\alpha}^\mu L_{\alpha} +  igW_b^\mu\sigma_b + ig^\prime B^\mu Y$ \cite{covariant derivative??}}

The above mentioned mathematical language is the so called
{\it lagrangian}\footnote{Lagrangians are elegant equations that discribe the dynamics of a system. They are used both in classical
systems such as motions of planets and in the quantum world of tiny distances.}
of the Standard Model and it has three main terms as shown in \equref{lagrangian}. The first {\it Kinnetic}, term describes the possible
interactions between the fermions. The second, {\it Higgs}, term is the one that contains the masses of the gauge and higgs bossons.
Whereas the last,
{\it Yukawa}\footnote{The {\it Kinnetic} and {\it Higgs} terms appear more naturealy in the lagrangian wheras the last one is added by hand, which is perfectly fine.},
includes the fermion masses. It is intreasting to point out that all the possible intractions between the fermions are a natural outcome of the symmetries that
\equref{lagrangian} has. By construction, the Standard Model lagrangian obeys the symmetry group\footnote{with the mathematical notion of a group implied}
$SU(3)_c\otimes SU(2)_L\otimes U(1)_Y$. This means that there are three distinct types of transformations of the fermion fields that leave \equref{lagrangian}
invariant. Each type introduces the electromagnetic, weak and strong interaction between the
fermion fields. The formal mathematical phrasing of the last senceses is spelled out by {\it Noether's theorem} and {\it local gauge invariance}.
Any more details are out of the scope of athe current thesis.

\begin{figure}[h]
  \begin{center}
    \includegraphics[trim=1.4cm 0cm 5.95cm 0cm, clip=true, width=\textwidth]{Figures/Chapter1/Standard_model_infographic.png}
    \caption{Standard Model matter particles and their possible interactions. The boxes indicate the allowed interactions
             between the gauge bossons and the fermions. The logic follows from the mahematical sets. For example the electon
             intects with the photon and the weak bosons but not with the gluon. {\color{red} Maybe this is confusing id does nto work for the gauge bosons.}}
    \label{sm_particles}
  \end{center}
\end{figure}

As shown in \figref{sm_particles} Standard Model particles come in three generations, illustrated by columns in the same fugure.
The weak intreaction operates with, or as it is commonly termed, {\it couples to} fermions in the same generation only, effectivelly
changing an up quark to down for example. Although quarks and leptons are both fermions, a quark can never become a lepton via
the weak interaction or vice versa in the Standard Model.The above paradigm with up and down quarks extends to the rest of the
generations as well. Resulting in the up-type, $u,c,t$ and the down-type $d,s,b$ quarks\footnote{Similarly for the  up-type, $\electron,\mu,\tau$
and the down-type $\neue,\neum,\neut$ leptons}. The above mentioned up-down type quark weak interactions are coverned by the
so called {\it flavour} quantum number. each quarck has a unique flavour number. Only the weak interaction can change the flaouvour
the other itenractions are blind to it. The situation is similar but more complex for th string interactions the corresponding
quantom number in that case is calleed {\it color}.

In addition nearly all of the ordinarry matter \footnote{Ordinary here means all the directly observed matter.} in the universe cosists of
particles from the first generation. Intreastingly enough there is no fundamental reason known as to why there are three generations though.


Within the accuracy of the current experiments the Standard Model has seen its predictions confirmed to a great extend.
The most recent and perhaps one of the most crucial meaning the discovery of the mechanism that particles acquire mass, makes the Standard
Model looks quite robust. However there are phenomena and observations that it can not account for. Perhaps the most stricking one is the
absence of any discription about the most familiar and strong force of nature, meaning gravity\footnote{Gravity is not renormalisable[\cite{}]
theory and thus cannot be described as a quantum field theory}. or the peculiar value of the cosmological constant leading to dark energy
interpretations[\cite{}].But even at the heart of the Standard Model there obscure aspects for example the well established fact that neutrinos
have non zero mass[\cite{}] or the unexplained amount of the observed matter-antimatter assymetry in the universe [\cite{}].
Lastly nearly all of the ordinarry matter \footnote{Ordinary here means all the directly observed matter.} in the universe cosists of
particles from the first generation of fermions. Intreastingly enough there is no fundamental reason known as to why there are three
generations though. For all of the above reasons plus our curiosity driven nature scientists are compled to continue testing the Standard Model
and look for deviations of its predictions.
