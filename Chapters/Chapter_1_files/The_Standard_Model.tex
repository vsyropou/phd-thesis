After a few decades of research in the subatomic scale it has been possible to account for a large number of phenomena
in nature, based on the existance of only a handfull of particles. Which is impressive given the vast dimmensions that the
universe spans over. Two distinct categories emerge from the above collection of particles, namelly the {\it gauge
bossons} responsible for mediating three out of the four know fundamental forces of nature, and quarks plus leptons which are the
constituents of matter
(and antimatter)\footnote{Anitmatter is not as exotic as it sounds. It is a state of matter where the
signs of all quantum numbers (like the electric charge) of a particle are fliped.}.
The resently discoverd higgs bosson [refff] is a special category by itself due to each spacial role of explaining how particles acquire mass.
Figure \figref{} illustrates those matter particles and their interactions via the gauge bossons (also known as force cariers).
The mathematical "language" necessary to describe those interactions is spelled out by the \textit{Standard Model} of Particle Physics
which is a quantum field
theory\footnote{Quantum field theories are[ref], roughly speaking, the result of combining Einstein's special theory of
relativity[reference] and quantum mechanics[ref].}.
In this framework particles are described by quantum fields which is a dificult
to grasp but very usefull way to represent a
particle\footnote{For further reading on quantum filed theories [refff]}.

\begin{equation}
\mathscr{L}_{SM} = \underbrace{i \bar\psi D^{\mu} \gamma _{\mu} \psi}_{\mathscr{L}_{\text{Kinetic}}} +
                   % \underbrace{(D^\mu\phi)^{\dagger} (D^\mu\phi) -\mu^2\phi^{\dagger}\phi - \lambda(\phi^{\dagger}\phi)^2}_{\mathscr{L}_{\text{Higgs}}} +
                   \underbrace{(D^\mu\phi)^{\dagger} (D^\mu\phi) + V(\phi)}_{\mathscr{L}_{\text{Higgs}}} +
                   \underbrace{Y_{ij}\bar\psi_{Li}\phi\psi_{Rj} \; + \; h.c.}_{\mathscr{L}_{\text{Yukawa}}},
\label{lagrangian}
\end{equation}

% \footnote{$D^\mu=\partial ^\mu + ig_sG_{\alpha}^\mu L_{\alpha} +  igW_b^\mu\sigma_b + ig^\prime B^\mu Y$ \cite{covariant derivative??}}

The above mentioned mathematical language in its most compact form i shown in \equref{lagrangian}. The last equation is the so called
{\it lagrangian}\footnote{Lgrangians are elegant equations that discribe the dynamics of a system. They are used both in classical physical
systems such as motions of planets and in the quantum world of tiny distances.}
of the Standard Model. The first {\it Kinnetic}, term of \equref{lagrangian} describes the possible interactions between the quarcks or the leptons.
The second, {\it Higgs}, term is the one that includes the masses of the gauge and higgs bossons. Whereas the last, {\it Yukawa}, attributes
masses to the quarks and
leptons\footnote{The {\it Kinnetic} and {\it Higgs} terms appear more naturealy in the lagrangian wheras the last one is added by hand, which is perfectly fine.}
Quarcks or leptons and the higgs fields are represented by $\psi$ and $\phi$ in \equref{lagrangian} respectivelly. The $\gamma_\mu$ is a Dirac matrix
that takes care of the additional structure of the fields due to their intrinsic spin. $V(\phi)$ is the potential term of the
higgs field and it contains its mass. The $L$ and $R$ subscripts denote the left and right handed projections of the quarck or lepton fields.
$Y_{ij}$ are called Yuakawa cuplings. Their meaning and relevance to the current thesis is shown in \secref{}.
The symbol $D_\mu$ introduces the interactions of the gauge bossons with the quarks or leptons quantum fields and it is called covariant derivative[ref????].

\begin{figure}[h]
  \begin{center}
    \includegraphics[width=0.5\textwidth]{Figures/Chapter1/Standard_Model_Particles.png}
    \caption{Standard Model matter particles. Combinations of three and two quarks build hadrons and messons respectivelly.
    Hadrons are for example protons and neutrons wheares a messon could be the \Bs particle which is relevant for the current thesis.
    Leptons can be charged (\electron,\mmu,\mtau) and uncharged (\neue,\neum,\neut). The first of the charged ones is the familiar
    electron that exists in the neucleous of every atom and the rest are heavier "brothers" of it. Netrinos on the other hand
    do not take part in any known nucleous-like formation they are also massless within the Standard Model making them extremelly (and notoriously)
    difficult to detect. They live in the least well known "neighborhoud"of the Standard Model. Their presence is indirectly
    implied in the radioactive decay of atoms and. The gauge bossons \g, g, \Wpm \Z  are responsible for mediating the electromagnetic,
    strong, and weak interactions between quarcks or leptons repsectivelly. These possible interactions are indicated by the blue
    lines. Some of the gauge bossons can interact with themselves which is illustrated by the blue closed loop lines.}
    \label{sm_particles}
  \end{center}
\end{figure}

The core element that makes \equref{lagrangian} elegant is the so called {\it local gauge invariance}[ref??].
Briefely speaking this means that \equref{lagrangian} is invariant (or symmetric) under spacetime dependant
phase\footnote{the quark and lepton quantum fields are complex valued quantities. Thus phase here means complex phase.} transformations of the quark or lepton fields.
By construction, the Lagrangian obeys the symmetry
group\footnote{with the mathematical notion of a group implied} $SU(3)_c\otimes SU(2)_L\otimes U(1)_Y$.
This means that there are three distinct transformations of the $\psi$ fields that \equref{lagrangian} is symetric to.
Each one introduces the electromagnetic, weak and strong interactions between the fields. The mathematics
behind the above mentioned transoformations really gives the current paragreaph proper meaning but it would
be completeley out of context here. Further reading on the symmetries of th Standard Model can be found in [?????????]

% In addition the quarks and the gluon quantum fields have an extra degre of freedom called color. Each quark can have three
% any of the three color degrees of freadom. Which implies that as far as the strong interaction is concerned the the quarks are
% actually 9 and not 3. In case of the gluons the color degree of freadom implies that there are eight gluons resulting from
% the symmetry structure of the strong interaction. Lastly, the vast majority of the observed matter is built from \uquark, \dquark
% quarks and \electron which makes the explanation of the triplet like structure of quarcks and leptons very enigmatic and difficult
% to understand from first principles.In addition the quarks and the gluon quantum fields have an extra degre of freedom called color. Each quark can have three
% any of the three color degrees of freadom. Which implies that as far as the strong interaction is concerned the the quarks are
% actually 9 and not 3. In case of the gluons the color degree of freadom implies that there are eight gluons resulting from
% the symmetry structure of the strong interaction.

Within the accuracy of the current experiments the Standard Model has seen its predictions confirmed to a great extend.
The most recent and perhaps one of the most crucial meaning the discovery of the mechanism that particles acquire mass, makes the Standard
Model looks quite robust. However there are phenomena and observations that it can not account for. Perhaps the most stricking one is the
absence of any discription about the most familiar and strong force of nature, meaning gravity\footnote{Gravity is not renormalisable[\cite{}]
theory and thus cannot be described as a quantum field theory}. or the peculiar value of the cosmological constant leading to dark energy
interpretations[\cite{}].But even at the heart of the Standard Model there obscure aspects for example the well established fact that neutrinos
have non zero mass[\cite{}] or the unexplained amount of the observed matter-antimatter assymetry in the universe [\cite{}]. For all of the
above reasons plus our curiosity driven nature scientists are compled to continue testing the Standard Model and look for deviations of
its predictions.
