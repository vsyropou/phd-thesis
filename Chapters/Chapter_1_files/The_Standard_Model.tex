After a few decades of research in the subatomic level a large number of natural phenomena have been accounted
for, based on the existance of only a handfull of particles. Which is impressive given the vast dimmensions that the
universe spans over. Two distinct categories emerge from the above collection of particles, namelly the {\it gauge
bossons} responsible for mediating three out of the four know fundamental forces of
nature\footnote{These are the gravity, electromagnetism, strong, and weak nuclear forces.}, and {\it quarks} plus {\it leptons} which are the
constituents of matter
(and antimatter)\footnote{Anitmatter is not as exotic as it sounds. It is a state of matter where the
signs of all quantum numbers (like the electric charge) of a particle are fliped.}. The last are collectivelly called {\it fermions}.
The resently discoverd higgs bosson [{\color{red}refff}] does not fall in any of the above clasification due to each spacial role of
explaining how particles acquire mass. \figref{sm_particles} illustrates those two types of particles.
The mathematical "language" necessary to describe those interactions is spelled out by the \textit{Standard Model} of Particle Physics
which is a model based on quantum field
theory\footnote{Quantum field theories are[{\color{red}ref}], roughly speaking, theories that combine Einstein's special theory of
relativity and quantum mechanics.}.
In this framework particles are described by quantum fields which is a dificult to grasp yet the simplest known way to represent a particle,
when it comes to the sub atomic scale.

\begin{equation}
\mathscr{L}_{\text{SM}} =
                  %  \underbrace{i \bar\psi D^{\mu} \gamma _{\mu} \psi}_{\mathscr{L}_{\text{Kinetic}}} +
                  %  % \underbrace{(D^\mu\phi)^{\dagger} (D^\mu\phi) -\mu^2\phi^{\dagger}\phi - \lambda(\phi^{\dagger}\phi)^2}_{\mathscr{L}_{\text{Higgs}}} +
                  %  \underbrace{(D^\mu\phi)^{\dagger} (D^\mu\phi) + V(\phi)}_{\mathscr{L}_{\text{Higgs}}} +
                  %  \underbrace{Y_{ij}\bar\psi_{Li}\phi\psi_{Rj} \; + \; h.c.}_{\mathscr{L}_{\text{Yukawa}}},
\mathscr{L}_{\text{Kinetic}} + \mathscr{L}_{\text{Higgs}} + \mathscr{L}_{\text{Yukawa}}
\label{lagrangian}
\end{equation}

% \footnote{$D^\mu=\partial ^\mu + ig_sG_{\alpha}^\mu L_{\alpha} +  igW_b^\mu\sigma_b + ig^\prime B^\mu Y$ \cite{covariant derivative??}}

The above mentioned mathematical language is the so called
{\it lagrangian}\footnote{Lagrangians are elegant equations that discribe the dynamics of a system. They are used both in classical
systems such as motions of planets and in the quantum world of tiny distances.}
of the Standard Model and it has three main terms as shown in \equref{lagrangian}. The first, {\it Kinnetic}, term describes the possible
interactions between the fermions. The second, {\it Higgs}, term is the one that contains the higgs potential, responsible for generating
masses for the gauge and higgs bossons which are also contained in the Higgs term.
Whereas the last,
{\it Yukawa}\footnote{The {\it Kinnetic} and {\it Higgs} terms appear more naturealy in the lagrangian wheras the last one is added by hand, which is perfectly fine.},
is introduced to couple the higgs field with the fermions and thus generate masses for them. It is intreasting to point out that all the possible intractions between the fermions are a natural outcome of the symmetries that
\equref{lagrangian} has. By construction, the Standard Model lagrangian obeys the symmetry group\footnote{with the mathematical notion of a group implied}
$SU(3)_c\otimes SU(2)_L\otimes U(1)_Y$. This means that there are three distinct types of transformations of the fermion fields that leave \equref{lagrangian}
invariant. Each type introduces the strong, weak and electromagnetic interactions respectivelly between the
fermion fields. The formal mathematical phrasing of the last senceses is spelled out by {\it Noether's theorem} and {\it local gauge invariance}[{\color{red}reff}],
which is a beutiful concept but goes beyond the scope of athe current thesis.

\begin{figure}[h]
  \begin{center}
    \includegraphics[trim=1.4cm 0cm 5.95cm 0cm, clip=true, width=\textwidth]{Figures/Chapter1/Standard_model_infographic.png}
    \caption{Standard Model matter particles and their possible interactions. The boxes indicate the allowed interactions
             between the gauge bossons and the fermions. The logic follows from the mahematical sets. For example the electon
             intects with the photon and the weak bosons but not with the gluon. {\color{red} Maybe this is confusing id does nto work for the gauge bosons.}}
    \label{sm_particles}
  \end{center}
\end{figure}

As shown in \figref{sm_particles} Standard Model particles come in three {\it generations}, illustrated by columns in the same fugure.
The weak intreaction operates with, or as it is commonly termed, {\it couples to} fermions in the same generation, effectivelly
changing an up quark to down for example. Although quarks and leptons are both fermions, a quark can never become a lepton via
the weak interaction or vice versa in the Standard Model.The above paradigm of up and down quarks extends to the rest of the
generations as well resulting in the up-type ($u,c,t$) and the down-type ($d,s,b$) quarks\footnote{Similarly for the  up-type ($\electron,\mu,\tau$)
and the down-type ($\neue,\neum,\neut$) leptons}. The above mentioned up-down type transitions are commonly called {\it Flavour} transitions.
and are based on the so called {\it weak isospin}. The last is a quantum number that each fermion has, depending on
wheather it is of up or down type. Weak isospin is a conserved quantity in the Standard Model. The charged weak bossons \Wpm have
the ability to change the flavour whereas the neutral \Z  does not, see \figref{ela}. These kind of interactions play a central role
in the matter-antimatter assymetry in the universe which is is called {\it CP-Violation}. For completness, strong interactions are
based on a similar quantum number namelly the {\it color}. As far as the strong interactions are conserned, quark fields carry one
of the three type of the color quantum number. Color transistions are mediated by the gluons, g. Lastly the electromagnetic interactions
is simpler in structure and it involves the familiar quantum number namelly the electric charge.

\begin{figure}[h]
  {\sffamily 
\hspace*{0.05\textwidth}
\begin{fmffile}{Figures/Chapter1/CCInteractions}
  \fmfframe(8,16)(8,16){
    \begin{fmfgraph*}(60,35)
      \fmfstraight
      \fmfleft{u}
      \fmfright{d,W}
      \fmf{fermion}{u,V,d}
      \fmf{boson}{V,W}
      \fmflabel{$u,c,t$}{u}
      \fmflabel{$d,s,b$}{d}
      \fmflabel{$\Wpm$}{W}
    \end{fmfgraph*}
  }
\end{fmffile}
\hfill
% \hspace{0.03\textwidth}
\begin{fmffile}{Figures/Chapter1/NCInteractions}
  \fmfframe(8,16)(8,16){
    \begin{fmfgraph*}(60,35)
      \fmfstraight
      \fmfright{q,qbar}
      \fmfleft{Z}
      \fmf{fermion}{q,V,qbar}
      \fmf{boson}{V,Z}
      \fmflabel{$q$}{qbar}
      \fmflabel{$\bar{q}$}{q}
      \fmflabel{$\Z$}{Z}
    \end{fmfgraph*}
    }
  \end{fmffile}
}
  \caption{Weak interactions of quarks in the Standard Model. Left: A charged weak boson changes the flavour of a up-type quark to a down-type within the same generation.
           Right: A neutral weak boson decays into quark anti quark pair. Time flows form left to right.}
  \label{WeakInteractions}
\end{figure}

Within the accuracy of the current experiments Standard Model has seen its predictions confirmed to a great extend.
The recent and crusial discovery of the higgs bosson, which is the mechanism that particles acquire mass, makes the Standard Model look
much more robust. However, the last is a model and not a compelte theory. There are established phenomena and observations that it can not
account for. Perhaps the most stricking one is the absence of any discription about the most familiar, yet the weakest, force of nature,
meaning gravity\footnote{Gravity cannot be incorporated in a quantum field theory because the necessary re-normalisation does not yield a
meaningful result.}.
Or the peculiar value of the cosmological constant leading to dark energy
interpretations[\cite{}].But even at the heart of the Standard Model there are obscure aspects for example the well established fact
that neutrinos have non zero mass[\cite{}] or the unexplained amount of the observed matter-antimatter assymetry in the universe [\cite{}].
Lastly nearly all of the ordinarry matter \footnote{Ordinary here means all the directly observed matter.} in the universe cosists of
particles from the first generation of fermions. Intreastingly enough there is no fundamental reason embeded in the Standard Model
as to why there are exactly three generations. For all of the above reasons plus our curiosity driven nature scientists are compled
to continue testing Standard Model predictions and look for ways to improve it.
