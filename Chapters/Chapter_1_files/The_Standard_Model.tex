After a few decades of research in the subatomic level a large number of natural phenomena have been accounted
for, based on the existance of only a handfull of idivisible-elementary particles. Which is impressive given the vast dimmensions that the
universe spans over. Two distinct categories emerge from the above collection of particles, namelly {\it gauge
bossons} responsible for mediating all the known fundamental forces of
nature\footnote{With the exception of gravity.}, and {\it quarks} plus {\it leptons} which are the
constituents of matter
(and antimatter\footnote{Anitmatter is a state of matter where the signs of all quantum numbers, like the electric charge, of a particle are fliped.}).
The last are collectivelly called {\it fermions}. The resently discoverd higgs bosson~\cite{higgs-cms,higgs-atlas} is does not mediate
any force but it plays a spacial role in explaining how particles acquire mass. \figref{sm_particles} illustrates those two types of particles.
The mathematical framework necessary to describe the interactions between elemntary particles is called \textit{Standard Model} of Particle
Physics~\cite{sm-glashow,sm-weinberg,sm-salam} which is a quantum field theory. The last is a theoretical framework for constructing quantum
mechanical models of subatomic particles. In this framework particles are treated as excited states of their underlying physical field.
This description of a particle is a dificult to grasp yet the simplest known way to represent a particle, when it comes to the subatomic scale.

\begin{equation}
\mathscr{L}_{\text{SM}} =
                  %  \underbrace{i \bar\psi D^{\mu} \gamma _{\mu} \psi}_{\mathscr{L}_{\text{Kinetic}}} +
                  %  % \underbrace{(D^\mu\phi)^{\dagger} (D^\mu\phi) -\mu^2\phi^{\dagger}\phi - \lambda(\phi^{\dagger}\phi)^2}_{\mathscr{L}_{\text{Higgs}}} +
                  %  \underbrace{(D^\mu\phi)^{\dagger} (D^\mu\phi) + V(\phi)}_{\mathscr{L}_{\text{Higgs}}} +
                  %  \underbrace{Y_{ij}\bar\psi_{Li}\phi\psi_{Rj} \; + \; h.c.}_{\mathscr{L}_{\text{Yukawa}}},
\mathscr{L}_{\text{Kinetic}} + \mathscr{L}_{\text{Higgs}} + \mathscr{L}_{\text{Yukawa}}
\label{lagrangian}
\end{equation}

% \footnote{$D^\mu=\partial ^\mu + ig_sG_{\alpha}^\mu L_{\alpha} +  igW_b^\mu\sigma_b + ig^\prime B^\mu Y$ \cite{covariant derivative??}}

The exact details of the elentary particle interactions are incorporated in the so called
{\it lagrangian}\footnote{ Lagrangians are elegant equations that discribe the dynamics of a system.} of the Standard Model,
 and it has three main terms as shown in \equref{lagrangian}. The first, {\it Kinnetic}, term describes the possible
interactions between the fermions. The second, {\it Higgs}, term is the one that contains the higgs potential, responsible for generating
masses for the gauge and higgs bossons which are also contained in the same term.
Whereas the last, {\it Yukawa}, is introduced to couple the higgs field with the fermions and thus generate masses for them.
It is intreasting to point out that the Kinetic and Higgs terms are introduced in a more natural way.
Specifically they are the result of exploiting the symmetries that \equref{lagrangian} has.
By construction, the Standard Model lagrangian obeys the symmetry group\footnote{the mathematical notion of a group is implied.}
$SU(3)_c\otimes SU(2)_L\otimes U(1)_Y$. This means that there are three distinct types of transformations of the fermion fields that leave \equref{lagrangian}
invariant. Each type introduces the strong, weak and electromagnetic interactions respectivelly between the
fermion fields following the consept of {\it local gauge invariance}~\cite{aitchison}.

\begin{figure}[h]
  \begin{center}
    \includegraphics[trim=1.4cm 0cm 5.95cm 0cm, clip=true, width=\textwidth]{Figures/Chapter1/Standard_model_infographic.png}
    \caption{Standard Model matter particles and their possible interactions. The boxes indicate the allowed interactions
             between the gauge bossons and the fermions. The logic follows from the mahematical sets. For example the electon
             intects with the photon and the weak bosons but not with the gluon. {\color{red} Maybe this is confusing id does nto work for the gauge bosons.}}
    \label{sm_particles}
  \end{center}
\end{figure}

As it was previously mentioned gauge bosons mediate fundamental forces. Theses are the electromagnetic, the strong and the weak nuclear forces.
The first one is mediated by the familiar photon, $\gamma$. The last one interacts with , or more appropriatelly {\it couples to}, any paticle that
caries an electric charge quantume number. The strong force is mediated by the gluons, $\gluon$, and it couples to different quantum number the so
called {\it color}. Only the quarks have a color quantum number which can be either red, green, or blue. Lastly the weak force is mediated
by the three weak gauge bosons, \Wpm, \Z. and can interact with all the fermions. The ascociated quantum number is the so called {\it weak isospin}
explained in the next paragraph.

Standard Model particles come in three {\it generations}, illustrated by columns in \figref{sm_particles}.
The weak intreaction couples to fermions in the same generation, effectivelly changing an up quark (\uquark) to a down (\dquark) for example.
Although quarks and leptons are both fermions, a quark can never become a lepton via the weak interaction or vice versa in the Standard Model.
The above paradigm of up and down quarks extends to the rest of the generations as well resulting in the up-type ($u,c,t$) and the down-type ($d,s,b$)
quarks\footnote{Similarly for the  up-type ($\electron,\mu,\tau$) and the down-type ($\neue,\neum,\neut$) leptons}. The above mentioned up-down type
transitions are commonly called {\it Flavour} transitions
and are based on the so called {\it weak isospin}. The last is a quantum number that each fermion has, depending on
wheather it is of up or down type. The charged weak bossons \Wpm have the ability to change the flavour whereas the neutral \Z  does not, see \figref{WeakInteractions}.

For completion it is mentioned that the weak interaction violates both the parity ($P$) and the charge ($C$) symmetries.
The first one is ascociated with left and right symmetry. Whereas the ohter with fliping the signs of all quantum numbers of a particle.
In both cases \equref{lagrangian} is not invariant under parity and charge transformations. What is intreasting experimentally
is that when it comes to neutrinos the weak interaction couples only to left handed\footnote{Handedness or {\it chirality} is the quantum number ascocited to the spin-momentum dot product.
Chirality can be $+1$, right-handed,  or $-1$, left-handed. } neutrinos and right anti-neutrinos~\cite{wu-parity,garwin-parity}.
Which means that the $P$ and $C$ are maximally violated indicating the peculiar structure of the weak interaction. It is reminded that
the electromagnetic and strong interactions respect those symetries. In hindshight of those observations the combined CP transformation
seems like a good symmetry of nature that could be ascociated with the matter-antimatter symmetry in the universe.
However this is not the case and nature has a way to distinguish between amatter and antimatter through the weak interaction.
The origin of CP-Violation in the Standard Model is driefly addressed in \secref{Flavour_Physics}.


\begin{figure}[h]
  {\sffamily 
\hspace*{0.05\textwidth}
\begin{fmffile}{Figures/Chapter1/CCInteractions}
  \fmfframe(8,16)(8,16){
    \begin{fmfgraph*}(60,35)
      \fmfstraight
      \fmfleft{u}
      \fmfright{d,W}
      \fmf{fermion}{u,V,d}
      \fmf{boson}{V,W}
      \fmflabel{$u,c,t$}{u}
      \fmflabel{$d,s,b$}{d}
      \fmflabel{$\Wpm$}{W}
    \end{fmfgraph*}
  }
\end{fmffile}
\hfill
% \hspace{0.03\textwidth}
\begin{fmffile}{Figures/Chapter1/NCInteractions}
  \fmfframe(8,16)(8,16){
    \begin{fmfgraph*}(60,35)
      \fmfstraight
      \fmfright{q,qbar}
      \fmfleft{Z}
      \fmf{fermion}{q,V,qbar}
      \fmf{boson}{V,Z}
      \fmflabel{$q$}{qbar}
      \fmflabel{$\bar{q}$}{q}
      \fmflabel{$\Z$}{Z}
    \end{fmfgraph*}
    }
  \end{fmffile}
}
  \caption{Weak interactions of quarks in the Standard Model. Left: A charged weak boson changes the flavour of a up-type quark to a down-type within the same generation.
           Right: A neutral weak boson decays into quark anti quark pair. Time flows form left to right.}
  \label{WeakInteractions}
\end{figure}

Within the accuracy of the current experiments Standard Model has seen its predictions confirmed to a great extend.
The recent and crusial discovery of the higgs bosson, which is the mechanism that particles acquire mass, makes the Standard Model look
more robust. However, the last is a model and not a compelte theory. There are established phenomena and observations that it can not
account for. Perhaps the most stricking one is the absence of any discription about the most familiar, yet the weakest, force of nature,
meaning gravity\footnote{Gravity cannot be expressed incorporated in a quantum field theory so far, due to issues in the renormalization procedure {\color{red} cite ...}}.
Or the fact that ordinary matter, meaning matter that consists of Standard Model particles, acounts for only a small fraction of the total matter
that is believed to be present in the universe~\cite{dmatter-Hinshaw}. But even at the heart of the Standard Model there are obscure aspects for example the well established fact
that neutrinos have non zero mass~\cite{nu-mass-superkam,nu-mass-kamland,nu-mass-sno,nu-mass-daya} or the unexplained amount of the observed
matter-antimatter assymetry in the universe~\cite{more-cpv-huet,more-cpv-gavela_I,more-cpv-gavela_II}.
Lastly nearly all of the ordinarry matter in the universe cosists of
particles from the first generation of fermions. Intreastingly enough there is no fundamental reason embeded in the Standard Model
as to why there are exactly three generations. For all of the above reasons the scientific method compeles scientists to continue
testing Standard Model predictions and look for ways to improve it.
