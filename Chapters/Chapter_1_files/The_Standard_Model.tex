After a decada of research at the subatomic level, a large number of natural phenomena have been accounted
for, based on the existence of only a handful of elementary particles. Given the vast dimensions that the
universe spans over, this is an impresive achievement. The above collection of particles can be calsified
in two distinct categories, namely {\it gauge bosons}, responsible for mediating all the known fundamental
forces of nature\footnote{With the exception of gravity.}, and {\it fermions} which are the constituents
of matter
(and antimatter\footnote{Antimatter is a state of matter where the signs of all quantum numbers, like the electric charge, of a particle are flipped.}).
The last can be devided further into {\it quarks} plus {\it leptons}, where the later do not interact through
the {\it strong} force, which is one of the above mentioned fundamental forces. The recently discovered higgs
boson~\cite{higgs-cms,higgs-atlas} does not mediate any force but it plays a spacial role in explaining how
particles acquire mass. \figref{sm_particles} illustrates those two types of particles.
The mathematical framework necessary to describe the interactions between these elementary particles is called
the \textit{Standard Model} of Particle Physics~\cite{sm-glashow,sm-weinberg,sm-salam} which is a quantum field theory.
The last is a theoretical framework for constructing quantum mechanical models of subatomic particles.
In this framework particles are treated as excited states of their underlying field. This description of a
particle is maybe difficult to grasp yet the simplest known way to represent a particle, when it comes to the subatomic scale.

% \footnote{$D^\mu=\partial ^\mu + ig_sG_{\alpha}^\mu L_{\alpha} +  igW_b^\mu\sigma_b + ig^\prime B^\mu Y$ \cite{covariant derivative??}}

The exact details of the elementary particle interactions are incorporated in the so called
{\it Lagrangian}\footnote{ Lagrangians are elegant equations that describe the dynamics of a system.} of the Standard Model,
and it has three main terms:

\begin{equation}
\mathscr{L}_{\text{SM}} =
                  %  \underbrace{i \bar\psi D^{\mu} \gamma _{\mu} \psi}_{\mathscr{L}_{\text{Kinetic}}} +
                  %  % \underbrace{(D^\mu\phi)^{\dagger} (D^\mu\phi) -\mu^2\phi^{\dagger}\phi - \lambda(\phi^{\dagger}\phi)^2}_{\mathscr{L}_{\text{Higgs}}} +
                  %  \underbrace{(D^\mu\phi)^{\dagger} (D^\mu\phi) + V(\phi)}_{\mathscr{L}_{\text{Higgs}}} +
                  %  \underbrace{Y_{ij}\bar\psi_{Li}\phi\psi_{Rj} \; + \; h.c.}_{\mathscr{L}_{\text{Yukawa}}},
\mathscr{L}_{\text{Kinetic}} + \mathscr{L}_{\text{Higgs}} + \mathscr{L}_{\text{Yukawa}}
\label{lagrangian}
\end{equation}

\noindent The first, or {\it Kinetic}, term describes the possible interactions between the fermions.
The second, or {\it Higgs}, term is the one that contains the higgs potential, responsible for generating
masses for gauge and Higgs bosons which are also contained in the same term.
The last, or {\it Yukawa}, is introduced to couple the higgs field with the fermions and thus generate masses for them.
It is interesting to point out that the Kinetic and Higgs terms are introduced in a more natural way.
Specifically, they are the result of exploiting the symmetries that \equref{lagrangian} has.
By construction, the Standard Model Lagrangian obeys the symmetry group\footnote{the mathematical notion of a group is implied.}
$SU(3)_c\otimes SU(2)_L\otimes U(1)_Y$. This means that there are three distinct types of transformations
that leave \equref{lagrangian} invariant. Each type of transformation introduces respectivelly the strong,
weak and electromagnetic interactions between the fermion fields following the concept of {\it local gauge invariance}~\cite{aitchison}.

\begin{figure}[h]
  \begin{center}
    \includegraphics[trim=1.4cm 0cm 5.95cm 0cm, clip=true, width=\textwidth]{Figures/Chapter1/Standard_model_infographic.png}
    \caption{Standard Model matter particles. {\color{red} Maybe use a less fancy picture.}}
    \label{sm_particles}
  \end{center}
\end{figure}

As previously mentioned, gauge bosons mediate fundamental forces. Theses are the electromagnetic, the strong and the weak nuclear forces.
The first one is mediated by the familiar photon, $\gamma$. The last one interacts with, or more appropriately, {\it couples to}, any particle that
caries an electric charge quantum number. The strong force is mediated by the gluons, $\gluon$, and it couples to different quantum number,
the so called {\it color}. Only the quarks have a color quantum number which can be labelled as either red, green, or blue. Lastly the
weak force is mediated by the three weak gauge bosons, \Wpm, \Z. and can interact with all fermions. The associated quantum number is
the so called {\it weak isospin} explained in the next paragraph.

Standard Model particles come in three {\it generations}, illustrated by columns in \figref{sm_particles}.
The weak interaction dominantly couples to fermions in the same generation, effectively changing an up quark (\uquark) to a down (\dquark) for example.
Although quarks and leptons are both fermions, a quark can never become a lepton via the weak interaction or vice versa in the Standard Model.
The above paradigm of up and down quarks extends to the rest of the generations as well resulting in the up-type ($u,c,t$) and the down-type ($d,s,b$)
quarks\footnote{Similarly for the  up-type ($\electron,\mu,\tau$) and the down-type ($\neue,\neum,\neut$) leptons}. The above mentioned up-down type
transitions are commonly called {\it Flavour} transitions and are based on the so called {\it weak isospin}. The last is a quantum number
that each fermion has, depending on whether it is of up or down type. 

The charged weak bosons \Wpm have the ability to change the flavour whereas the neutral \Z  does not, see \figref{WeakInteractions}.

For completion it is mentioned that the weak interaction violates both the parity ($P$) and the charge ($C$) symmetries.
The first one is associated with left and right symmetry. Whereas the other with flipping the signs of all quantum numbers of a particle.
In both cases \equref{lagrangian} is not invariant under parity and charge transformations. What is interesting experimentally
is that when it comes to neutrinos the weak interaction couples only to left handed\footnote{Handedness or {\it chirality} is the quantum number associated to the spin-momentum dot product.
Chirality can be $+1$, right-handed,  or $-1$, left-handed. } neutrinos and right anti-neutrinos~\cite{wu-parity,garwin-parity}.
Which means that the $P$ and $C$ are maximally violated indicating the peculiar structure of the weak interaction. It is reminded that
the electromagnetic and strong interactions respect those symmetries. In hindsight of those observations the combined CP transformation
seems like a good symmetry of nature that could be associated with the matter-antimatter symmetry in the universe.
However this is not the case and nature has a way to distinguish between matter and antimatter through the weak interaction.
The origin of CP-Violation in the Standard Model is briefly addressed in \secref{Flavour_Physics}.


\begin{figure}[h]
  \centering
  {\sffamily 
\hspace*{0.05\textwidth}
\begin{fmffile}{Figures/Chapter1/CCInteractions}
  \fmfframe(8,16)(8,16){
    \begin{fmfgraph*}(60,35)
      \fmfstraight
      \fmfleft{u}
      \fmfright{d,W}
      \fmf{fermion}{u,V,d}
      \fmf{boson}{V,W}
      \fmflabel{$u,c,t$}{u}
      \fmflabel{$d,s,b$}{d}
      \fmflabel{$\Wpm$}{W}
    \end{fmfgraph*}
  }
\end{fmffile}
\hfill
% \hspace{0.03\textwidth}
\begin{fmffile}{Figures/Chapter1/NCInteractions}
  \fmfframe(8,16)(8,16){
    \begin{fmfgraph*}(60,35)
      \fmfstraight
      \fmfright{q,qbar}
      \fmfleft{Z}
      \fmf{fermion}{q,V,qbar}
      \fmf{boson}{V,Z}
      \fmflabel{$q$}{qbar}
      \fmflabel{$\bar{q}$}{q}
      \fmflabel{$\Z$}{Z}
    \end{fmfgraph*}
    }
  \end{fmffile}
}
  \caption{Weak interactions of quarks in the Standard Model. Left: A charged weak boson changes the flavour of a up-type quark to a down-type within the same generation.
           Right: A neutral weak boson decays into quark anti quark pair. Time flows form left to right.}
  \label{WeakInteractions}
\end{figure}

Within the accuracy of the current experiments Standard Model has seen its predictions confirmed to a great extend.
The recent discovery of the higgs boson, which is the mechanism that particles acquire mass, makes the Standard Model look
more robust. However, the last is a model and not a complete theory. There are established phenomena and observations that it can not
account for. Perhaps the most striking one is the absence of any description about the most familiar, yet the weakest, force of nature,
meaning gravity\footnote{Gravity cannot be expressed incorporated in a quantum field theory so far.}.
Or the fact that ordinary matter, meaning matter that consists of Standard Model particles, accounts for only a small fraction of the total matter
that is believed to be present in the universe~\cite{dmatter-Hinshaw}. But even at the heart of the Standard Model there are obscure aspects for example the well established fact
that neutrinos have non zero mass~\cite{nu-mass-superkam,nu-mass-kamland,nu-mass-sno,nu-mass-daya} or the unexplained amount of the observed
matter-antimatter asymmetry in the universe~\cite{more-cpv-huet,more-cpv-gavela_I,more-cpv-gavela_II}.
Lastly nearly all of the ordinary matter in the universe consists of
particles from the first generation of fermions. Interestingly enough there is no fundamental reason embedded in the Standard Model
as to why there are exactly three generations. For all of the above reasons the scientific method compels scientists to continue
testing Standard Model predictions and look for ways to improve it.
