After a century of research at the subatomic level, a large number of natural phenomena have been accounted
for, based on the existence of only a handful of elementary particles. Given the vast dimensions that the
universe spans, this is an impressive achievement. The above collection of particles can be classified
in two distinct categories, namely {\it gauge bosons}, responsible for mediating all the known fundamental
forces of nature\footnote{With the exception of gravity.}, and {\it fermions} which are the constituents
of matter and antimatter\footnote{Antimatter is a state of matter where the signs of all quantum numbers,
like the electric charge, of a particle are flipped.}. Fermions can be divided further into {\it quarks}
and {\it leptons}, where the later do not interact through the {\it strong} force, one of the above-mentioned
fundamental forces. The recently discovered Higgs boson \cite{higgs-cms,higgs-atlas} is not a gauge bosson and
it does not mediate any force. Hoewver, it plays a special role in explaining how particles acquire mass.
Standard Model fermions are illustrated in \tabref{quarksLeptons}. The mathematical framework necessary to
describe the interactions between these elementary particles is called the \textit{Standard Model} of Particle
Physics \cite{sm-glashow,sm-weinberg,sm-salam} which is a quantum field theory. In this framework particles are
treated as excited states of their underlying field.

% \footnote{$D^\mu=\partial ^\mu + ig_sG_{\alpha}^\mu L_{\alpha} +  igW_b^\mu\sigma_b + ig^\prime B^\mu Y$ \cite{covariant derivative??}}

\begin{table}[h!]
  \centering
 \begin{tabular}{cccc}
   \hline
                            &  1st generation                     &   2nd generation              &  3rd generation    \\
   \hline
   \multirow{2}{*}{quarks}  &  down (d)                           &   strange (s)                 &  beauty (b)        \\
                            &  up (u)                             &   charm (c)                   &  top (t)           \\
   \hline
   \multirow{2}{*}{leptons} &   electron ($\en$)                  &   muon ($\mmu$)               &  tau ($\taum$)     \\
                            &   $\electron$ - neutrino $(\neue)$  &  $\mmu$ - neutrino $(\neum)$  &  $\mtau$ - neutrino $(\neut)$  \\
   \hline
 \end{tabular}
 \caption{Standard Model fermions.}
 \label{quarksLeptons}
\end{table}

The exact details of the elementary particle interactions are incorporated in the so called
{\it Lagrangian} of the Standard Model, and it has three main terms:

\begin{equation}
\mathscr{L}_{\text{SM}} =
                  %  \underbrace{i \bar\psi D^{\mu} \gamma _{\mu} \psi}_{\mathscr{L}_{\text{Kinetic}}} +
                  %  % \underbrace{(D^\mu\phi)^{\dagger} (D^\mu\phi) -\mu^2\phi^{\dagger}\phi - \lambda(\phi^{\dagger}\phi)^2}_{\mathscr{L}_{\text{Higgs}}} +
                  %  \underbrace{(D^\mu\phi)^{\dagger} (D^\mu\phi) + V(\phi)}_{\mathscr{L}_{\text{Higgs}}} +
                  %  \underbrace{Y_{ij}\bar\psi_{Li}\phi\psi_{Rj} \; + \; h.c.}_{\mathscr{L}_{\text{Yukawa}}},
\mathscr{L}_{\text{Kinetic}} + \mathscr{L}_{\text{Higgs}} + \mathscr{L}_{\text{Yukawa}}
\label{lagrangian}
\end{equation}

\noindent The first, or {\it Kinetic}, term describes the possible interactions between the fermions.
The second, or {\it Higgs}, term is the one that contains the Higgs potential, responsible for generating
masses for gauge and Higgs bosons which are also contained in the same term.
The last, or {\it Yukawa}, term is introduced to couple the Higgs field with the fermions and thus generate masses for them.
It is interesting to point out that the Kinetic and Higgs terms orignate in a natural way.
Specifically, they are the result of exploiting the symmetries that \equref{lagrangian} exhibits.
By construction, the Standard Model Lagrangian obeys the symmetry group\footnote{the mathematical notion of a group is implied.}
$SU(3)_c\otimes SU(2)_L\otimes U(1)_Y$. This means that there are three distinct types of transformations
that leave \equref{lagrangian} invariant. Each type of transformation introduces respectively the strong,
weak and electromagnetic interactions between the fermion fields following the concept of {\it local gauge invariance}\cite{aitchison,halzen1984quarks}.

As previously mentioned, gauge bosons mediate fundamental forces. Theses are the electromagnetic plus the strong and weak nuclear forces.
The first one is mediated by the familiar photon, $\gamma$, which interacts with, or more appropriately, {\it couples to}, any particle that
caries an electric charge quantum number. The strong force is mediated by gluons, $\gluon$,
and it couples to a different quantum number, the so called {\it color}.
Only quarks have a non-zero color quantum number which is typically labelled as either red, green, or blue. Lastly the
weak force is mediated by the three weak gauge bosons, \Wpm, \Z and interact with all fermions.
The associated quantum number is the so called {\it weak isospin}, explained in the next paragraph.

Standard Model particles come in three {\it generations}, illustrated in \figref{quarksLeptons}.
The weak interaction dominantly couples to fermions in the same generation, effectively changing an up quark (\uquark) to a down (\dquark) for example.
Although quarks and leptons are both fermions, a quark can never become a lepton via the weak interaction or \viceversa within the Standard Model.
The above paradigm of up and down quarks extends to the rest of the generations as well resulting in the up-type (\uquark,\cquark,\tquark)
and the down-type (\dquark,\squark,\bquark) quarks. Similarly for the  up-type ($\electron,\mu,\tau$) and the down-type (\neue,\neum,\neut)
leptons. The above-mentioned up-down type transitions are commonly called {\it Flavour transitions} and are based on the so called
{\it weak isospin}. The latter is a quantum number that each fermion has; The value of this quantum number depends on whether it is of up or down type.
The charged weak bosons \Wpm have the ability to change the whereas the neutral \Z  does not, see \figref{WeakInteractions}.
An important subtlety is related to the right handed\footnote{Handedness or {\it chirality} is the quantum number associated to the
spin-momentum dot product. Chirality can be $+1$, right-handed, or $-1$, left-handed. } anti-neutrinos, see next paragraph,
for which there is no up-down type structure, \ie they form a so called {\it singlet representation}.

\begin{figure}[t]
  \centering
  \begin{subfigure}{0.49\textwidth}
    \hspace{1.3cm}
    \tikzsetnextfilename{CCInteractions}
    \scalebox{1.}{\begin{fmffile}{Figures/Chapter1/CCInteractions}
  \fmfframe(8,16)(8,16){
    \begin{fmfgraph*}(60,35)
      \fmfstraight
      \fmfleft{u}
      \fmfright{d,W}
      \fmf{fermion}{u,V,d}
      \fmf{boson}{V,W}
      \fmflabel{\uquark,\cquark,\tquark}{u}
      \fmflabel{\dquark,\squark,\bquark}{d}
      \fmflabel{\Wpm}{W}
    \end{fmfgraph*}
  }
\end{fmffile}
}
    \caption{}
    \label{CC_WeakInteractions}
  \end{subfigure}%
  \hfill%
  \begin{subfigure}{0.49\textwidth}
    \hspace{1.3cm}
    \tikzsetnextfilename{NCInteractions}
    \scalebox{1.}{\begin{fmffile}{Figures/Chapter1/NCInteractions}
  \fmfframe(8,16)(8,16){
    \begin{fmfgraph*}(60,35)
      \fmfstraight
      \fmfright{q,qbar}
      \fmfleft{Z}
      \fmf{fermion}{q,V,qbar}
      \fmf{boson}{V,Z}
      \fmflabel{$q$}{qbar}
      \fmflabel{$\bar{q}$}{q}
      \fmflabel{\Z}{Z}
    \end{fmfgraph*}
    }
  \end{fmffile}
}
    \caption{}
    \label{NC_WeakInteractions}
  \end{subfigure}
  \caption{Weak interactions of quarks in the Standard Model. Left: A charged weak boson changes the flavour
           of an up-type quark to a down-type within the same generation.
           Right: A neutral weak boson decays into a quark anti-quark pair. Time flows form left to right.}
\label{WeakInteractions}
\end{figure}

It should be noted that the weak interaction violates both the parity ($P$) and the charge ($C$) symmetries.
The first one is associated with left-right symmetry in three dimensional space.
The second is ascociated with inverting the signs of all quantum numbers of a particle.
In both cases \equref{lagrangian} is not invariant under these parity and charge transformations. What is interesting experimentally
is the observation \cite{wu-parity,garwin-parity} that the weak force interacts only with left handed neutrinos or right-handed
anti-neutrinos\footnote{In addition, given that neutrinos interact only with the weak interaction, it follows that there are no
right-handed neutrinos (and left-handed anti-neutrinos), according to the Standard Model.}. This has to do with the nature of
the weak force which couples only to the left-handed projection of a particle's
wave-function. The implication of the above is that $P$ and $C$ are maximally violated, indicating the
peculiar structure of the weak interaction. It is reminded that the electromagnetic and
strong\footnote{The \CP conservation of the strong force is still to be fully understood.} interactions
respect these symmetries. In view of these observations the combined \CP transformation
seems like a symmetry of nature which seems to imply a matter-antimatter symmetry in the universe \cite{Sakharov:1967dj}.
However the later is not consistent with observations and nature does distinguish between matter and antimatter through
the weak interaction. The origin of \CP-violation in the Standard Model is briefly addressed in \secref{Flavour_Physics}.

Within the accuracy of current experiments the Standard Model has seen its predictions confirmed.
The recent discovery of the Higgs boson, which is the mechanism through which particles acquire mass, makes the Standard Model looks
complete. However, there are established phenomena and observations that it can not
account for. Perhaps the most striking one is the absence of any description about the most familiar, yet the weakest, force of nature,
meaning gravity\footnote{Gravity cannot be incorporated in a quantum field theory so far.}.
Or the fact that ordinary matter, \ie matter that consists of Standard Model particles, accounts for only a small
fraction of the total matter density inferred to be present in the universe \cite{dmatter-Hinshaw}. But even at the heart
of the Standard Model there are aspects for example the well established fact that neutrinos have non zero
mass \cite{nu-mass-superkam,nu-mass-kamland,nu-mass-sno,nu-mass-daya} or the unexplained observed
matter-antimatter asymmetry in the universe \cite{more-cpv-huet,more-cpv-gavela_I,more-cpv-gavela_II}.
Lastly, nearly all of the ordinary matter in the universe consists of
particles from the first generation of fermions. Interestingly enough there is no fundamental reason embedded in the Standard Model
as to why there are exactly three generations. For all of the above reasons, the scientific method compels scientists to continue
testing Standard Model predictions and look for ways to improve it.
