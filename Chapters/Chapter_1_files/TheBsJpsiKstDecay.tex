As it was mentioned in \secref{WeakPhase} \BsJpsiPhi decays are dominated by the tree diagram of \figref{bs2jpsiphi}.
However the same decay could also take place via a higher order supressed process, for example the so called {\it penguin} diagram (or topology)
shown in \figref{bs2jpsiphi_peng}. The \phis measurement of \equref{phis_lhcb} ignores the contribution of the last penguing diagram.

\begin{figure}[h]
  \centering
  {\sffamily %%BoundingBox: -5 0 121 170
%%HiResBoundingBox: -5 0 120.57008 169.36447

\begin{fmffile}{Figures/Chapter1/penguin}
  \fmfframe(17,-25)(31,-25){
    \begin{fmfgraph*}(115,170)
      \fmfstraight
      \fmfleft{i0,i1,i2,i3,i4,i5}
      \fmfright{o0,o1,o2,o3,o4,o5}
      \fmf{fermion,tension=1.8,label.side=left,label=b}{v5,i3}
      \fmf{fermion,tension=1.5,right=0.2,label.side=left,label={\hspace*{18pt}u,,c,,t}}{v2,v5}
      \fmf{gluon,tension=2}{v4,v2}
      \fmf{dbl_dashes,tension=0}{v4,v2}
      \fmf{fermion,tension=0.3,right=0.2,label.side=left }{v3,v2}
      \fmf{boson,tension=0.6,left=0.3,label=W$^+$,label.side=left}{v3,v5}
      \fmf{fermion,label=c,tension=0.9,right=0.3,label.side=left}{o4,v4}
      \fmf{fermion,label=c,tension=0.9,right=0.3,label.side=left}{v4,o3}
      \fmf{fermion,label=s,label.side=left}{o2,v3}
      \fmffreeze
      \fmf{fermion,tension=0.7,label=s,label.side=left}{v1,o1}
      \fmf{fermion,tension=1,label.side=left,label=s}{i2,v1}
      %\fmf{phantom,tension=0.4}{v4,v1}
      \fmf{phantom,tension=0.4}{v3,v1,v5}
      \fmf{plain,right=0.2}{i2,i3}
      \fmf{plain,left=0.2,label=$\Bs$}{i2,i3}
      \fmf{plain,right=0.2,label=$\phi$}{o1,o2}
      \fmf{plain,left=0.2}{o1,o2}
      \fmf{plain,right=0.2,label=$\jpsi$}{o3,o4}
      \fmf{plain,left=0.2}{o3,o4}
    \end{fmfgraph*}
  }
\end{fmffile}
}
  \caption{{\color{red} Fix this to look more like \figref{QuarkMixing} and put ckm elements on the vertices}. Explain tha this is called tree diagram}
  \label{bs2jpsiphi_peng}
\end{figure}

\noindent This has been a reasonable assumtion before the the \lhcb measurement which showed that the measured \phiS{eff} value is consistent with the Standard
Model prediction \phiS{SM,tree}, within the current experimental accuracy. However with increasing accuracy {\color{red} ref lhcb upgrade \phis sensitivy}the last assumption needs to be revisited.
Specifically, because of the fact that penguin topology contributions shift the leading tree level Standard Model prediction.
The last shift could be misinterpreted as $\Delta\phiS{NP}$ if those penguin contributions are not properly esimated.
Thus in order to correctly probe New Physics contributions in fututre measurements the penguin topology contributions $\Delta\phiS{SM,peng}$
have to be estimated.

\begin{equation}
\phis^{\text {eff}} = \phis^{\tiny \text{SM,tree}} + \Delta\phiS{SM,peng} + \Delta\phis^{\tiny \text{NP}}
 \label{phis_sm_peng}
\end{equation}

\noindent The above situation is spelled out in \equref{phis_sm_peng}. The current section gives a brief discription
of the strategy followed in order to estimate those penguin shifts in \phis. The full suggested framework for the
current and future constrains on penguin shifts can be found in {\color{red} Krystofs thesis and roberts papers.}

\subsubsection{Amplitude structure}
Follwoing {\color{red} ref 106 in krystofs thesis.} the \BsJpsiPhi decay amplitude can take place via four different topologies.
Two of them were already introduced in \figref{bs2jpsiphi} and \figref{bs2jpsiphi_peng} which are called {\it color suppressed }$(C)$ and {\it penguin}$(P)$
topologies respectivelly. The other two types namelly {\it exchage} and {\it penguin-anihilation} can be neglected according to {\color{red} krystofs thesis}
given the current experimental precision. Given this assumption the \BsJpsiPhi amplitude is decomposed in \equref{bsjpsiphi_amp}, taking into account the
relevat CKM elements involved in each topology.

\begin{equation}
A \parenthesis{\BsJpsiPhi} = \Vus\Vub^*P_{\uquark} + \Vcs\Vcb^*\brackets{C +P_{\cquark}} + \Vts\Vtb^*P_{\tquark},
 \label{bsjpsiphi_amp}
\end{equation}

\noindent where the suscribts in the penguin topologies, $P$, denote the flavour of the quark present inside the loop of \figref{bs2jpsiphi_peng}.
The above expresion \equref{bsjpsiphi_amp} has to be re-expressed in such a way that it is posible to probe the penguin contributions to
the \BsJpsiPhi decay amplitude. Given the unitarity feature of the CKM matrix plus the Wolfenstein paramtrization of \equref{CKMwolfenstein},
the decay amplitude in \equref{bsjpsiphi_amp} can be rewritten as shown in \equref{bsjpsiphi_amp_param}.

\begin{equation}
  A \parenthesis{\BsJpsiPhi} = \parenthesis{1-\frac{\lambda^2}{2}} \polFrac{f} \brackets{ 1 + \epsilon a_f e^{i\theta_f} e^{i\gamma} },
 \label{bsjpsiphi_amp_param}
\end{equation}

\noindent where the following definitions are used:

\begin{equation}
  \polFrac{f} \equiv \VcbMag \brackets{C + P_c - P_t}, \;\;\;\; a_f e^{i\theta_f} \equiv R_b \brackets{ \frac{P_c - P_t}{C + P_c - P_t} },
  \label{bsjpsiphi_amp_param_defs}
\end{equation}

\noindent with

\begin{equation}
  \epsilon = \frac{\lambda^2}{1-\lambda^2} \;\;\text{and} \;\;  R_b = \parenthesis{1-\frac{\lambda^2}{2}} \frac{1}{\lambda} \modulo{\frac{\Vub}{\Vcb}},
  \label{bsjpsiphi_amp_param_defs}
\end{equation}

Calculations for the penguin contributions to $A \parenthesis{\BsJpsiPhi}$ are available {\color{red} 40 and 41 in kyrstofs thesis}.
However these cacluations are difficult to perform, since they involve non-perturbative long-distance QCD effects. Thus an alternative
approach, according to {\color{red} krystof and rob}, is follwed in order to relate those penguin contributions to similar decay cahanells
as \BsJpsiPhi using the $SU(3)_F$ falvour symmetry, see {\color{red} Krystof thesis section 3.4 ana elsewere.}.
A minimal desctription of the necessesary formalism to extract the penguin shift, $\Delta\phiS{SM,peng}$, is given the rest of the current section.

\subsubsection{Formalism}

Accroding to blah the penguin parameters beta and alpha can be related to the CP asymmetries, \equref{acp_mixing} and \equref{cpv_decay},
as shown in \equref{}.


In addition for modes that include are related to \BsJpsiPhi via the $SU(3)_F$ symmetry


 Given the experimentally measured values of the above CP assymetries {\color{red} ref blah}, a $\chi^2$ fit is
used, see ref {\color{red} krystof} to determine the penguin parameters alpha and beta.

\subsubsection{Including more chanells}
Using su3 flavour we can include more chanels.
In order to increase sensitivuty or for a flavour sepcific finals atetes the Acp mix is nto there thus the H observable is constructed
according to

\begin{itemize}
  \item Acp mix and Acp dir which are measured are realeated to alpha and theta as in equation
  \item Thus for flavour specific final states you need branching ratio information.
  \item footnote(In principle you can solve it with only jpsiphi but you want to introduce more chanells to reduce uncertainty).
\end{itemize}


\begin{itemize}
\item $H_i$, related to the branching ratios and polarisation fractions,
\begin{eqnarray}\label{Eq:Hobs_Vector}
H_i & \equiv &  \frac{1}{\epsilon} \left|\frac{\mathcal{A}'_i}{\mathcal{A}_i}\right|^2
\frac{\text{PhSp}\left(\BsJpsiPhi\right)}{\text{PhSp}(\BsJpsiKst)}
\frac{\BR{\BsJpsiKst}_{\text{theo}}}{\BR{\BsJpsiPhi}_{\text{theo}}}
\frac{f_i}{f'_i}\:,  \\
  & = & \frac{1-2a_i \cos(\theta_i) \cos\gamma+a_i^{2}}{1+2\epsilon a'_i \cos\theta'_i\cos\gamma +\epsilon^2 a_i^{\prime 2}} \:, \nonumber
\end{eqnarray}
\item $A^{CP}_i$, the direct CP violation asymmetries%
\footnote{Conventions: $A^{CP}_i = -\mathcal{A}_{\text{dir}}^{\CP}$ used in Ref.~\cite{DeBruyn:2014oga}}.
\begin{equation}\label{eq:ACPpeng}
A^{\CP}_i= -\frac{2a_i\sin\theta_i\sin\gamma}{1-2a_i\cos\theta_i\cos\gamma+a_i^{2}}\:.
\end{equation}
\end{itemize}
\begin{equation}
\text{PhSp}(B\to V_1V_2) \equiv \frac{1}{16\pi m_B}\Phi\left(\frac{m_{V_1}}{m_B}, \frac{m_{V_2}}{m_B}\right)\:,
\end{equation}
\begin{equation}\label{Eq:penguin_relation}
a_i = a'_i\:,\qquad \theta_i = \theta'_i\:,
\end{equation}
\begin{equation}\label{tandelta}
\tan(\Delta\,\phi_{s,i}) = \frac{2\epsilon a'_i \cos\theta'_i \sin\gamma+\epsilon^2 a^{\prime 2}_i \sin2\gamma}{1+2\epsilon a'_i \cos\theta'_i \cos\gamma+ \epsilon^2 a^{\prime 2}_i \cos2\gamma}.
\end{equation}



\subsubsection{The \BsJpsiKst and Bsjpsirho control chanells}

\begin{itemize}
  \item H observable is necesary.
  \item amplitude ratio is required introducing more uncertainty.
  \item SU3 symmetry is used. but it increases systematic uncertainties.
\end{itemize}

%
% {\small
% \begin{eqnarray}
% A^{CP} (B^0_{(s)} \to f_{(s)})  = &  \nonumber \\
% \frac{ \displaystyle \int_0^\infty \Big[ \Gamma(B^0_{(s)} \to \bar{f_{(s)}}) + \Gamma(\bar{B^0_{(s)}} \to \bar{f_{(s)}}) \Big]
% {\rm d}t \;-\; \int_0^\infty \Big[ \Gamma(B^0_{(s)} \to f_{(s)}) + \Gamma(\bar{B^0_{(s)}} \to f_{(s)}) \Big]
% {\rm d}t}{ \displaystyle \int_0^\infty \Big[ \Gamma(B^0_{(s)} \to \bar{f_{(s)}}) + \Gamma(\bar{B^0_{(s)}} \to \bar{f_{(s)}}) \Big] {\rm d}t
%  \;+\; \int_0^\infty \Big[ \Gamma(B^0_{(s)} \to f_{(s)}) + \Gamma(\bar{B^0_{(s)}} \to f_{(s)}) \Big] {\rm d}t}  &
% \label{decayrates}
% \end{eqnarray}
%  }
%
% \begin{equation}
% %A_i^{CP} = \frac{ |\cAbarfbar(0)|^2 - |\cAf(0)|^2 }{ |\cAbarfbar(0)|^2 + |\cAf(0)|^2 }
% A^{CP} = \frac{\Gamma(\bar{B^0_{(s)}} \to \bar{f_{(s)}})-\Gamma(B^0_{(s)} \to f_{(s)})}{\Gamma(\bar{B^0_{(s)}} \to \bar{f_{(s)}})+ \Gamma(B^0_{(s)} \to f_{(s)})}
% \end{equation}
%
% {\color{red} Assuming negligible cpv in mixing, Mention this.}
%
% \begin{equation}
% A_{CP}^{\rm raw}(B^0_{(s)} \to f_{(s)}) = \frac{N^{\rm obs}(\bar{f_{(s)}}) -N^{\rm obs}(f_{(s)}) }{
% N^{\rm obs}(\bar{f_{(s)}}) + N^{\rm obs}(f_{(s)})}
% \label{acp_mes}
% \end{equation}
