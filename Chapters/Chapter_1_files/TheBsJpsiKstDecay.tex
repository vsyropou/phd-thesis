As it was mentioned in \secref{WeakPhase} \BsJpsiPhi decays are dominated by the tree diagram of \figref{bs2jpsiphi}.
However the same decay could also take place via a higher order supressed process, for example the so called {\it penguin} diagram (or topology)
shown in \figref{bs2jpsiphi_peng}. The \phis measurement of \equref{phis_lhcb} ignores the contribution of the last penguing diagram.

\begin{figure}[h]
  \centering
  {\sffamily %%BoundingBox: -5 0 121 170
%%HiResBoundingBox: -5 0 120.57008 169.36447

\begin{fmffile}{Figures/Chapter1/penguin}
  \fmfframe(17,-25)(31,-25){
    \begin{fmfgraph*}(115,170)
      \fmfstraight
      \fmfleft{i0,i1,i2,i3,i4,i5}
      \fmfright{o0,o1,o2,o3,o4,o5}
      \fmf{fermion,tension=1.8,label.side=left,label=b}{v5,i3}
      \fmf{fermion,tension=1.5,right=0.2,label.side=left,label={\hspace*{18pt}u,,c,,t}}{v2,v5}
      \fmf{gluon,tension=2}{v4,v2}
      \fmf{dbl_dashes,tension=0}{v4,v2}
      \fmf{fermion,tension=0.3,right=0.2,label.side=left }{v3,v2}
      \fmf{boson,tension=0.6,left=0.3,label=W$^+$,label.side=left}{v3,v5}
      \fmf{fermion,label=c,tension=0.9,right=0.3,label.side=left}{o4,v4}
      \fmf{fermion,label=c,tension=0.9,right=0.3,label.side=left}{v4,o3}
      \fmf{fermion,label=s,label.side=left}{o2,v3}
      \fmffreeze
      \fmf{fermion,tension=0.7,label=s,label.side=left}{v1,o1}
      \fmf{fermion,tension=1,label.side=left,label=s}{i2,v1}
      %\fmf{phantom,tension=0.4}{v4,v1}
      \fmf{phantom,tension=0.4}{v3,v1,v5}
      \fmf{plain,right=0.2}{i2,i3}
      \fmf{plain,left=0.2,label=$\Bs$}{i2,i3}
      \fmf{plain,right=0.2,label=$\phi$}{o1,o2}
      \fmf{plain,left=0.2}{o1,o2}
      \fmf{plain,right=0.2,label=$\jpsi$}{o3,o4}
      \fmf{plain,left=0.2}{o3,o4}
    \end{fmfgraph*}
  }
\end{fmffile}
}
  \caption{{\color{red} Fix this to look more like \figref{QuarkMixing} and put ckm elements on the vertices}. Explain tha this is called tree diagram}
  \label{bs2jpsiphi_peng}
\end{figure}

\noindent This has been a reasonable assumtion before the \lhcb measurement which showed that the measured \phiS{eff} value is consistent with the Standard
Model prediction \phiS{SM,tree}, within the current experimental accuracy. However with increasing accuracy {\color{red} ref lhcb upgrade \phis sensitivy}the last assumption needs to be revisited.
Specifically, because of the fact that penguin topology contributions shift the leading tree level Standard Model prediction.
The last shift could be misinterpreted as $\Delta\phiS{NP}$ if those penguin contributions are not properly esimated.
Thus in order to correctly probe New Physics contributions in fututre measurements the penguin topology contributions $\Delta\phiS{SM,peng}$
have to be estimated.

\begin{equation}
\phis^{\text {eff}} = \phis^{\tiny \text{SM,tree}} + \Delta\phiS{SM,peng} + \Delta\phis^{\tiny \text{NP}}
 \label{phis_sm_peng}
\end{equation}

\noindent The above situation is spelled out in \equref{phis_sm_peng}.
Calculations for the penguin contributions to $\Delta\phiS{SM,peng}$ are available {\color{red} 40 and 41 in kyrstofs thesis}.
However, these cacluations are difficult to perform, since they involve non-perturbative long-distance QCD effects. Thus an alternative
approach, according to {\color{red} krystof and rob}, is followed in order to experimentally extract $\Delta\phiS{SM,peng}$ by relying to
similar decay cahanells as \BsJpsiPhi as well. By doing so the presition on the penguin contributions increases. A desctription
of the necessesary formalism to extract the penguin shift, $\Delta\phiS{SM,peng}$, using also the chanells \BsJpsiKst and \BsJpsiRho is
given in \chapref{Penguins}.
