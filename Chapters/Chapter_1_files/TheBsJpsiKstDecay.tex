As it was mentioned in \secref{WeakPhase} \BsJpsiPhi decays are dominated by the tree diagram of \figref{bs2jpsiphi}.
However the same decay could also take place via a higher order supressed diagram called {\it penguin} diagram (or topology)
shown in \figref{bs2jpsiphi_peng}. The \phis measurement of \equref{phis_lhcb} ignores the contribution of the last penguing diagram.

\begin{figure}[h]
  \centering
  {\sffamily %%BoundingBox: -5 0 121 170
%%HiResBoundingBox: -5 0 120.57008 169.36447

\begin{fmffile}{Figures/Chapter1/penguin}
  \fmfframe(17,-25)(31,-25){
    \begin{fmfgraph*}(115,170)
      \fmfstraight
      \fmfleft{i0,i1,i2,i3,i4,i5}
      \fmfright{o0,o1,o2,o3,o4,o5}
      \fmf{fermion,tension=1.8,label.side=left,label=b}{v5,i3}
      \fmf{fermion,tension=1.5,right=0.2,label.side=left,label={\hspace*{18pt}u,,c,,t}}{v2,v5}
      \fmf{gluon,tension=2}{v4,v2}
      \fmf{dbl_dashes,tension=0}{v4,v2}
      \fmf{fermion,tension=0.3,right=0.2,label.side=left }{v3,v2}
      \fmf{boson,tension=0.6,left=0.3,label=W$^+$,label.side=left}{v3,v5}
      \fmf{fermion,label=c,tension=0.9,right=0.3,label.side=left}{o4,v4}
      \fmf{fermion,label=c,tension=0.9,right=0.3,label.side=left}{v4,o3}
      \fmf{fermion,label=s,label.side=left}{o2,v3}
      \fmffreeze
      \fmf{fermion,tension=0.7,label=s,label.side=left}{v1,o1}
      \fmf{fermion,tension=1,label.side=left,label=s}{i2,v1}
      %\fmf{phantom,tension=0.4}{v4,v1}
      \fmf{phantom,tension=0.4}{v3,v1,v5}
      \fmf{plain,right=0.2}{i2,i3}
      \fmf{plain,left=0.2,label=$\Bs$}{i2,i3}
      \fmf{plain,right=0.2,label=$\phi$}{o1,o2}
      \fmf{plain,left=0.2}{o1,o2}
      \fmf{plain,right=0.2,label=$\jpsi$}{o3,o4}
      \fmf{plain,left=0.2}{o3,o4}
    \end{fmfgraph*}
  }
\end{fmffile}
}
  \caption{{\color{red} Fix this to look more like \figref{QuarkMixing} and put ckm elements on the vertices}. Explain tha this is called tree diagram}
  \label{bs2jpsiphi_peng}
\end{figure}

\noindent This has been a reasonable assumtion before the the \lhcb measurement which showed that the measured \phiS{eff} value is consistent with the Standard
Model prediction \phiS{SM,tree}, within the current experimental accuracy. However with increasing accuracy {\color{red} ref lhcb upgrade \phis sensitivy}the last assumption needs to be revisited.
Specifically, because of the fact that penguin topology contributions shift the leading tree level Standard Model prediction.
The last shift could be misinterpreted as $\Delta\phiS{NP}$ if those penguin contributions are not properly esimated.
Thus in order to correctly probe New Physics contributions in fututre measurements the penguin topology contributions $\Delta\phiS{SM,penguin}$
have to be estimated.

\begin{equation}
\phis^{\text {eff}} = \phis^{\tiny \text{SM,tree}} + \Delta\phiS{SM, penguin} + \Delta\phis^{\tiny \text{NP}}
 \label{phis_sm_peng}
\end{equation}

\noindent The above situation is spelled out in \equref{phis_sm_peng}. The current section gives a brief discription of
one [{\color{red} ref bs2jpsikst}] of the suggested strategies [{\color{red}{ref robs papers}} ]to estimate those penguin shifts.
The necessary observables to extract the penguin shifts $\Delta\phiS{SM,penguin}$ are also introduced.


\subsubsection{Controling higher order effects in \phis}
We rely on SU3 symmetry to estimate penguin shifts. Theoretical calulations include difficult calculations due to non perturbative qcd effects.


\subsubsection{The \BsJpsiKst decay as a control chanell}

\begin{itemize}
  \item Why do we need a control chanell
  \item Point out the 2x2 system structure
  \item Why Bs2jpsiKst
  \item Flavour specific final state (no oscilations, the sign tags the flavour at production)
\end{itemize}

\subsubsection{Observables for controling penguin shifts}


\begin{itemize}
  \item Define parameters of interest
\end{itemize}


{\small
\begin{eqnarray}
A^{CP} (B^0_{(s)} \to f_{(s)})  = &  \nonumber \\
\frac{ \displaystyle \int_0^\infty \Big[ \Gamma(B^0_{(s)} \to \bar{f_{(s)}}) + \Gamma(\bar{B^0_{(s)}} \to \bar{f_{(s)}}) \Big]
{\rm d}t \;-\; \int_0^\infty \Big[ \Gamma(B^0_{(s)} \to f_{(s)}) + \Gamma(\bar{B^0_{(s)}} \to f_{(s)}) \Big]
{\rm d}t}{ \displaystyle \int_0^\infty \Big[ \Gamma(B^0_{(s)} \to \bar{f_{(s)}}) + \Gamma(\bar{B^0_{(s)}} \to \bar{f_{(s)}}) \Big] {\rm d}t
 \;+\; \int_0^\infty \Big[ \Gamma(B^0_{(s)} \to f_{(s)}) + \Gamma(\bar{B^0_{(s)}} \to f_{(s)}) \Big] {\rm d}t}  &
\label{decayrates}
\end{eqnarray}
 }

\begin{equation}
%A_i^{CP} = \frac{ |\cAbarfbar(0)|^2 - |\cAf(0)|^2 }{ |\cAbarfbar(0)|^2 + |\cAf(0)|^2 }
A^{CP} = \frac{\Gamma(\bar{B^0_{(s)}} \to \bar{f_{(s)}})-\Gamma(B^0_{(s)} \to f_{(s)})}{\Gamma(\bar{B^0_{(s)}} \to \bar{f_{(s)}})+ \Gamma(B^0_{(s)} \to f_{(s)})}
\end{equation}

{\color{red} Assuming negligible cpv in mixing, Mention this.}

\begin{equation}
A_{CP}^{\rm raw}(B^0_{(s)} \to f_{(s)}) = \frac{N^{\rm obs}(\bar{f_{(s)}}) -N^{\rm obs}(f_{(s)}) }{
N^{\rm obs}(\bar{f_{(s)}}) + N^{\rm obs}(f_{(s)})}
\label{acp_mes}
\end{equation}
