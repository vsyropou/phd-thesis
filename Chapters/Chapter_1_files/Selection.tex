

\subsection{Multivariate Based Selection}
\label{Multivariate_Based_Selection}

The \BsJpsiKst signal yield out of the full 3 \invfb data is expected to be low {\color{red} Footnote on the exp. yield or reference to a 
previous calculation, (maybe in th eintroduction.)} Thus, one would like to get the most signal yield while rejecting as much background
as possible. One way to do that would be a cut-based analysis, in which case certain cuts are applied to a set of variables like the \Bs 
mass range for example. Alternatevely a multivariate (hearafter MVA) approach is addopted. In that case a set of variables are 
combined by the MVA algorithm to produce one output variable. This variable ranges from -1 to 1 and clasifies each event to be more 
signal-like (closer to 1) or background-like (closer to -1).

For the current analysics the TMVA toolkit~\cite{TMVA} was used for the MVA procedure. In order to compute the MVA clasifier some input
needed. Particularly two sets of signal-like sample and background-like samples are needed, one for the MVA clasifier to be trained with
and the another one to be tested with. For the signal-like sample, \BsJpsiKst Monte-Carlo simulated data (hearafter MC) are used. The \Bs 
mass widnow for that sample is $\pm 25 \MeVcc$ around the \Bs peak. As for the background-like sample, candidates from the high mass sideband
with invariant masses between $5401.3\mevcc$ and $5700\mevcc$ are used. Note that the simulated samples are treated exactly the same way as the
the real data sample when it comes to any selection cuts applied. Half of both signal and background like samples were used for training and the
other half for testing. This is done because the testing samples need to be statistically indepedant from the ones that were used to train the 
MVA clasifier with. A boosted decision tree with gradient boosting (BDTG){\color{red}{what is gradiaent boosting})} method was trained and tested
over those samples using the following kinematic variables as discriminating variables for the multivariate procedure (\Bs meson variables are 
named here as \texttt{B0}):

\begin{itemize}
\item{} \texttt{max\_DOCA}: maximum of all distances between pairs of tracks from daughter particles.
\item{} \texttt{B0\_LOKI\_DTF\_CTAU}: time of flight $ct$ of the \Bs meson candidate, where
$t$ is the decay time of the \Bs meson candidate measured in its proper reference frame.
\item{} \texttt{lessIPS}: minimum of all significances on the impact parameter of a daughter particle (kaons, muons and pions) with respect to the \Bs meson candidate.
\item{} \texttt{B0\_PT}: transverse momentum of the \Bs meson candidate.
\item{} \texttt{B0\_IP\_OWNPV}: impact parameter of the \Bs meson candidate with respect to its best own parent vertex.
\item{} \texttt{B0\_ENDVERTEX\_CHI2}: reconstruction significance of a reconstructed decay vertex of the \Bs meson candidate.
\end{itemize}

Signal and background distributions for discriminating variables among with the BDTG method response are shown in \appref{AppendixE}.
No overtraining is observed in the BDTG classifier, showing a good discrimination power for the chosen MVA method between both signal
and background distributions. A cut on the BDTG is applied so that it maximises the following figure of merit (FoM)~\cite{Yuehong_fom}:

\begin{equation}
\label{eqn:fom}
F(\sWeights) = \frac{\left(\sum{w_{i}}\right)^2}{\sum{w_{i}^2}},
\end{equation}

\noindent where $w_i$ are \sWeights associated to each event, and calculated with the \sPlot technique~\cite{splot}, considering \Bs candidate events as signal yield. 
This FoM can be understood as an {\it{effective signal value}}, which is inversely proportional to the square root of the total number of events. 

For a range of cut values applied on the BDTG, a \sPlot can be performed and a value for the FoM can be obtained. A plot containing
the value of $F(\sWeights)$ versus the cut value applied on the corresponding BDTG method is shown in \figref{fig:FOM_optimisation_2011}
for 2011 conditions and in \figref{fig:FOM_optimisation_2012} for 2012 conditions. Optimal BDTG cut values are chosen as those which maximise $F(\sWeights)$. 
These values are 0.2 for 2011 conditions and 0.12 for 2012 conditions.

Signal efficiencies and background rejections values are calculated for both 2012 and 2011 data conditions separately and individually for three 
different subsets of cuts in the final selection presented in \tabref{tab:Bs2JpsiKstSelection}, resulting in three individual efficiencies (rejections). 
As a first step, $\varepsilon_{\rm sel}$ is calculated selecting signal and background samples with all the final selection cuts excepting (**) and (*). 
As a second step, $\varepsilon_{\rm BDTG}$ is calculated selecting previous samples also with cut (**). As a third step, $\varepsilon_{\Lb}$ is obtained 
selecting previous samples also with cut (*). Finally, a total efficiency (rejection) is obtained, containing the information from all the efficiencies 
(rejections) computed in previous steps: $\varepsilon_{\rm tot}$ uses signal and background samples selected with all the cuts from the final selection.


\subsection{Peaking Backgrounds}

\subsection{Combinatorial Background}

\subsubsection{The \BJpsiKpi Invariant Mass Distribution}
\label{The_Invariant_Mass_Distribution}

\subsubsection{The \sPlot technique}

