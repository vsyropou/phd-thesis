% Chapter 4

\chapter{Data Analysis}
\label{Data_Analysis}

The parameters of interest as described in \chapref{Introduction} are extracted from the analysis of the
\runone \lhcb data accumulated over the years 2011 and 2012, which correspond to a total integrated luminosity
of 3\invfb. The fit to these data is a weighted Maximum Mikelihood fit to the angular distributions of \BsJpsiKst,
where the weights have been computed with the \sPlot technique \cite{splot}. The extracted parameters of interest are polarization
fractions and amplitude phases of the previously mentioned decay chanell. Prior to the angular analysis the data as 
they come straight from the detector need to be filterd to reduce the presence of background events. Thus the current
chapter addresses first the event selection and then the angular analysis of the selected data.

\secref{Event_Selection} describes all the steps involded to reduce background in the data. For this purpose
it is necessary to model the invariant mass distribution of the \BJpsiKpi decay chanells, which is addressed
in \secref{The_Invariant_Mass_Distribution}. In \secref{Angular_Analysis} the details of estimating the parameters
of intrest are presented in steps.

% Expected signal yield
% The center of mass energy was 3 and 4 \tev respectivelly for 2011 and 2012.

% Section Event Selection
\section{Event Selection}
\label{Event_Selection}

There are effectivelly four stages of offline event selection taking place after the data has been written out by the detector. 
First a set of rough selection criteria is applied. Those cuts are quantities like track and vertex quality {\color{red}(jargon:track and vertex?)},
track impact parameter, mass ranges of mother and daugther particles, particle identification and transverse momentum. A detalied 
table of those cuts is given in \appref{Stripping}. The purpose of those cuts is to perform a first rough rejection of the combinatorial background
and prepare the data for the next stage of selection. The rest of the selection steps are described in each of the subsequent subsections.



\subsection{Multivariate Based Selection}
\label{Multivariate_Based_Selection}

The \BsJpsiKst signal yield out of the full 3 \invfb data is expected to be low {\color{red} Footnote on the exp. yield or reference to a 
previous calculation, (maybe in the introduction.)} Thus, one would like to get as much signal yield as possible while rejecting as much background
as possible. One way to do that would be a cut-based analysis, in which case certain cuts are applied to a set of variables like the \Bs 
mass range for example (essentially a tighter Stripping). Alternatevely a multivariate approach (hearafter {\it MVA}) is addopted. 
In that case a set of variables are combined by the MVA algorithm to produce one output variable, the {\it classifing variable}. This variable ranges from -1 to 1 and 
clasifies each event to be more signal-like (closer to 1) or background-like (closer to -1).

For the current analysics the TMVA toolkit~\cite{TMVA} was used for the MVA selection. In order to construct the classifing variable certain 
input datasets are needed, particularly two sets of signal-like plus two background-like datasets. First a pair of a singal and background like
datasets is fed to the MVA algorithm to construct the classifing variable (this is step is called {\it trainning}). The second pair is used to
asses how well the trainning step was and it is called {\it testing}. For the signal-like samples, \BsJpsiKst Monte-Carlo simulated data (hearafter MC)
are used. The \Bs mass widnow for that sample is $\pm 25 \MeVcc$ around the \Bs peak. As for the background-like sample, candidates from the high mass sideband
with invariant masses between $5401.3\mevcc$ and $5700\mevcc$ are used. Note that the simulated samples are treated exactly the same way as the
real data sample when it comes to any selection cuts applied. A boosted decision tree with gradient boosting (BDTG){\color{red}{what is gradiaent boosting})}
was used as the classifing variable. The following kinematic variables were provided as input variables for the multivariate procedure (\Bs meson variables are 
named here as \texttt{B0}):

\begin{itemize}
\item{} \texttt{max\_DOCA}: maximum of all distances between pairs of tracks from daughter particles.
\item{} \texttt{B0\_LOKI\_DTF\_CTAU}: time of flight $ct$ of the \Bs meson candidate, where
$t$ is the decay time of the \Bs meson candidate measured in its proper reference frame.
\item{} \texttt{lessIPS}: minimum of all significances on the impact parameter {\color{red} what is impact parameter} of a daughter particle (kaons, muons and pions) with respect to the \Bs meson candidate.
\item{} \texttt{B0\_PT}: transverse momentum of the \Bs meson candidate.
\item{} \texttt{B0\_IP\_OWNPV}: impact parameter of the \Bs meson candidate with respect to its best own parent vertex.
\item{} \texttt{B0\_ENDVERTEX\_CHI2}: reconstruction significance of a reconstructed decay vertex of the \Bs meson candidate.
\end{itemize}

The BDTG shows a good discrimination power over signal and background distributions \figref{BTDG_performance}. It was also checked for potential overtraining {\color{red} what is overtraining}.
Once the trainning and testing steps are complete a cut on the BDTG is applied so that it maximises the following figure of merit
(FoM)~\cite{Yuehong_fom}:

\begin{equation}
\label{eqn:fom}
F(\sWeights) = \frac{\left(\sum{w_{i}}\right)^2}{\sum{w_{i}^2}},
\end{equation}

\noindent where $w_i$ are weights associated to each event (hearafter \sWeights), and calculated with the \sPlot technique. 
This FoM can be understood as an {\it{effective signal yield}}, which is inversely proportional to the square root of the total number of events. 


For a range of BDTG values a corresponding \sPlot can be performed and a value for the FoM can be obtained.
The optimum BDTG value is chosen as the one that maximises $F(\sWeights)$. 
This value is 0.2 for 2011 conditions and 0.12 for 2012 conditions.

\begin{figure}[h]
\begin{center}
% \includegraphics[width=1.1\textwidth]{Figures/AppendixE/input_variables_2011.pdf}
\caption{BDTG distribution for signal and background.}
\label{BTDG_performance}
\end{center}
\end{figure}

\begin{figure}[h]
\begin{center}
% \includegraphics[width=1.1\textwidth]{Figures/AppendixE/input_variables_2011.pdf}
\caption{Mass distribution before and after BDTG selection.}
\label{mass_BDTG_selection}
\end{center}
\end{figure}

% ----------------------------------------------------------------------------------------------------------------------------------------------------------------------------------------
\subsection{Peaking Backgrounds}
\label{peaking_backgrounds}

After appling the BTDG cut there is still some combinatorial background remaining which is removed
further using the \sPlot technique described in \secref{Combinatorial_Background}. However, there is
one more crusial step that is necessary to address beofre namelly that of the peaking backgrounds.
Studies of fully simulated samples show contributions from several specific backgrounds, such as \BsJpsiKK, \BsJpsipipi and \BdJpsipipi.
Those backgrounds are the result of assigning the wrong mass hypothesis to a given track during reconstruction.  
The invariant mass distribution of misidentified \BdJpsipipi and \BsJpsipipi peaks near the \BsJpsiKpi signal peak
and the misidentified \BsJpsiKK events are located almost under the \BdJpsiKpi signal peak. 
This behavior makes the invariant mass of the \Jpsi$K\pi$ system not a good variable to reject further the combinatorial background,
which is what the \sPlot technique requires. In order to remove these peaking backgrounds simulated events with negative weights are
injected to the data thus cancelling out the likelihood contribution from peaking background events. The current section briefly addresses
the treatment of peaking backgrounds with negative weights as well as the special treatment of the \LbJpsippi peaking background.

The expected yields of the peaking backgrounds on data are estimated with simulated data based on the expression:
\begin{equation}
N_{\rm exp} = 2 \times \sigma_{b\bar{b}}\times f_q \times BR \times \varepsilon \times {\mathcal{L}}
\end{equation}
\noindent Where $\sigma_{b\bar{b}}$ is the corss section for the production of a pair of bottom quarks, $f_q$ is the hadronization fraction
(probability that the $b$ quarck forms a hadron with another quarck type), $\varepsilon$ the total efficiency (reconstruction, selection and trigger)
and ${\mathcal{L}}$ the luminosity of the data. BR stands for the branching fraction of the particular peaking background. Since simulated data are used
it is necessary to estimate the effective luminosity of that simulated sample and scale it to the luminosoty of the data. After that the number of
peaking background events from the simulated sample is a valid estimate of the one in real data.

The last step of the peaking background removal is to apply an angular correction factor to account for the fact that 
simulated events are generated in PHSP and hence do not contain the proper decay amplitudes. This can cause the peaking 
backgound yield estimations to be inaccurate because the simulated events can be distributed in the $(\Omega, m_{K\pi})$ space
in a different way than the actual peaking background of the data. The amplitude analysis of \BdJpsipipi, \BsJpsipipi, \BsJpsiKK and \LbJpsipK 
has been performed in~\cite{SheldonBdpipi},~\cite{SheldonBspipi},~\cite{SheldonKK} and~\cite{Gao:1701984} respectively. Therefore the simulated events 
are weighted with

\begin{equation}
w_{\rm MC} = \frac{P_{\rm DATA} (\Omega, m_{hh}  | {A_i})}{P_{\rm MC}(\Omega, m_{hh})},
\end{equation}

\noindent $P_{\rm DATA}$ and $P_{\rm MC}$ are normalised PDFs {\color{red} check if you have defined pdf before} and $A_i$ stands for
the particular amplutde structure of a certain peaking background mode. The complete description of the above steps can be found in \cite{BsJpsiKst_ANA}.

\begin{table}
\begin{center}
%\scriptsize
\begin{tabular}{c|c}%|c|c|c|c}
Sample & $\pm70\mevcc$ window \\
\hline 
\BdJpsipipi 2011 & $51 \pm 10$ \\
\BdJpsipipi 2012 & $115\pm 23$ \\  
\BsJpsipipi 2011 & $9.3\pm 2.1$ \\
\BsJpsipipi 2012 & $25.0\pm 5.4$\\
\BsJpsiKK 2011 & $10.1 \pm 2.3$ \\
\BsJpsiKK 2012 & $19.2 \pm 4.0$ \\ 
\LbJpsipK 2011 & $36 \pm 17$ \\
\LbJpsipK 2012 & $90 \pm 43$ \\ 
%\LbJpsippi 2011 & $13.8 \pm 4.4$ \\
%\LbJpsippi 2012 & $27.3 \pm 6.9$ \\ 
\LbJpsippi 2011 & $13.8 \pm 5.3$ \\
\LbJpsippi 2012 & $27.3 \pm 9.0$ \\
\hline
\end{tabular}
\caption{Approximated expected yields in $\pm 70\mevcc$ $K\pi$ mass window of each background after reweighting of 
the phase space (the \LbJpsippi decay is not weighted since no amplitude analysis for that decay is published).}
\label{tab:peakingSummary}
\end{center}
\end{table}


A final note the special treatment of the \LbJpsippi peaking background. Instead of following the above procedure for this
particular background, the last is treated the same way as the remaining cobinatorial background in the next section.
The reasons for this different treatment with respect to other peaking backgrounds are two:
\begin{itemize}
\item The full amplitude structure of \LbJpsippi decays is not yet known, and thus the weighting of the simulated samples 
      would have to be done by looking at data sidebands. Which is not correct.
\item The peak of the misidentified \LbJpsippi decays in the \Jpsi$K\pi$ mass spectrum is broader than those of the other
      peaking backgrounds see \figref{fig:amoroso}, making the \sPlot technique to still be effective.  
\end{itemize}
More details on the removal of that background are presented after the \sPlot technique is introduced.

% ----------------------------------------------------------------------------------------------------------------------------------------------------------------------------------------
\subsection{Combinatorial Background}
\label{Combinatorial_Background}

\noindent The \Jpsi$K\pi$ mass line shape of the misidentified \LbJpsippi using the Amoroso distribution~\cite{Amoroso}, which provides a good description of the data as can be seen in \figref{fig:amoroso}. 
The parameters of the distribution are obtained from simulation for each bin of \mkpi and then fixed in the fit to data. Despite
of the fact that there is no full amplitude analysis for \LbJpsippi decays yet, the 2D Dalitz plane and projections are
measured in Ref.~\cite{LHCb-PAPER-2014-020}, where some $p\pi^-$ resonances are identified. A conservative systematic (absolute
difference between the original value and the re-estimated value) due to the effect of not taking into account these resonances
(see~\tabref{tab:LbJpsippiResonances}) is estimated following the same weighting procedure as in \secref{ssec:peak_rw}, and
added in quadrature to the total uncertainty of \LbJpsippi yields from~\tabref{tab:peaking_MC}. The relative magnitude of
this
systematic uncertainty for the \LbJpsippi yield is approximately of $21.3\%$ ($21.7\%$) for 2011 (2012) conditions.

\subsubsection{The \BJpsiKpi Invariant Mass Distribution}

\subsubsection{The \sPlot technique}



% Section Angular Analysis
\section{Angular Analysis}
\label{Angular_Analysis}

\subsection{Diferential Decay Rate}
\subsection{\Kpi Invariant mass}
\subsection{Acceptance}

Describe Efficiency moments formalism\\
Dump eff moments\\

\subsection{Acceptance Corrections}
Motivation\\
Data - Simulation Differences\\
Validation\\
\subsection{\CP Assymetries}
\subsection{Maximum Likelihood Fit}

