
\subsection{Diferential Decay Rate}
\label{Diferential_Decay_Rate}

\subsection{\Kpi Invariant mass}
\label{Kpi_Invariant_mass}

\subsubsection{sWeighted mKpi distribution}
While the \Bs or \Bd di-muon \sPlot s have very similar shape, the \mkpi weighted spectrums exhibit different shapes. 
Indeed, the \Bs \mkpi \sPlot seems to be slightly distorted. This could be due to the presence of interference between
 the $K\pi$ \swave and the \Kstarz, which would appear to be stronger in the \Bs decays compared to the \Bd. 

In order to check the validity of this hypothesis, we perform two additional studies. 
It was checked that the peaking background treatment propagated to the \sWeights was not responsible for this behavior.
Indeed, we found no significant difference between the \Bs \mkpi spectrum using \sWeights computed with and without MC 
data injection. We perform an additional study looking at the \mkpi structure after correcting for efficiency effects 
using the normalisation weights coming from the angular acceptance study. Indeed, this way the interference between the 
$K\pi$ \swave and the \Kstarz \pwave vanishes, since we integrate over the helicity angles. Figure~\ref{fig:Kpi_invMass_BsBd_w_and_wo_acceptance}
 gives the efficiency corrected \Bs and \Bd \mkpi spectra using the nominal sets of \sWeights. We observe that the \Bs \mkpi distribution
 is closer to the one of the \Bd after applying the efficiency correction. This is a clear indication of the presence of stronger interference 
in the \Bs case compared to the \Bd one. 

We also checked the effect of allowing the mean and sigma of the \Bs and \Bd Hypatia functions to share common values over the 20 bins from
a simultaneous fit. No significant gain was observed. The corresponding results are presented in \appref{app:massFitSimultaneous}. 
\subsubsection{CSP factors}

ela more CSP

\subsection{Acceptance}
\label{Accceptance}
Describe Efficiency moments formalism\\
Dump eff moments\\

\subsection{Acceptance Corrections}
\label{Accceptance_Corrections}
Motivation\\
Data - Simulation Differences\\
Validation\\


\subsection{Maximum Likelihood Fit}
\label{Maximum_Likelihood_Fit}

\subsubsection{\CP Assymetries}
