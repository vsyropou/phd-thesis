
\subsection{Angular Dependence}
\label{Diferential_Decay_Rate}
The decay of intreast, \BJpsiKst, is a P2VV process\footnote{P2VV is an acronim that characterises the spin-parity of the particles involved in the dacay.
The $\Bs$ (and $\Bd$) is spin-0 parity minus (Pseudoscalar) particle, whereas the intermediate resonances $\Jpsi$ and $\phi$ are spin-1 parity minus (Vector) particles. Hence the
acronim P2VV} with 4 particles in the final state. There are at least two ways to describe the angular dependence of that decay, namelly the transversity framework \cite{transvFrameworkI,transvFrameworkII}
and the helicity formalism \cite{helicityFormI,helicityFormII}. In both cases the goal is to come up with an angular dependant description of the total decay amplitude.
In the transversity framework the decay amplitude is decomposed in the three possible polarization states of the intermediate particles \Jpsi and \Kst witgh resct to the \Bs rest frame. For the current analysis the helicity
formalism is addopted where the angular dependance is introduced by summing all possible spin conficurations of the intermediate vector paticles relative to their 
momentum direction and squaring the sum. Or in more comapct wording, from summing up all possible helicity configurations of the intermediate vecror particles. A detailed derivation of both 
approaches within the scope of a P2VV decay can be found in \cite{daanThesis} and \cite{jeroenThesis} respectivelly for the trasvercity and helicity formalism.
 
The current paragraph aims at very briefly explaining the steps needed to arive at an angular pdf for \BsJpsiKst decays, which is the sum of ten terms, corresponding to
all the possible \pwave amplitude polarization states longitudinal ($0$), perpendicular ($\perp$), parallel ($\parallel$) plus \pwave self interference terms as well as 
the \swave amplitude (S) and the \spwave interference terms, \equref{ang_terms}.

\begin{equation}
\mathcal{A(\text{\BJpsiKst})} \propto \sum_n a_n h_n M_n
\label{ang_terms}
\end{equation}

\noindent Where the terms $a_n$ denote expresions releated the the angular amplitudes. The exact espresions of $a_n$ are shown in the first column of \tabref{ang_distr}. The functions $h$ 
represent the angular dependance of each term, whereas $M_n$ stands for the \mkpi dependence of the anplitude and its explanation and special treatment is postponed for the next section. 
After squaring the amplitude and applying the \emph{helicity formalistm} the ten terms of \equref{ang_terms} are realised in table \tabref{ang_distr}.

\begin{table}[h]
  \centering 
  \caption{ Angular functions corresponding to each term in \equref{ang_terms} for the \BJpsiKst decay. Pure and interference \pwave terms are shown in the upper part, 
    whereas the \swave plus \spwave interference in the lower. The angular functions are expressed in the helicity basis. The angles $\costhetax{K},\costhetax{\mu},\phihel$
    are called \emph{helicity angles} and in \figref{helAngles} are put into perspective. The $P$ and $Y$ symbols denote ascosiated legendre polynomials and real valued sperical harmonics
    respectively. The middle column express the angular dependence in an orthogonal basis and it is equivalent to the last column. }
  \renewcommand{\arraystretch}{1.2}
  \label{ang_distr}
  \begin{tabular}{ccc}
    \hline
    $a_n$                             &
    %$hh'$                                  &
      $h_n(\Omega) \times 16\sqrt{\pi}$      &
      $h_n(\Omega) \times \tfrac{32\pi}{9}$  \\

    \hline
    $\AmpSq{0}$  &
    %00  &
      $4\, (P_0^0 + 2\, P_2^0)\, (Y_{0,\,0} - \tfrac{1}{\sqrt{5}}\, Y_{2,\,0})$  &
      $2\, \cos^2\thetaK\, \sin^2\thetamu$  \\

    $\AmpSq{\parallel}$  &
    %$\parallel\parallel$  &
      $P_2^2\, (2\, Y_{0,\,0} + \tfrac{1}{\sqrt{5}}\, Y_{2,\,0} - \sqrt{\tfrac{3}{5}}\, Y_{2,\,+2})$  &
      $\sin^2\thetaK\, (1 - \sin^2\thetamu\, \cos^2\phihel)$  \\

    $\AmpSq{\perp}$  &
    %$\perp\perp$  &
      $P_2^2\, (2\, Y_{0,\,0} + \tfrac{1}{\sqrt{5}}\, Y_{2,\,0} + \sqrt{\tfrac{3}{5}}\, Y_{2,\,+2})$  &
      $\sin^2\thetaK\, (1 - \sin^2\thetamu\, \sin^2\phihel)$  \\

    $\ReAmp{0}{\parallel}$  &
    %0$\parallel$ ($\Re$)  &
      $+2\sqrt{2}\sqrt{\tfrac{3}{5}}\, P_2^1\, Y_{2,\,+1}$  &
      $+\frac{1}{\sqrt{2}}\, \sin2\thetaK\, \sin2\thetamu\, \cos\phihel$  \\

    $\ImAmp{0}{\perp}$  &
    %0$\perp$ ($\Im$)  &
      $-2\sqrt{2}\sqrt{\tfrac{3}{5}}\, P_2^1\, Y_{2,\,-1}$  &
      $-\frac{1}{\sqrt{2}}\, \sin2\thetaK\, \sin2\thetamu\, \sin\phihel$  \\


    $\ImAmp{\parallel}{\perp}$  &
    %$\parallel\perp$ ($\Im$)  &
      $+2\sqrt{\tfrac{3}{5}}\, P_2^2\, Y_{2,\,-2}$  &
      $+\sin^2\thetaK\, \sin^2\thetamu\, \sin2\phihel$  \\

    \hline
    $\AmpSq{{\text{S}}}$  &
      $4\, P_0^0\, (Y_{0,\,0} - \tfrac{1}{\sqrt{5}}\, Y_{2,\,0})$  &
      $\tfrac{2}{3}\, \sin^2\thetamu$  \\

    $\ReAmp{0}{{\text{S}}}$  &
      $8\sqrt{3}\, P_1^0\, (Y_{0,\,0} - \tfrac{1}{\sqrt{5}}\, Y_{2,\,0})$  &
      $\tfrac{4}{3}\sqrt{3}\, \cos\thetaK\, \sin^2\thetamu$  \\

    $\ReAmp{\parallel}{{\text{S}}}$  &
      $+6\sqrt{2}\tfrac{1}{\sqrt{5}}\, P_1^1\, Y_{2,\,+1}$  &
      $+\tfrac{1}{3}\sqrt{6}\, \sin\thetaK\, \sin2\thetamu\, \cos\phihel$  \\

    $\ImAmp{\perp}{{\text{S}}}$  &
      $+6\sqrt{2}\tfrac{1}{\sqrt{5}}\, P_1^1\, Y_{2,\,-1}$  &
      $+\tfrac{1}{3}\sqrt{6}\, \sin\thetaK\, \sin2\thetamu\, \sin\phihel$  \\
    \hline
  \end{tabular}
\end{table}  

Three angles are required in a 4 body decay corresponding to the angular degrees of freedom necessary to describe the angular dependence of the system.
The angles $\costhetax{K},\costhetax{\mu},\phihel$ are called \emph{heliciy angles} and are defined in \figref{helAngles}. In that figure it can be seen
that $\costhetax{\mu}$ is the angle of the positive $\mu$ with respect to the momentum of the \Jpsi in the \Bs rest frame. Similarly for $\costhetax{K}$,
the \kaon is used to define the angle with respect to the momentum of the intermediate \Kpi resonance. Finally $\phihel$ is 
the relative angle between the \Kpi and dimuon decay plances, where each plane is again defined in the rest frame of the respective intermediate resonance.  

Lastly, the mathematically elegant orthonormal angular basis of $P$ and $Y$
\footnote{The decompotition of angular functions in an orthogonal basis follwos naturally from the fact that sperical harmonics can be expressed as 
Wigner-$D$ matrices. The last are an essential part of the helicity formalism. More details on this mapping can be found in \cite{jeroenThesis} } 
is adopted in the current analysis, since the integrals of type $\int_{-1}^{1}Pd\varphi$ and $\int_{-1}^{1}Yd\costhetax{K}d\costhetax{\mu}$ are known 
analitically thus significantly reducing the fiting time and simplifing the implementation of the angular fucntions.


\begin{figure}[h]
\begin{center}
  \includegraphics[width=\textwidth]{Figures/Chapter4/helAngles.pdf}
  \caption{Definition of the decay angles in the helicity formalism.}
  \label{helAngles}  
\end{center}
\end{figure}



\subsection{Acceptance}
\label{Accceptance}
Describe Efficiency moments formalism\\
Dump eff moments\\

\subsection{Acceptance Corrections}
\label{Accceptance_Corrections}
Motivation\\
Data - Simulation Differences\\
Validation\\

\subsection{\Kpi Invariant mass}
\label{Kpi_Invariant_mass}

\subsubsection{CSP factors}
Explain CSP factors\\

\subsubsection{sWeighted mKpi distribution}
While the \Bs or \Bd di-muon \sPlot s have very similar shape, the \mkpi weighted spectrums exhibit different shapes. 
Indeed, the \Bs \mkpi \sPlot seems to be slightly distorted. This could be due to the presence of interference between
 the $K\pi$ \swave and the \Kstarz, which would appear to be stronger in the \Bs decays compared to the \Bd. 

In order to check the validity of this hypothesis, we perform two additional studies. 
It was checked that the peaking background treatment propagated to the \sWeights was not responsible for this behavior.
Indeed, we found no significant difference between the \Bs \mkpi spectrum using \sWeights computed with and without MC 
data injection. We perform an additional study looking at the \mkpi structure after correcting for efficiency effects 
using the normalisation weights coming from the angular acceptance study. Indeed, this way the interference between the 
$K\pi$ \swave and the \Kstarz \pwave vanishes, since we integrate over the helicity angles. Figure~\ref{Kpi_invMass_BsBd_w_and_wo_acceptance}
 gives the efficiency corrected \Bs and \Bd \mkpi spectra using the nominal sets of \sWeights. We observe that the \Bs \mkpi distribution
 is closer to the one of the \Bd after applying the efficiency correction. This is a clear indication of the presence of stronger interference 
in the \Bs case compared to the \Bd one. 

We also checked the effect of allowing the mean and sigma of the \Bs and \Bd Hypatia functions to share common values over the 20 bins from
a simultaneous fit. No significant gain was observed. The corresponding results are presented in \appref{app:massFitSimultaneous}. 


\subsection{Total Decay Rate}
\label{Total_Decay_Rate}

\subsubsection{\CP Assymetries}

\subsubsection{Maximum Likelihood Fit}


