
The current section addresses the issue of the branching ratio normalization.
One of the necessary elements towards estimating the penguin contributions to \phis is the \BsJpsiKst branching ratio measurement,
hearafter $\BRof\BsJpsiKst$, which is used as input to \secref{penguins_formalism}. Absolute branching ratio measurements are 
more complecated to perform compared to a relative one. Thus $\BRof\BsJpsiKst$ is normalised with respect to $\BRof\BsToJPsiPhi$,
which is suitable given the relevant formalism, see \secref{penguins_formalism}, for the penguin contribution determination. 

\begin{align}
\frac{\BRof\BsJpsiKst}{\BRof\BsJpsiPhi} = \frac{N_{\BsJpsiKpi}}{N_{\BsJpsiKK}} &\times \frac{\varepsilon_{\BsJpsiPhi}^{MC}}{\varepsilon_{\BsJpsiKst}^{MC}} \times \nonumber \\
                                                                          &\times \frac{\omega_{\BsJpsiPhi}}{\omega_{\BsJpsiKst}} \times \frac{\BRof{\phi\to\kaon^+\kaon^-}}{\BRof{\Kstar\to\kaon\pion}} 
\label{norm_jpsiPhi}
\end{align}

The $\BRof\BsJpsiKst$ measurement as shown in \equref{norm_jpsiPhi} is detached from the angular analisis. It requires the observed number 
\BsJpsiKst and \BsToJPsiPhi events. The first one is taken from \equref{signal_yields} as for the later a bit more efford is needed.
Particularly the selection steps applied to $\BsToJPsiPhi$ (also refered to as reference chanell) are chosen such that they are as 
similar as possible to that of \BsJpsiKst in \tabref{Bs2JpsiKstSelection}. Also the BDTG clasifier is applied to the refference chanell
with the same cut values as in the \BsJpsiKst chanell. The ratio of observed number of events is represented by the first fraction in \equref{norm_jpsiPhi},
Whereas the ratio of total efficiencies between the two chanells by the second. The last is corrected for differenceses in efficiencies between real and 
simulated data. Addionally there is a correction factor to account for the presence of \swave in the two 
chanells\footnote{The penguin estiamtion assumes no \swave contributiuon to \BsJpsiKst. The factors $\omega$ in \equref{norm_jpsiPhi} correspond
to the ratio of two angular \pdfs, one with the \swave parameters set to zero and \pwave to those obtaind from the angular fit,  whereas the other
with both \pwave and \swave parameters as obtained from the same fit.
}. The branching fractions of the intermediate resonance to the final state di-hadron system are obtained from \cite{PDG}
\footnote{ $\BRof{\Kstar\to\kaon\pion} = 2/3$ whereas $\BRof{\phi\to\kaon^+\kaon^-}\simeq 1/2$}.
The $\BRof\BsToJPsiPhi$ \cite{SheldonKK} is updated with the latest hadronization \bquark quark fraction ratio  
ratio\footnote{The quantity \fdfs expresses the ratio of probabilities that a \bquark quark will form a messon with a \dquark quark or an \squark quark}
\cite{LHCb-CONF-2013-011}. Whereas the efficiency and the angular correction ratios are about $2$ and $0.9$ respectivelly. 

\begin{equation}
\label{eq:norm_jpsiPhi_result}
\frac{\BRof\BsJpsiKst}{\BRof\BsJpsiPhi} = (4.09 \pm 0.20 ({\rm stat}) \pm 0.12 ({\rm syst}))\% \\
\end{equation}


\begin{equation}
\label{Br_JpsiPhi}
\BR{\BsJpsiKst}_{\phi} = (4.25 \pm 0.20  \text{(stat)} \pm  0.13  \text{(syst)}  \pm 0.36 (\BRof\BsJpsiPhi))\times 10^{-5}
\end{equation}


%%%%% We use the fomrula blah.
%%%%% Explain formula.
%%%%% Epsilon means offline trigger and generator lvl 
%%%%% Epsilon rations is a weighted average over polarity and periods
%%%%% Show epsilon ratios
%%%%% Angular corrections
%%%%% data mc corrections
%%%%% Final choice
%%%%% dump numbers
