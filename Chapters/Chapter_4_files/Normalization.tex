
The current section addresses the determination of the \BsJpsiKst branching fraction, hereafter $\BRof\BsJpsiKst$.
This is one of the necessary ingredients for estimating the penguin contributions to \phis, addressed in \chapref{Penguins}.
Absolute branching fraction measurements are more complicated to perform compared to relative ones. Because a
dedicated analysis must be built up and optimize for the channel of interest. Also in a relative branching fraction
measurement some systematic uncertainties drop out, \ie the systematic related to the luminosity estimate. Thus, the $\BRof\BsJpsiKst$ is
normalized with respect to two different channels, namely $\BsJpsiPhi$ and $\BdJpsiKst$ and a single average is computed.

\subsubsection{Normalization with respect to $\BsJpsiPhi$}
Equation \ref{norm_jpsiPhi} shows how $\BRof\BsJpsiKst$ is normalized with respect to the \BsJpsiPhi channel.
The branching fraction defined in the previous equation will hereafter be denoted as $\BRof{\BsJpsiKst}_\Pphi$.
It requires as input the observed number of \BsJpsiKst and \BsJpsiPhi decays. The first one is taken from
\equref{signal_yields} whereas the latter requires more effort. Particularly, the selection steps applied
to $\BsJpsiPhi$ are chosen such that they are as similar as possible to that of \BsJpsiKst, see \tabref{Bs2JpsiKstSelection}.

\begin{align}
  \centering
\frac{\BRof\BsJpsiKst}{\BRof\BsJpsiPhi} = \frac{N_{\BsJpsiKpi}}{N_{\BsJpsiKK}}
                                  &\times \frac{\varepsilon_{\BsJpsiPhi}^{MC}}{\varepsilon_{\BsJpsiKst}^{MC}}
                                   \times \nonumber \\
                                  \times \frac{\omega_{\BsJpsiPhi}}{\omega_{\BsJpsiKst}}
                                  & \times \frac{\BRof{\Pphi\to\kaon^+\kaon^-}}{\BRof{\Kstarzb\to\kaon^-\pion^+}}.
\label{norm_jpsiPhi}
\end{align}

\noindent Also the BDT classifier is applied to the reference channel
with the same cut values as in the \BsJpsiKst channel. The ratio of observed number of candidates is represented by the
first fraction in \equref{norm_jpsiPhi}, whereas the ratio of total efficiency between the two channels is denoted
by the second. The latter fraction is corrected for known differences in efficiency between data and simulated data.
Additionally, there is a correction factor $\omega$ to account for the presence of \swave in the two
channels, since the penguin estimation assumes no \swave contribution to \BsJpsiKst. Each of the factors $\omega$ in
\equref{norm_jpsiPhi} correspond to the ratio of two angular \pdfs. One with the \swave parameters set to zero and
\pwave to these obtained in the angular fit,  whereas the other with both \pwave and \swave parameters as obtained
from the same fit. The $\omega$ factors can thus be understood as an overall \swave fraction.

\noindent The branching fractions of the intermediate resonance decaying to a dihadron system are obtained from \cite{PDG}.
The $\BRof\BsJpsiPhi$ in \cite{SheldonKK} is updated with the latest \bquark quark hadronization
fraction, \fdfs \cite{LHCb-CONF-2013-011}. This quantity expresses the ratio of probabilities that a \bquark quark will form a meson with a
\dquark quark over an \squark quark. The efficiency and the angular correction ratios are about $2$ and $0.9$ respectively.
The above information is used to determine the following branching fraction:

\begin{equation}
  \centering
\BRof\BsJpsiKst_{\Pphi} = (4.20 \pm 0.20 \pm 0.13 \pm 0.36) \times 10^{-5},
\label{Br_JpsiPhi}
\end{equation}

\noindent where the uncertainties are statistical, systematic and due to $\BRof\BsJpsiPhi$ respectively.

\subsubsection{Normalization with respect to $\BdJpsiKst$}
Alternatively $\BRof\BsJpsiKst$ can be normalized with respect to the \BdJpsiKst channel to obtain $\BRof{\BsJpsiKst}_\dquark$:

\begin{align}
  \centering
\frac{\BRof\BsJpsiKst}{\BRof\BdJpsiKst} = \frac{N_{\BsJpsiKpi}}{N_{\BdJpsiKst}} \times & \nonumber \\
                                   \times \fdfs
                                   \times \frac{\varepsilon_{\BdJpsiKst}^{MC}}{\varepsilon_{\BsJpsiKst}^{MC}}
                                   \times &\frac{\omega_{\BdJpsiKst}}{\omega_{\BsJpsiKst}},
\label{norm_jpsiKst}
\end{align}

\noindent The previous equation is similar to \equref{norm_jpsiPhi} as far as the way efficiency and angular correlation ratios are obtained.
The observed number of candidates are both obtained from \equref{signal_yields} and the latest measurement of the hadronization
fraction from \cite{LHCb-CONF-2013-011}. The $\BRof\BdJpsiKst$ value is taken from \cite{Abe:2002haa}, where the \swave component
is explicitly modeled and does contribute in the $\BRof\BdJpsiKst$ estimation. The final result is:

\begin{equation}
  \centering
\BRof\BsJpsiKst_\dquark = (3.95 \pm 0.18 \pm 0.16 \pm 0.23 \pm 0.43 )\times 10^{-5},
\label{Br_JpsiKst}
\end{equation}

 \noindent where the first two uncertainties are statistical and systematic and the rest are coming from \fdfs and $\BRof\BdJpsiKst$ respectively.

\subsubsection{Averaged $\BRof\BsJpsiKst$}
Both estimates of \equref{Br_JpsiPhi} and \equref{Br_JpsiKst} are compatible within uncorrelated systematic uncertainties.
A weighted average of these two estimates is performed where correlations are taken into account.
Correlations are introduced via the common parameters $N_{\BsJpsiKpi}$ and $\omega_{\BsJpsiKst}$.
In addition the efficiency ratios cannot be treated separately and they have a common factor,
namely $\varepsilon_{\BsJpsiKst}^{MC}$. The averaged $\BRof\BsJpsiKst$ is given by:

\begin{equation}
  \centering
\BRof\BsJpsiKst = (4.14 \pm 0.18 \pm 0.26 \pm 0.24 )\times 10^{-5}.
\label{Br_total}
\end{equation}

\noindent The first two uncertainties are statistical and systematic and the last is due to \fdfs.
The result of \equref{Br_total} is in good agreement with previous measurements \cite{Aaij:2012nh}.
