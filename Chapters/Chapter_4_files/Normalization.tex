
The current section addresses the deterination of the \BsJpsiKst branching fraction, hereafter $\BRof\BsJpsiKst$.
This is one of the necessary ingredients for estimating the penguin contributions to \phis, addressed in \secref{penguin_formalism}.
Absolute branching fraction measurements are more complicated to perform compared to relative ones. A dedicated analysis
muyst be built up and optimise for the channel of intererest. Thus, the $\BRof\BsJpsiKst$ is normalized with respect
to two different channels, namely $\BsJpsiPhi$ and $\BdJpsiKst$ and a single average is extracted.

\subsubsection{Normalization with respect to $\BsJpsiPhi$}
Equation \ref{norm_jpsiPhi} shows how $\BRof\BsJpsiKst$ is normalized with respect to $\BRof\BsJpsiPhi$, hereafter $\BRof{\BsJpsiKst}_\phi$.
It requires as input the observed number of \BsJpsiKst and \BsJpsiPhi events. The first one is taken from \equref{signal_yields} as for
the latter a bit more effort is needed. Particularly the selection steps applied to $\BsJpsiPhi$
are chosen such that they are as similar as possible to that of \BsJpsiKst see in \tabref{Bs2JpsiKstSelection}.

\begin{align}
  \centering
\frac{\BRof\BsJpsiKst}{\BRof\BsJpsiPhi} = \frac{N_{\BsJpsiKpi}}{N_{\BsJpsiKK}}
                                  &\times \frac{\varepsilon_{\BsJpsiPhi}^{MC}}{\varepsilon_{\BsJpsiKst}^{MC}}
                                   \times \nonumber \\
                                  \times \frac{\omega_{\BsJpsiPhi}}{\omega_{\BsJpsiKst}}
                                  & \times \frac{\BRof{\phi\to\kaon^+\kaon^-}}{\BRof{\Kstar\to\kaon\pion}}.
\label{norm_jpsiPhi}
\end{align}

\noindent Also the BDTG classifier is applied to the reference channel
with the same cut values as in the \BsJpsiKst channel. The ratio of observed number of events is represented by the first fraction in \equref{norm_jpsiPhi},
whereas the ratio of total efficiencies between the two channels by the second. The last is corrected for differences in efficiencies between real and
simulated data. Additionally, there is a correction factor to account for the presence of \swave in the two
channels\footnote{The penguin estimation assumes no \swave contribution to \BsJpsiKst. Each of the factors $\omega$ in \equref{norm_jpsiPhi} correspond
to the ratio of two angular \pdfs. One with the \swave parameters set to zero and \pwave to those obtained in the angular fit,  whereas the other
with both \pwave and \swave parameters as obtained from the same fit. The $\omega$ factors can be understood as an overall \swave fraction which
corrects the observed number of events.
}.

\begin{equation}
  \centering
\BRof\BsJpsiKst_{\phi} = (4.25 \pm 0.20 \pm 0.13 \pm 0.36) \times 10^{-5},
\label{Br_JpsiPhi}
\end{equation}

\noindent The branching fractions of the intermediate resonance decaying to a di-hadron system are obtained from \cite{PDG}
The $\BRof\BsJpsiPhi$~\cite{SheldonKK} is updated with the latest \bquark quark hadronization
fraction, \fdfs~\cite{LHCb-CONF-2013-011}. The lastquantity expresses the ratio of probabilities that a \bquark quark will form a meson with a
\dquark quark over an \squark quark. The efficiency and the angular correction ratios are about $2$ and $0.9$ respectively.
All the above information is used to extract $\BRof{\BsJpsiKst}_\phi$ shown in \equref{Br_JpsiPhi}, where the uncertainties
are statistical systematic and due to  $\BRof\BsJpsiPhi$ respectively.

\subsubsection{Normalization with respect to $\BdJpsiKst$}
Equation \ref{norm_jpsiKst} shows how $\BRof\BsJpsiKst$ is normalized with respect to $\BRof\BdJpsiKst$, hereafter $\BRof{\BsJpsiKst}_d$.

\begin{equation}
  \centering
\frac{\BRof\BsJpsiKst}{\BRof\BdJpsiKst} = \frac{N_{\BsJpsiKpi}}{N_{\BdJpsiKst}} \times \fdfs \times \frac{\varepsilon_{\BdJpsiKst}^{MC}}{\varepsilon_{\BsJpsiKst}^{MC}}
                                                                          \times \frac{\omega_{\BdJpsiKst}}{\omega_{\BsJpsiKst}},
\label{norm_jpsiKst}
\end{equation}

\noindent The last is similar to \equref{norm_jpsiPhi} as far as the way efficiency and angular correlation ratios are obtained.
The observed number of events are both obtained from \equref{signal_yields} and the latest measurement of the hadronization
fraction from \cite{LHCb-CONF-2013-011}. The $\BRof\BdJpsiKst$ value is taken from \cite{Abe:2002haa}. The final result for $\BRof{\BsJpsiKst}_d$
is shown in \equref{Br_JpsiKst}

\begin{equation}
  \centering
\BRof\BsJpsiKst_{d} = (3.85 \pm 0.18 \pm 0.15 \pm 0.22 \pm 0.42 )\times 10^{-5},
\label{Br_JpsiKst}
\end{equation}

 \noindent where the first two uncertainties are statistical and systematic and the rest are coming from \fdfs and $\BRof\BdJpsiKst$ respectively.

\subsubsection{Averaged $\BRof\BsJpsiKst$}
Both estimates of \equref{Br_JpsiPhi} and \equref{Br_JpsiKst} are compatible within uncorrelated systematics. A weighted average of these two
estimates is performed where correlations are taken into account. Correlations between $\omega_{\BdJpsiKst}$ and $\omega_{\BsJpsiKst}$ originate from the
expected number of events estimated with the mass fit in \secref{Event_Selection}. In addition the efficiency ratios cannot be treated separately
and they have a common factor, namely $\varepsilon_{\BsJpsiKst}^{MC}$. The averaged  $\BRof\BsJpsiKst$ is shown in \equref{Br_total}

\begin{equation}
  \centering
\BRof\BsJpsiKst_{d} = (4.13 \pm 0.16 \pm 0.25 \pm 0.24 \pm 0.42 )\times 10^{-5},
\label{Br_total}
\end{equation}

\noindent The last is in good agreement with previous measurements {\color{red} this reference is wrong}\cite{LHCb-PAPER-2015-006}
