The results of the \chisq fit to the penguin parameters, described in \secref{penguin_more_chanells}
are the following:

\begin{subequations}
\label{peng_a_theta_results}
\begin{align}
    \aPeng{0}         & = 0.01^{+0.10}_{-0.01} & \thPeng{0}         & = -\parenthesis{82^{+262}_{-97}}^\circ & \hadRatio{0}{}{\prime}         = 1.212^{+0.075}_{-0.057},\\
    \aPeng{\parallel} & = 0.07^{+0.11}_{-0.05} & \thPeng{\parallel} & = -\parenthesis{84^{+93}_{-129}}^\circ  & \hadRatio{\parallel}{}{\prime} = 1.230^{+0.102}_{-0.078},\\
    \aPeng{\perp}     & = 0.05^{+0.12}_{-0.05} & \thPeng{\perp}     & = +\parenthesis{53^{+127}_{-232}}^\circ & \hadRatio{\perp}{}{\prime}     = 1.038^{+0.083}_{-0.065}.
\end{align}
\end{subequations}

\noindent The quoted uncertainties are the quadratic sum of statistical and systematic uncertainties.
A $20\%$ $\grpsuthree_F$ breaking is assumed for the systematic, according to what was mentioned at
the end of \secref{had_pars_suthree}, by performing a special fit explained in \secref{su3_breaking}.
Two dimensional contours are shown in Figures \ref{pengPlot_long} to \ref{pengPlot_para}, where the
parameters of interest, $(\akPeng,\thkPeng)$, are transformed to rectangular coordinates for better
visibility. There it can be seen that the $\Hobs{k}$ observables have a smaller impact on the best fit
result due to their associated large uncertainty. Also note the tension between the $\Hobs{k}$ observables
and the best fit result, which suggests that the assumption \equref{had_ratios_assuption} is probably
not fully satisfied.

\begin{figure}[!t]
  \centering
  \includegraphics[trim=0.0cm 0.0cm 0.0cm 0.0cm, clip=true,scale=0.4]{Figures/Chapter5/Penguin_Contribution_Re_vs_Im_allB2VV_Long_withH.pdf}
  \caption{$\aPeng{0}$, $\thPeng{0}$ confidence levels. Measurements of \equref{chisq_form_with} are illustrated with
           colored bands, which correspond to one std. deviation. Figure from \cite{DeBruyn-thesis}}
  \label{pengPlot_long}
\end{figure}

The best fit estimates of \equref{peng_a_theta_results} are subsequently translated, using \equref{tandelta},
to the final estimate of the $\DeltaPhisJpsiPhi{k}$ phase shift:

\begin{subequations}
\label{delta_phis_result}
\begin{align}
\DeltaPhisJpsiPhi{0}         & = 0.000 \; ^{+0.009}_{-0.011} \; {\rm (stat)} \; ^{+0.006}_{-0.007} \; ({\rm syst}) = 0.000 \; ^{+0.010}_{-0.014},\\
\DeltaPhisJpsiPhi{\parallel} & = 0.001 \; ^{+0.010}_{-0.014} \; {\rm (stat)} \; ^{+0.006}_{-0.007} \; ({\rm syst)} = 0.001 \; ^{+0.012}_{-0.016},\\
\DeltaPhisJpsiPhi{\perp}     & = 0.003 \; ^{+0.010}_{-0.014} \; {\rm (stat)} \; ^{+0.006}_{-0.007} \; ({\rm syst)} = 0.003 \; ^{+0.012}_{-0.016},
\end{align}
\end{subequations}

\begin{figure}[t!]
\centering
  \includegraphics[trim=0.0cm 0.0cm 0.0cm 0.0cm, clip=true,scale=0.4]{Figures/Chapter5/Penguin_Contribution_Re_vs_Im_allB2VV_Para_withH.pdf}
  \caption{$\aPeng{\parallel}$, $\thPeng{\parallel}$ confidence levels. Measurements of \equref{chisq_form_with} are illustrated with
           colored bands, which correspond to one std. deviation. Figure from \cite{DeBruyn-thesis}}
  \label{pengPlot_para}
\end{figure}

\noindent where the uncertainties are statistical and systematic due to \grpsuthree breaking respectively.
Statistical uncertainty dominates the $\DeltaPhisJpsiPhi{k}$ measurement.
Note that the penguin parameters $(\akPeng,\thkPeng)$ are ratios of hadronic amplitudes.
Given that, it is interesting to point out that any factorizable $\grpsuthree_F$ breaking
effects entering through the assumption of \equref{su3_apply} cancel out. Thus the results
reported here suffer from non

\begin{figure}[!t]
  \centering
  \includegraphics[trim=0.0cm 0.0cm 0.0cm 0.0cm, clip=true,scale=0.4]{Figures/Chapter5/Penguin_Contribution_Re_vs_Im_allB2VV_Perp_withH.pdf}
  \caption{Limits on the penguin parameters for: $\aPeng{\perp}$ and $\thPeng{\perp}$.}
  \label{pengPlot_perp}
\end{figure}

\noindent factorizable $\grpsuthree_F$ breaking which are small,
according to \cite{DeBruyn-thesis}. The previous statement is supported by the values of the fitted hadronic parameter
ratios which are close to the ones calculated assuming factorization holds. As mentioned in \cite{DeBruyn-thesis},
either the non factorization $\grpsuthree_F$ effects or their ratio with respect to the factorizable $\grpsuthree_F$
ones are small.

Lastly, the shifts $\DeltaPhisJpsiPhi{k}$ quoted in \equref{delta_phis_result}
suggest that contributions of penguin topologies to the \BsJpsiPhi decay amplitude are
small, $<1^\circ$. Given the also small \phis measured value of \equref{phis_lhcb}
it becomes mandatory to control penguin contributions in future \phis measurements.
Furthermore, the sensitivity from the experimental side is interesting with respect to
the Standard Model prediction, see \equref{phis_lhcb_theo}. Thus, potential deviations from these
predictions are going to play a central role in future and more precise \phis measurements.
Increasing the amount of data in the \lhc \runtwo might not be enough to yield
a significant claim on the presence of physics beyond the Standard Model and hence
the upgraded \lhcb detector becomes important in the pursuit for New Physics in the future.

\subsection{Further Crosschecks}
There are two important issues that the fitting strategy, as described in \secref{penguin_chi2_fit},
does not take into account. Namely, $\grpsuthree_F$ breaking effects and correlations from the
experimental measurement between the observables \equref{chisq_form_with}. Both of these issues
are addressed in the current section.

\subsubsection{\grpsuthree breaking}
\label{su3_breaking}
Potential $\grpsuthree_F$ effects manifest themselves in the calculations of the hadronic parameters ratios
when computing the $\Hobs{k}$ observables and in the assumption of \equref{su3_apply}. However, due to the
strategy followed the hadronic parameter ratios are not affected by $\grpsuthree_F$ breaking since they
are determined by the fit. As far as the assumption of \equref{su3_apply} is concerned, a special fit is
performed using only observables related to \BdJpsiRho channel, to investigate the effect of $\grpsuthree_F$.
For that, $(\akPeng,\thkPeng)$ are re-expressed as:

\begin{equation}
  \centering
  \akPeng \to \xi_k \; \akPeng'' \quad \thkPeng \to \delta_k + \thkPeng''.
\label{su3_breaking_invoke}
\end{equation}

\noindent The quantities $\xi_k$ and $\delta_k$ are allowed to vary in the fit and can thus absorb
potential $\grpsuthree_F$ effects that will break the assumption of \equref{su3_apply}.
The central values of $\xi_k$ and $\delta_k$ assume perfect $\grpsuthree_F$ symmetry, meaning $\xi_k=1$ and $\delta_k=0$.
However, a Gaussian constraint allows them to vary within a range of up to $50\%$ breaking.
$\grpsuthree_F$. Specifically, the uncertainties within which $\xi_k$ and $\delta_k$ are allowed to
vary are $[0,0.5]$ and $[0,40]^\circ$ respectively.
Repeating the above-mentioned special shows that the value of $\DeltaPhis{k}$
does not depend on the amount of allowed $\grpsuthree_F$ symmetry breaking.
This is most likely due to the particular structure of \equref{tandelta}, \ie the value of
$\akPeng$, or $\xi_k$ in that case becomes less pronounced for values of $\thkPeng$ which are close to $90^\circ$.

\subsubsection{Correlations of polarization states}
Correlations between the $k$ polarizations of the observables entering the \chisq fit are ignored in the
current analysis. These correlations enter mainly via the $\Acp{k}$ and $\fP{k}$. For example, see the
correlation matrix \tabref{correlation_matrix} of the angular analysis performed in \chapref{Data_Analysis}.
In order to check the effect of these correlations on the estimation of the penguin parameters $(\akPeng,\thkPeng)$,
an additional \chisq fit is performed. In that fit the correlations are indeed taken into account.
The results show a small increase on the uncertainty of the penguin parameter $\thPeng{0}$.
This result suggest that the impact of the correlations on the parameters of interest is negligible.
