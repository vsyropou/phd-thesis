

% \begin{alignat}{2}
% H_0 & = 0.99 \pm 0.07\:\text{(stat)} \pm 0.06\:\text{(syst)} \pm 0.27\:(\text{hadr}) && = 0.99 \pm 0.28\:,\\
% H_\parallel & = 0.91 \pm 0.14\:\text{(stat)} \pm 0.08\:\text{(syst)} \pm 0.21\:(\text{hadr}) && = 0.91 \pm 0.27\:,\\
% H_\perp & = 1.47 \pm 0.14\:\text{(stat)} \pm 0.11\:\text{(syst)} \pm 0.28\:(\text{hadr}) && = 1.47 \pm 0.33\:.
% \end{alignat}
%


% \begin{align}
% \Re[a_0] & = \phantom{-}0.01_{-0.32}^{+0.97}\:, & \Im[a_0] & = \phantom{-}0.025_{-0.031}^{+0.035}\:, & \chi^2_{\text{min}} & = 1.1 \times 10^{-7}\:,\label{Eq:Pen_Re_Im_Long}\\
% %
% \Re[a_\parallel] & = \phantom{-}0.31_{-0.51}^{+0.58}\:, & \Im[a_\parallel] & = -0.082_{-0.087}^{+0.074}\:,& \chi^2_{\text{min}} & = 1.2 \times 10^{-3}\:,\label{Eq:Pen_Re_Im_Para}\\
% %
% \Re[a_\perp] & = -0.44_{-0.21}^{+0.27}\:, & \Im[a_\perp] & = \phantom{-}0.037_{-0.076}^{+0.079}\:,& \chi^2_{\text{min}} & = 1.5 \times 10^{-6}\:,\label{Eq:Pen_Re_Im_Perp}
% \end{align}

\begin{align}
a_0 & = 0.03^{+0.97}_{-0.03}\:, & \theta_0 & = \phantom{-}\left(64^{+116}_{-244}\right)^{\circ}\:,\\
%
a_\parallel & = 0.32^{+0.58}_{-0.32}\:, & \theta_\parallel & = -\left(15^{+150}_{-14}\right)^{\circ}\:,\\
%
a_\perp & = 0.45^{+0.21}_{-0.27}\:, & \theta_\perp & = \phantom{-}\left(175 \pm 10\right)^{\circ}\:.
\end{align}



\begin{alignat}{2}
\Delta\phi^{\jpsi{}\phi}_{s,0} & =
\phantom{-}0.001^{+0.087}_{-0.011}\:\text{(stat)}^{+0.013}_{-0.008}\:\text{(syst)}^{+0.048}_{-0.030}\:(\text{hadr})
&& = \phantom{-}0.001^{+0.100}_{-0.033}\:,\\
%
\Delta\phi^{\jpsi{}\phi}_{s,\parallel} & =
\phantom{-}0.031^{+0.049}_{-0.038}\:\text{(stat)}^{+0.013}_{-0.013}\:\text{(syst)}^{+0.031}_{-0.033}\:(\text{hadr})
&& = \phantom{-}0.031^{+0.059}_{-0.052}\:,\\
%
\Delta\phi^{\jpsi{}\phi}_{s,\perp} & =
-0.046^{+0.012}_{-0.012}\:\text{(stat)}^{+0.007}_{-0.008}\:\text{(syst)}^{+0.017}_{-0.024}\:(\text{hadr})
&& = -0.046^{+0.022}_{-0.028}\:,
\end{alignat}


\begin{figure}[h]
\begin{center}
  \begin{subfigure}{1\textwidth}
    \includegraphics[trim=0.0cm 0.0cm 0.0cm 0.0cm, clip=true,scale=0.4]{Figures/Chapter5/Penguin_Contribution_Ang_vs_Abs_allB2VV_Long.pdf}
    \caption{}
    \label{pengPlot_long}
  \end{subfigure}\\
  \begin{subfigure}{1\textwidth}
    \includegraphics[trim=0.0cm 0.0cm 0.0cm 0.0cm, clip=true,scale=0.4]{Figures/Chapter5/Penguin_Contribution_Ang_vs_Abs_allB2VV_Perp.pdf}
    \caption{}
    \label{pengPlot_perp}
  \end{subfigure}
  \caption{Penguin parameter contours. Figures from~\cite{DeBruyn-thesis}}
\end{center}
\end{figure}

\begin{figure}[h]
\begin{center}
  \includegraphics[trim=0.0cm 0.0cm 0.0cm 0.0cm, clip=true,scale=0.33]{Figures/Chapter5/Penguin_Contribution_Ang_vs_Abs_allB2VV_Para.pdf}
  \caption{Penguin parameter contours. Figures from~\cite{DeBruyn-thesis}}
  \label{pengPlot_para}
\end{center}
\end{figure}

\section{Further Corsechecks}
Here the su3 breaking and the polarization correlations are dressed.

\subsubsection{\grpsuthree breaking}
\label{su3_breaking}

The fact that the \grpsuthree syummetry is a broken symmetry is taken into account in the fit by re parametrizing
$\alpha_f$ and $\theta_f$ as shown in \equref{su3_breaking}.

\begin{equation}
a_i \to \xi \; \alpha_i \quad \theta_i \to \delta + \theta'_i
\label{su3_breaking}
\end{equation}

As there are no dedicated studies for \grpsuthree breaking. It is assumed that $\xi=1$ and $\delta=0$ but include uncertainties
on these paorameters as gaussian constrains in the $\chi^2$ fit. The dependance of $\Delta\phiS{SM,peng}$ on the $\xi$ and $\delta$
uncertainties was found to be irrelavant, {\color{red} see jpsiKst paper}, due to the particular structure of \equref{tanddelta}.

\subsubsection{Correlations of Polarization States}
Correlation between the $k$ polarizations of $\Acp{k}$ and $\fP{k}$ are ignored in the coming from between the polarizations.
A correlated chisq fit was perfomed showin a slightly increase in the uncertainties for some parameters, see jpsiKst paper.
