The results of the \chisq fit described in \secref{penguin_more_chanells} are shown in \equref{peng_a_theta_results}.

\begin{subequations}
\label{peng_a_theta_results}
\begin{align}
    \aPeng{0}         & = 0.01^{+0.10}_{-0.01}\:, & \thPeng{0}         & = -\parenthesis{76^{+256}_{-103}}^\circ\:, & \hadRatio{0}{}{\prime}         = 1.221^{+0.073}_{-0.054}\:,\\
    \aPeng{\parallel} & = 0.07^{+0.11}_{-0.52}\:, & \thPeng{\parallel} & = -\parenthesis{84^{+93}_{-130}}^\circ\:,  & \hadRatio{\parallel}{}{\prime} = 1.233^{+0.102}_{-0.078}\:,\\
    \aPeng{\perp}     & = 0.05^{+0.12}_{-0.05}\:, & \thPeng{\perp}     & = +\parenthesis{53^{+127}_{-232}}^\circ\:, & \hadRatio{\perp}{}{\prime}     = 1.027^{+0.078}_{-0.06}\:.
\end{align}
\end{subequations}

\noindent The quoted uncertainties are the quadratic sum of statistical and systematic ones.
The results of \equref{peng_a_theta_results} are then translated, using \equref{tandelta},
to the final estimate of the phase shift $\DeltaPhisJpsiPhi{k}$, due to penguin topologies which is shown in \equref{delta_phis_result}.

\begin{alignat}{2}
\DeltaPhisJpsiPhi{0} & =
\phantom{-}0.001^{+0.087}_{-0.011}\:\text{(stat)}^{+0.013}_{-0.008}\:\text{(syst)}^{+0.048}_{-0.030}\:(\text{hadr})
&& = \phantom{-}0.001^{+0.100}_{-0.033}\:,\\
%
\DeltaPhisJpsiPhi{\parallel} & =
\phantom{-}0.031^{+0.049}_{-0.038}\:\text{(stat)}^{+0.013}_{-0.013}\:\text{(syst)}^{+0.031}_{-0.033}\:(\text{hadr})
&& = \phantom{-}0.031^{+0.059}_{-0.052}\:,\\
%
\DeltaPhisJpsiPhi{\perp} & =
-0.046^{+0.012}_{-0.012}\:\text{(stat)}^{+0.007}_{-0.008}\:\text{(syst)}^{+0.017}_{-0.024}\:(\text{hadr})
&& = -0.046^{+0.022}_{-0.028}\:,
\label{delta_phis_result}
\end{alignat}

{\color{red} I still need to comment on the uncertainties. I have not decicded which I am goin to quote.}
A systematic uncertainty to $\DeltaPhis{k}$ is asigned by assuming $20\%$ $\grpsuthree_F$ breaking,
according to what was mentioned at the end of \secref{had_pars_suthree}. The systematic is asigned
by performing a special fit explained in \secref{su3_breaking}. Note that the penguin paramters
$(\akPeng,\thkPeng)$ are ratios of hadronic amplitues. Given that it is
intreasting to point out that any factorizable $\grpsuthree_F$ breaking effets entering through the asumption
of \equref{su3_apply} cancell out.
Thus the results reported here suffer from non facroraizable $\grpsuthree_F$ breaking which are small,
accroding to ~\cite{DeBruyn-thesis}. The last statement is supported by the values of the fitted hadronic parameter
ratios which are close to the once calculated assuming factorization holds. As it is mentioned in~\cite{DeBruyn-thesis},
either the non factorization $\grpsuthree_F$ effects or their ratio with respct to the factorizable $\grpsuthree_F$
ones are small.

The results of the paramters $(\akPeng,\thkPeng)$ are summarised in the contours of
Figs \ref{pengPlot_long}, \ref{pengPlot_perp} and \ref{pengPlot_para}. Note that the $\Hobs{k}$ observables have almost no
effect on the result due to their asscociated large uncertainty. Also note that, the $\Hobs{k}$ observables are in
tention with respect to the cetral value of the fit result. This suggests probably that the assuption
\equref{had_ratios_assuption} probably not true. {\color{red} I will rewrite-polish this once I update the final contours.}

\begin{figure}[h]
\begin{center}
  \begin{subfigure}{1\textwidth}
    \includegraphics[trim=0.0cm 0.0cm 0.0cm 0.0cm, clip=true,scale=0.4]{Figures/Chapter5/Penguin_Contribution_Ang_vs_Abs_allB2VV_Long.pdf}
    \caption{}
    \label{pengPlot_long}
  \end{subfigure}\\
  \begin{subfigure}{1\textwidth}
    \includegraphics[trim=0.0cm 0.0cm 0.0cm 0.0cm, clip=true,scale=0.4]{Figures/Chapter5/Penguin_Contribution_Ang_vs_Abs_allB2VV_Perp.pdf}
    \caption{}
    \label{pengPlot_perp}
  \end{subfigure}
  \caption{Penguin parameter contours. Figures from~\cite{DeBruyn-thesis}}
\end{center}
\end{figure}

\begin{figure}[h]
\begin{center}
  \includegraphics[trim=0.0cm 0.0cm 0.0cm 0.0cm, clip=true,scale=0.33]{Figures/Chapter5/Penguin_Contribution_Ang_vs_Abs_allB2VV_Para.pdf}
  \caption{Penguin parameter contours. Figures from~\cite{DeBruyn-thesis}}
  \label{pengPlot_para}
\end{center}
\end{figure}

% \begin{align}
% \Re[a_0] & = \phantom{-}0.01_{-0.32}^{+0.97}\:, & \Im[a_0] & = \phantom{-}0.025_{-0.031}^{+0.035}\:, & \chi^2_{\text{min}} & = 1.1 \times 10^{-7}\:,\label{Eq:Pen_Re_Im_Long}\\
% %
% \Re[a_\parallel] & = \phantom{-}0.31_{-0.51}^{+0.58}\:, & \Im[a_\parallel] & = -0.082_{-0.087}^{+0.074}\:,& \chi^2_{\text{min}} & = 1.2 \times 10^{-3}\:,\label{Eq:Pen_Re_Im_Para}\\
% %
% \Re[a_\perp] & = -0.44_{-0.21}^{+0.27}\:, & \Im[a_\perp] & = \phantom{-}0.037_{-0.076}^{+0.079}\:,& \chi^2_{\text{min}} & = 1.5 \times 10^{-6}\:,\label{Eq:Pen_Re_Im_Perp}
% \end{align}

The results of the penguin shift $\DeltaPhisJpsiPhi{k}$ quoted in \equref{delta_phis_result}
suggest that they controbutions of penguin topologies to the \BsJpsiPhi decay amplitude are small.
{\color{red} Maybe add some comments on future prospet, in view of the intreated precition.}

\subsection{Further Corsechecks}
There are two important issues that the fitting strategy as described in \secref{penguin_chi2_fit}
does not take into account. Namelly, $\grpsuthree_F$ breaking effects and correlations, from the
experimental measurement, between the observables \equref{chisq_form_with}. Both of these issues
addressed in the current section.

\subsubsection{\grpsuthree breaking}
\label{su3_breaking}
Potential $\grpsuthree_F$ effects manifest themeselfs in the calculations of the hadronic parameters ratios
when computing the $\Hobs{k}$ observable and in the assumption of \equref{su3_apply}. However, due to the
followed strategy the hadronic parameter ratios are not affected by $\grpsuthree_F$ breaking since the are fitted for.
As far as the assumption of \equref{su3_apply} is concerned, a special fit is performed using only observables
releated to \BdJpsiRho channel, to investigate the effect of $\grpsuthree_F$. For that, $(\akPeng,\thkPeng)$
are re-expressed as shown in \equref{su3_breaking}

\begin{equation}
  \akPeng \to \xi_k \; \akPeng'' \quad \thkPeng \to \delta_k + \thkPeng''.
\label{su3_breaking}
\end{equation}

\noindent The quantities $\xi_k$ and $\delta_k$ are allowed to vary in the fit and can thus absorb
potential $\grpsuthree_F$ effects that will break the assuption of \equref{su3_apply}.
The central vaues of $\xi_k$ and $\delta_k$ assume perfect $\grpsuthree_F$ symmetry, meaning $\xi_k=1$ and $\delta_k=0$.
However, a gausian contrain allows them to vary within a range of up to $50\%$
$\grpsuthree_F$\footnote{Meaning $\sigma{\xi_k}\in\brackets{0,0.5}$ and $\sigma_{\delta_k}\in\brackets{0,40}^\circ$ }.
Repeating the above mentioned special fit showed that the extracted value of $\DeltaPhis{k}$
does not depend on the amount of allowed $\grpsuthree_F$ symmetrty breaking.
This is most likely due to the particular structure of \equref{tandelta}, meaning that
the value of $\akPeng$, or $\xi_k$ in that case, becomes less pronounced by values of
$\thkPeng$ which are close to $90^\circ$.

\subsubsection{Correlations of Polarization States}
Correlations between the $k$ polarizations of the observables entering the \chisq fit were ignored in the
current analysis. These correlations enter mainly via the $\Acp{k}$ and $\fP{k}$. For example, see the
correlation matrix \tabref{correlation_matrix} of the angular analysis performed in \chapref{Data_Analysis}.
In order to check the effect of those corrlations on the estimation of the penguin paramters $(\akPeng,\thkPeng)$,
an additonal \chisq fit is performed. In that fit the correlations are indeed taken into account.
The results show a small increase on the error of the penguin paramter $\thPeng{0}$.
The last result suggest that the impact of the coorelations on the parameters of intereset is negligible.
