The results of the \chisq fit described in \secref{penguin_more_chanells} are shown in \equref{peng_a_theta_results}.
which are then translated, using \equref{tandelta}, to the final estimate for the penguin phase shift on \phis,
shown in \equref{delta_phis_result}.

\begin{subequations}
\label{peng_a_theta_results}
\begin{align}
    \aPeng{0}         & = 0.03^{+0.97}_{-0.03}\:, & \thPeng{0}         & = +\parenthesis{64^{+116}_{-244}}^\circ\:,\\
    \aPeng{\parallel} & = 0.32^{+0.58}_{-0.32}\:, & \thPeng{\parallel} & = -\parenthesis{15^{+150}_{-14}}^\circ\:,\\
    \aPeng{\perp}     & = 0.45^{+0.21}_{-0.27}\:, & \thPeng{\perp}     & = +\parenthesis{175 \pm 10}^\circ\:.
\end{align}
\end{subequations}

\noindent were the uncertainties are the quadratic sum of statistical and systematic ones.

\begin{alignat}{2}
\Delta\phi^{\jpsi{}\phi}_{s,0} & =
\phantom{-}0.001^{+0.087}_{-0.011}\:\text{(stat)}^{+0.013}_{-0.008}\:\text{(syst)}^{+0.048}_{-0.030}\:(\text{hadr})
&& = \phantom{-}0.001^{+0.100}_{-0.033}\:,\\
%
\Delta\phi^{\jpsi{}\phi}_{s,\parallel} & =
\phantom{-}0.031^{+0.049}_{-0.038}\:\text{(stat)}^{+0.013}_{-0.013}\:\text{(syst)}^{+0.031}_{-0.033}\:(\text{hadr})
&& = \phantom{-}0.031^{+0.059}_{-0.052}\:,\\
%
\Delta\phi^{\jpsi{}\phi}_{s,\perp} & =
-0.046^{+0.012}_{-0.012}\:\text{(stat)}^{+0.007}_{-0.008}\:\text{(syst)}^{+0.017}_{-0.024}\:(\text{hadr})
&& = -0.046^{+0.022}_{-0.028}\:,
\label{delta_phis_result}
\end{alignat}

Note that the penguin paramters $(\akPeng,\thkPeng)$ are ratios of hadronic amplitues. Given that it is
intreasting to point out that any factorizable $\grpsuthree_F$ breaking effets entering through the asumption
of \equref{} cancell out. (A small subtlity enterign through the second schanell jspsirho).
Thus the results reported here suffer from non facroraizable $\grpsuthree_F$ breaking which are probably
small, accroding to blah. The last statement is supported by the values of t the fitted hadronic parameter
ratios which are close to the once calculated assuming factorization. Thus, as it is mentioned in krystof,
it is either that the non factorization $\grpsuthree_F$ effects are small or thei ratio with respct to
the factorizable $\grpsuthree_F$ effects is small.

Note also that the H observables have almost no effect on the result.
It was also noticed that the H observables are in tention with respct to the cetral value of the fit result.
This suggests probably that the assuption \equref{} probably not true.

\begin{itemize}
  \item put results with hadronic ratios
  \item interpretaion of the hadronic ratio result formt he fit.
  \item explain plots
  \item investigate the C = -Acp
  \item modify plot legend to
  \item maybe
\end{itemize}

\begin{figure}[h]
\begin{center}
  \begin{subfigure}{1\textwidth}
    \includegraphics[trim=0.0cm 0.0cm 0.0cm 0.0cm, clip=true,scale=0.4]{Figures/Chapter5/Penguin_Contribution_Ang_vs_Abs_allB2VV_Long.pdf}
    \caption{}
    \label{pengPlot_long}
  \end{subfigure}\\
  \begin{subfigure}{1\textwidth}
    \includegraphics[trim=0.0cm 0.0cm 0.0cm 0.0cm, clip=true,scale=0.4]{Figures/Chapter5/Penguin_Contribution_Ang_vs_Abs_allB2VV_Perp.pdf}
    \caption{}
    \label{pengPlot_perp}
  \end{subfigure}
  \caption{Penguin parameter contours. Figures from~\cite{DeBruyn-thesis}}
\end{center}
\end{figure}

\begin{figure}[h]
\begin{center}
  \includegraphics[trim=0.0cm 0.0cm 0.0cm 0.0cm, clip=true,scale=0.33]{Figures/Chapter5/Penguin_Contribution_Ang_vs_Abs_allB2VV_Para.pdf}
  \caption{Penguin parameter contours. Figures from~\cite{DeBruyn-thesis}}
  \label{pengPlot_para}
\end{center}
\end{figure}

% \begin{align}
% \Re[a_0] & = \phantom{-}0.01_{-0.32}^{+0.97}\:, & \Im[a_0] & = \phantom{-}0.025_{-0.031}^{+0.035}\:, & \chi^2_{\text{min}} & = 1.1 \times 10^{-7}\:,\label{Eq:Pen_Re_Im_Long}\\
% %
% \Re[a_\parallel] & = \phantom{-}0.31_{-0.51}^{+0.58}\:, & \Im[a_\parallel] & = -0.082_{-0.087}^{+0.074}\:,& \chi^2_{\text{min}} & = 1.2 \times 10^{-3}\:,\label{Eq:Pen_Re_Im_Para}\\
% %
% \Re[a_\perp] & = -0.44_{-0.21}^{+0.27}\:, & \Im[a_\perp] & = \phantom{-}0.037_{-0.076}^{+0.079}\:,& \chi^2_{\text{min}} & = 1.5 \times 10^{-6}\:,\label{Eq:Pen_Re_Im_Perp}
% \end{align}


\subsection{Further Corsechecks}
There are two important issues that the fitting strategy as described in \secref{penguin_chi2_fit}
does not take into account. Namelly, $\grpsuthree_F$ breaking effects and correlations between
the measurements of the observables \equref{chisq_form_with}. Bothe of these issues addressed in the current section.

\subsubsection{\grpsuthree breaking}
\label{su3_breaking}
Potential $\grpsuthree_F$ effects manifest themeselfs in the calculations of the hadronic parameters ratios
when computing the $\Hobs{}$ and in the assumption of \equref{su3_apply}. However, due to the followed strategy
the hadronic parameter ratios do not rely on any $\grpsuthree_F$ since the are fitted for.
As far as the assumption of \equref{su3_apply} is concerned, a special fit is performed using only observables
releated to \BdJpsiRho channel. For that, $(\akPeng,\thkPeng)$ are re-expressed as shown in \equref{su3_breaking}

\begin{equation}
  \akPeng \to \xi_k \; \akPeng'' \quad \thkPeng \to \delta_k + \thkPeng''.
\label{su3_breaking}
\end{equation}

\noindent The quantities $\xi_k$ and $\delta_k$ are free parameters with a gaiussian constrian and can absorb
potential $\grpsuthree_F$ effects that will break the assuption of \equref{su3_apply}. The effect of various
amounts of $\grpsuthree_F$ breaking is investigated by means of a $\xi_k-\DeltaPhis{k}$ contour. The errors on the
$\xi_k$ and $\delta_k$ parameters, tha constrain their central valuee, are varied from $0-50\%$ and $(0-40)^{\circ}$.
Then the above mentioned special fit is pepreated several times. The results showed no dependence on the value
of $\DeltaPhis{k}$ on $\xi_k$. This is most likely due to the particular structure of \equref{tanddelta}. {\color{red}strong theta pahse maybe. what do you eman here.}

In addition, a systematic uncertainty to $\DeltaPhis{k}$ is asigned by assuming $20\%$ $\grpsuthree_F$ breaking,
according to what was mentioned at the end of \secref{had_pars_suthree}.

\subsubsection{Correlations of Polarization States}
Correlations between the $k$ polarizations of the observables entering the \chisq fit were ignored in the
current analysis. These correlations enter mainly via the $\Acp{k}$ and $\fP{k}$. For example, see the
correlation matrix \tabref{correlation_matrix} of the angular analysis performed in \chapref{Data_Analysis}.
In order to check the effect of those corrlations on the estimation of the penguin paramters $(\akPeng,\thkPeng)$,
an additonal \chisq fit is performed. In that fit the correlations are indeed taken into account.
The results show a small increase on the error of the penguin paramter $\thPeng{0}$.
The last result suggest that the impact of the coorelations on the parameters of intereset is negligible.
