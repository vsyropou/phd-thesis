


Follwoing {\color{red} ref 106 in krystofs thesis.} the \BsJpsiPhi decay amplitude can take place via four different topologies.
Two of them were already introduced in \figref{bs2jpsiphi} and \figref{bs2jpsiphi_peng} which are called {\it color suppressed }$(C)$ and {\it penguin}$(P)$
topologies respectivelly. The other two types namelly {\it exchage} and {\it penguin-anihilation} can be neglected according to {\color{red} krystofs thesis}
given the current experimental precision. Also it is assumed that the $\phi$ meson is a pure ($\squark\squarkbar$) state {\color{red}neglecting ref 131 krystofs}. Given these assumptions the \BsJpsiPhi amplitude
is decomposed in \equref{bsjpsiphi_amp}, taking into account the relevat CKM elements involved in each topology.

\begin{equation}
\mathcal{A} \parenthesis{\BsJpsiPhiPolState{k}} = \Vus\Vub^*P_{\uquark} + \Vcs\Vcb^*\brackets{C +P_{\cquark}} + \Vts\Vtb^*P_{\tquark},
 \label{bsjpsiphi_amp}
\end{equation}

\noindent where the suscribts in the penguin topologies, $P$, denote the flavour of the quark present inside the loop of \figref{bs2jpsiphi_peng}.
The above expresion has to be parameterized in such a way that it is posible to probe the penguin contributions to
the \BsJpsiPhi decay amplitude. Given the unitarity feature of the CKM matrix and using the Wolfenstein parametrization of \equref{CKMwolfenstein},
the decay amplitude in \equref{bsjpsiphi_amp} can be rewritten as shown in \equref{bsjpsiphi_amp_param}.

\begin{equation}
  \mathcal{A} \parenthesis{\BsJpsiPhiPolState{k}} = \eta_k  \parenthesis{1-\frac{\lambda^2}{2}} \formFctr{k} \brackets{ 1 - \epsilon \akPeng e^{i\thkPeng} e^{i\gamma} },
 \label{bsjpsiphi_amp_param}
\end{equation}

\noindent where the following definitions are used:

\begin{equation}
  \formFctr{k} \equiv \VcbMag \brackets{C + P_c - P_t}, \;\;\;\; \akPeng e^{i\thkPeng} \equiv R_b \brackets{ \frac{P_c - P_t}{C + P_c - P_t} },
  \label{bsjpsiphi_amp_param_defs}
\end{equation}

\noindent with

\begin{equation}
  \epsilon = \frac{\lambda^2}{1-\lambda^2} \;\;\text{and} \;\;  R_b = \parenthesis{1-\frac{\lambda^2}{2}} \frac{1}{\lambda} \modulo{\frac{\Vub}{\Vcb}},
  \label{bsjpsiphi_amp_param_defs_II}
\end{equation}

\noindent The quantities in $\akPeng$ and $\thkPeng$ in \equref{bsjpsiphi_amp_param} parameterize the penguin contribution to the overall \BsJpsiPhi
decay amplitude. Whereas the hadronic parameters $\formFctr{k}$ are discussed in \secref{penguin_more_chanells}.
The subsctipt $k$ denotes the \BsJpsiPhi amplitude polarization which is introduced for reasons mentioned at the end of
\secref{Flavour_Physics}\footnote{Subscript $k$ is identical to the one defined in \equref{amps_param} of \chapref{Data_Analysis}.}.
The complex phase $\gamma$ is introduced via the $\Vub$ element of the CKM matrix. The parameter $\lambda$ here is the one of the
CKM matrix parametrization of \secref{CKMwolfenstein}. Whrease the eigenvalue, $\eta_k$, of the final state $f$ is identified with the one in \equref{lambda_cpv}.
Note the suppression factor $\epsilon$ in \equref{bsjpsiphi_amp_param} which implies that penguin contributions to the \BsJpsiPhi amilitude are smaller
by rouglty 2 orders of magnitude with respect to the ones from tree level.
