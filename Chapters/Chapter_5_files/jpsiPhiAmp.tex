


Follwoing {\color{red} ref 106 in krystofs thesis.} the \BsJpsiPhi decay amplitude can take place via four different topologies.
Two of them were already introduced in \figref{bs2jpsiphi} and \figref{bs2jpsiphi_peng} which are called {\it color suppressed }$(C)$ and {\it penguin}$(P)$
topologies respectivelly. The other two types namelly {\it exchage} and {\it penguin-anihilation} can be neglected according to {\color{red} krystofs thesis}
given the current experimental precision. Given this assumption the \BsJpsiPhi amplitude is decomposed in \equref{bsjpsiphi_amp}, taking into account the
relevat CKM elements involved in each topology.

\begin{equation}
A \parenthesis{\BsJpsiPhi} = \Vus\Vub^*P_{\uquark} + \Vcs\Vcb^*\brackets{C +P_{\cquark}} + \Vts\Vtb^*P_{\tquark},
 \label{bsjpsiphi_amp}
\end{equation}

\noindent where the suscribts in the penguin topologies, $P$, denote the flavour of the quark present inside the loop of \figref{bs2jpsiphi_peng}.
The above expresion \equref{bsjpsiphi_amp} has to be re-expressed in such a way that it is posible to probe the penguin contributions to
the \BsJpsiPhi decay amplitude. Given the unitarity feature of the CKM matrix plus the Wolfenstein paramtrization of \equref{CKMwolfenstein},
the decay amplitude in \equref{bsjpsiphi_amp} can be rewritten as shown in \equref{bsjpsiphi_amp_param}.

\begin{equation}
  A \parenthesis{\BsJpsiPhi} = \parenthesis{1-\frac{\lambda^2}{2}} \polFrac{f} \brackets{ 1 + \epsilon a_f e^{i\theta_f} e^{i\gamma} },
 \label{bsjpsiphi_amp_param}
\end{equation}

\noindent where the following definitions are used:

\begin{equation}
  \polFrac{f} \equiv \VcbMag \brackets{C + P_c - P_t}, \;\;\;\; a_f e^{i\theta_f} \equiv R_b \brackets{ \frac{P_c - P_t}{C + P_c - P_t} },
  \label{bsjpsiphi_amp_param_defs}
\end{equation}

\noindent with

\begin{equation}
  \epsilon = \frac{\lambda^2}{1-\lambda^2} \;\;\text{and} \;\;  R_b = \parenthesis{1-\frac{\lambda^2}{2}} \frac{1}{\lambda} \modulo{\frac{\Vub}{\Vcb}},
  \label{bsjpsiphi_amp_param_defs_II}
\end{equation}

\noindent The quantities in $\alpha_f$ and $\theta_f$ in \equref{bsjpsiphi_amp_param} parameterize the penguin contribution to the overall \BsJpsiPhi
decay amplitude. The subsctipt $f$ denotes the polarization state of the total decay amplitude. They are introduced for reasons mentioned at the end
of \secref{Flavour_Physics} and a bit more details can be found in \secref{Diferential_Decay_Rate}.
Also the complex phase $\gamma$ is introduced via the $\Vub$ element of the CKM matrix.

Calculations for the penguin contributions to $A \parenthesis{\BsJpsiPhi}$ are available {\color{red} 40 and 41 in kyrstofs thesis}.
However, these cacluations are difficult to perform, since they involve non-perturbative long-distance QCD effects. Thus an alternative
approach, according to {\color{red} krystof and rob}, is follwed in order to relate those contributions to similar decay cahanells
as \BsJpsiPhi. By doing so the presition on the penguin contributions increases. A minimal desctription of the necessesary formalism
to extract the penguin shift, $\Delta\phiS{SM,peng}$, using also the chanells \BsJpsiKst and \BsJpsiRho is given in the rest of the
current section along
