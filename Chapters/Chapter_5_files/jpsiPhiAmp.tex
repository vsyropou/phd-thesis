The \BsJpsiPhi decay amplitude can take place via four different topologies.
Two of them were already introduced in \figref{bs2jpsiphi} and \figref{bs2jpsiphi_peng};
the {\it color suppressed }$(C)$ and {\it penguin} $(P)$ topologies respectively.
According to \cite{DeBruyn-thesis,DeBruyn:2014oga} the other two types, {\it penguin-annihilation}
and {\it exchange} illustrated in \figref{peng_ana_exchange}, can be neglected given the current experimental precision.
It is also assumed that the $\phi$ meson is a pure ($\squark\squarkbar$) state, based on \cite{Faller:2008gt}.
Given these assumptions the \BsJpsiPhi amplitude is decomposed in \equref{bsjpsiphi_amp},
taking into account the relevant CKM elements involved in each topology.

\begin{equation}
\mathcal{A} \parenthesis{\BsJpsiPhiPolState{k}} = \Vus\Vub^*P_{\uquark}^k + \Vcs\Vcb^*\brackets{C^k +P_{\cquark}^k} + \Vts\Vtb^*P_{\tquark}^k,
 \label{bsjpsiphi_amp}
\end{equation}

\noindent where the subscripts in the penguin topologies, $P$, denote the flavor of the quark present inside
the loop of \figref{bs2jpsiphi_peng}. The superscript $k$ is identical to the one defined in \equref{amps_param}
of \chapref{Data_Analysis} and denotes the \BsJpsiPhi amplitude polarization.
The above expression has to be parameterized in such a way that it is possible to probe the penguin contributions to
the \BsJpsiPhi decay amplitude. Given the unitarity nature of the CKM matrix and using the Wolfenstein parametrization
of \equref{CKMwolfenstein}, the decay amplitude of \equref{bsjpsiphi_amp} can be rewritten as shown in the following equation:

\begin{equation}
  \mathcal{A} \parenthesis{\BsJpsiPhiPolState{k}} = \eta_k  \parenthesis{1-\frac{\lambda^2}{2}} \formFctr{k} \brackets{ 1 - \epsilon \akPeng e^{i\thkPeng} e^{i\gamma} },
 \label{bsjpsiphi_amp_param}
\end{equation}

\noindent where the following definitions are used:

\begin{equation}
  \formFctr{k} \equiv \VcbMag \brackets{C^k + P_\cquark^k - P_\tquark^k}, \;\;\;\; \akPeng e^{i\thkPeng} \equiv R_\bquark \brackets{ \frac{P_\cquark^k - P_\tquark^k}{C^k + P_\cquark^k - P_\tquark^k} },
  \label{bsjpsiphi_amp_param_defs}
\end{equation}

\noindent with,

\begin{equation}
  \epsilon = \frac{\lambda^2}{1-\lambda^2} \;\;\text{and} \;\;  R_\bquark = \parenthesis{1-\frac{\lambda^2}{2}} \frac{1}{\lambda} \modulo{\frac{\Vub}{\Vcb}}.
  \label{bsjpsiphi_amp_param_defs_II}
\end{equation}

\noindent The quantities $\akPeng$ and $\thkPeng$ in \equref{bsjpsiphi_amp_param} parameterize the penguin
contribution to the overall \BsJpsiPhi decay amplitude. The hadronic parameters $\formFctr{k}$ are combinations
of $C$ and $P_\quark$ decay amplitudes and are discussed in more detail, in \secref{had_pars_suthree}.
The CKM angle $\gamma$ is introduced via the $\Vub$ matrix element.
The parameter $\lambda$ here is the one of the CKM matrix parametrization of \secref{CKMwolfenstein}.
The eigenvalue, $\eta_k$, of the final state $f$ is identified with the one in \equref{lambda_cpv}.
Note the suppression factor $\epsilon \sim 0.05$ in \equref{bsjpsiphi_amp_param} which implies that penguin contributions
to the \BsJpsiPhi amplitude are smaller by approximately two orders of magnitude with respect to tree level contributions.

\begin{figure}[t]
  \centering
  \begin{subfigure}{0.5\textwidth}
    \raggedright
    \includegraphics[width=1.02\textwidth, trim=1.3cm 0cm 0.5cm 0.5cm, clip=true]{Figures/Chapter5/B2JpsiX_PenguinAnnihilation}
    \caption{}
    \label{peng_exchange}
  \end{subfigure}%
  \hfill%
  \begin{subfigure}{0.5\textwidth}
    \raggedleft
    \includegraphics[width=1.02\textwidth,trim=1cm 0cm 0.5cm 0.5cm, clip=true]{Figures/Chapter5/B2JpsiX_Exchange}
    \caption{}
    \label{peng_anihilation}
  \end{subfigure}
  \caption{Higher order, penguin annihilation (A) and exchange (B), \BJpsiX decay topologies. Figures from \cite{DeBruyn-thesis}.}
  \label{peng_ana_exchange}
\end{figure}
