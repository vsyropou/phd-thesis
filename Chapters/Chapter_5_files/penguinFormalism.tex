Accroding to blah the penguin parameters $\alpha_f$ and $\beta_f$ can be related to the CP asymmetries, \equref{acp_mixing} and \equref{cpv_decay},
as shown in \equref{}.

\begin{align}
  \Acp{\rm dir}\parenthesis{\Bqtof} &= \frac{ 2a_f \sin\theta_f \sin\gamma} {1-2a_f \cos\theta_f \cos\gamma+a_f^{2}} \\
  \Acp{\rm mix}\parenthesis{\Bqtof} &= \eta_f \brackets{\frac{\sin\phiq - 2\alpha_f \cos\theta_f\sin(\phiq+\gamma) + \alpha_f^2 \sin(\phiq+2\gamma)} {1-2a_f \cos\theta_f \cos\gamma+a_f^{2}} }
  \label{bsjpsiphi_amp_param_defs_II}
\end{align}

In addition for modes that include are related to \BsJpsiPhi via the $SU(3)_F$ symmetry


 Given the experimentally measured values of the above CP assymetries {\color{red} ref blah}, a $\chi^2$ fit is
used, see ref {\color{red} krystof} to determine the penguin parameters alpha and beta.

\subsection{Including more chanells}
\subsection{The \BsJpsiKst and Bsjpsirho control chanells}
For this relation to take place it is assumed that, as far as the strong interaction is concerned, quark flavor is irrelevant,
since the strong interaction couples to color and not to flavor. This assumption is based on the so called $SU(3)_F$ or $U$-sppin symmetry,
see {\color{red} Krystof thesis section 3.4 ana elsewere.}.

In order to increase sensitivuty or for a flavour sepcific finals atetes the Acp mix is nto there thus the H observable is constructed
according to

\begin{itemize}
  \item Acp mix and Acp dir which are measured are realeated to alpha and theta as in equation
  \item Thus for flavour specific final states you need branching ratio information.
  \item footnote(In principle you can solve it with only jpsiphi but you want to introduce more chanells to reduce uncertainty).
\end{itemize}


\begin{itemize}
\item $H_i$, related to the branching ratios and polarisation fractions,
\begin{eqnarray}\label{Eq:Hobs_Vector}
H_i & \equiv &  \frac{1}{\epsilon} \left|\frac{\mathcal{A}'_i}{\mathcal{A}_i}\right|^2
\frac{\text{PhSp}\left(\BsJpsiPhi\right)}{\text{PhSp}(\BsJpsiKst)}
\frac{\BR{\BsJpsiKst}_{\text{theo}}}{\BR{\BsJpsiPhi}_{\text{theo}}}
\frac{f_i}{f'_i}\:,  \\
  & = & \frac{1-2a_i \cos(\theta_i) \cos\gamma+a_i^{2}}{1+2\epsilon a'_i \cos\theta'_i\cos\gamma +\epsilon^2 a_i^{\prime 2}} \:, \nonumber
\end{eqnarray}
\item $A^{CP}_i$, the direct CP violation asymmetries%
\footnote{Conventions: $A^{CP}_i = -\mathcal{A}_{\text{dir}}^{\CP}$ used in Ref.~\cite{DeBruyn:2014oga}}.
\begin{equation}\label{eq:ACPpeng}
A^{\CP}_i= -\frac{2a_i\sin\theta_i\sin\gamma}{1-2a_i\cos\theta_i\cos\gamma+a_i^{2}}\:.
\end{equation}
\end{itemize}
\begin{equation}
\text{PhSp}(B\to V_1V_2) \equiv \frac{1}{16\pi m_B}\Phi\left(\frac{m_{V_1}}{m_B}, \frac{m_{V_2}}{m_B}\right)\:,
\end{equation}
\begin{equation}\label{Eq:penguin_relation}
a_i = a'_i\:,\qquad \theta_i = \theta'_i\:,
\end{equation}
\begin{equation}\label{tandelta}
\tan(\Delta\,\phi_{s,i}) = \frac{2\epsilon a'_i \cos\theta'_i \sin\gamma+\epsilon^2 a^{\prime 2}_i \sin2\gamma}{1+2\epsilon a'_i \cos\theta'_i \cos\gamma+ \epsilon^2 a^{\prime 2}_i \cos2\gamma}.
\end{equation}


\begin{itemize}
  \item H observable is necesary.
  \item amplitude ratio is required introducing more uncertainty.
  \item SU3 symmetry is used. but it increases systematic uncertainties.
\end{itemize}

\subsection{\grpsuthree breaking}
\label{su3_breaking}

\begin{align}
\left|\frac{\mathcal{A}'_0(\BsJpsiPhi)}{\mathcal{A}_0(\BsJpsiKst)}\right| & = 1.23 \pm 0.16\:,\label{Eq:AmpRat_JpsiKstar_long}\\
%%%
\left|\frac{\mathcal{A}'_{\parallel}(\BsJpsiPhi)}{\mathcal{A}_{\parallel}(\BsJpsiKst)}\right| & = 1.28 \pm 0.15\:,\\
%%%
\left|\frac{\mathcal{A}'_{\perp}(\BsJpsiPhi)}{\mathcal{A}_{\perp}(\BsJpsiKst)}\right| & = 1.20 \pm 0.12\:,\label{Eq:AmpRat_JpsiKstar_perp}
\end{align}
