Following {\color{red} 0810.4248v1} the penguin parameters $\alpha_f$ and $\beta_f$ of \equref{bsjpsiphi_amp_param}
can be related to the CP asymmetries of \equref{cp_asym_lambda} as shown in \equref{bsjpsiphi_peng_acp_obs}.

\begin{subequations}
  \label{bsjpsiphi_peng_acp_obs}
\begin{align}
  \Acp{\rm dir}\parenthesis{\Bq\to (f)^k} &= \frac{ \akPeng \sin\thkPeng \sin\gamma} {1 - 2 a_k \cos\thkPeng \cos\gamma + a_k^{2}}
  \label{bsjpsiphi_peng_acp_dir} \\
  \Acp{\rm mix}\parenthesis{\Bq\to (f)^k} &= \nonumber \\
   = \eta_k & \brackets{\frac{\sin\phiq - 2\akPeng \cos\thkPeng\sin(\phiq+\gamma) + \akPeng^2 \sin(\phiq+2\gamma)}
                                                                 {1 - 2 a_k \cos\thkPeng \cos\gamma + a_k^{2}} },
  \label{bsjpsiphi_peng_acp_mix}
\end{align}
\end{subequations}

\noindent where $\phiq$ represents the Standard Model prediction for either the weak phase \phis of \equref{phis_theo},
or the equivalent weak phase \phid of the \Bd meson defined in \equref{phid_theo}.
Equations \ref{bsjpsiphi_peng_acp_obs} form a $2x2$ system in terms of $(\akPeng,\thkPeng)$ that can in principle be solved
for. Note that in the case where \eqref{bsjpsiphi_peng_acp_obs} refer to \BsJpsiPhi, the penguin supression factor
needs to be appied. This implies that following transformation has to take place $(\akPeng\to-\epsilon\akPeng)$.
After an estimation of $(\akPeng,\thkPeng)$ is available the penguin shift to \phis could be determined from \equref{tandelta}.

\begin{equation}
\centering
\tan(\Delta\phis^k) = \frac{ 2\akPeng\epsilon\cos\thkPeng \sin\gamma + \epsilon^2\akPeng^2 \sin2\gamma}
                             {1 + 2\epsilon\akPeng \cos\thkPeng \cos\gamma + \epsilon^2\akPeng^2 \cos2\gamma},
\label{tandelta}
\end{equation}

\noindent where $\Delta\phis^k$ denotes the polarisation
dependant version of $\Delta\phiS{SM, peng}$ of \equref{phis_sm_peng}.

However, as it was metnioned in the begining of the currrent chapter, the goal is to increase the precision on $\Delta{\phis}$
by relating similar to \BsJpsiPhi chanells in the penguin estimation on grounds of the $\grpsuthree$ quark symmetry.
Thus, the measured values of the observables \equref{bsjpsiphi_peng_acp_obs} from the chanells \BsJpsiKst and \BdJpsiRho
are combined in a $\chi^2$ fit to estimate $(\akPeng,\thkPeng)$. The details of this fit are given in \secref{penguin_chi2_fit}.

\subsubsection{Invoking \grpsuthree Symmetry}
The $\grpsuthree_{F}$ symmetry is used to relate chanells like \BsJpsiKst and \BsJpsiRho to \BsJpsiPhi.
In practice this relation is implemented by assuming that the penguin parameters $(\akPeng,\thkPeng)$
are the same between all the above chanells, as shown in \equref{invoke_su3}. In additon $\grpsuthree_{F}$
is also exploited in hadronic parameter ratios between different cahnells, as explained in \secref{penguin_chi2_fit}.

\subsubsection{Information from Branching ratios}
Addional information from branching ratios can be exploited via the $H$ observable, shown in \equref{hobs_def}

\begin{equation}
\centering
  \Hobs{k} \equiv   \frac{1}{\epsilon}
            \modulo{\frac{\formFctr{k}}{\formFctr{k}'}}^2
                    \frac{\PhSp{\BsJpsiPhi}} {\PhSp{\Bq\to f}}
                    \frac{\tau_{\Bs}}{\tau_{\Bq}}
                    \frac{\BRof{\Bq\to f}_{\rm theo}}{\BRof{\BsJpsiPhi}_{\rm theo}}
                    \frac{\fP{k}'}{{\fP{k}}},
\label{hobs_def}
\end{equation}

\noindent where the superscript prime$({}^\prime)$ labels quantities related to $\Bq\to f$. The subscript "theo" is there to distinguish
between the concept of branching fraction used in theoretical calculations and the ones that are measured experimentally,
see \secref{penguin_more_chanells} for more details. The polarization fractions $\fP{k}$ are identical to the ones
efined in \equref{amps_param}. Whereas the average decay time of the \Bs and \Bq are denoted by $\tau_{\Bs}$ and $\tau_{\Bq}$
respectivelly. The phase space factor $\PhSp{X}$ is defined in \equref{phase_space_eq_def}.

\begin{equation}
\centering
   \PhSp{\BJpsiX}  = \brackets{ \mass{\Bq} \PhSpPhi{ \nicefrac{\mass{\jpsi}}{\mass{\Bq}}, \nicefrac{\mass{X}}{\mass{\Bq}}  } }^3,
\label{phase_space_eq_def}
\end{equation}

\noindent where,

\begin{equation}
\centering
   \PhSpPhi{x,y} = \sqrt{ (1-(x+y)^2)(1-(x-y)^2) },
\label{phase_space_phi_eq_def}
\end{equation}

\noindent is the standard two body decay phase space function. Lastly, the $H$ observable is releated to the penguin parameters
$(\akPeng,\thkPeng)$ as shown in \equref{Hobs_peng_param}.

\begin{equation}
\centering
  \Hobs{k} = \frac{1-2\akPeng\cos\thkPeng\cos\gamma + \akPeng^2}{1+2\epsilon\akPengPrime\cos\thkPeng\cos\gamma + \epsilon^2\akPeng^2}
\label{Hobs_peng_param}
\end{equation}

The observable is usefull for final states where the \Acp{\rm mix} observable vanishes. Such final states are called {\it flavour specific}
where CP-Violation in the interference is not active, implying that either \Bs or \Bsb can decay to this final state but not both.
Despite providing the second equation in flavour specific states the $\Hobs{}$ relies in external input, particularly the hadronic quantities $\formFctr{k}$.
The last come from theoretical calculations and introduce additional uncertainty to the exrtaction of the penguin paramters.
For this reason the $\Hobs{}$ is not always prefared.

Lastly note that the $\Hobs{}$ observable are constructed in terms of the theoretical branching fractions
defined at zero decay time, which difer from the measured time-integrated branching fractions~\cite{DeBruyn:2012wj}
due to the non-zero decay-width diference of the \Bs meson~\cite{hfag-2014}. The calculation of the convertion factor
is done as in~\cite{bsjpsikst-paper}. The results denpend on the CP eigenvalue of the final state and shown in
\tabref{br_conversions}

\begin{table}[!h]
  \center
  \begin{tabular}{c c }
    \hline
                        & $\BRof{\B\to f}_{\rm theo} / \BRof{\B\to f}_{\rm exp} $ \\
    \hline
      CP-even          &  $1.0608 \pm 0.0045$ \\
      CP-odd           &  $0.9392 \pm 0.0045$ \\
      flavor specific  &  $0.9963 \pm 0.0006$ \\

    \hline
  \end{tabular}
  \caption{\small Conversion facrors.}
  \label{br_conversions}
\end{table}

\noindent The results for the CP-even and CP-odd are $1.0608 \rm 0.0045$ and  $0.9392 \rm 0.0045$.
While for the flavor specific final states is  $0.9963 \rm 0.0006$
Note that kst is a flavlour specific final state
