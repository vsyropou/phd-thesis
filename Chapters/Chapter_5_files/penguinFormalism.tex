Following {\color{red} 0810.4248v1} the penguin parameters $\alpha_f$ and $\beta_f$ of \equref{bsjpsiphi_amp_param}
can be related to the CP asymmetries, \equref{acp_mixing} and \equref{cpv_decay},
as shown in \equref{}.

\begin{align}
  \Acp{\rm dir}\parenthesis{\BsJpsiPhi}_f &= \frac{ -2\epsilon a_f \sin\theta_f \sin\gamma} {1 + 2\epsilon a_f \cos\theta_f \cos\gamma + \epsilon^2 a_f^{2}} \\
  \Acp{\rm mix}\parenthesis{\BsJpsiPhi}_f &= \eta_f \brackets{\frac{\sin\phis - 2\epsilon\alpha_f \cos\theta_f\sin(\phis+\gamma) + \epsilon^2\alpha_f^2 \sin(\phis+2\gamma)}
                                                                 {1 + 2\epsilon a_f \cos\theta_f \cos\gamma + \epsilon^2a_f^{2}} }
  \label{bsjpsiphi_peng_acp_obs}
\end{align}

Given the experimentally measured values of the above CP assymetries {\color{red} ref blah}, a $\chi^2$ fit could be used,
see ref {\color{red} krystof} to determine the penguin parameters alpha and beta. Subsequently and according to rob
the penguin parameters $\alpha_a$ and $\theta_f$ can be related to $\Delta\phiS{SM,peng}$ as shown in \equref{tandelta}.

\begin{equation}
\tan(\Delta{\phis}_f) = \frac{2\epsilon a_f \cos\theta_f \sin\gamma+\epsilon^2 a^2_f \sin2\gamma}{1+2\epsilon a_f \cos\theta_f \cos\gamma+ \epsilon^2 a^2_f \cos2\gamma}.
\label{tandelta}
\end{equation}

However, the preciosion on $\Delta{\phis}_f$ increases by releating it to other chanells, described in the subsequent section.
Thus the final $\chi^2$ fit is postponed untill after the \BsJpsiKst and \BsJpsiRho chanells are introduced in \secref{penguin_more_chanells}

\subsection{The \BsJpsiKst and \BsJpsiRho chanells}
\label{penguin_more_chanells}

For this relation it is assumed that, as far as the strong interaction is concerned, quark flavor is irrelevant,
since the strong interaction couples to color and not to flavor. This assumption is based on the so called $SU(3)_F$ or $U$-sppin symmetry,
see {\color{red} Krystof thesis section 3.4 ana elsewere.}.

In order to increase sensitivuty or for a flavour sepcific finals atetes the Acp mix is nto there thus the H observable is constructed
according to

\subsubsection{The \BsJpsiKst chanell}

\begin{equation}
  \Acp{\rm dir}\parenthesis{\BsJpsiKst}_f = \frac{ 2a'_f \sin\theta'_f \sin\gamma} {1 - 2a'_f \cos\theta'_f \cos\gamma +  {a'_f}^{2}}
  \label{bsjpsikst_peng_acp_obs}
\end{equation}

Note the absence of the suppresion factor $\epsilon$

Additional information from \BsJpsiKst can be imported via
\begin{equation}
  H_f \equiv \frac{1}{\epsilon} \left|\frac{\mathcal{A}'_f}{\mathcal{A}_f}\right|^2
  \frac{\text{PhSp}\left(\BsJpsiPhi\right)}{\text{PhSp}(\BsJpsiKst)}
  \frac{\BR{\BsJpsiKst}_{\text{theo}}}{\BR{\BsJpsiPhi}_{\text{theo}}}
  \frac{f_i}{f'_i}\:,  \\
\label{bsjpsikst_Hobs_def}
\end{equation}
{\color{red} SITE PAPER HERE. avoid defining phase space functions, and the isue with the teoretical vs expermental branching rations.}

% \begin{equation}
% \text{PhSp}(B\to V_1V_2) \equiv \frac{1}{16\pi m_B}\Phi\left(\frac{m_{V_1}}{m_B}, \frac{m_{V_2}}{m_B}\right)\:,
% \label{bsjpsikst_Hobs_def_params}
% \end{equation}

\begin{equation}
  H_f = \frac{1-2a_i \cos(\theta_i) \cos\gamma+a_i^{2}}{1+2\epsilon a'_i \cos\theta'_i\cos\gamma +\epsilon^2 a_i^{\prime 2}}
\label{bsjpsikst_Hobs_param}
\end{equation}

\begin{align}
\left|\frac{\mathcal{A}'_0(\BsJpsiPhi)}{\mathcal{A}_0(\BsJpsiKst)}\right| & = 1.23 \pm 0.16\:,\label{Eq:AmpRat_JpsiKstar_long}\\
%%%
\left|\frac{\mathcal{A}'_{\parallel}(\BsJpsiPhi)}{\mathcal{A}_{\parallel}(\BsJpsiKst)}\right| & = 1.28 \pm 0.15\:,\\
%%%
\left|\frac{\mathcal{A}'_{\perp}(\BsJpsiPhi)}{\mathcal{A}_{\perp}(\BsJpsiKst)}\right| & = 1.20 \pm 0.12\:,\label{Eq:AmpRat_JpsiKstar_perp}
\end{align}


\subsubsection{The \BsJpsiRho chanell}

This is addes in a similar way as to \BsJpsiKst but including Acp mix.
Ref reff krystof and paper

Plus several hypothesis {\color{red} study a bit from ana ana paper}
% \begin{align}
%   \Acp{\rm dir}\parenthesis{\BsJpsiRho}_f &= \frac{ 2a''_f \sin\theta''_f \sin\gamma} {1 - 2a''_f \cos\theta''_f \cos\gamma +  {a''_f}^{2}} \\
%   \Acp{\rm mix}\parenthesis{\BsJpsiPhi}_f &= \eta_f \brackets{\frac{\sin\phis - 2\epsilon\alpha_f \cos\theta_f\sin(\phis+\gamma) + \epsilon^2\alpha_f^2 \sin(\phis+2\gamma)}
%                                                                  {1 + 2\epsilon a_f \cos\theta_f \cos\gamma + \epsilon^2a_f^{2}} }
%   \label{bsjpsirho_peng_acp_obs}
% \end{align}



\subsubsection{Fitting for the $\alpha_f$ and $\theta_f$}

By invoking \grpsuthree symmetry, as shown in \equref{su3_apply}, the acpmix and acp dir and ah obserbales from the chanells \BsJpsiPhi \BsJpsiRho \BsJpsiKst
are used to define a $\chi^2$, see ref. Minimizing the $\chi^2$ yields the penguin paramtgers $\alpha_f$ $\theta_f$ whic are
then translated, using \equref{tandelta}, to the final estimate for the penguin phase shift on \phis.

\begin{equation}
a_i = a'_i\:,\qquad \theta_i = \theta'_i\:,
\label{su3_apply}
\end{equation}

and similartly for \BsJpsiRho

The following inputs have been used for the fit.

\subsubsection{\grpsuthree breaking}
\label{su3_breaking}

The fact that the \grpsuthree syummetry is a broken symmetry is taken into account in the fit by re parametrizing
$\alpha_f$ and $\theta_f$ as shown in \equref{su3_breaking}.

\begin{equation}
a_i \to \xi \; \alpha_i \quad \theta_i \to \delta + \theta'_i
\label{su3_breaking}
\end{equation}

As there are no dedicated studies for \grpsuthree breaking. It is assumed that $\xi=1$ and $\delta=0$ but include uncertainties
on these paorameters as gaussian constrains in the $\chi^2$ fit. The dependance of $\Delta\phiS{SM,peng}$ on the $\xi$ and $\delta$
uncertainties was found to be irrelavant, {\color{red} see jpsiKst paper}, due to the particular structure of \equref{tanddelta}.
