Following ~\cite{Faller:2008gt} the penguin parameters $\alpha_f$ and $\beta_f$ of \equref{bsjpsiphi_amp_param}
can be related to the CP asymmetries of \equref{cp_asym_lambda} as shown in \equref{bsjpsiphi_peng_acp_obs}.

\begin{subequations}
  \label{bsjpsiphi_peng_acp_obs}
\begin{align}
  \Acp{\rm dir}\parenthesis{\Bq\to (f)^k} &= \frac{ \akPeng \sin\thkPeng \sin\gamma} {1 - 2 a_k \cos\thkPeng \cos\gamma + a_k^{2}}
  \label{bsjpsiphi_peng_acp_dir} \\
  \Acp{\rm mix}\parenthesis{\Bq\to (f)^k} &= \nonumber \\
   = \eta_k & \brackets{\frac{\sin\phiq - 2\akPeng \cos\thkPeng\sin(\phiq+\gamma) + \akPeng^2 \sin(\phiq+2\gamma)}
                                                                 {1 - 2 a_k \cos\thkPeng \cos\gamma + a_k^{2}} },
  \label{bsjpsiphi_peng_acp_mix}
\end{align}
\end{subequations}

\noindent where $\phiq$ represents the Standard Model prediction for either the weak phase \phis of \equref{phis_theo},
or the equivalent weak phase \phid in the \Bd meson~\cite{PDG}. Equations \ref{bsjpsiphi_peng_acp_obs} form a $2x2$ system in
terms of $(\akPeng,\thkPeng)$ that can in principle be solved for. After an estimation of $(\akPeng,\thkPeng)$ is available the
penguin shift to \phis could be determined from \equref{tandelta}.

\begin{equation}
\centering
\tan(\Delta\phis^k) = \frac{ 2\akPeng\epsilon\cos\thkPeng \sin\gamma + \epsilon^2\akPeng^2 \sin2\gamma}
                             {1 + 2\epsilon\akPeng \cos\thkPeng \cos\gamma + \epsilon^2\akPeng^2 \cos2\gamma},
\label{tandelta}
\end{equation}

\noindent where $\Delta\phis^k$ denotes the polarization dependent version of $\Delta\phiS{SM, peng}$ of \equref{phis_sm_peng}.
Note that in the case where \equref{bsjpsiphi_peng_acp_obs} refer to the \BsJpsiPhi channel, the penguin suppression factor
$\epsilon$ needs to be applied. This implies that the transformation $(\akPeng\to-\epsilon\akPeng)$ has to take place.

As already mentioned in the current chapter so far, the aim is to increase the precision on $\Delta{\phis}$
by relating similar to \BsJpsiPhi channels in the penguin estimation on grounds of the $\grpsuthree$ quark symmetry.
Thus, the estimation of $(\akPeng,\thkPeng)$ will come from a \chisq fit to the measured values of the observables
\equref{bsjpsiphi_peng_acp_obs} where the channels \BsJpsiKst and \BdJpsiRho enter the fit.
The details of this fit are given in \secref{penguin_chi2_fit}.

\subsubsection{Information from Branching ratios}
Additional information from branching ratios can be exploited via the $H$ observable~\cite{Fleischer:1999zi}, shown in \equref{hobs_def}

\begin{equation}
\centering
  \Hobs{k} \equiv   \frac{1}{\epsilon}
            \modulo{\frac{\formFctr{k}}{\formFctr{k}'}}^2
                    \frac{\PhSp{\BsJpsiPhi}} {\PhSp{\Bq\to f}}
                    \frac{\tau_{\Bs}}{\tau_{\Bq}}
                    \frac{\BRof{\Bq\to f}_{\rm theo}}{\BRof{\BsJpsiPhi}_{\rm theo}}
                    \frac{\fP{k}'}{{\fP{k}}},
\label{hobs_def}
\end{equation}

\noindent where the superscript prime$({}^\prime)$ labels quantities related to $\Bq\to f$. The subscript "theo" is there
to distinguish between definitions of branching fraction, more details later in the current section.
The polarization fractions $\fP{k}$ are identical to the ones defined in \equref{amps_param}.
Whereas the average decay time of the \Bs and \Bq are denoted by $\tau_{\Bs}$ and $\tau_{\Bq}$
respectively. The phase space factor $\PhSp{X}$ is defined in \equref{phase_space_eq_def}.

\begin{equation}
\centering
   \PhSp{\BJpsiX}  = \brackets{ \mass{\Bq} \PhSpPhi{ \nicefrac{\mass{\jpsi}}{\mass{\Bq}}, \nicefrac{\mass{X}}{\mass{\Bq}}  } }^3,
\label{phase_space_eq_def}
\end{equation}

\noindent where,

\begin{equation}
\centering
   \PhSpPhi{x,y} = \sqrt{ (1-(x+y)^2)(1-(x-y)^2) },
\label{phase_space_phi_eq_def}
\end{equation}

\noindent is the standard two body decay phase space function.

The $\Hobs{k}$ observable of \equref{hobs_def} is related~\cite{Fleischer:1999zi}
to the penguin parameters $(\akPeng,\thkPeng)$ as shown in \equref{Hobs_peng_param}.

\begin{equation}
\centering
  \Hobs{k} = \frac{1-2\akPeng\cos\thkPeng\cos\gamma + \akPeng^2}{1+2\epsilon\akPengPrime\cos\thkPeng\cos\gamma + \epsilon^2\akPeng^2}
\label{Hobs_peng_param}
\end{equation}

\noindent the above observable is useful for decay channels, such as \BsJpsiKst, where the \Acp{\rm mix} observable vanishes,
thus providing the second equation necessary to solve for $(\akPeng,\thkPeng)$. Such channels
where CP-Violation in the interference is not active, are called {\it flavor specific}, implying
that either \Bs or \Bsb can decay to this final state but not both.
Despite providing the second equation the $\Hobs{}$ relies in external input, specifically the hadronic quantities $\formFctr{k}$.
The last come from theoretical calculations and introduce additional uncertainty to the extraction of the penguin parameters
, as explained in \secref{had_pars_suthree}. For this reason the $\Hobs{}$ is not always preferred.

Lastly, note that the $\Hobs{k}$ observable is constructed in terms of the theoretical branching fractions
defined at zero decay time, which differ from the experimentally measured time-integrated branching fractions~\cite{DeBruyn:2012wj}
due to the non-zero decay-width difference, $\Delta\Gamma_s$, of the \Bs meson~\cite{hfag-2014}. The necessary conversion factor
is shown in \equref{br_conversion_formula}.

\begin{equation}
  \centering
  \frac{\BRof{\B\to f}_{\rm theo}}{\BRof{\B\to f}_{\rm exp}} = \frac{1-y_s^2}{1-y_s\cos\parenthesis{\phiS{\rm SM, tree}}},
  \label{br_conversion_formula}
\end{equation}

\noindent with $y_s = \Delta\Gamma_s / \Gamma_s$. The parameter $\Gamma_s$ is the average decay width of the two
mass eigenstates of the $\BBbarSyst$ introduced in \secref{the_bbar_system}. The exact numbers for the CP-even,
CP-odd, and flavor specific final states are reported in Section 9.1 of~\cite{bsjpsikst-paper}.

\subsubsection{Invoking \grpsuthree Symmetry}
The $\grpsuthree_{F}$ symmetry will be used to relate channels like \BsJpsiKst and \BsJpsiRho to \BsJpsiPhi.
In practice this relation is implemented by assuming that the penguin parameters $(\akPeng,\thkPeng)$
are the same between all the above channels, as shown in \equref{su3_apply}. In addition $\grpsuthree_{F}$
is also exploited in hadronic parameter ratios between different channels, as explained in \secref{penguin_chi2_fit}.
