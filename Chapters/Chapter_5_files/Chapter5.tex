% Chapter 5

\chapter{Controlling Penguins in \phis}
\label{Penguins}

As explained in \secref{TheBsJpsiKstDecay} higher order effects become increasingly relevant in the pursuit
of New Physics from flavor physics experiments. The issue of estimating the contributions from
penguin topologies to the total \BsJpsiPhi amplitude is addressed in the current chapter.
The adopted formalism focuses on involving as many channels similar to \BsJpsiPhi as possible.
This way the sensitivity to the parameters that quantify penguin topologies increases.
From \secref{jpsiphi_amp_struct} to \secref{penguin_more_chanells} the fundamentals of
the necessary formalism are introduced. Details of the penguin parameter estimation are
explained in \secref{penguin_more_chanells}. Results are given in \secref{penguin_results}.
Note that the strategy followed in this chapter along with the quoted equations are
suggested by \cite{Fleischer:1999zi,Faller:2008gt,DeBruyn:2014oga}.

\section{Amplitude Structure of \BsJpsiPhi}
\label{jpsiphi_amp_struct}
The \BsJpsiPhi decay amplitude can take place via four different topologies.
Two of them were already introduced in \figref{bs2jpsiphi} and \figref{bs2jpsiphi_peng};
the {\it color suppressed }$(C)$ and {\it penguin} $(P)$ topologies respectively.
According to \cite{DeBruyn-thesis,DeBruyn:2014oga} the other two types, {\it penguin-annihilation}
and {\it exchange} illustrated in \figref{peng_ana_exchange}, can be neglected given the current experimental precision.
It is also assumed that the $\phi$ meson is a pure ($\squark\squarkbar$) state, based on \cite{Faller:2008gt}.
Given these assumptions the \BsJpsiPhi amplitude is decomposed in \equref{bsjpsiphi_amp},
taking into account the relevant CKM elements involved in each topology.

\begin{equation}
\mathcal{A} \parenthesis{\BsJpsiPhiPolState{k}} = \Vus\Vub^*P_{\uquark}^k + \Vcs\Vcb^*\brackets{C^k +P_{\cquark}^k} + \Vts\Vtb^*P_{\tquark}^k,
 \label{bsjpsiphi_amp}
\end{equation}

\noindent where the subscripts in the penguin topologies, $P$, denote the flavor of the quark present inside
the loop of \figref{bs2jpsiphi_peng}. The superscript $k$ is identical to the one defined in \equref{amps_param}
of \chapref{Data_Analysis} and denotes the \BsJpsiPhi amplitude polarization.
The above expression has to be parameterized in such a way that it is possible to probe the penguin contributions to
the \BsJpsiPhi decay amplitude. Given the unitarity nature of the CKM matrix and using the Wolfenstein parametrization
of \equref{CKMwolfenstein}, the decay amplitude of \equref{bsjpsiphi_amp} can be rewritten as shown in the following equation:

\begin{equation}
  \mathcal{A} \parenthesis{\BsJpsiPhiPolState{k}} = \eta_k  \parenthesis{1-\frac{\lambda^2}{2}} \formFctr{k} \brackets{ 1 - \epsilon \akPeng e^{i\thkPeng} e^{i\gamma} },
 \label{bsjpsiphi_amp_param}
\end{equation}

\noindent where the following definitions are used:

\begin{equation}
  \formFctr{k} \equiv \VcbMag \brackets{C^k + P_\cquark^k - P_\tquark^k}, \;\;\;\; \akPeng e^{i\thkPeng} \equiv R_\bquark \brackets{ \frac{P_\cquark^k - P_\tquark^k}{C^k + P_\cquark^k - P_\tquark^k} },
  \label{bsjpsiphi_amp_param_defs}
\end{equation}

\noindent with,

\begin{equation}
  \epsilon = \frac{\lambda^2}{1-\lambda^2} \;\;\text{and} \;\;  R_\bquark = \parenthesis{1-\frac{\lambda^2}{2}} \frac{1}{\lambda} \modulo{\frac{\Vub}{\Vcb}}.
  \label{bsjpsiphi_amp_param_defs_II}
\end{equation}

\noindent The quantities $\akPeng$ and $\thkPeng$ in \equref{bsjpsiphi_amp_param} parameterize the penguin
contribution to the overall \BsJpsiPhi decay amplitude. The hadronic parameters $\formFctr{k}$ are combinations
of $C$ and $P_\quark$ decay amplitudes and are discussed in more detail, in \secref{had_pars_suthree}.
The CKM angle $\gamma$ is introduced via the $\Vub$ matrix element.
The parameter $\lambda$ here is the one of the CKM matrix parametrization of \secref{CKMwolfenstein}.
The eigenvalue, $\eta_k$, of the final state $f$ is identified with the one in \equref{lambda_cpv}.
Note the suppression factor $\epsilon \sim 0.05$ in \equref{bsjpsiphi_amp_param} which implies that penguin contributions
to the \BsJpsiPhi amplitude are smaller by approximately two orders of magnitude with respect to tree level contributions.

\begin{figure}[t]
  \centering
  \begin{subfigure}{0.5\textwidth}
    \raggedright
    \includegraphics[width=1.02\textwidth, trim=1.3cm 0cm 0.5cm 0.5cm, clip=true]{Figures/Chapter5/B2JpsiX_PenguinAnnihilation}
    \caption{}
    \label{peng_exchange}
  \end{subfigure}%
  \hfill%
  \begin{subfigure}{0.5\textwidth}
    \raggedleft
    \includegraphics[width=1.02\textwidth,trim=1cm 0cm 0.5cm 0.5cm, clip=true]{Figures/Chapter5/B2JpsiX_Exchange}
    \caption{}
    \label{peng_anihilation}
  \end{subfigure}
  \caption{Higher order, penguin annihilation (A) and exchange (B), \BJpsiX decay topologies. Figures from \cite{DeBruyn-thesis}.}
  \label{peng_ana_exchange}
\end{figure}


\section{Hadronic Factors and \grpsuthree Symmetry}
\label{had_pars_suthree}
Hadronic factors $\formFctr{k}$ are introduced by the \Hobs{k} observable of \equref{hobs_def}.
Despite the fact that these factors can be calculated theoretically, as for example
in \cite{DeBruyn-thesis}. The calculations involve the computation of \BJpsiX decay amplitudes
which is difficult. The difficulty comes from the {\it hadronization} process, which is the
process of quarks forming bound states, hadrons, such as the $\jpsi$ and $\Pphi$.
During this process quarks might interact with each other via the strong interaction.
This class of strong interactions, also referred to as {\it long distance QCD effects}
cannot be computed within the context of {\it perturbation theory}.
The computation of the \BJpsiX decay amplitude becomes easier under the so called
{\it factorization} assumption \cite{HAAN1970448,Wirbel1985,CABIBBO1978418,FAKIROV1978315}.
However, according to \cite{DeBruyn-thesis},
the power of this approach is also limited in the case of \BJpsiX decays and corrections
are required to account for {\it non-factorizable} effects during the process of hadronization.
Hence, the hadronic factors $\formFctr{k}$ computed from theory suffer from large uncertainties.

The strategy described in the current chapter avoids, as much as possible, the use of hadronic
factors $\formFctr{k}$ from theory. However, the \Hobs{k} observable, does require the explicit
calculation of ratios of hadronic factors. Thus on the one hand
introducing the previous observable provides additional information for estimating the penguin
parameters, on the other hand it introduces further uncertainty from non-factorizable QCD effects.
As will be explained in \secref{penguin_more_chanells} introducing the previous the \Hobs{k}
observable can also be avoided.

An important feature that the current chapter makes use of is the so called
$\grpsuthree_{\rm F}$ {\it flavor symmetry} \cite{GELLMANN1964214,NEEMAN1961222}.
In a naive interpretation, $\grpsuthree_{\rm F}$ flavor symmetry implies that the three lightest quarks (\uquark,\dquark,\squark)
are indistinguishable by the strong interaction. The last symmetry is broken by the fact that the above-mentioned
quarks have indeed different mass. However, these differences are small with respect
to the hadronization scale \lqcd of the strong interaction, and as a result $\grpsuthree_{\rm F}$
remains a useful approximate symmetry. Particularly, in the case of $\Bq$ meson decays, hadronic
parameters from different decay modes can be related to one another and thus their calculation can be avoided
in the first place. The penguin parameter estimation of the current chapter is an example of $\grpsuthree_{\rm F}$.

According to \cite{Nagashima:2007qn,Gronau:2013mda} the estimated amount of $\grpsuthree_{\rm F}$ breaking is about $20\%$.
The impact of the symmetry breaking has implications in the penguin parameter estimation
of the current chapter. These implications are addressed in \secref{su3_breaking}.
Other $\grpsuthree_{\rm F}$ breaking estimates can be found in \cite{Charles:2015gya,PDG}.


\section{Formalism}
\label{penguin_formalism}
Following {\color{red} 0810.4248v1} the penguin parameters $\alpha_f$ and $\beta_f$ of \equref{bsjpsiphi_amp_param}
can be related to the CP asymmetries of \equref{cp_asym_lambda} as shown in \equref{bsjpsiphi_peng_acp_obs}.

\begin{subequations}
  \label{bsjpsiphi_peng_acp_obs}
\begin{align}
  \Acp{\rm dir}\parenthesis{\Bq\to (f)^k} &= \frac{ \akPeng \sin\thkPeng \sin\gamma} {1 - 2 a_k \cos\thkPeng \cos\gamma + a_k^{2}}
  \label{bsjpsiphi_peng_acp_dir} \\
  \Acp{\rm mix}\parenthesis{\Bq\to (f)^k} &= \nonumber \\
   = \eta_k & \brackets{\frac{\sin\phiq - 2\akPeng \cos\thkPeng\sin(\phiq+\gamma) + \akPeng^2 \sin(\phiq+2\gamma)}
                                                                 {1 - 2 a_k \cos\thkPeng \cos\gamma + a_k^{2}} },
  \label{bsjpsiphi_peng_acp_mix}
\end{align}
\end{subequations}

\noindent where $\phiq$ represents the Standard Model prediction for either the weak phase \phis of \equref{phis_theo},
or the equivalent weak phase \phid of the \Bd meson defined in \equref{phid_theo}.
Equations \ref{bsjpsiphi_peng_acp_obs} form a $2x2$ system in terms of $(\akPeng,\thkPeng)$ that can in principle be solved
for. Note that in the case where \eqref{bsjpsiphi_peng_acp_obs} refer to \BsJpsiPhi, the penguin supression factor
needs to be appied. This implies that following transformation has to take place $(\akPeng\to-\epsilon\akPeng)$.
After an estimation of $(\akPeng,\thkPeng)$ is available the penguin shift to \phis could be determined from \equref{tandelta}.

\begin{equation}
\centering
\tan(\Delta\phis^k) = \frac{ 2\akPeng\epsilon\cos\thkPeng \sin\gamma + \epsilon^2\akPeng^2 \sin2\gamma}
                             {1 + 2\epsilon\akPeng \cos\thkPeng \cos\gamma + \epsilon^2\akPeng^2 \cos2\gamma},
\label{tandelta}
\end{equation}

\noindent where $\Delta\phis^k$ denotes the polarisation
dependant version of $\Delta\phiS{SM, peng}$ of \equref{phis_sm_peng}.

However, as it was metnioned in the begining of the currrent chapter, the goal is to increase the precision on $\Delta{\phis}$
by relating similar to \BsJpsiPhi chanells in the penguin estimation on grounds of the $\grpsuthree$ quark symmetry.
Thus, the measured values of the observables \equref{bsjpsiphi_peng_acp_obs} from the chanells \BsJpsiKst and \BdJpsiRho
are combined in a $\chi^2$ fit to estimate $(\akPeng,\thkPeng)$. The details of this fit are given in \secref{penguin_chi2_fit}.

\subsubsection{Invoking \grpsuthree Symmetry}
The $\grpsuthree_{F}$ symmetry is used to relate chanells like \BsJpsiKst and \BsJpsiRho to \BsJpsiPhi.
In practice this relation is implemented by assuming that the penguin parameters $(\akPeng,\thkPeng)$
are the same between all the above chanells, as shown in \equref{invoke_su3}. In additon $\grpsuthree_{F}$
is also exploited in hadronic parameter ratios between different cahnells, as explained in \secref{penguin_chi2_fit}.

\subsubsection{Information from Branching ratios}
Addional information from branching ratios can be exploited via the $H$ observable, shown in \equref{hobs_def}

\begin{equation}
\centering
  \Hobs{k} \equiv   \frac{1}{\epsilon}
            \modulo{\frac{\formFctr{k}}{\formFctr{k}'}}^2
                    \frac{\PhSp{\BsJpsiPhi}} {\PhSp{\Bq\to f}}
                    \frac{\tau_{\Bs}}{\tau_{\Bq}}
                    \frac{\BRof{\Bq\to f}_{\rm theo}}{\BRof{\BsJpsiPhi}_{\rm theo}}
                    \frac{\fP{k}'}{{\fP{k}}},
\label{hobs_def}
\end{equation}

\noindent where the superscript prime$({}^\prime)$ labels quantities related to $\Bq\to f$. The subscript "theo" is there to distinguish
between the concept of branching fraction used in theoretical calculations and the ones that are measured experimentally,
see \secref{penguin_more_chanells} for more details. The polarization fractions $\fP{k}$ are identical to the ones
efined in \equref{amps_param}. Whereas the average decay time of the \Bs and \Bq are denoted by $\tau_{\Bs}$ and $\tau_{\Bq}$
respectivelly. The phase space factor $\PhSp{X}$ is defined in \equref{phase_space_eq_def}.

\begin{equation}
\centering
   \PhSp{\BJpsiX}  = \brackets{ \mass{\Bq} \PhSpPhi{ \nicefrac{\mass{\jpsi}}{\mass{\Bq}}, \nicefrac{\mass{X}}{\mass{\Bq}}  } }^3,
\label{phase_space_eq_def}
\end{equation}

\noindent where,

\begin{equation}
\centering
   \PhSpPhi{x,y} = \sqrt{ (1-(x+y)^2)(1-(x-y)^2) },
\label{phase_space_phi_eq_def}
\end{equation}

\noindent is the standard two body decay phase space function. Lastly, the $H$ observable is releated to the penguin parameters
$(\akPeng,\thkPeng)$ as shown in \equref{Hobs_peng_param}.

\begin{equation}
\centering
  \Hobs{k} = \frac{1-2\akPeng\cos\thkPeng\cos\gamma + \akPeng^2}{1+2\epsilon\akPengPrime\cos\thkPeng\cos\gamma + \epsilon^2\akPeng^2}
\label{Hobs_peng_param}
\end{equation}

The observable is usefull for final states where the \Acp{\rm mix} observable vanishes. Such final states are called {\it flavour specific}
where CP-Violation in the interference is not active, implying that either \Bs or \Bsb can decay to this final state but not both.
Despite providing the second equation in flavour specific states the $\Hobs{}$ relies in external input, particularly the hadronic quantities $\formFctr{k}$.
The last come from theoretical calculations and introduce additional uncertainty to the exrtaction of the penguin paramters.
For this reason the $\Hobs{}$ is not always prefared.

Lastly note that the $\Hobs{}$ observable are constructed in terms of the theoretical branching fractions
defined at zero decay time, which difer from the measured time-integrated branching fractions~\cite{DeBruyn:2012wj}
due to the non-zero decay-width diference of the \Bs meson~\cite{hfag-2014}. The calculation of the convertion factor
is done as in~\cite{bsjpsikst-paper}. The results denpend on the CP eigenvalue of the final state and shown in
\tabref{br_conversions}

\begin{table}[!h]
  \center
  \begin{tabular}{c c }
    \hline
                        & $\BRof{\B\to f}_{\rm theo} / \BRof{\B\to f}_{\rm exp} $ \\
    \hline
      CP-even          &  $1.0608 \pm 0.0045$ \\
      CP-odd           &  $0.9392 \pm 0.0045$ \\
      flavor specific  &  $0.9963 \pm 0.0006$ \\

    \hline
  \end{tabular}
  \caption{\small Conversion facrors.}
  \label{br_conversions}
\end{table}

\noindent The results for the CP-even and CP-odd are $1.0608 \rm 0.0045$ and  $0.9392 \rm 0.0045$.
While for the flavor specific final states is  $0.9963 \rm 0.0006$
Note that kst is a flavlour specific final state


\section{Estimating Penguin Parameters}
\label{penguin_more_chanells}
In the recent \lhcb publication \cite{phis-3fb-paper} of the \phis measurement one can find the best fit value for the $\lambda_{\jpsi\phi}$
parameter, mentioned in \equref{lambda_cpv}. This parameter along with the quoted \phis, shown in \equref{phis_lambda_result},
can be used to estimate the penguin parameters $(\aPeng{},\thPeng{})$.

\begin{subequations}
  \label{phis_lambda_result}
  \begin{align}
    \centering
    \phis^{\jpsi\phi}     &=  -0.058 \pm 0.049(\text{stat})  \;\; \text{rad},\\
    \lambda^{\jpsi\phi}   &=  +0.964 \pm 0.019(\text{stat}).
  \end{align}
\end{subequations}

\noindent Based on \equref{cp_asym_lambda} and according to the formalism in \secref{penguin_formalism}, the $\DeltaPhis{}$ estimation
due to penguin topologies based on \BsJpsiPhi decays comes with an uncertainty of about $0.05$ rad, see Eq. 5.125 of \cite{DeBruyn-thesis},
which is not precise enough given the uncertainty of $\phis^{\jpsi\phi}$. This is related to the suppression factor $\epsilon$, mentioned in
\secref{jpsiphi_amp_struct}, that the penguin topologies receive in the \BsJpsiPhi decay.

In order to increase the precision on the penguin parameters the \grpsuthree flavor symmetry is
invoked such that more channels similar to \BsJpsiPhi are involved in the computation of $(\akPeng-\thkPeng)$.
Note that the information from these channels entering through the observables in \equref{bsjpsiphi_peng_acp_obs} and \equref{hobs_def}
has to be polarization dependent. This is due to the fact that new physics dynamics might enter in a different way in each polarization,
as mentioned in \cite{DeBruyn-thesis}.
This section introduces the additional channels, \BsJpsiKst and \BdJpsiRho and provides
details of the fitting strategy to estimate the penguin parameters.

\subsection{The \BsJpsiKst Channel}
\label{bsjpsikst_chanell}

The \BsJpsiKst channel is a flavor specific decay with the same topology as \BsJpsiPhi.
The corresponding Feynman diagrams are illustrated in the following figure:

\begin{figure}[h]
  \centering
  \scalebox{0.9}{\sffamily \input{Figures/Chapter5/tree_penguin_jpsikst}}
  \caption{Leading order diagrams of the decay \BsJpsiKst. Left: Color-suppressed tree topology. Right: Penguin topology.}
  \label{bs2jpsikst}
\end{figure}

\noindent The \BsJpsiKst amplitude is parameterized following the same concept as in the case of \BsJpsiPhi resulting in:

\begin{equation}
  \mathcal{A} \parenthesis{\BsJpsiKstPolState{k}} = -\lambda \formFctr{k}' \brackets{ 1 - \akPeng' e^{i\thkPeng'} e^{i\gamma} },
  \label{bsjpsikst_amp}
\end{equation}

\noindent where primed ${}^\prime$ quantities from here are associated with the \BsJpsiKst decay only.
Note the absence of the suppression factor $\epsilon$ in \equref{bsjpsikst_amp}, which implies that the penguin diagram
contributes as much as the color suppressed tree diagram does to the total amplitude, contrary to the case of the \BsJpsiPhi decay.

As mentioned in \secref{penguin_formalism} the \BsJpsiKst channel provides access to \Acp{\rm dir} only.
Thus, additional information is required, via the $\Hobs{k}'$ observable, in order to probe $(\akPeng,\thkPeng)$.
Both observables are based on measurements that took place in \chapref{Data_Analysis} of the current thesis.
The first observable is reported in \tabref{bestFitResult}, with the associated systematic quoted in \tabref{systematics_acp}.
Whereas the $\Hobs{k}'$ observable is constructed from the \BRof{\BsJpsiKst} reported in \equref{Br_total} and shown in
the following equations:

\begin{subequations}
  \label{hobs_jpsikst}
  \begin{alignat}{2}
  \Hobs{0}'         & = 0.99 \pm 0.07\:\text{(sta)} \; 0.06\:\text{(sys)} \; 0.27\:(\text{had}) && = 0.99 \pm 0.28\:, \label{hobs_jpsikst_long}\\
  \Hobs{\parallel}' & = 0.91 \pm 0.14\:\text{(sta)} \; 0.08\:\text{(sys)} \; 0.21\:(\text{had}) && = 0.91 \pm 0.27\:, \label{hobs_jpsikst_para} \\
  \Hobs{\perp}'     & = 1.47 \pm 0.14\:\text{(sta)} \; 0.11\:\text{(sys)} \; 0.28\:(\text{had}) && = 1.47 \pm 0.33\:. \label{hobs_jpsikst_perp}
  \end{alignat}
\end{subequations}

\noindent Where the last uncertainty is due to the hadronization factor $|\formFctr{k}/\formFctr{k}'|$ necessary for the construction of $\Hobs{k}'$.
The hadronization factors are calculated theoretically based on the method suggested in \cite{eff-hamiltonian-bs-syst}.
The exact numbers used can be found in \cite{DeBruyn-thesis}. Note that the uncertainty on $\Hobs{k}'$ is dominated by these factors.

% \begin{align}
% \left|\frac{\mathcal{A}'_0(\BsJpsiPhi)}{\mathcal{A}_0(\BsJpsiKst)}\right| & = 1.23 \pm 0.16\:,\label{Eq:AmpRat_JpsiKstar_long}\\
% %%%
% \left|\frac{\mathcal{A}'_{\parallel}(\BsJpsiPhi)}{\mathcal{A}_{\parallel}(\BsJpsiKst)}\right| & = 1.28 \pm 0.15\:,\\
% %%%
% \left|\frac{\mathcal{A}'_{\perp}(\BsJpsiPhi)}{\mathcal{A}_{\perp}(\BsJpsiKst)}\right| & = 1.20 \pm 0.12\:,\label{Eq:AmpRat_JpsiKstar_perp}
% \end{align}

Prior to any combination the penguin parameters have been estimated based on
the \BsJpsiKst channel only \cite{bsjpsikst-paper}, resulting in:

\begin{subequations}
  \label{delta_phis_jpsikst}
  \begin{align}
    \DeltaPhisJpsiPhi{0}         & = +0.001^{+0.100}_{-0.033} \:\text{rad},\\
    \DeltaPhisJpsiPhi{\parallel} & = +0.031^{+0.059}_{-0.052} \:\text{rad},\\
    \DeltaPhisJpsiPhi{\perp}     & = -0.046^{+0.022}_{-0.028} \:\text{rad},
  \end{align}
\end{subequations}

\noindent where most of the uncertainty is statistical in nature. The $\chisq$ fit performed here is similar
to the one described in \secref{penguin_chi2_fit}, hence details of the fitting strategy are postponed for later.

\subsection{The \BdJpsiRho Channel}
\label{bsjpsirho_chanell}

The topology of the \BdJpsiRho decay is shown in \figref{bs2jpsirho_diagram}.
The amplitude structure is identical to that of the \BsJpsiKst mode:

\begin{equation}
  \mathcal{A} \parenthesis{\BdJpsiRhoPolState{k}} = -\lambda \formFctr{k}'' \brackets{ 1 - \akPeng'' e^{i\thkPeng''} e^{i\gamma} },
  \label{bsjpsirho_amp}
\end{equation}

\noindent where the double primed symbol ${}^{\prime\prime}$ from here on labels parameters related to the \BdJpsiRho decay.
Note again the absence of the suppression factor $\epsilon$.

\begin{figure}[t]
  \centering
  \scalebox{0.9}{\sffamily %%BoundingBox: -5 0 121 170
%%HiResBoundingBox: -5 0 120.57008 169.36447

\begin{fmffile}{Figures/Chapter5/tree_jpsirho}
  \fmfframe(17,-25)(31,-25){
    \begin{fmfgraph*}(115,170)
      \fmfstraight
      \fmfleft{i0,i1,i2,i3,i4,i5}
      \fmfright{o0,o1,o2,o3,o4,o5}
      \fmf{fermion,tension=3.5,label.side=left,label=$\bquark$}{v2,i3}
      \fmf{fermion,label=$\cquark$,label.side=left}{o4,v2}
      \fmf{fermion,label=$\cquark$,label.side=left}{v3,o3}
      \fmf{fermion,label=$\dquark$,label.side=left,tension=2}{o2,v3}
      \fmf{boson,tension=2.4,label=\Wp,label.side=right,right=0.3}{v2,v3}
      \fmffreeze
      \fmf{phantom,tension=0.3}{v2,v1,v3}
      \fmf{fermion,tension=0.5,label=$\dquark$,label.side=left}{v1,o1}
      \fmf{fermion,tension=0.5,label.side=left,label=$\dquark$}{i2,v1}
      \fmf{plain,right=0.2}{i2,i3}
      \fmf{plain,left=0.2,label=$\Bs$}{i2,i3}
      \fmf{plain,right=0.2,label=$\rho^0$}{o1,o2}
      \fmf{plain,left=0.2}{o1,o2}
      \fmf{plain,right=0.2,label=$\jpsi$}{o3,o4}
      \fmf{plain,left=0.2}{o3,o4}
      \fmflabel{\hspace{-1cm}$\Vcb^*$}{v2}
      \fmflabel{\hspace{-0.7cm} \vspace{0.5cm}$\Vcd$}{v3}
    \end{fmfgraph*}
  }
\end{fmffile}%
\hfill
\begin{fmffile}{Figures/Chapter5/penguin_jpsirho}
  \fmfframe(17,-25)(31,-25){
    \begin{fmfgraph*}(115,170)
      \fmfstraight
      \fmfleft{i0,i1,i2,i3,i4,i5}
      \fmfright{o0,o1,o2,o3,o4,o5}
      \fmf{fermion,tension=1.8,label.side=left,label=$\bquark$}{v5,i3}
      \fmf{fermion,tension=1.5,right=0.2,label.side=left,label={\hspace*{18pt}\uquark,,\cquark,,\tquark}}{v2,v5}
      \fmf{gluon,tension=2}{v4,v2}
      \fmf{dbl_dashes,tension=0}{v4,v2}
      \fmf{fermion,tension=0.3,right=0.2,label.side=left }{v3,v2}
      \fmf{boson,tension=0.6,left=0.3,label=\Wp,label.side=left}{v3,v5}
      \fmf{fermion,label=$\cquark$,tension=0.9,right=0.3,label.side=left}{o4,v4}
      \fmf{fermion,label=$\cquark$,c,tension=0.9,right=0.3,label.side=left}{v4,o3}
      \fmf{fermion,label=$\dquark$,label.side=left}{o2,v3}
      \fmffreeze
      \fmf{fermion,tension=0.7,label=$\dquark$,label.side=left}{v1,o1}
      \fmf{fermion,tension=1,label.side=left,label=$\dquark$}{i2,v1}
      %\fmf{phantom,tension=0.4}{v4,v1}
      \fmf{phantom,tension=0.4}{v3,v1,v5}
      \fmf{plain,right=0.2}{i2,i3}
      \fmf{plain,left=0.2,label=$\Bs$}{i2,i3}
      \fmf{plain,right=0.2,label=$\rho^0$}{o1,o2}
      \fmf{plain,left=0.2}{o1,o2}
      \fmf{plain,right=0.2,label=$\jpsi$}{o3,o4}
      \fmf{plain,left=0.2}{o3,o4}
    \end{fmfgraph*}
  }
\end{fmffile}
}
  \caption{Leading order tree diagram of the decay \BdJpsiRho.}
  \label{bs2jpsirho_diagram}
\end{figure}

The \BdJpsiRho channel provides access to both \Acp{\rm dir} and \Acp{\rm mix} since the final state $\jpsi\rho$ is
a \CP eigenstate and thus $\BdBdbarSyst$ oscillations are active. Both of the above observables are measured in the
time dependent analysis of \BdJpsipipi decays from \lhcb \cite{Aaij:2014vda}. The penguin parameters as determined
in Section 5.5.3. of \cite{DeBruyn-thesis} are:

\begin{subequations}
  \label{delta_phis_jpsirho}
  \begin{align}
    \DeltaPhisJpsiPhi{0}         & = -0.000^{+0.011}_{-0.014}\:\text{rad},\\
    \DeltaPhisJpsiPhi{\parallel} & = +0.001^{+0.012}_{-0.017}\:\text{rad},\\
    \DeltaPhisJpsiPhi{\perp}     & = +0.003^{+0.012}_{-0.016}\:\text{rad},
  \end{align}
\end{subequations}

\noindent where most of the uncertainty is statistical in nature. Note the small uncertainty on
$\DeltaPhisJpsiPhi{k}$ compared to \equref{delta_phis_jpsikst}. According to \cite{DeBruyn-thesis}
this is attributed to the particular value of the penguin parameter $\thkPeng{k}''$, which is
$\sim 90^\circ$ (while in the case of \BsJpsiKst, $\thkPeng{k}' \sim 10^\circ$).
Specifically, as it can be seen in \equref{tandelta}, the algebraic structure of the
same equation is such that, $\tan(\Delta\phis^k)$ becomes minimal for values of \thkPeng{}
around odd multiples of $\pi/2$.

\subsection{Fitting Strategy}
\label{penguin_chi2_fit}

The parameters $(\akPeng,\thkPeng)$ of \equref{bsjpsiphi_amp_param}, which quantify the penguin topology contributions
to the \BsJpsiPhi decay, are estimated by means of a $\chisq$ fit. The $\chisq$ is defined after some assumptions have
been made. First, perfect $\grpsuthree_{\rm F}$ symmetry between the \BsJpsiPhi and the additional channels \BsJpsiKst
and \BdJpsiRho is assumed, implying:

\begin{equation}
\akPeng= \akPeng' = \akPeng'' \qquad \thkPeng = \thkPeng' = \thkPeng''.
\label{su3_apply}
\end{equation}

\noindent Second, the hadronic ratios necessary for building the $\Hobs{k}$ is assumed to be the same between
the two additional channels \BsJpsiKst and \BdJpsiRho, which translates in:

\begin{equation}
  \centering
  % \modulo{ \frac{\formFctr{k}\parenthesis{\BsJpsiPhi}}{\formFctr{k}'\parenthesis{\BsJpsiKst}}  } =
  % \modulo{ \frac{\formFctr{k}\parenthesis{\BsJpsiPhi}}{\formFctr{k}''\parenthesis{\BdJpsiRho}}  }.
  \hadRatio{k}{}{\prime} =
  \hadRatio{k}{}{\prime\prime}.
    \label{had_ratios_assuption}
\end{equation}

\noindent Doing so makes it possible to avoid using theoretical calculations of the hadronic factors.
Instead the single hadronic ratio, based on the assumption of \equref{had_ratios_assuption}, is directly
estimated from the fit, since it is treated as a free parameter. This way the uncertainty from the theoretical
calculations on the hadronic parameters does not enter the fit. This choice is supported by \cite{DeBruyn-thesis}
where it can be seen that the hadronic parameters estimated form experimental data are more precise
that the ones coming from theoretical calculations.

Under the above assumptions the observables related to the \BsJpsiKst and \BdJpsiRho channels along
with their corresponding experimental measurements, as described in \secref{bsjpsirho_chanell} and
\secref{bsjpsikst_chanell}, are used to define a $\chisq$ in the following equation:

\begin{equation}
  \centering
  \chisq_k = \sum \frac{O^{\rm theo}_k - O^{\rm exp}_k} {\sigma(O^{\rm exp}_k)},
  \label{chisq_form}
\end{equation}

\noindent{where,}

\begin{equation}
  \centering
  O_k \in \left\{ \parenthesis{\Acp{\rm dir}}'_k, \Hobs{k}', \parenthesis{\Acp{\rm dir}}''_k, \parenthesis{\Acp{\rm mix}}''_k, \Hobs{k}'' \right\}.
%   O_k \in \left{ {\Acp{\rm dir}}', \Hobs{k}', {\Acp{\rm dir}}'', {\Acp{\rm mix}}'', \Hobs{k}'' \right}
  \label{chisq_form_with}
\end{equation}

\noindent Note also that the CKM angle $\gamma$ and the weak phase $\phid$ are necessary for computing the ${\Acp{\rm dir}}'_k$,
${\Acp{\rm dir}}'_k$ and ${\Acp{\rm mix}}''_k$ observables. The previous parameters are allowed to vary in the
$\chisq$ fit with a Gaussian constrain around their measured value which are shown in \tabref{chi2_fit_constrains}.

\begin{table}[!h]
  \center
  \begin{tabular}{c c c}
    \hline
    parameter & value & source \\
    \hline
    $\gamma$      & $\left(73.2_{-7.0}^{+6.3}\right)^{\circ}$ & CKMfitter \cite{Charles:2015gya} \\
    $\phid$       & $0.767 \pm 0.029$ rad & De Bruyn K.\cite{DeBruyn-thesis} \\
    \hline
  \end{tabular}
  \caption{\small Constrains entering the $\chisq$ fit.}
  \label{chi2_fit_constrains}
\end{table}

\noindent Minimizing the above $\chisq$ yields the penguin parameters $(\akPeng$,$\thkPeng)$.
The results of the fit presented in \secref{penguin_results}.


\section{Results}
\label{penguin_results}
The results of the \chisq fit described in \secref{penguin_more_chanells} are shown in \equref{peng_a_theta_results}.
which are then translated, using \equref{tandelta}, to the final estimate for the penguin phase shift on \phis,
shown in \equref{delta_phis_result}.

\begin{subequations}
\label{peng_a_theta_results}
\begin{align}
    \aPeng{0}         & = 0.03^{+0.97}_{-0.03}\:, & \thPeng{0}         & = +\parenthesis{64^{+116}_{-244}}^\circ\:,\\
    \aPeng{\parallel} & = 0.32^{+0.58}_{-0.32}\:, & \thPeng{\parallel} & = -\parenthesis{15^{+150}_{-14}}^\circ\:,\\
    \aPeng{\perp}     & = 0.45^{+0.21}_{-0.27}\:, & \thPeng{\perp}     & = +\parenthesis{175 \pm 10}^\circ\:.
\end{align}
\end{subequations}

\noindent were the uncertainties are the quadratic sum of statistical and systematic ones.

\begin{alignat}{2}
\Delta\phi^{\jpsi{}\phi}_{s,0} & =
\phantom{-}0.001^{+0.087}_{-0.011}\:\text{(stat)}^{+0.013}_{-0.008}\:\text{(syst)}^{+0.048}_{-0.030}\:(\text{hadr})
&& = \phantom{-}0.001^{+0.100}_{-0.033}\:,\\
%
\Delta\phi^{\jpsi{}\phi}_{s,\parallel} & =
\phantom{-}0.031^{+0.049}_{-0.038}\:\text{(stat)}^{+0.013}_{-0.013}\:\text{(syst)}^{+0.031}_{-0.033}\:(\text{hadr})
&& = \phantom{-}0.031^{+0.059}_{-0.052}\:,\\
%
\Delta\phi^{\jpsi{}\phi}_{s,\perp} & =
-0.046^{+0.012}_{-0.012}\:\text{(stat)}^{+0.007}_{-0.008}\:\text{(syst)}^{+0.017}_{-0.024}\:(\text{hadr})
&& = -0.046^{+0.022}_{-0.028}\:,
\label{delta_phis_result}
\end{alignat}

Note that the penguin paramters $(\akPeng,\thkPeng)$ are ratios of hadronic amplitues. Given that it is
intreasting to point out that any factorizable $\grpsuthree_F$ breaking effets entering through the asumption
of \equref{} cancell out. (A small subtlity enterign through the second schanell jspsirho).
Thus the results reported here suffer from non facroraizable $\grpsuthree_F$ breaking which are probably
small, accroding to blah. The last statement is supported by the values of t the fitted hadronic parameter
ratios which are close to the once calculated assuming factorization. Thus, as it is mentioned in krystof,
it is either that the non factorization $\grpsuthree_F$ effects are small or thei ratio with respct to
the factorizable $\grpsuthree_F$ effects is small.

Note also that the H observables have almost no effect on the result.
It was also noticed that the H observables are in tention with respct to the cetral value of the fit result.
This suggests probably that the assuption \equref{} probably not true.

\begin{itemize}
  \item put results with hadronic ratios
  \item interpretaion of the hadronic ratio result formt he fit.
  \item explain plots
  \item investigate the C = -Acp
  \item modify plot legend to
  \item maybe
\end{itemize}

\begin{figure}[h]
\begin{center}
  \begin{subfigure}{1\textwidth}
    \includegraphics[trim=0.0cm 0.0cm 0.0cm 0.0cm, clip=true,scale=0.4]{Figures/Chapter5/Penguin_Contribution_Ang_vs_Abs_allB2VV_Long.pdf}
    \caption{}
    \label{pengPlot_long}
  \end{subfigure}\\
  \begin{subfigure}{1\textwidth}
    \includegraphics[trim=0.0cm 0.0cm 0.0cm 0.0cm, clip=true,scale=0.4]{Figures/Chapter5/Penguin_Contribution_Ang_vs_Abs_allB2VV_Perp.pdf}
    \caption{}
    \label{pengPlot_perp}
  \end{subfigure}
  \caption{Penguin parameter contours. Figures from~\cite{DeBruyn-thesis}}
\end{center}
\end{figure}

\begin{figure}[h]
\begin{center}
  \includegraphics[trim=0.0cm 0.0cm 0.0cm 0.0cm, clip=true,scale=0.33]{Figures/Chapter5/Penguin_Contribution_Ang_vs_Abs_allB2VV_Para.pdf}
  \caption{Penguin parameter contours. Figures from~\cite{DeBruyn-thesis}}
  \label{pengPlot_para}
\end{center}
\end{figure}

% \begin{align}
% \Re[a_0] & = \phantom{-}0.01_{-0.32}^{+0.97}\:, & \Im[a_0] & = \phantom{-}0.025_{-0.031}^{+0.035}\:, & \chi^2_{\text{min}} & = 1.1 \times 10^{-7}\:,\label{Eq:Pen_Re_Im_Long}\\
% %
% \Re[a_\parallel] & = \phantom{-}0.31_{-0.51}^{+0.58}\:, & \Im[a_\parallel] & = -0.082_{-0.087}^{+0.074}\:,& \chi^2_{\text{min}} & = 1.2 \times 10^{-3}\:,\label{Eq:Pen_Re_Im_Para}\\
% %
% \Re[a_\perp] & = -0.44_{-0.21}^{+0.27}\:, & \Im[a_\perp] & = \phantom{-}0.037_{-0.076}^{+0.079}\:,& \chi^2_{\text{min}} & = 1.5 \times 10^{-6}\:,\label{Eq:Pen_Re_Im_Perp}
% \end{align}


\subsection{Further Corsechecks}
There are two important issues that the fitting strategy as described in \secref{penguin_chi2_fit}
does not take into account. Namelly, $\grpsuthree_F$ breaking effects and correlations between
the measurements of the observables \equref{chisq_form_with}. Bothe of these issues addressed in the current section.

\subsubsection{\grpsuthree breaking}
\label{su3_breaking}
Potential $\grpsuthree_F$ effects manifest themeselfs in the calculations of the hadronic parameters ratios
when computing the $\Hobs{}$ and in the assumption of \equref{su3_apply}. However, due to the followed strategy
the hadronic parameter ratios do not rely on any $\grpsuthree_F$ since the are fitted for.
As far as the assumption of \equref{su3_apply} is concerned, a special fit is performed using only observables
releated to \BdJpsiRho channel. For that, $(\akPeng,\thkPeng)$ are re-expressed as shown in \equref{su3_breaking}

\begin{equation}
  \akPeng \to \xi_k \; \akPeng'' \quad \thkPeng \to \delta_k + \thkPeng''.
\label{su3_breaking}
\end{equation}

\noindent The quantities $\xi_k$ and $\delta_k$ are free parameters with a gaiussian constrian and can absorb
potential $\grpsuthree_F$ effects that will break the assuption of \equref{su3_apply}. The effect of various
amounts of $\grpsuthree_F$ breaking is investigated by means of a $\xi_k-\DeltaPhis{k}$ contour. The errors on the
$\xi_k$ and $\delta_k$ parameters, tha constrain their central valuee, are varied from $0-50\%$ and $(0-40)^{\circ}$.
Then the above mentioned special fit is pepreated several times. The results showed no dependence on the value
of $\DeltaPhis{k}$ on $\xi_k$. This is most likely due to the particular structure of \equref{tanddelta}. {\color{red}strong theta pahse maybe. what do you eman here.}

In addition, a systematic uncertainty to $\DeltaPhis{k}$ is asigned by assuming $20\%$ $\grpsuthree_F$ breaking,
according to what was mentioned at the end of \secref{had_pars_suthree}.

\subsubsection{Correlations of Polarization States}
Correlations between the $k$ polarizations of the observables entering the \chisq fit were ignored in the
current analysis. These correlations enter mainly via the $\Acp{k}$ and $\fP{k}$. For example, see the
correlation matrix \tabref{correlation_matrix} of the angular analysis performed in \chapref{Data_Analysis}.
In order to check the effect of those corrlations on the estimation of the penguin paramters $(\akPeng,\thkPeng)$,
an additonal \chisq fit is performed. In that fit the correlations are indeed taken into account.
The results show a small increase on the error of the penguin paramter $\thPeng{0}$.
The last result suggest that the impact of the coorelations on the parameters of intereset is negligible.

