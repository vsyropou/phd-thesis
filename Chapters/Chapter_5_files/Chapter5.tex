% Chapter 5

\chapter{Results}
\label{Results}

\section{Parameters of Interest}
\section{Normalization of \BsJpsiKpi}

%% Signal efficiencies and background rejections values are calculated for both 2012 and 2011 data conditions separately and individually for three 
%% different subsets of cuts in the final selection presented in \tabref{tab:Bs2JpsiKstSelection}, resulting in three individual efficiencies (rejections). 
%% As a first step, $\varepsilon_{\rm sel}$ is calculated selecting signal and background samples with all the final selection cuts excepting (**) and (*). 
%% As a second step, $\varepsilon_{\rm BDTG}$ is calculated selecting previous samples also with cut (**). As a third step, $\varepsilon_{\Lb}$ is obtained 
%% selecting previous samples also with cut (*). Finally, a total efficiency (rejection) is obtained, containing the information from all the efficiencies 
%% (rejections) computed in previous steps: $\varepsilon_{\rm tot}$ uses signal and background samples selected with all the cuts from the final selection.



\section{Systematic Uncertainties}
\section{Likelihood scans}
\section{Toy Experiments Study}


