In the penguin contribution estimation of the current chapter, the hadronic factors $\formFctr{k}$ become relevant when building
one of the neccessarry observables, see \equref{hobs_def}. These factors involve the computation of a
\BJpsiX decay amplitude which is dificult{\color{red} ref} to perform and the ascociated uncertainty is large.
The dificulty comes in the {\it hadronization} process, which is the process of quarks forming bound
states\footnote{ {\it Hadrons} are bound, by the strong interaction, states of two, three and recently{\color{red} ref pentaquarks} of more quarks.},
like the $\jpsi$ and $\phi$ for example. During this process quarks might interact with each other via
the storng force before hadronizing. This class of strong interactions, also refered to as {\it long distance QCD effcts}
are difficult to compute within the context of {\it perturbation theory}. An alternative approach
to computing the \BJpsiX decay amplitude is the {\it factorization} assumption{\color{red} ref},
However its power is limited. Hence the last computation suffers from large uncertainties necessary
to account for these {\it non-factorizable} effects{\color{red} make sure this is the case, read}.

The strategy described in the current chapter avoids the use of the full decay amplitude, of \equref{bsjpsiphi_amp},
when building the observables of \equref{bsjpsiphi_peng_acp_obs}. This way the hadronic factors $\formFctr{k}$
are not needed and the above mentioned calculations are not necessary. However, the observable of \equref{hobs_def}
does require the explicit calculation of ratios of hadronic factors. Thus, introducing the last observable
provides additional information on estimating the penguin parameters on the one hand, on the other hand
it itroduces further uncertainty from non-factorizable qcd effects. As it is explained in \secref{penguin_formalism}
the observable \equref{hobs_def} can be sometimes avoided.

{\color{red} hadronic ratios were calcualted based on table  }
Show tables with form factors \\


An important feature that the current chapter makes use of is the so called $\grpsuthree_{\rm F}$ {it flavour symetry} {\color{red} ref}.
In a naive interpretation $\grpsuthree_{\rm F}$ flavour symmetry implies that the three lightest quarks (\uquark,\dquark,\squark)
are indistinguishable by the strong interaction. The last symmetry is broken by the fact that the above
mentioned qurks have indeed different mass. However, those differences in mass are small with respect
to the hadronization scale \lqcd of the strong iteraction, such that $\grpsuthree_{\rm F}$
remains a usefull approximate symmetry. Particularly in the case of $\Bq$ meson decays where hadronic
parameters can be releated between different decay modes and thus avoiding calculating them in the first place.
The penguin parameter estimation of the current chapter is an example use case of $\grpsuthree_{\rm F}$.


According to~\cite{ela} the estimated amount of $\grpsuthree_{\rm F}$ breaking is about $20\%$.
The imapact of the last symemtry breaking is crucial to the penguin paramter estimation of the current chapter
and needs to be investigated as a source of systematic uncertainty. The last is adressed in \secref{su3_breaking}.
For completion other estiamtes on the $\grpsuthree_{\rm F}$ breaking can be found in {\color{red} blahhh}.
