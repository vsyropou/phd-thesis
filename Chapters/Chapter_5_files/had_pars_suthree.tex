In the penguin parameter estimation, the hadronic factors $\formFctr{k}$ become relevant when building
the neccessarry observables, see \equref{hobs_def}. Calculating these factrors is very dificult{\color{red} ref}
and the ascociated uncertainties are large. These calculations rely on the so called {\it factorization} assumption
when computing the decay amplitude of a $\B\to M_1M_2$ transition, where $M_1$ and $M_2$ are two mesons.
In the factorization framework it is assumed that the $\B\to M_1M_2$ decay amplitude can be written as
the product of a {\it decay constant} and a {\it form factor}. Especially for \BJpsiX decay modes, the
last assupmtion leads to large corrections from {\it non-factorizable} effects that are not taken into
account. The strategy described in \secref{penguin_formalism} and \secref{penguin_more_chanells}, uses
ratios of decay amplitudes, as much as possible, when building the necessary observables such that hadronic
factors would cancell out. An important aspect on achieving this is by invoking the \grpsuthree flavour symetry,
explained in the next paragraph.

In a naive interpretation of the actual \grpsuthree flavour symmetry{ref} the dynamics of the strong
interaction are the same for the three lightest quarks (\uquark,\dquark,\squark). This is attributed
to the small masses of those quakrs with respect to the hadronization scale of the QCD.
In the context of the current chapter the invoking \grpsuthree allows to relate hadronic quantities
between different \BJpsiX modes.

%
%
% \begin{itemize}
%   \item The eff hamiltoniasn decompozedecay amplitudes into wilson coeef and a matrix elements
%   \item Calculating matirx elements is very dificult, invloving long disntace - -non perturbative qcd.
%   \item Facrtorization makes these calculations easirer, whne aplicable.( by decompozing the matrix element in a decay constant and a form afactor.(do not mention this)
%   \item In JpsiX facrorization does not realy work. Expected large correltuions from non factorizable.
%   \item A way out is to exploit ratios where amplitudes cancell out
%   \item For that it helps to invoke su3 symmetry. to relate a chanell of intreast to decays with similar topology.
%   \item Drawback non facrtorizable long distance qcd effects that do not cancell in ratios.
%   \item Experimental info is usefull and more precise.
% \end{itemize}
