In the current chapter, the hadronic factors $\formFctr{k}$ become relevant when building
one of the neccessarry observables, see \equref{hobs_def}. These factors can be calculated
theoretically, as for example in ~\cite{DeBruyn-thesis}. However, these calculations involve
the computation of \BJpsiX decay amplitude which is dificult. The dificulty comes in the
{\it hadronization} process, which is the process of quarks forming bound states, named
hadrons\footnote{ {\it Hadrons} are bound, by the strong interaction, states of two,
three and recently~\cite{Aaij:2016nsc} of five quarks.}, like the $\jpsi$ and $\phi$ for example.
During this process quarks might interact with each other via the storng force before hadronizing.
This class of strong interactions, also refered to as {\it long distance QCD effects}
are difficult to compute within the context of {\it perturbation theory}.
The computation of the \BJpsiX decay amplitude becomes easier under the so called
{\it factorization} assumption~\cite{HAAN1970448,Wirbel1985,CABIBBO1978418,FAKIROV1978315}.
However, according to ~\cite{DeBruyn-thesis},
the power of this approach is also limited in the case of \BJpsiX decays and corrections
are requred to to account for {\it non-factorizable} during the process of hadronization.
Hence, the hadronic factors $\formFctr{k}$ computed from theory suffer from large uncertainties.

The strategy described in the current chapter avoids, as much as posible, the use of hdronic
factors $\formFctr{k}$ from theory. However, the observable of \equref{hobs_def}
does require the explicit calculation of ratios of hadronic factors. Thus on the one hand
introducing the last observable provides additional information for estimating the penguin
parameters, on the other hand it itroduces further uncertainty from non-factorizable qcd effects.
As explained in \secref{penguin_more_chanells} the observable \equref{hobs_def} can also be avoided.

An important feature that the current chapter makes use of is the so called
$\grpsuthree_{\rm F}$ {\it flavour symetry}~\cite{GELLMANN1964214,NEEMAN1961222}.
In a naive interpretation, $\grpsuthree_{\rm F}$ flavour symmetry implies that the three lightest quarks (\uquark,\dquark,\squark)
are indistinguishable by the strong interaction. The last symmetry is broken by the fact that the above
mentioned qurks have indeed different mass. However, those differences in mass are small with respect
to the hadronization scale \lqcd of the strong iteraction, such that $\grpsuthree_{\rm F}$
remains a usefull approximate symmetry. Particularly in the case of $\Bq$ meson decays where hadronic
parameters can be releated between different decay modes and thus avoiding calculating them in the first place.
The penguin parameter estimation of the current chapter is an example use case of $\grpsuthree_{\rm F}$.

According to~\cite{Nagashima:2007qn,Gronau:2013mda} the estimated amount of $\grpsuthree_{\rm F}$ breaking is about $20\%$.
The imapact of the last symemtry breaking has implications in the penguin paramter estimation
of the current chapter. These implications are adressed in \secref{su3_breaking}.
Other $\grpsuthree_{\rm F}$ breaking estimates can be found in~\cite{Charles:2015gya,PDG}.
