Hadronic factors $\formFctr{k}$ are introduced by the \Hobs{k} observable of \equref{hobs_def}.
Despite the fact that these factors can be calculated theoretically, as for example
in \cite{DeBruyn-thesis}. The calculations involve the computation of \BJpsiX decay amplitudes
which is difficult. The difficulty comes from the {\it hadronization} process, which is the
process of quarks forming bound states, hadrons, such as the $\jpsi$ and $\Pphi$.
During this process quarks might interact with each other via the strong interaction.
This class of strong interactions, also referred to as {\it long distance QCD effects}
cannot be computed within the context of {\it perturbation theory}.
The computation of the \BJpsiX decay amplitude becomes easier under the so called
{\it factorization} assumption \cite{HAAN1970448,Wirbel1985,CABIBBO1978418,FAKIROV1978315}.
However, according to \cite{DeBruyn-thesis},
the power of this approach is also limited in the case of \BJpsiX decays and corrections
are required to account for {\it non-factorizable} effects during the process of hadronization.
Hence, the hadronic factors $\formFctr{k}$ computed from theory suffer from large uncertainties.

The strategy described in the current chapter avoids, as much as possible, the use of hadronic
factors $\formFctr{k}$ from theory. However, the \Hobs{k} observable, does require the explicit
calculation of ratios of hadronic factors. Thus on the one hand
introducing the previous observable provides additional information for estimating the penguin
parameters, on the other hand it introduces further uncertainty from non-factorizable QCD effects.
As will be explained in \secref{penguin_more_chanells} introducing the previous the \Hobs{k}
observable can also be avoided.

An important feature that the current chapter makes use of is the so called
$\grpsuthree_{\rm F}$ {\it flavor symmetry} \cite{GELLMANN1964214,NEEMAN1961222}.
In a naive interpretation, $\grpsuthree_{\rm F}$ flavor symmetry implies that the three lightest quarks (\uquark,\dquark,\squark)
are indistinguishable by the strong interaction. The last symmetry is broken by the fact that the above-mentioned
quarks have indeed different mass. However, these differences are small with respect
to the hadronization scale \lqcd of the strong interaction, and as a result $\grpsuthree_{\rm F}$
remains a useful approximate symmetry. Particularly, in the case of $\Bq$ meson decays, hadronic
parameters from different decay modes can be related to one another and thus their calculation can be avoided
in the first place. The penguin parameter estimation of the current chapter is an example of $\grpsuthree_{\rm F}$.

According to \cite{Nagashima:2007qn,Gronau:2013mda} the estimated amount of $\grpsuthree_{\rm F}$ breaking is about $20\%$.
The impact of the symmetry breaking has implications in the penguin parameter estimation
of the current chapter. These implications are addressed in \secref{su3_breaking}.
Other $\grpsuthree_{\rm F}$ breaking estimates can be found in \cite{Charles:2015gya,PDG}.
