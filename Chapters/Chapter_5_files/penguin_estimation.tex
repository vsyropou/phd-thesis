In the recent \lhcb publication \cite{phis-3fb-paper} of the \phis measurement one can find the best fit value for the $\lambda_{\jpsi\Pphi}$
parameter, mentioned in \equref{lambda_cpv}. These two parameters are summarized as in the following equation:

%shown in \equref{phis_lambda_result},

\begin{subequations}
  \label{phis_lambda_result}
  \begin{align}
    \centering
    \phis^{\jpsi\Pphi}     &=  -0.058 \pm 0.049(\text{stat})  \;\; \text{rad},\\
    \lambda^{\jpsi\Pphi}   &=  +0.964 \pm 0.019(\text{stat}),
  \end{align}
\end{subequations}

\noindent and can be used to estimate the penguin parameters $(\aPeng{},\thPeng{})$.
Based on \equref{cp_asym_lambda} and according to the formalism in \secref{penguin_formalism}, the $\DeltaPhis{}$ estimation
due to penguin topologies based on \BsJpsiPhi decays comes with an uncertainty of about $0.05$ rad, see Eq. 5.125 of \cite{DeBruyn-thesis},
which is not precise enough given the uncertainty of $\phis^{\jpsi\Pphi}$. This is related to the suppression factor $\epsilon$,
mentioned in \secref{jpsiphi_amp_struct}, which the penguin topologies receive in the \BsJpsiPhi decay.

In order to increase the precision on the penguin parameters the \grpsuthree flavor symmetry is
invoked such that more channels similar to \BsJpsiPhi are involved in the computation of $(\akPeng-\thkPeng)$.
Note that the information from these channels entering through the observables in \equref{bsjpsiphi_peng_acp_obs} and \equref{hobs_def}
has to be polarization dependent. This is due to the fact that new physics dynamics might enter in a different way in each polarization,
as mentioned in \cite{DeBruyn-thesis}.
This section introduces the additional channels, \BsJpsiKst and \BdJpsiRho and provides
details of the fitting strategy to estimate the penguin parameters.

\subsection{The \BsJpsiKst Channel}
\label{bsjpsikst_chanell}

The \BsJpsiKst channel is a flavor specific decay with the same topology as \BsJpsiPhi.
The corresponding diagrams are illustrated in \figref{bs2jpsikst}.
The \BsJpsiKst amplitude is parameterized following the same concept as in the case of \BsJpsiPhi resulting in \cite{DeBruyn:2014oga}:

\begin{equation}
  \mathcal{A} \parenthesis{\BsJpsiKstPolState{k}} = -\lambda \formFctr{k}' \brackets{ 1 - \akPeng' e^{i\thkPeng'} e^{i\gamma} },
  \label{bsjpsikst_amp}
\end{equation}

\noindent where primed ${}^\prime$ quantities from here are associated with the \BsJpsiKst decay only.
Note the absence of the suppression factor $\epsilon$ in \equref{bsjpsikst_amp}, which implies that the penguin diagram
contributes as much as the color suppressed tree diagram does to the total amplitude, contrary to the case of the \BsJpsiPhi decay.

\begin{figure}[t]
  \centering
  \scalebox{0.9}{\sffamily \input{Figures/Chapter5/tree_penguin_jpsikst}}
  \caption{Leading order diagrams of the decay \BsJpsiKst. Left: Color-suppressed tree topology. Right: Penguin topology.}
  \label{bs2jpsikst}
\end{figure}

As mentioned in \secref{penguin_formalism} the \BsJpsiKst channel provides access to \Acp{\rm dir} only.
Thus, additional information is required, via the $\Hobs{k}'$ observable, in order to probe $\akPeng$ and $\thkPeng$.
Both observables are based on measurements that are described in \chapref{Data_Analysis} of the current thesis.
The first observable is reported in \tabref{bestFitResult}, with the associated systematic quoted in \tabref{systematics_acp}.
The $\Hobs{k}'$ observable, constructed from \BRof{\BsJpsiKst}, is reported in \equref{Br_total}
and is determined to be:

\begin{subequations}
  \label{hobs_jpsikst}
  \begin{align}
  \Hobs{0}'         & = 0.99 \pm 0.07\:\text{(stat)} \; 0.06\:\text{(syst)} \; 0.27\:(\text{hadr}) , \label{hobs_jpsikst_long} \\
  \Hobs{\parallel}' & = 0.91 \pm 0.14\:\text{(stat)} \; 0.08\:\text{(syst)} \; 0.21\:(\text{hadr}) , \label{hobs_jpsikst_para} \\
  \Hobs{\perp}'     & = 1.47 \pm 0.14\:\text{(stat)} \; 0.11\:\text{(syst)} \; 0.28\:(\text{hadr}) . \label{hobs_jpsikst_perp}
  \end{align}
\end{subequations}

\noindent The last uncertainty is due to the hadronization factor $|\formFctr{k}/\formFctr{k}'|$ necessary for the construction of $\Hobs{k}'$.
The hadronization factors are calculated theoretically based on the method suggested in \cite{eff-hamiltonian-bs-syst}.
The exact numbers used can be found in \cite{DeBruyn-thesis}. Note that the uncertainty on $\Hobs{k}'$ is dominated by these factors.

% \begin{align}
% \left|\frac{\mathcal{A}'_0(\BsJpsiPhi)}{\mathcal{A}_0(\BsJpsiKst)}\right| & = 1.23 \pm 0.16\:,\label{Eq:AmpRat_JpsiKstar_long}\\
% %%%
% \left|\frac{\mathcal{A}'_{\parallel}(\BsJpsiPhi)}{\mathcal{A}_{\parallel}(\BsJpsiKst)}\right| & = 1.28 \pm 0.15\:,\\
% %%%
% \left|\frac{\mathcal{A}'_{\perp}(\BsJpsiPhi)}{\mathcal{A}_{\perp}(\BsJpsiKst)}\right| & = 1.20 \pm 0.12\:,\label{Eq:AmpRat_JpsiKstar_perp}
% \end{align}

Prior to any combination the penguin parameters have been estimated based on
the \BsJpsiKst channel only \cite{bsjpsikst-paper}, resulting in \equref{delta_phis_jpsikst},
where most of the uncertainty is statistical in nature. The $\chisq$ fit performed here is similar
to the one described in \secref{penguin_chi2_fit}, hence details of the fitting strategy are postponed until later.

\begin{subequations}
  \label{delta_phis_jpsikst}
  \begin{align}
    \DeltaPhisJpsiPhi{0}         & = +0.001^{+0.100}_{-0.033} \:\text{rad},\\
    \DeltaPhisJpsiPhi{\parallel} & = +0.031^{+0.059}_{-0.052} \:\text{rad},\\
    \DeltaPhisJpsiPhi{\perp}     & = -0.046^{+0.022}_{-0.028} \:\text{rad},
  \end{align}
\end{subequations}

\subsection{The \BdJpsiRho Channel}
\label{bsjpsirho_chanell}

The topology of the \BdJpsiRho decay is shown in \figref{bs2jpsirho_diagram}.
The amplitude structure is identical to that of the \BsJpsiKst mode \cite{Fleischer:1999zi}:

\begin{equation}
  \mathcal{A} \parenthesis{\BdJpsiRhoPolState{k}} = -\lambda \formFctr{k}'' \brackets{ 1 - \akPeng'' e^{i\thkPeng''} e^{i\gamma} },
  \label{bsjpsirho_amp}
\end{equation}

\noindent where the double primed symbol ${}^{\prime\prime}$ from here on labels parameters related to the \BdJpsiRho decay.
Note again the absence of the suppression factor $\epsilon$.

\begin{figure}[h]
  \centering
  \scalebox{0.9}{\sffamily %%BoundingBox: -5 0 121 170
%%HiResBoundingBox: -5 0 120.57008 169.36447

\begin{fmffile}{Figures/Chapter5/tree_jpsirho}
  \fmfframe(17,-25)(31,-25){
    \begin{fmfgraph*}(115,170)
      \fmfstraight
      \fmfleft{i0,i1,i2,i3,i4,i5}
      \fmfright{o0,o1,o2,o3,o4,o5}
      \fmf{fermion,tension=3.5,label.side=left,label=$\bquark$}{v2,i3}
      \fmf{fermion,label=$\cquark$,label.side=left}{o4,v2}
      \fmf{fermion,label=$\cquark$,label.side=left}{v3,o3}
      \fmf{fermion,label=$\dquark$,label.side=left,tension=2}{o2,v3}
      \fmf{boson,tension=2.4,label=\Wp,label.side=right,right=0.3}{v2,v3}
      \fmffreeze
      \fmf{phantom,tension=0.3}{v2,v1,v3}
      \fmf{fermion,tension=0.5,label=$\dquark$,label.side=left}{v1,o1}
      \fmf{fermion,tension=0.5,label.side=left,label=$\dquark$}{i2,v1}
      \fmf{plain,right=0.2}{i2,i3}
      \fmf{plain,left=0.2,label=$\Bs$}{i2,i3}
      \fmf{plain,right=0.2,label=$\rho^0$}{o1,o2}
      \fmf{plain,left=0.2}{o1,o2}
      \fmf{plain,right=0.2,label=$\jpsi$}{o3,o4}
      \fmf{plain,left=0.2}{o3,o4}
      \fmflabel{\hspace{-1cm}$\Vcb^*$}{v2}
      \fmflabel{\hspace{-0.7cm} \vspace{0.5cm}$\Vcd$}{v3}
    \end{fmfgraph*}
  }
\end{fmffile}%
\hfill
\begin{fmffile}{Figures/Chapter5/penguin_jpsirho}
  \fmfframe(17,-25)(31,-25){
    \begin{fmfgraph*}(115,170)
      \fmfstraight
      \fmfleft{i0,i1,i2,i3,i4,i5}
      \fmfright{o0,o1,o2,o3,o4,o5}
      \fmf{fermion,tension=1.8,label.side=left,label=$\bquark$}{v5,i3}
      \fmf{fermion,tension=1.5,right=0.2,label.side=left,label={\hspace*{18pt}\uquark,,\cquark,,\tquark}}{v2,v5}
      \fmf{gluon,tension=2}{v4,v2}
      \fmf{dbl_dashes,tension=0}{v4,v2}
      \fmf{fermion,tension=0.3,right=0.2,label.side=left }{v3,v2}
      \fmf{boson,tension=0.6,left=0.3,label=\Wp,label.side=left}{v3,v5}
      \fmf{fermion,label=$\cquark$,tension=0.9,right=0.3,label.side=left}{o4,v4}
      \fmf{fermion,label=$\cquark$,c,tension=0.9,right=0.3,label.side=left}{v4,o3}
      \fmf{fermion,label=$\dquark$,label.side=left}{o2,v3}
      \fmffreeze
      \fmf{fermion,tension=0.7,label=$\dquark$,label.side=left}{v1,o1}
      \fmf{fermion,tension=1,label.side=left,label=$\dquark$}{i2,v1}
      %\fmf{phantom,tension=0.4}{v4,v1}
      \fmf{phantom,tension=0.4}{v3,v1,v5}
      \fmf{plain,right=0.2}{i2,i3}
      \fmf{plain,left=0.2,label=$\Bs$}{i2,i3}
      \fmf{plain,right=0.2,label=$\rho^0$}{o1,o2}
      \fmf{plain,left=0.2}{o1,o2}
      \fmf{plain,right=0.2,label=$\jpsi$}{o3,o4}
      \fmf{plain,left=0.2}{o3,o4}
    \end{fmfgraph*}
  }
\end{fmffile}
}
  \caption{Leading order tree diagram of the decay \BdJpsiRho.}
  \label{bs2jpsirho_diagram}
\end{figure}

The \BdJpsiRho channel provides access to both \Acp{\rm dir} and \Acp{\rm mix} since the final state
$\jpsi\rho$, just like $\jpsi\Pphi$ (see \secref{WeakPhase}), is an admixture of \CP-odd and \CP-even
eigenstates and thus $\BdBdbarSyst$ oscillations are active. Both of the above observables are measured
in the time dependent angular analysis of \BdJpsipipi decays from \lhcb \cite{Aaij:2014vda}. The penguin
parameters as determined in Section 5.5.3. of \cite{DeBruyn-thesis} are shown in the following equations:

\begin{subequations}
  \label{delta_phis_jpsirho}
  \begin{align}
    \DeltaPhisJpsiPhi{0}         & = -0.000^{+0.011}_{-0.014}\:\text{rad},\\
    \DeltaPhisJpsiPhi{\parallel} & = +0.001^{+0.012}_{-0.017}\:\text{rad},\\
    \DeltaPhisJpsiPhi{\perp}     & = +0.003^{+0.012}_{-0.016}\:\text{rad},
  \end{align}
\end{subequations}

\noindent where most of the uncertainty is statistical in nature. Note the small uncertainty on
$\DeltaPhisJpsiPhi{k}$ compared to \equref{delta_phis_jpsikst}. According to \cite{DeBruyn:2014oga,DeBruyn-thesis}
this is attributed to the particular value of the penguin parameter $\thkPeng''$, which is
$\sim 90^\circ$ (while in the case of \BsJpsiKst, $\thkPeng' \sim 10^\circ$).
Specifically, as it can be seen in \equref{tandelta}, the algebraic structure of the
same equation is such that, $\tan(\Delta\phis^k)$ becomes minimal for values of \thkPeng{}
around odd multiples of $\pi/2$.

\subsection{Fitting Strategy}
\label{penguin_chi2_fit}

The parameters $(\akPeng,\thkPeng)$ of \equref{bsjpsiphi_amp_param}, which quantify the penguin topology contributions
to the \BsJpsiPhi decay, are estimated by means of a $\chisq$ fit. The $\chisq$ is defined after some assumptions have
been made. First, perfect $\grpsuthree_{\rm F}$ symmetry between the \BsJpsiPhi and the additional channels \BsJpsiKst
and \BdJpsiRho is assumed, implying:

\begin{equation}
\akPeng= \akPeng' = \akPeng'' \qquad \thkPeng = \thkPeng' = \thkPeng''.
\label{su3_apply}
\end{equation}

\noindent Second, the hadronic ratios necessary for building $\Hobs{k}$ are assumed to be the same between
the two additional channels \BsJpsiKst and \BdJpsiRho, which translates in:

\begin{equation}
  \centering
  % \modulo{ \frac{\formFctr{k}\parenthesis{\BsJpsiPhi}}{\formFctr{k}'\parenthesis{\BsJpsiKst}}  } =
  % \modulo{ \frac{\formFctr{k}\parenthesis{\BsJpsiPhi}}{\formFctr{k}''\parenthesis{\BdJpsiRho}}  }.
  \hadRatio{k}{}{\prime} =
  \hadRatio{k}{}{\prime\prime}.
    \label{had_ratios_assuption}
\end{equation}

\noindent Doing so makes it possible to avoid using theoretical calculations of the hadronic factors.
Instead the single hadronic ratio, based on the assumption of \equref{had_ratios_assuption}, is directly
estimated from the fit, since it is treated as a free parameter. This way the uncertainty from the theoretical
calculations on the hadronic parameters does not enter the fit. This choice is supported by \cite{DeBruyn:2014oga,DeBruyn-thesis}
where it can be seen that the hadronic parameters estimated form experimental data are more precise
that the ones coming from theoretical calculations.

Under the above assumptions the observables related to the \BsJpsiKst and \BdJpsiRho channels along
with their corresponding experimental measurements, as described in \secref{bsjpsirho_chanell} and
\secref{bsjpsikst_chanell}, are used to define a $\chisq$ in the following equation:

\begin{equation}
  \centering
  \chisq_k = \sum \left(\frac{O^{\rm theo}_k - O^{\rm exp}_k} {\sigma(O^{\rm exp}_k)}\right)^2,
  \label{chisq_form}
\end{equation}

\noindent{where,}

\begin{equation}
  \centering
  O_k \in \left\{ \parenthesis{\Acp{\rm dir}}'_k, \Hobs{k}', \parenthesis{\Acp{\rm dir}}''_k, \parenthesis{\Acp{\rm mix}}''_k, \Hobs{k}'' \right\}.
%   O_k \in \left{ {\Acp{\rm dir}}', \Hobs{k}', {\Acp{\rm dir}}'', {\Acp{\rm mix}}'', \Hobs{k}'' \right}
  \label{chisq_form_with}
\end{equation}

\noindent Note also that the CKM angle $\gamma$ and the weak phase $\phid$ are necessary for computing
${\Acp{\rm dir}}'_k$, ${\Acp{\rm dir}}'_k$ and ${\Acp{\rm mix}}''_k$. The previous parameters are allowed to vary in the
$\chisq$ fit with a Gaussian constrain around their measured values which are shown in \tabref{chi2_fit_constrains}.

\begin{table}[!h]
  \center
  \begin{tabular}{c c c}
    \hline
    parameter & value & source \\
    \hline
    $\gamma$      & $\left(73.2_{-7.0}^{+6.3}\right)^{\circ}$ & CKMfitter \cite{Charles:2015gya} \\
    $\phid$       & $0.767 \pm 0.029$ rad & De Bruyn K.\cite{DeBruyn-thesis} \\
    \hline
  \end{tabular}
  \caption{\small Constrains entering the $\chisq$ fit.}
  \label{chi2_fit_constrains}
\end{table}

\noindent Minimizing the above $\chisq$ yields the penguin parameters $(\akPeng$,$\thkPeng)$.
The results of the fit presented in \secref{penguin_results}.
