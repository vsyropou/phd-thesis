

The recent \lhcb measurement~\cite{phis-3fb-paper} of \phis comes along with $\lambda_{\jpsi\phi}$, mentioned \equref{lambda_cpv}.
These two parameters, shown in \equref{phis_lambda_result}, can be used to estimate the penguin parameters $(\aPeng{},\thPeng{})$.

\begin{equation}
  \centering
  \phis^{\jpsi\phi}     =  -0.058 \pm 0.049(\text{stat})  \;\; \text{rad}, \;\;\;\;\;
  \lambda^{\jpsi\phi}   =  +0.964 \pm 0.019(\text{stat}).
  \label{phis_lambda_result}
\end{equation}

\noindent Based on \equref{cp_asym_lambda} and according to the formalism in \secref{penguin_formalism}, the $\DeltaPhis{}$ estimation
from \BsJpsiPhi decays comes with an uncertainty of about $0.05$ rad, see Eq. 5.125 of~\cite{DeBruyn-thesis}, which is not precise
enough given the central value of $\phis^{\jpsi\phi}$.

In order to increase the precision on \DeltaPhis{} the \grpsuthree flavour symmetry,
is invoked in order to extract \DeltaPhis{} from similar to \BsJpsiPhi chanells.
Note that the information from those chanells entering through the observables, \equref{bsjpsiphi_peng_acp_obs} and \equref{hobs_def}
has to be polarization dependant. This is due to the fact that new physics dynamics might enter in a different way in each polarization as mentioned {\color{red}somewhere in krystof.}
This section introduces the aditional chanells, \BsJpsiKst and \BdJpsiRho. The details of the $\chi^2$ fit are presented in \secref{penguin_chi2_fit}.


\subsection{The \BsJpsiKst chanell}
The \BsJpsiKst chanell is a flavour specific decay with the same topology as \BsJpsiPhi, see \figref{bs2jpsikst}.
The amplitude of \BsJpsiKst, is parametrised following the same concept as in the case of \BsJpsiPhi and is shown in \equref{bsjpsikst_amp}.

\begin{figure}[h]
  \centering
  {\sffamily \input{Figures/Chapter5/tree_penguin_jpsikst}}
  \caption{Leading order tree diagram of the decay \BsJpsiPhi.}
  \label{bs2jpsikst}
\end{figure}

\begin{equation}
  \mathcal{A} \parenthesis{\BsJpsiKstPolState{k}} = -\lambda \formFctr{k}' \brackets{ 1 - \akPeng' e^{i\thkPeng'} e^{i\gamma} },
  \label{bsjpsikst_amp}
\end{equation}

\noindent where primed$({}^\prime$) quantities from here and on will be ascociated with the \BsJpsiKst decay only.
Note the absence of the subression factor $\epsilon$ in \equref{bsjpsikst_amp}, which implies that the penguin diagram
contributes as much as the tree diagram does to the total amplitude, in contrast to the \BsJpsiPhi decay.

As mentioned in \secref{penguin_formalism} the \BsJpsiKst chanell provides acess to \Acp{\rm dir} only.
Thus, additional information is required, via the $\Hobs{k}'$ observable, in order to probe $(\akPeng,\thkPeng)$.
Both observables are based on measurements that took place in \chapref{Data_Analysis} of the current thesis.
The first one\footnote{Note the notation convension \Acp{\rm dir}\parenthesis{\BsJpsiKstPolState{k}} = \Acp{k}.}
is reported in \tabref{bestFitResult}, with the ascociated systematic quoted in \tabref{systematics_acp}.
Whereas the $\Hobs{k}'$ observable is constructed from the \BRof{\BsJpsiKst} reported in \equref{Br_total} and shown in \equref{hobs_jpsikst}.

\begin{subequations}
  \label{hobs_jpsikst}
  \begin{alignat}{2}
  % \Hobs{0}         & = 0.99 \pm 0.07\:\text{(stat)} \pm 0.06\:\text{(syst)} \pm 0.27\:(|\formFctr{0}'/\formFctr{0}|) && = 0.99 \pm 0.28\:,\\
  % \Hobs{\parallel} & = 0.91 \pm 0.14\:\text{(stat)} \pm 0.08\:\text{(syst)} \pm 0.21\:(|\formFctr{\parallel}'/\formFctr{\parallel}|) && = 0.91 \pm 0.27\:,\\
  % \Hobs{\perp}     & = 1.47 \pm 0.14\:\text{(stat)} \pm 0.11\:\text{(syst)} \pm 0.28\:(|\formFctr{\perp}'/\formFctr{\perp}|) && = 1.47 \pm 0.33\:.
  \Hobs{0}'         & = 0.99 \pm 0.07\:\text{(stat)} \pm 0.06\:\text{(syst)} \pm 0.27\:(\text{hadr}) && = 0.99 \pm 0.28\:, \label{hobs_jpsikst_long}\\
  \Hobs{\parallel}' & = 0.91 \pm 0.14\:\text{(stat)} \pm 0.08\:\text{(syst)} \pm 0.21\:(\text{hadr}) && = 0.91 \pm 0.27\:, \label{hobs_jpsikst_para} \\
  \Hobs{\perp}'     & = 1.47 \pm 0.14\:\text{(stat)} \pm 0.11\:\text{(syst)} \pm 0.28\:(\text{hadr}) && = 1.47 \pm 0.33\:. \label{hobs_jpsikst_perp}
  \end{alignat}
\end{subequations}

\noindent Where "hadr" referes to the hadronisation factor $|\formFctr{k}/\formFctr{k}'|$ necessary for the construction of $\Hobs{k}'$.
The hadronization factors are calculated theoretically. Such calcualtions are available in {\color{red} somewhere} and were
used in \equref{hobs_jpsikst}. Note how the error on $\Hobs{k}'$ is dominated by the hadronization factor.

% \begin{align}
% \left|\frac{\mathcal{A}'_0(\BsJpsiPhi)}{\mathcal{A}_0(\BsJpsiKst)}\right| & = 1.23 \pm 0.16\:,\label{Eq:AmpRat_JpsiKstar_long}\\
% %%%
% \left|\frac{\mathcal{A}'_{\parallel}(\BsJpsiPhi)}{\mathcal{A}_{\parallel}(\BsJpsiKst)}\right| & = 1.28 \pm 0.15\:,\\
% %%%
% \left|\frac{\mathcal{A}'_{\perp}(\BsJpsiPhi)}{\mathcal{A}_{\perp}(\BsJpsiKst)}\right| & = 1.20 \pm 0.12\:,\label{Eq:AmpRat_JpsiKstar_perp}
% \end{align}

Prior to any combination the penguin parameters have been estimated based on the \BsJpsiKst channell only.
The results on \DeltaPhis{k}~\cite{bsjpsikst-paper}, are shown in \equref{delta_phis_jpsikst},

\begin{subequations}
  \label{delta_phis_jpsikst}
  \begin{align}
    \DeltaPhisJpsiPhi{0}         & = +0.001^{+0.100}_{-0.033} \:\text{rad},\\
    \DeltaPhisJpsiPhi{\parallel} & = +0.031^{+0.059}_{-0.052} \:\text{rad},\\
    \DeltaPhisJpsiPhi{\perp}     & = -0.046^{+0.022}_{-0.028} \:\text{rad}.
  \end{align}
\end{subequations}

\noindent Where most of the error is statistical in nature.The $\chi^2$ fit performed here is similar
to the one described later. Hence its details are postopned untill \secref{penguin_chi2_fit}.


\subsection{The \BdJpsiRho chanell}
The topology of the \BdJpsiRho decay is shown in \figref{bs2jpsirho_diagram}. The amplitude structure is identical
to that of the \BsJpsiKst mode resulting in \equref{bsjpsirho_amp}. same The finals state is a cp ei the chanells

\begin{figure}[h]
  \centering
  {\sffamily %%BoundingBox: -5 0 121 170
%%HiResBoundingBox: -5 0 120.57008 169.36447

\begin{fmffile}{Figures/Chapter5/tree_jpsirho}
  \fmfframe(17,-25)(31,-25){
    \begin{fmfgraph*}(115,170)
      \fmfstraight
      \fmfleft{i0,i1,i2,i3,i4,i5}
      \fmfright{o0,o1,o2,o3,o4,o5}
      \fmf{fermion,tension=3.5,label.side=left,label=$\bquark$}{v2,i3}
      \fmf{fermion,label=$\cquark$,label.side=left}{o4,v2}
      \fmf{fermion,label=$\cquark$,label.side=left}{v3,o3}
      \fmf{fermion,label=$\dquark$,label.side=left,tension=2}{o2,v3}
      \fmf{boson,tension=2.4,label=\Wp,label.side=right,right=0.3}{v2,v3}
      \fmffreeze
      \fmf{phantom,tension=0.3}{v2,v1,v3}
      \fmf{fermion,tension=0.5,label=$\dquark$,label.side=left}{v1,o1}
      \fmf{fermion,tension=0.5,label.side=left,label=$\dquark$}{i2,v1}
      \fmf{plain,right=0.2}{i2,i3}
      \fmf{plain,left=0.2,label=$\Bs$}{i2,i3}
      \fmf{plain,right=0.2,label=$\rho^0$}{o1,o2}
      \fmf{plain,left=0.2}{o1,o2}
      \fmf{plain,right=0.2,label=$\jpsi$}{o3,o4}
      \fmf{plain,left=0.2}{o3,o4}
      \fmflabel{\hspace{-1cm}$\Vcb^*$}{v2}
      \fmflabel{\hspace{-0.7cm} \vspace{0.5cm}$\Vcd$}{v3}
    \end{fmfgraph*}
  }
\end{fmffile}%
\hfill
\begin{fmffile}{Figures/Chapter5/penguin_jpsirho}
  \fmfframe(17,-25)(31,-25){
    \begin{fmfgraph*}(115,170)
      \fmfstraight
      \fmfleft{i0,i1,i2,i3,i4,i5}
      \fmfright{o0,o1,o2,o3,o4,o5}
      \fmf{fermion,tension=1.8,label.side=left,label=$\bquark$}{v5,i3}
      \fmf{fermion,tension=1.5,right=0.2,label.side=left,label={\hspace*{18pt}\uquark,,\cquark,,\tquark}}{v2,v5}
      \fmf{gluon,tension=2}{v4,v2}
      \fmf{dbl_dashes,tension=0}{v4,v2}
      \fmf{fermion,tension=0.3,right=0.2,label.side=left }{v3,v2}
      \fmf{boson,tension=0.6,left=0.3,label=\Wp,label.side=left}{v3,v5}
      \fmf{fermion,label=$\cquark$,tension=0.9,right=0.3,label.side=left}{o4,v4}
      \fmf{fermion,label=$\cquark$,c,tension=0.9,right=0.3,label.side=left}{v4,o3}
      \fmf{fermion,label=$\dquark$,label.side=left}{o2,v3}
      \fmffreeze
      \fmf{fermion,tension=0.7,label=$\dquark$,label.side=left}{v1,o1}
      \fmf{fermion,tension=1,label.side=left,label=$\dquark$}{i2,v1}
      %\fmf{phantom,tension=0.4}{v4,v1}
      \fmf{phantom,tension=0.4}{v3,v1,v5}
      \fmf{plain,right=0.2}{i2,i3}
      \fmf{plain,left=0.2,label=$\Bs$}{i2,i3}
      \fmf{plain,right=0.2,label=$\rho^0$}{o1,o2}
      \fmf{plain,left=0.2}{o1,o2}
      \fmf{plain,right=0.2,label=$\jpsi$}{o3,o4}
      \fmf{plain,left=0.2}{o3,o4}
    \end{fmfgraph*}
  }
\end{fmffile}
}
  \caption{Leading order tree diagram of the decay \BdJpsiRho.}
  \label{bs2jpsirho_diagram}
\end{figure}

\begin{equation}
  \mathcal{A} \parenthesis{\BdJpsiRhoPolState{k}} = -\lambda \formFctr{k}'' \brackets{ 1 - \akPeng'' e^{i\thkPeng''} e^{i\gamma} },
  \label{bsjpsirho_amp}
\end{equation}

\noindent where the ${}^{\prime\prime}|$ from here and on labels parameters related to the \BdJpsiRho decay.
Note again the absence of the subression factor $\epsilon$.

The \BdJpsiRho chanell provides access to both \Acp{\rm dir}, \Acp{\rm mix} since the final state $\jpsi\rho$ is
a CP eigenstate and thus $\BdBdbarSyst$ oscilations are active. Both of the above observables are measured in the
time dependant analysis of \BdJpsipipi decays from \lhcb~\cite{Aaij:2014vda}. The penguin parameters as extracted
in Section 5.5.3. of ~\cite{DeBruyn-thesis} are shown in \equref{delta_phis_jpsirho}.

\begin{subequations}
  \label{delta_phis_jpsirho}
  \begin{align}
    \DeltaPhisJpsiPhi{0}         & = -0.000^{+0.011}_{-0.014}\:\text{rad},\\
    \DeltaPhisJpsiPhi{\parallel} & = +0.001^{+0.012}_{-0.017}\:\text{rad},\\
    \DeltaPhisJpsiPhi{\perp}     & = +0.003^{+0.012}_{-0.016}\:\text{rad}.
  \end{align}
\end{subequations}

\noindent Where the most of the uncertainty is statistical in nature. The rest comes from the assumption that \grpsuthree is
unbroken. The exact treatment of this uncertainty is postponed for \secref{su3_breaking}. Note the small uncertainty on
$\DeltaPhisJpsiPhi{k}$ comapred to \equref{delta_phis_jpsikst}. According to~\cite{DeBruyn-thesis} the last is attributed
to the small value of the penguin parameter $\thkPeng{k}''$, which is around $90^{\degrees}$.

\subsection{Combined $\chi^2$ Fit}
\label{penguin_chi2_fit}

\begin{itemize}
 \item Say what goes int he chi2
 \item show formula
 \item mention constrains on phis and phid and gamma
 \item mention the hadronic function assumption for the 2 contrl chanells
 \item point ou the reduced uncertainty wrt to teory calcuations.
\end{itemize}

By invoking \grpsuthree symmetry, as shown in \equref{su3_apply}, the acpmix and acp dir and ah obserbales from the chanells \BsJpsiPhi \BdJpsiRho \BsJpsiKst
are used to define a $\chi^2$, see ref. Minimizing the $\chi^2$ yields the penguin paramtgers $\alpha_f$ $\theta_f$ whic are
then translated, using \equref{tandelta}, to the final estimate for the penguin phase shift on \phis.

\begin{equation}
\akPeng= \akPeng' = \akPeng'' \qquad \thkPeng = \thkPeng' = \thkPeng'',
\label{su3_apply}
\end{equation}

\begin{equation}
  \centering
  \chi^2_k = \sum \frac{O^{\rm theo}_k - O^{\rm exp}_k} {\sigma(O^{\rm exp}_k)},
  \label{chisq_form}
\end{equation}

\begin{equation}
  \centering
  O_k \in \left\{ \parenthesis{\Acp{\rm dir}}'_k, \Hobs{k}', \parenthesis{\Acp{\rm dir}}''_k, \parenthesis{\Acp{\rm mix}}''_k, \Hobs{k}'' \right\}
%   O_k \in \left{ {\Acp{\rm dir}}', \Hobs{k}', {\Acp{\rm dir}}'', {\Acp{\rm mix}}'', \Hobs{k}'' \right}
  \label{chisq_form_with}
\end{equation}

Hypothesis of the combined fit will go here.

\subsubsection{Assumptions and Constrains}
The following inputs have been used for the fit.
Also the teoretical vs expermental branching rations is corrected for.


\begin{table}[!h]
  \center
  \begin{tabular}{c c c}
    \hline
    parameter & value & source \\
    \hline
    $\gamma$           & $\left(73.2_{-7.0}^{+6.3}\right)^{\circ}$ & CKMfitter \cite{Charles:2015gya} \\
    $\phiS{k}$         & some value & \lhcb \cite{phis-3fb-paper} \\
    $\phid^k$            & some value & \lhcb \cite{phis-3fb-paper} \\
    \hline
  \end{tabular}
  \caption{\small Constrains enterign the $\chi^2$ fit.}
  \label{chi2_fit_constrains}
\end{table}

\begin{equation}\label{Eq:Gamma_GC}
\gamma  = \left(73.2_{-7.0}^{+6.3}\right)^{\circ} \quad\text{(CKMfitter \cite{Charles:2015gya})}\:,
\end{equation}
