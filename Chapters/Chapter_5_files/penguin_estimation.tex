

The recent \lhcb measurement of \phis comes with the measurement of the $\lambda_{\jpsi\phi}$ of \equref{lambda_cpv}~\cite{phis-3fb-paper}.
These two parameters, shown in \equref{phis_lambda_result}, can be used to estimate the penguin parameters $(\aPeng{},\thPeng{})$.
Based on \equref{cp_asym_lambda} and according to the formalism in \secref{penguin_formalism}, the $\DeltaPhis{}$ estimation
from \BsJpsiPhi decays comes with an uncertainty of about $0.05$ rad, see Eq. 5.125 of~\cite{DeBruyn-thesis}, which is not precise
enough given the central value of $\phis^{\jpsi\phi}$.

In order to increase the precision on \DeltaPhis{} the \grpsuthree flavour symmetry, also known as $U-spin$ symmetry {\color{red} rob or sth},
is invoked in order to extract \DeltaPhis{} from similar to \BsJpsiPhi chanells.
Note that the information from those chanells entering through the observables, \equref{bsjpsiphi_peng_acp_obs} and \equref{hobs_def}
is polarization dependant. This is due to the fact that new physics dynamics might enter in a different way in each polarization as mentioned {\color{red}somewhere in krystof.}
This section summarizes the details of each of the aditional chanells, \BsJpsiKst and \BdJpsiRho as well as the assumptions and external inputs of the $\chi^2$ fit.
Results are discussed at the end.

\begin{equation}
  \centering
  \phis^{\jpsi\phi}     =  -0.058 \pm 0.049(\text{stat})  \;\; \text{rad}, \;\;\;\;\;
  \lambda^{\jpsi\phi}   =  +0.964 \pm 0.019(\text{stat}).
  \label{phis_lambda_result}
\end{equation}


\subsection{The \BsJpsiKst chanell}

The \BsJpsiKst chanell is a flavour specific decay with the same topology as \BsJpsiPhi, see \figref{bs2jpsikst_tree}.
The amplitude of \BsJpsiKst, shown in \equref{bsjpsikst_amp}, is parametrised following the same concept as in the case of \BsJpsiPhi.

\begin{figure}[h]
  \centering
  {\sffamily \input{Figures/Chapter5/tree_penguin_jpsikst}}
  \caption{Leading order tree diagram of the decay \BsJpsiPhi.}
  \label{bs2jpsikst_tree}
\end{figure}

\begin{equation}
  \mathcal{A} \parenthesis{\BsJpsiKstPolState{k}} = -\lambda \formFctr{k}' \brackets{ 1 - \akPeng' e^{i\thkPeng'} e^{i\gamma} },
  \label{bsjpsikst_amp}
\end{equation}

\noindent where the ${}^\prime$ from here and on labels parameters related to the \BsJpsiKst decay. Note the absence of the
subression factor $\epsilon$, which implies that the penguin diagram contributes equally to the total amplitude, in contrast
to the \BsJpsiPhi decay.

As mentioned in \secref{penguin_formalism} \BsJpsiKst provides acess to \Acp{\rm dir} only. Additional information is required via the $\Hobs{}$,
in order to probe $(\akPeng,\thkPeng)$. Both observables are measured in \chapref{Data_Analysis} of the current thesis.
The first one, \Acp{\rm dir}\parenthesis{\BsJpsiKstPolState{k}} is reported in \tabref{bestFitResult}\footnote{including systematics from \tabref{systematics_acp}}.
Whereas the $\Hobs{k}$ observable is constructed from the \BRof{\BsJpsiKst} reported in \equref{Br_total} and shown in \equref{hobs_jpsikst}.

\begin{alignat}{2}
  % \Hobs{0}         & = 0.99 \pm 0.07\:\text{(stat)} \pm 0.06\:\text{(syst)} \pm 0.27\:(|\formFctr{0}'/\formFctr{0}|) && = 0.99 \pm 0.28\:,\\
  % \Hobs{\parallel} & = 0.91 \pm 0.14\:\text{(stat)} \pm 0.08\:\text{(syst)} \pm 0.21\:(|\formFctr{\parallel}'/\formFctr{\parallel}|) && = 0.91 \pm 0.27\:,\\
  % \Hobs{\perp}     & = 1.47 \pm 0.14\:\text{(stat)} \pm 0.11\:\text{(syst)} \pm 0.28\:(|\formFctr{\perp}'/\formFctr{\perp}|) && = 1.47 \pm 0.33\:.
  \Hobs{0}         & = 0.99 \pm 0.07\:\text{(stat)} \pm 0.06\:\text{(syst)} \pm 0.27\:(\text{hadr}) && = 0.99 \pm 0.28\:,\\
  \Hobs{\parallel} & = 0.91 \pm 0.14\:\text{(stat)} \pm 0.08\:\text{(syst)} \pm 0.21\:(\text{hadr}) && = 0.91 \pm 0.27\:,\\
  \Hobs{\perp}     & = 1.47 \pm 0.14\:\text{(stat)} \pm 0.11\:\text{(syst)} \pm 0.28\:(\text{hadr}) && = 1.47 \pm 0.33\:.
  \label{hobs_jpsikst}
\end{alignat}

\noindent Where "hadr" referes to the hadronisation factor $|\formFctr{k}'/\formFctr{k}|$ necessary for the construction of $\Hobs{k}$.
The hadronization factors need to be calculated theoretically. Such calcualtions are available in {\color{red} somewhere} and were
used in \equref{hobs_jpsikst}. Note how the hadronization factor dominates the error on the $\Hobs{k}$, as it was mentioned in \secref{penguin_formalism}.

% \begin{align}
% \left|\frac{\mathcal{A}'_0(\BsJpsiPhi)}{\mathcal{A}_0(\BsJpsiKst)}\right| & = 1.23 \pm 0.16\:,\label{Eq:AmpRat_JpsiKstar_long}\\
% %%%
% \left|\frac{\mathcal{A}'_{\parallel}(\BsJpsiPhi)}{\mathcal{A}_{\parallel}(\BsJpsiKst)}\right| & = 1.28 \pm 0.15\:,\\
% %%%
% \left|\frac{\mathcal{A}'_{\perp}(\BsJpsiPhi)}{\mathcal{A}_{\perp}(\BsJpsiKst)}\right| & = 1.20 \pm 0.12\:,\label{Eq:AmpRat_JpsiKstar_perp}
% \end{align}


\subsection{The \BsJpsiRho chanell}
The topology of the \BsJpsiRho decay is shown in \figref{jpsirho}. The amplitude structure is very similar to that
of \BsJpsiKst resulting in \equref{bsjpsirho_amp}. same The finals state is a cp ei the chanells

\begin{figure}[h]
  \centering
  {\sffamily %%BoundingBox: -5 0 121 170
%%HiResBoundingBox: -5 0 120.57008 169.36447

\begin{fmffile}{Figures/Chapter5/tree_jpsirho}
  \fmfframe(17,-25)(31,-25){
    \begin{fmfgraph*}(115,170)
      \fmfstraight
      \fmfleft{i0,i1,i2,i3,i4,i5}
      \fmfright{o0,o1,o2,o3,o4,o5}
      \fmf{fermion,tension=3.5,label.side=left,label=$\bquark$}{v2,i3}
      \fmf{fermion,label=$\cquark$,label.side=left}{o4,v2}
      \fmf{fermion,label=$\cquark$,label.side=left}{v3,o3}
      \fmf{fermion,label=$\dquark$,label.side=left,tension=2}{o2,v3}
      \fmf{boson,tension=2.4,label=\Wp,label.side=right,right=0.3}{v2,v3}
      \fmffreeze
      \fmf{phantom,tension=0.3}{v2,v1,v3}
      \fmf{fermion,tension=0.5,label=$\dquark$,label.side=left}{v1,o1}
      \fmf{fermion,tension=0.5,label.side=left,label=$\dquark$}{i2,v1}
      \fmf{plain,right=0.2}{i2,i3}
      \fmf{plain,left=0.2,label=$\Bs$}{i2,i3}
      \fmf{plain,right=0.2,label=$\rho^0$}{o1,o2}
      \fmf{plain,left=0.2}{o1,o2}
      \fmf{plain,right=0.2,label=$\jpsi$}{o3,o4}
      \fmf{plain,left=0.2}{o3,o4}
      \fmflabel{\hspace{-1cm}$\Vcb^*$}{v2}
      \fmflabel{\hspace{-0.7cm} \vspace{0.5cm}$\Vcd$}{v3}
    \end{fmfgraph*}
  }
\end{fmffile}%
\hfill
\begin{fmffile}{Figures/Chapter5/penguin_jpsirho}
  \fmfframe(17,-25)(31,-25){
    \begin{fmfgraph*}(115,170)
      \fmfstraight
      \fmfleft{i0,i1,i2,i3,i4,i5}
      \fmfright{o0,o1,o2,o3,o4,o5}
      \fmf{fermion,tension=1.8,label.side=left,label=$\bquark$}{v5,i3}
      \fmf{fermion,tension=1.5,right=0.2,label.side=left,label={\hspace*{18pt}\uquark,,\cquark,,\tquark}}{v2,v5}
      \fmf{gluon,tension=2}{v4,v2}
      \fmf{dbl_dashes,tension=0}{v4,v2}
      \fmf{fermion,tension=0.3,right=0.2,label.side=left }{v3,v2}
      \fmf{boson,tension=0.6,left=0.3,label=\Wp,label.side=left}{v3,v5}
      \fmf{fermion,label=$\cquark$,tension=0.9,right=0.3,label.side=left}{o4,v4}
      \fmf{fermion,label=$\cquark$,c,tension=0.9,right=0.3,label.side=left}{v4,o3}
      \fmf{fermion,label=$\dquark$,label.side=left}{o2,v3}
      \fmffreeze
      \fmf{fermion,tension=0.7,label=$\dquark$,label.side=left}{v1,o1}
      \fmf{fermion,tension=1,label.side=left,label=$\dquark$}{i2,v1}
      %\fmf{phantom,tension=0.4}{v4,v1}
      \fmf{phantom,tension=0.4}{v3,v1,v5}
      \fmf{plain,right=0.2}{i2,i3}
      \fmf{plain,left=0.2,label=$\Bs$}{i2,i3}
      \fmf{plain,right=0.2,label=$\rho^0$}{o1,o2}
      \fmf{plain,left=0.2}{o1,o2}
      \fmf{plain,right=0.2,label=$\jpsi$}{o3,o4}
      \fmf{plain,left=0.2}{o3,o4}
    \end{fmfgraph*}
  }
\end{fmffile}
}
  \caption{Leading order tree diagram of the decay \BsJpsiRho.}
  \label{bs2jpsikst_penguin}
\end{figure}

\begin{equation}
  \mathcal{A} \parenthesis{\BsJpsiRhoPolState{k}} = -\lambda \formFctr{k}'' \brackets{ 1 - \akPeng'' e^{i\thkPeng''} e^{i\gamma} },
  \label{bsjpsirho_amp}
\end{equation}

\noindent where the ${}^{\prime\prime}|$ from here and on labels parameters related to the \BsJpsiRho decay. Note again
the absence of the subression factor $\epsilon$.

The \BsJpsiRho chanells provides acess to both \Acp{\rm dir}, \Acp{\rm mix} since the final state $\jpsi\rho$ is a CP eigenstate
and thus $\BdBdbarSyst$ oscilations are active. Both of the above observables are available from \lhcb, based on the
time dependant analysis of \BdJpsiRho decays~\cite{Aaij:2014vda} and are listed in table 5.5 of~\cite{DeBruyn-thesis}.

H observable is also used, listed in blah
Mention details on the factoriazable vs non facroizable shit.

All of the above is again polarization dependant. in a polarization dependent


\subsection{$\chi^2$ fit}
\label{penguin_chi2_fit}

By invoking \grpsuthree symmetry, as shown in \equref{su3_apply}, the acpmix and acp dir and ah obserbales from the chanells \BsJpsiPhi \BsJpsiRho \BsJpsiKst
are used to define a $\chi^2$, see ref. Minimizing the $\chi^2$ yields the penguin paramtgers $\alpha_f$ $\theta_f$ whic are
then translated, using \equref{tandelta}, to the final estimate for the penguin phase shift on \phis.

\begin{equation}
a_i = a'_i\:,\qquad \theta_i = \theta'_i\:,
\label{su3_apply}
\end{equation}

and similartly for \BsJpsiRho

Hypothesis of the combined fit will go here.
\subsubsection{Constrains}
The following inputs have been used for the fit.
Also the teoretical vs expermental branching rations is corrected for.


\subsubsection{\grpsuthree breaking}
\label{su3_breaking}

The fact that the \grpsuthree syummetry is a broken symmetry is taken into account in the fit by re parametrizing
$\alpha_f$ and $\theta_f$ as shown in \equref{su3_breaking}.

\begin{equation}
a_i \to \xi \; \alpha_i \quad \theta_i \to \delta + \theta'_i
\label{su3_breaking}
\end{equation}

As there are no dedicated studies for \grpsuthree breaking. It is assumed that $\xi=1$ and $\delta=0$ but include uncertainties
on these paorameters as gaussian constrains in the $\chi^2$ fit. The dependance of $\Delta\phiS{SM,peng}$ on the $\xi$ and $\delta$
uncertainties was found to be irrelavant, {\color{red} see jpsiKst paper}, due to the particular structure of \equref{tanddelta}.
