
The \muonID \cite{LHCb-PUB-2009-013,LHCb-PUB-2010-002} that is used both in the High Level Trigger (\hlt) \cite{LHCb-PUB-2011-017}
and in the offline event processing, or simply {\it offline}, has been revisited in view of the \lhc \runtwo.
Note that during \runone the \hlt version of the \muonID was different from the one used offline in order
to conform with the limitations imposed by the available computing resources. The main reasons for revisiting
the software are; First to unify the \muonID used in \hlt and offline.
Second to optimize the trigger efficiency of low transverse momentum muons.
In particular, the \muonID has undergone a significant re-factory resulting in a modularized common code base
between \hlt and offline \cite{kevinThesis}. Because of that \muonID in the \hltone and offline
are identical \footnote{The \muonID in \hlttwo was the same as offline already in \runone.},
which combined with the novel online detector alignment and calibration endeavour, introduced in \secref{det_trigger},
allows for the trigger to produce offline quality data. After the modularization readability and maintainability
of the code is significantly improved. An increase in performance of the \muonID code both in CPU and memory usage
is also achieved.

The algorithm sequence of the \hltone muon trigger lines\footnote{The term trigger line is defined in \secref{det_trigger}}
during \runone was tuned in order to comply with the output rate limitations given the existing computing infrastructure,
named {\it Event Filter Farm} (EFF). Since then the EFF has undergone a significant upgrade resulting in an
increased processing power along with increased storage capabilities. In view of this infrastructure upgrade
as well as the \runtwo \lhc running conditions the total output rate of the entire trigger system was increased
from 5\khz to 12.5\khz, see \figref{det_trigger_scheams}. The impact on the \hltone is that its output rate is
roughly doubled and additional processing time is available. This boost in computing power allowed for changes
in the \muonID algorithm sequence described in the following subsection.

\subsection{\hltone muon lines algorithm sequence}
\label{hlt1run2}

The role of the trigger system and the decisions that it provides is presented in \secref{det_trigger},
where the concept of a trigger line is also introduced. The \hltone muon trigger lines are
mainly organized in single muon and dimuon ones; both exploit the same muon identification procedure which
is described in the current subsection. The efficiency of the muon lines during \runone is shown in \figref{det_run_one_muon_line_eff}.
It can be observed that a low \pt turn on and a large efficiency loss in the dimuon lines are present.
In order to understand the origin of this efficiency loss the \hltone algorithm sequence \cite{LHCb-PUB-2011-017}
is analyzed further. The sequence is shown  in \figref{hlt1_algo_seq}. A brief summary of the
updated sequence is given in what follows, whereas the full description can be found in \cite{kevinThesis}.

During \runone the muon identification starts directly from \veloTracks\footnote{These are tracks built only with the \velo sub-detector, see \figref{track_types}}
which are then filtered out keeping good muon candidates. The filtering is done by the \mvm algorithm \cite{LHCb-PUB-2011-017}.
Subsequently the \FwD algorithm uses information from the rest of the tracking stations, see \secref{det_tracking}, and
upgrades the \veloTracks to long tracks\footnote{These are tracks built using all the tracking sub-detectors, see \figref{track_types}}
After this, the \isMuon\footnote{\isMuon is an alias for the \muonID software.} algorithm is applied.
In addition there are some quality cuts between each algorithm step, such as the number of hits produced in the
\velo sub-detector by a track or the fitted track $\chisq/\nDoF$ as well as momentum cuts applied after the \FwD
tracking algorithm. The minimum momentum and transverse momentum is 6 \gevc and 0.5 \gevc respectively.

\begin{figure}[t]
  \centering
  \tikzsetnextfilename{hlt1_alg_seq_old}
  \scalebox{1}{\begin{tikzpicture}[node distance=.5cm and 2.cm]
       \node [block, source] (test-0) {\veloTracks};
       \node [block, op, below=of test-0, ] (test-1) {\mvm};
        \node [block, op,  below=of test-1] (test-2) {\FwD};
        \node [block, op, below=of test-2] (test-3) {\isMuon};
        \node [block, op, below=of test-3] (test-4) {\fitTrack};
        \node [block, op, below=of test-4] (test-5) {\diMuons };
        \node [block, cut, below=of test-5] (test-6) {Additional cuts};
        \node [block, sink, below=of test-6] (test-7) {Decision};
       \path [line] (test-0) -- (test-1);
       \path [line] (test-1) -- (test-2);
       \path [line] (test-2) -- (test-3);
       \path [line] (test-3) -- (test-4);
       \path [line] (test-4) -- (test-5);
       \path [line] (test-5) -- (test-6);
       \path [line] (test-6) -- (test-7);
\end{tikzpicture}
}
  \tikzsetnextfilename{hlt1_alg_seq_new}
  \scalebox{1}{\begin{tikzpicture}[node distance=.5cm and 2.cm, every node/.style={rectangle,fill=white}]
       \node [block, source] (test-0) {Velo Tracks};
       \node [block, cut, below=of test-0, ] (test-1) {$\pt > 500$ MeV/c};
       \node [block, op, below=of test-1,  xshift=2cm] (test-2) {MatchVeloTTMuon};
       \node [block, op, below=of test-2] (test-20) {Forward Tracking};
       \node [block, op, below=of test-20] (test-4) {Fit track};
        \node [block, source, below=of test-1,  xshift=-2cm] (test-3) {Get long track};
        \node [block, cut, below=of test-4,   xshift=-2cm] (test-5) {Cuts on track};
        \node [block, op, below=of test-5] (test-6) {IsMuon};
        \node [block, op, below=of test-6] (test-7) {Make dimuons };
	\node [block, cut, below=of test-7] (test-8) {Additional cuts};
	\node [block, sink,below=of test-8] (test-9) {Decision};
       \path [line] (test-0) -- (test-1);
       \path [line] (test-1) -- node {no} (test-2);
       \path [line] (test-1) -- node {yes} (test-3);
       \path [line] (test-2) -- (test-20);
       \path [line] (test-20) -- (test-4);
       \path [line] (test-3) -- (test-5);
       \path [line] (test-4) -- (test-5);
       \path [line] (test-5) -- (test-6);
       \path [line] (test-6) -- (test-7);
       \path [line] (test-7) -- (test-8);
       \path [line] (test-8) -- (test-9);
\end{tikzpicture}
}
 \caption{ \runone and \runtwo \hltone algorithms compared. }
  \label{hlt1_algo_seq}
\end{figure}

The efficiency breakdown, performed in \cite{kevinThesis}, of the \runone \hltone muon lines showed that the
main source of efficiency loss is due to quality and momentum cuts. Also the \mvm algorithm reduces the
efficiency by roughly $4\%$. Given these diagnostics plus the \runtwo capabilities of the EFF, the updated
\hltone muon lines algorithm sequence is described in the following paragraph.

For the majority of the tracks, which have transverse momentum larger than 500 \mevc, the sequence is different with respect to
Run I: full tracking is performed for these tracks, so that the start for muon finding is already with long tracks.
After soft quality cuts these get passed through the standard \isMuon algorithm in order to be identified;
positively identified muons are then combined into dimuons as in Run I.
The combination of the use of long tracks and standard muon identification is much more efficient than the
procedure in Run I, and is exactly the same as the one of offline reconstruction.
Tracks with transverse momentum smaller than 500 \mevc cannot be immediately upgraded to long tracks for timing reasons.
However some of them will be muons, and the efficiency for these could be recovered using a different strategy,
explained in \secref{mvm_algorrithm}.
