
The muon identification software \cite{LHCb-PUB-2009-013,LHCb-PUB-2010-002} (\muonID),
that is used both in the High Level Trigger (\hlt) \cite{LHCb-PUB-2011-017} and offline reconstruction has been
revisited in view of the \lhc \runtwo. The main goals were: First to use the same \muonID and offline as much as
possible. Second to optimize the efficiency of muon identification of low momentum muons in the \hlt.
In particular, the \muonID has undergone a significant refactory resulting in a modularised common code base
between \hlt and offline event processing {\color{red} Site kevin or some pub note.}. Bacause of that
muon indentification in the \hltone and offline are identical \footnote{The \muonID in \hlttwo was the same as offline already in \runone.},
which combined with the novel {\it online detector alignment and calibration} endeavour, {\color{red} sitation} allows for the trigger to produce offline quality data.
After the modularization readability and maintenance of the code were improved. An increase in performance
of the \muonID code was also achieved both in CPU and memory usage.

The algorithm sequence of the \hltone muon trigger lines{\it defined in blah} during \runone was tuned in order to
comply with the output rate limitations given the existinng computing infrastructure,
named {\it Event Filter Farm} (EFF). Since then the EFF has undergone a significant upgrade resulting in an
increased processing power along with increased storage capabilities. In view of this infrastructure upgrade
as well as the \runtwo \lhc running conditions the total output rate{\color{red} define rate in the detector chapter.} of the entire trigger system was increased
from 5\khz to 12.5\khz. The impact on the \hltone is that its output rate is roughly doubled and additional processing time is available.
This boost in computing power allowed for changes in the \muonID alogorithm sequence regarding muons above 500 \mevc.
Particularly the preselection of the above tracks that was present during \runone, see \figref{hlt1_algo_seq},
was removed. Instead all these muons are passed directly to the next stages of the \muonID.

\subsection{\hltone muon lines algorithm sequence}
\label{hlt1run2}

The role of the trigger system and the decitions that it provides were presented in \secref{}.
The concept of calssifing and selecting events based on different physics signatures is implemented by
the so called {\it trigger line}, also introcude in the same section. The \hltone muon trigger lines are
mainly organized in single muon and dimuon ones; both exploit the same muon identification procedure which
is described in the current subsection. Between \runone and \runtwo, as mentioned earlier, both muon identification
and also some of the muon line's algorithms were reoptimized and others were added. Only the differences due
to the MuonID are underlined in the current chapter. { More details for the lines can be found in blah.}

The efficiency of the muon lines during \runone is shown in \figref{fig:hlt1_eff_run1}.
It can be observed that a low \pt turn on and a large efficiency loss in the dimuon lines are present.
In order to understand the origin of this efficiency loss the \hltone algorithm sequence is analysed further.


\begin{figure}[h!]
  \centering
  \includegraphics[trim=11cm 20.5cm 2cm 3cm, clip=true,scale=1.]{Figures/Chapter3/hlt1_muon_eff_run1.pdf}
  \caption{ Single and dimuon \hltone line efficiencies in \runone \cite{LHCb-PROC-2014-005}. }
  \label{fig:hlt1_eff_run1}
\end{figure}


The \hltone muon trigger lines sequence during \runone is shown in \figref{fig:hlt1_algo_seq}(a).
Only a brief summary of this procedure is given here,  while more details can be found in \cite{LHCb-PUB-2011-017}.
During \runone the muon identification started directly from \velo tracks,{\color{red} define in blah}
and identified muons with a simple extrapolation, done by the \mvm algorithm\cite{LHCb-PUB-2011-017}.
Subsequently those potential muons were upgraded to long tracks, {\color{red} define in blah}, by the
\FwD tracking algorithm. After this, the \kw{IsMuon} algorithm was applied {\color{red} site isMuon algo}.
In addition there were some quality cuts between each algorithm step, such as the number of \velo hits
of the track or the fitted track $\chisq_{\text {nDoF}}$ as well as momentum cuts applied after the \FwD
tracking algorithm. The minimum momentum and transverse momentum was 6 \gevc and 0.5 \gevc respectively.

\begin{figure}[h!]
  \centering
  \tikzsetnextfilename{hlt1_alg_seq_old}
  \scalebox{1}{\begin{tikzpicture}[node distance=.5cm and 2.cm]
       \node [block, source] (test-0) {\veloTracks};
       \node [block, op, below=of test-0, ] (test-1) {\mvm};
        \node [block, op,  below=of test-1] (test-2) {\FwD};
        \node [block, op, below=of test-2] (test-3) {\isMuon};
        \node [block, op, below=of test-3] (test-4) {\fitTrack};
        \node [block, op, below=of test-4] (test-5) {\diMuons };
        \node [block, cut, below=of test-5] (test-6) {Additional cuts};
        \node [block, sink, below=of test-6] (test-7) {Decision};
       \path [line] (test-0) -- (test-1);
       \path [line] (test-1) -- (test-2);
       \path [line] (test-2) -- (test-3);
       \path [line] (test-3) -- (test-4);
       \path [line] (test-4) -- (test-5);
       \path [line] (test-5) -- (test-6);
       \path [line] (test-6) -- (test-7);
\end{tikzpicture}
}
  \tikzsetnextfilename{hlt1_alg_seq_new}
  \scalebox{1}{\begin{tikzpicture}[node distance=.5cm and 2.cm, every node/.style={rectangle,fill=white}]
       \node [block, source] (test-0) {Velo Tracks};
       \node [block, cut, below=of test-0, ] (test-1) {$\pt > 500$ MeV/c};
       \node [block, op, below=of test-1,  xshift=2cm] (test-2) {MatchVeloTTMuon};
       \node [block, op, below=of test-2] (test-20) {Forward Tracking};
       \node [block, op, below=of test-20] (test-4) {Fit track};
        \node [block, source, below=of test-1,  xshift=-2cm] (test-3) {Get long track};
        \node [block, cut, below=of test-4,   xshift=-2cm] (test-5) {Cuts on track};
        \node [block, op, below=of test-5] (test-6) {IsMuon};
        \node [block, op, below=of test-6] (test-7) {Make dimuons };
	\node [block, cut, below=of test-7] (test-8) {Additional cuts};
	\node [block, sink,below=of test-8] (test-9) {Decision};
       \path [line] (test-0) -- (test-1);
       \path [line] (test-1) -- node {no} (test-2);
       \path [line] (test-1) -- node {yes} (test-3);
       \path [line] (test-2) -- (test-20);
       \path [line] (test-20) -- (test-4);
       \path [line] (test-3) -- (test-5);
       \path [line] (test-4) -- (test-5);
       \path [line] (test-5) -- (test-6);
       \path [line] (test-6) -- (test-7);
       \path [line] (test-7) -- (test-8);
       \path [line] (test-8) -- (test-9);
\end{tikzpicture}
}
 \caption{ \runone and \runtwo \hltone algoriths compared. }
  \label{hlt1_algo_seq}
\end{figure}

A detailed efficiency breakdown{\color{red} site kevin} of the \runone \hltone muon lines showed that the main source of efficiency
loss is due to the quality and momentum cuts. In addition the \mvm algorithm reduces the efficiency by roughly $4\%$.
As mentioned earlier the updated scheme of \hltone for \runtwo includes two major changes. One in the code sequence of the
\hltone muon lines, more details in {\color{red} site shit}. Two it upgrades the \mvm algorithm for softer muons. The focus of the current chapter is on the later.

The updated \hltone algorith sequence for \runtwo is shown in \figref{fig:hlt1_algo_seq}(b).
For the majority of the tracks, which have transverse momentum larger than 500 MeV, the sequence is different with respect to
Run I: full tracking is performed for these tracks, so that the start for muon finding is already with long tracks.
After soft quality cuts these get passed through the standard \kw{IsMuon} algorithm in order to be identified;
positively identified muons are then combined into dimuons as in Run I.
The combination of the use of long tracks and standard muon identification is much more efficient than the
procedure in Run I, and is exactly the same as the one of offline reconstruction.
Tracks with transverse momentum smaller than 500 MeV cannot be immediately upgraded to long tracks for timing reasons.
However some of them will be muons, and the efficiency for these could be recovered using a different strategy,
explained in \secref{mvm_algorrithm}.
