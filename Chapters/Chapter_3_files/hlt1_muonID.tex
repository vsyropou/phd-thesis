\section{\hltone muon identification}
\label{sec:hlt1_rev}

The algorithm sequence of the \hltone muon lines during \runone was tuned in order to
comply with the output rate limitations given the existinng computing infrastructure,
namely the Event Filter Farm (EFF).
Since then the EFF has undergone a significant upgrade resulting in an increased processing
power along with increased storage capabilities.
In view of this infrastructure upgrade as well as the \runtwo \lhc running conditions
the total output rate of the entire trigger system was increased
from 5\khz to 12.5\khz.
The impact on the \hltone is that its output rate is roughly doubled and additional processing time is available.
This offered chances for optimization of the \hltone muon identificationa and muon trigger lines.

\subsection{\hltone muon lines algorithm sequence}
\label{sec:hlt1run2}

The role of \hltone is to select interesting events based on reduced event information.
The software is divided in lines selecting different physics signatures.
The \hltone muon lines are mainly organized in single muon and dimuon ones; both
exploit the same muon identification procedure which will be described in the following.
Between \runone and \runtwo not only the muon-identification but also some of the lines algorithms were
reoptimized and others were added; the change in the lines will not be described in the present document,
where only the differences due to the MuonID will be underlined.

The efficiency of the muon lines during \runone is shown in \figref{fig:hlt1_eff_run1}.
It can be observed that a low \pt turn on and a large efficiency loss in the dimuon lines are present.
In order to understand the origin of this efficiency loss
the \hltone algorith sequence is analysed further in the current section.


\begin{figure}[h!]
  \centering
  \includegraphics[trim=11cm 20.5cm 2cm 3cm, clip=true,scale=1.]{Figures/Chapter3/hlt1_muon_eff_run1.pdf}
  \caption{ Single and dimuon \hltone line efficiencies in \runone \cite{LHCb-PROC-2014-005}. }
  \label{fig:hlt1_eff_run1}
\end{figure}


The \hltone muon trigger lines sequence during \runone is shown in \figref{fig:hlt1_algo_seq}(a).
Here we only give a brief summary of this procedure while more details can be found in Ref.\cite{LHCb-PUB-2011-017}.
During \runone the muon identification started directly from VELO tracks
which were identified as muons with a simple extrapolation through the \mvm algorithm\cite{LHCb-PUB-2011-017}
which will be detailed in \secref{sec:muon_matching},
before being upgraded to long tracks by the \FwD tracking algorithm.
After this, the \kw{IsMuon} algorithm was applied.
In addition there were some quality cuts between each algorithm step, such as the number of VELO hits
of the track or the fitted track $\chi^2_{\text {dof}}$ as well as momentum cuts applied after the \FwD tracking algorithm.
The minimum momentum and transverse momentum was 6 \gevc and 0.5 \gevc respectively.

\begin{figure}[h!]
  \centering
  \tikzsetnextfilename{hlt1_alg_seq_old}
  \scalebox{1}{\begin{tikzpicture}[node distance=.5cm and 2.cm]
       \node [block, source] (test-0) {\veloTracks};
       \node [block, op, below=of test-0, ] (test-1) {\mvm};
        \node [block, op,  below=of test-1] (test-2) {\FwD};
        \node [block, op, below=of test-2] (test-3) {\isMuon};
        \node [block, op, below=of test-3] (test-4) {\fitTrack};
        \node [block, op, below=of test-4] (test-5) {\diMuons };
        \node [block, cut, below=of test-5] (test-6) {Additional cuts};
        \node [block, sink, below=of test-6] (test-7) {Decision};
       \path [line] (test-0) -- (test-1);
       \path [line] (test-1) -- (test-2);
       \path [line] (test-2) -- (test-3);
       \path [line] (test-3) -- (test-4);
       \path [line] (test-4) -- (test-5);
       \path [line] (test-5) -- (test-6);
       \path [line] (test-6) -- (test-7);
\end{tikzpicture}
}
  \tikzsetnextfilename{hlt1_alg_seq_new}
  \scalebox{1}{\begin{tikzpicture}[node distance=.5cm and 2.cm, every node/.style={rectangle,fill=white}]
       \node [block, source] (test-0) {Velo Tracks};
       \node [block, cut, below=of test-0, ] (test-1) {$\pt > 500$ MeV/c};
       \node [block, op, below=of test-1,  xshift=2cm] (test-2) {MatchVeloTTMuon};
       \node [block, op, below=of test-2] (test-20) {Forward Tracking};
       \node [block, op, below=of test-20] (test-4) {Fit track};
        \node [block, source, below=of test-1,  xshift=-2cm] (test-3) {Get long track};
        \node [block, cut, below=of test-4,   xshift=-2cm] (test-5) {Cuts on track};
        \node [block, op, below=of test-5] (test-6) {IsMuon};
        \node [block, op, below=of test-6] (test-7) {Make dimuons };
	\node [block, cut, below=of test-7] (test-8) {Additional cuts};
	\node [block, sink,below=of test-8] (test-9) {Decision};
       \path [line] (test-0) -- (test-1);
       \path [line] (test-1) -- node {no} (test-2);
       \path [line] (test-1) -- node {yes} (test-3);
       \path [line] (test-2) -- (test-20);
       \path [line] (test-20) -- (test-4);
       \path [line] (test-3) -- (test-5);
       \path [line] (test-4) -- (test-5);
       \path [line] (test-5) -- (test-6);
       \path [line] (test-6) -- (test-7);
       \path [line] (test-7) -- (test-8);
       \path [line] (test-8) -- (test-9);
\end{tikzpicture}
}
 \caption{ \runone and \runtwo \hltone algoriths compared. }
  \label{hlt1_algo_seq}
\end{figure}

The detailed efficiency of the \runone \hltone muon lines is shown in \autoref{fig:effPie} and \autoref{tab:effBefore},
where it can be seen that the main source of efficiency loss is due to the quality and momentum cuts.
In addition the \mvm algorithm reduces the efficiency by roughly $4\%$.

The updated scheme of \hltone for \runtwo includes two major changes.
One in the code sequence of the \hltone muon lines. Two it upgrades the \mvm algorithm for softer muons. The first one is adressed
in the currect section whereas the later in \secref{sec:muon_matching}.

During the course of this work it became apparent that the HLT1 \kw{IsMuon} calculation in \runone
did not use the correct pad sizes for uncrossed hits.
The tool used an internal definition of the pad sizes for each station and
region instead of the information from the hit object itself.
The use of these incorrect pad sizes resulted simply in uncrossed hits being wrongly
rejected, transforming \emph{de facto} the \kw{IsMuon} request into an \kw{IsMuonTight} one.
This does not have any impact in the performances of LHCb, corresponding only to a lower efficiency:
in LHCb physics publications the trigger efficiency is always measured directly on data so that this
does not cause any shift in measurable quantities.

The updated \hltone algorith sequence for \runtwo is shown in \figref{fig:hlt1_algo_seq}(b).
For the majority of the tracks, which have transverse momentum larger than 500 MeV, the sequence is different with respect to
Run I:
full tracking is performed for these tracks, so that the start for muon finding is already with long tracks.
After soft quality cuts these get passed through the standard \kw{IsMuon} algorithm in order to be identified;
positively identified muons are then combined into dimuons as in Run I.
The combination of the use of long tracks and standard muon identification is much more efficient than the
procedure in Run I, and is exactly the same as the one of offline reconstruction.
Furtheremore, correct pad sizes are being used so that the inclusion of uncrossed increases the efficiency further on.

Tracks with transverse momentum smaller than 500 MeV cannot be immediately upgraded to long tracks for timing reasons.
However some of them will be muons, and the efficiency for these could be recovered using a different strategy,
explained in the following.

% \S\ref{sec:matchvelottmuon}.

%% \begin{figure}[!h]
%%   \centering
%%   \includegraphics[trim=0.5cm 1cm 8cm 1cm, clip=true,scale=0.9]{ineff_budget_single.pdf}
%%   \hspace{0.5cm}
%% \caption{ \hltone single muon trigger line effciency versus transverse momentum.
%%   The efficiency improves each time an algorithm of the old \hltone sequence, left of \figref{fig:hlt1_algo_seq}, is removed.
%% {\color{red}, } }
%%   \label{fig:ineff_budget}
%% \end{figure}
