% $Id: introduction.tex 87042 2016-02-03 10:54:01Z fdettori $

\section{Introduction}
\label{sec:Introduction}

A large fraction of the LHCb experiment physics programme is based on the identification of muons in the final state: for example
the measurement of \phis through \BsJpsiPhi decays, the study of CP violation through semileptonic decays, and the measurement
in several rare decays such as \BdKstmumu and \Bsmm.
It is therefore of major importance to maintain and improve the efficiency and purity of identified muons in the Run II data-taking of LHCb.

During the shutdown period between Run I and II, effort has been put to re-optimize the software performing the muon identification.
A unified software that can be run both online in the LHCb software trigger and in the offline reconstruction has been obtained.
In the chapter we describe the current muon identification procedure and the comparison with the previous software.

The muon identification software \cite{LHCb-PUB-2009-013,LHCb-PUB-2010-002} (\muonID),
that is used both in the High Level Trigger (\hlt) \cite{LHCb-PUB-2011-017} and offline reconstruction has been
revisited in view of the \lhc \runtwo.
The main goals of this revisiting were the following two. First to use the same \muonID in \hlt and offline as much as possible.
Second to optimize the efficiency of muon identification in the new HLT software.
In particular, the \muonID has undergone a significant refactory resulting in a modularised common code base
between \hlt and offline event processing.
Bacause of that the \muonID in \hlt and offline is now identical. After the modularization readability and maintenance
of the code were improved.  An increase in performance was also achieved both in CPU and memory usage.

Specifically for \hltone due to looser timing constrains in \runtwo all the tracks above 500 \mevc
are passed directly to the forward trackng and track fitting algorithms without any preselection. This was not the
case in \runone.

The above mentioned changes of the \muonID and in the \hltone are presented in sections \ref{sec:hlt2_rev} and \ref{sec:hlt1_rev} respectively.

Improved performances in terms of trigger efficiency and timing have been obtained and already exploited in recent Run II publications.
\tred{TODO: add citations. }
