The trigger system is responsible for keeping the amount of data that the detector writes to offline storage
at a manageable level, since it is not feasible to save every collision-{\it event}. The three levels of
the \lhcb trigger system decide weather a certain event, is interesting from physics point
of view and should thus be written out to storage. The criteria used to filter out the interesting events 
get progressively more stringent at each trigger level. The whole trigger chain is optimized for maximum
signal purity given the amount of computing and storage infrastructure. In more detail the first trigger
level, \lzero, reads part of the detector at a rate of $40\mhz$, while the subsequent {\it Higher Level Trigger},
\hltone and \hlttwo, wrote out to storage at $5\khz$ (at the end of the 2012 run). The organization of the trigger 
system is done in {\it trigger lines}. Essentially, each trigger line is a collection of selection criteria similar 
to the ones applied during the offline data analysis, but looser. Furthermore, the trigger lines are software
implementations and can thus be quite general as to what they trigger on, or on the other hand they can
be very sophisticated. For example a trigger line could require the presence of two muons forming a vertex 
with an invariant mass window around the nominal \jpsi mass. Or in case of an explicitly decay chanell a 
trigger line could reconstruct the entire $n-\text{body}$ paricle cascade and perhaps require from 
particular quantities to satisfy some criteria, \ie the distance between the primary and secodnary vetex.   

\begin{figure}[t]
  \centering
  \begin{subfigure}{0.5\textwidth}
    \raggedright
    \includegraphics[width=\textwidth]{Figures/Chapter2/LHCb_Trigger_RunIAlgDetail_May2015}
    \caption{}
    \label{det_run_one_trigger}
  \end{subfigure}%
  \hfill%
  \begin{subfigure}{0.5\textwidth}
    \raggedleft
    \includegraphics[width=\textwidth]{Figures/Chapter2/LHCb_Trigger_RunII_May2015}
    \caption{}
    \label{det_run_two_trigger}
  \end{subfigure}
  \caption{ \runone (left) and \runtwo (right) trigger system schemes.}
  \label{det_trigger_scheams}
\end{figure}

The first trigger level is a pure hardware implementation whilst the other two are software only.
Complicated selection criteria are not affordable from computation power point of view, given the
$40\mhz$ input rate of \lzero. After the rate is reduced down to $1\mhz$ the data is read out and
passed to the subsequent higher level trigger, where tracking and PID information become available.

The overall performance of the \lhcb trigger system is $\sim 90\%$ for dimuon channels and $\sim 30\%$ for
multi-body hadronic final states. A complete performance study of the trigger system can be found in chapter
5 of \cite{Aaij:2014jba}. The current section focuses on the trigger performance involving muons;
The corresponding efficiencies are shown in \figref{det_run_one_muon_line_eff}.

\begin{figure}[t]
  \centering
  \begin{subfigure}{0.5\textwidth}
    \raggedright
    \includegraphics[width=\textwidth,trim=0.45cm 0cm 0.4cm 0cm, clip=true]{Figures/Chapter2/l0_muon_eff}
    \caption{}
    \label{det_run_one_l0_muon_line_eff}
  \end{subfigure}%
  \hfill%
  \begin{subfigure}{0.5\textwidth}
    \raggedleft
    \includegraphics[width=\textwidth,trim=0.45cm 0cm 0.4cm 0cm, clip=true]{Figures/Chapter2/hlt1_muon_eff}
    \caption{}
    \label{det_run_one_hlt1_muon_line_eff}
  \end{subfigure}%
  \hfill%
  \begin{subfigure}{0.5\textwidth}
    \raggedright
    \includegraphics[width=\textwidth,trim=0.45cm 0cm 0.4cm 0cm, clip=true]{Figures/Chapter2/hlt2_muon_eff}
    \caption{}
    \label{det_run_one_hlt2_muon_line_eff}
  \end{subfigure}
  \caption{\runone muon line efficiencies of the \lzero (A), \hltone (B) and \hlttwo (C) trigger stages.}
  \label{det_run_one_muon_line_eff}
\end{figure}

Lastly there are two more features of the \lhcb trigger system worth-mentioning. First is the so called
{\it deferred} trigger, where $20\%$ of the data triggered by \lzero are temporarily stored and processed by the \hlt
during the periods where the \lhc does not provide colliding beams and thus the \lhcb trigger infrastructure
would otherwise idle. This effectively increases the amount of available computing power which can be used
to improve other features of the trigger system. The second interesting feature is the novel technique of
{\it online detector alignment and calibration} that was enabled in \runtwo, see \cite{Aaij:2016rxn}.
The technique makes it possible that the output data quality of the trigger becomes identical to the offline data quality.
This means that offline data analysis can be performed straight from the output of the trigger, which saves
a lot of computing resources and accelerates the analysis.
