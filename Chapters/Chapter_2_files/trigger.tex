The trigger system is responsible for keeping the amount of data that the detector writes to offline storage
at a manageable level, since it is not feasible to save every event. The three levels of the \lhcb trigger system
are able to quickly, $\sim 25\ns$, decide weather a certain collision, {\it event}, is interesting from physics point
of view and should thus be written out to storage. The criteria used to filter out the interesting events, get
progressively more stringent at each trigger level. The whole trigger chain is optimized for maximum
signal purity given the amount of computing and storage infrastructure. In more detail the first trigger
level, \lzero, reads the detector at a rate of $40\mhz$, while the subsequent {\it Higher Level Trigger},
\hltone and \hlttwo, write out to storage at $5\khz$. The first trigger level is a pure hardware
implementation whilst the other two are software only.

\begin{figure}[t]
  \centering
  \begin{subfigure}{0.5\textwidth}
    \raggedright
    \includegraphics[width=\textwidth]{Figures/Chapter2/LHCb_Trigger_RunIAlgDetail_May2015}
    \caption{}
    \label{det_run_one_trigger}
  \end{subfigure}%
  \hfill%
  \begin{subfigure}{0.5\textwidth}
    \raggedleft
    \includegraphics[width=\textwidth]{Figures/Chapter2/LHCb_Trigger_RunII_May2015}
    \caption{}
    \label{det_run_two_trigger}
  \end{subfigure}
  \caption{ \runone (left) and \runtwo (right) trigger system schemes.}
  \label{det_trigger_scheams}
\end{figure}

The organization of the trigger system is done in {\it streams} and {\it trigger lines}.
The four streams, {\it Hadron}, {\it Muon}, {\it DiMuon} and {\it Photon}, correspond to the possible
ways that an event can be pass through the \lzero trigger level. Complicated selection criteria are
not affordable from computation power point of view at this stage, since the $40\mhz$ input rate of \lzero is quite
big. After the rate is reduced by $\sim 1$ order of magnitude the data are fed to the subsequent
higher level trigger, where tracking and PID information become available. The \hltone and \hlttwo
trigger levels organize their output in trigger lines. Essentially, each trigger line is a collection
of selection criteria similar to the ones applied during the offline data analysis, but looser.
Furthermore, the trigger lines are software implementations and can thus be quite general as to what
they trigger on, or on the other hand they can be very dedicated.

The overall performance of the \lhcb trigger system is $\sim 90\%$ for dimuon channels and $\sim 30\%$ for
multi-body hadronic final states. A complete performance study of the trigger system can be found in chapter
5 of \cite{Aaij:2014jba}. The current section focuses on the trigger performance involving muons;
The corresponding efficiencies are shown in \figref{det_run_one_muon_line_eff}.

\begin{figure}[t]
  \centering
  \begin{subfigure}{0.5\textwidth}
    \raggedright
    \includegraphics[width=\textwidth,trim=0.45cm 0cm 0.4cm 0cm, clip=true]{Figures/Chapter2/l0_muon_eff}
    \caption{}
    \label{det_run_one_l0_muon_line_eff}
  \end{subfigure}%
  \hfill%
  \begin{subfigure}{0.5\textwidth}
    \raggedleft
    \includegraphics[width=\textwidth,trim=0.45cm 0cm 0.4cm 0cm, clip=true]{Figures/Chapter2/hlt1_muon_eff}
    \caption{}
    \label{det_run_one_hlt1_muon_line_eff}
  \end{subfigure}%
  \hfill%
  \begin{subfigure}{0.5\textwidth}
    \raggedright
    \includegraphics[width=\textwidth,trim=0.45cm 0cm 0.4cm 0cm, clip=true]{Figures/Chapter2/hlt2_muon_eff}
    \caption{}
    \label{det_run_one_hlt2_muon_line_eff}
  \end{subfigure}
  \caption{\runone muon line efficiencies of the \lzero (A), \hltone (B) and \hlttwo (C) trigger stages.}
  \label{det_run_one_muon_line_eff}
\end{figure}

Lastly there are two more worth mentioning features of the \lhcb trigger system. First is the so called
{\it deferral} trigger, where $20\%$ of the data triggered by \lzero are temporarily stored and proceed by the \hltone
during the periods where the \lhc does not provide colliding beams and thus the \lhcb computing infrastructure
would normally idle. This effectively increases the amount of available computing power which can be used
to improve other features of the trigger system. The second interesting feature is the novel technique of
{\it online detector alignment and calibration} that was enabled in view of the \runtwo, see \cite{Aaij:2016rxn}.
The technique makes it possible that the output data quality of the trigger becomes identical to the offline data quality.
This means that offline data analysis can be performed straight from the output of the trigger, which saves
a lot of computing resources, accelerates the analysis and minimizes the chances of introducing mistakes.
