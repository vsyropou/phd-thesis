The trigger system is responsible for keeping the amount of data that the detector writes to offline storage
at a managable level, since it is not feesible to save every event. The three levels of the \lhcb trigger system
are able to quickly, $\sim 25\ns$, decide weather a certain colsion, {\it event}, is intreating from physics point
of view and should thus be writen out to storage. The creteria used to filter out the intreasting events, get
progresively more stringent at each trigger level. The whole trigger chain is optimised for maximum
signal purity given the amount of computing and storage infrastructure. In more detail the first trigger
level, \lzero, reads the detector at a rate of $40\mhz$, while the subsequent {\it Higher Level Trigger},
\hltone and \hlttwo, write the out to storage at $5\khz$. The first trigger level is a pure hardware
implementation whilst the other two are software only.

\begin{figure}[t]
  \centering
  \includegraphics[width=0.45\textwidth]{Figures/Chapter2/LHCb_Trigger_RunIAlgDetail_May2015}
  \hspace{0.2cm}
  \includegraphics[width=0.45\textwidth]{Figures/Chapter2/LHCb_Trigger_RunII_May2015}
  \caption{\runone trigger scheam.}
  \label{run_one_trigger}
\end{figure}

The organization of the trigger system is done in the so called {\it streams} and {\it trigger lines} (or simply lines).
The streams are relevant for the \lzero level where the triggered data are categorised as {\it Hadron}, {\it Muon},
{\it DiMuon} and {\it Photon}. Since \lzero has to reduce the output rate to \hltone by aproximatelly one order of
magnitute, there is not enough computing time available to perform complecated selection creteria. On the other
hand in the subsequent trigger levels more advanced selection can be required since tracking and PID information
become available. The \hltone and \hlttwo trigger levels organise their output in trigger lines. Essentially,
each trigger line is a collection of selection criteria similar, but looser, to the ones applied during the
offline data analysis.

The overal performance of the \lhcb trigger system is $\sim 90\%$ for dimuon chanells and $\sim 30\%$ for
multibody hadronic final states. A complete performance study of the trigger system can be found in \cite{}.
The current section focuses on the trigger performance involving muons; The correspondin efficiencies can
are shown in \figref{}.

Lastly there are two more worth mentioning features of the \lhcb trigger system. First is the so called
{\it deferal} trigger, where $20\%$ of the data triggered by \lzero are temporarily stored and procced by the \hltone
during the periods where the \lhc does not provide coliding beams and thus the \lhcb computing infrastructure
would normally idle. This effectively increases the amount of available computing power which can be used
to improve other feutures of the trigger system. The second intreasting feature is the novel tecnique of
{\it online detector alignment and calibration} that was enabled in view of the \runtwo. The technique
makes it possible that the output data quality of the trigger becomes identical to the offline data quality.
This means that offline data analysis can be performed straight from the output of the trigger, which saves
a lot of computing resources, accelerates the analysis and minimises the chances of introducing mistakes.
