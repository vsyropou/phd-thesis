
The \lhcb experiment is an international colaboration and is located at the {\it European Organization for Nuclear Research}, \cern.
There are four major experiments at that location; Two are general purpose experiments, \atlas and \cms, whereas
\alice and \lhcb, are dedicated to {\it heavy ion} and {\it flavour} physics respectively. Heavy ion physics
studies the very energetic state of matter where quarks behave almost as free particles. 
While flavour physics is introduced in \secref{Flavour_Physics}. All of the above experiments aim at detecting, 
recording and analyzing as many particle collisions as possible, produced by the {\it Large Hadron Collider}, \lhc.
The latter is the world's most powerful particle accelerating machine.
The \lhc machine accelerates, stores and collides two beams of protons. It is also capable of handling
heavy ion beams, \eg lead, thus serving the heavy ion physics program of the experiments.
In both cases storage is achieved by magnetically forcing the beams to follow a closed, almost circular,
trajectory. The two beam trajectories collide with each other at four specific {\it interaction points},
where each of the four experiments are located. The \lhc accelerator, as well as the four previously
mentioned experiments, are about $100\m$ underground. Given the complexity and technical difficulties, 
building, operating and maintaining all these machines is a remarkable achievement.

\begin{figure}[t]
  \centering
  \includegraphics[width=1\textwidth]{Figures/Chapter2/detector_cross_cmyk}
  \caption{Schematic side view (vertical cross section) of the \lhcb detector.}
  \label{lhcb_detector_cross_section}
\end{figure}

As mentioned in the previous paragraph, the \lhcb experiment is dedicated in studding flavor physics, 
and particularly \B mesons, introduced in \secref{Phenomenology}. The geometry of the \lhcb detector, 
shown in \figref{lhcb_detector_cross_section}, is such that it exploits a particular feature of the 
\bquark quark production (which is one of the constituents of the \B mesons). Specifically, in \figref{bb_roduction_angles} 
it can be seen that \bquark quarks produced in such a collision, cluster mainly along the proton beam axis. 
In other words approximately $50\%$ of the \bquark quarks {\it fly} inside a region of $\sim 45^\circ$ solid 
angle along either directions of the beam axis. This region is also called {\it forward} region. 
As a result, a $4\pi$ detector geometry like the rest of the \lhc experiments, where the center of collision 
is surrounded by layers of detectors, is not necessary. Instead the \lhcb detector covers one of the two 
forward regions along the beam axis, and thus approximately $25\%$ of the produced \bquark hadrons are 
detectable by \lhcb. The full \lhcb detector specifications, design and performance are described in \cite{Aaij:2014jba}.
The approach of the current chapter is to briefly address the function and performance of the detector components
throughout \secref{det_tracking} - \secref{det_calo}. Emphasis is given to the sub-detectors that are relevant
to the analysis described in \chapref{Data_Analysis} as well as to the trigger system which is of crucial importance
to the \lhcb experiment.

\begin{figure}[t]
  \centering
  \includegraphics[width=0.8\textwidth, trim=0cm 0cm 0cm 2.5cm, clip=true]{Figures/Chapter2/08_rad_acc_scheme_right}
  \caption{Decay angles distribution of \bquark quark pair produced in \lhcb. Angles are defined with
           respect to the beam axis $z$.}
  \label{bb_roduction_angles}
\end{figure}
