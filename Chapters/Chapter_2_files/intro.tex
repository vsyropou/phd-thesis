
The \lhcb experiment is an international collaboration located in the {\it European Organization for Nuclear Research}, \cern.
There are four major experiments at that location; Two of them, \atlas and \cms, are general purpose experiments whereas the other
two \alice and \lhcb, are dedecated to {\it heavy ion} and {\it flavour} physics respectively\footnote{Heavy ion physics studies
the very energetic state of matter where quarks behave almost as free partciles. While flavour physics in introduced in \secref{Flavour_Physics}.}.
All of the above experiments aim at detecting, recording and analysig as many as
posible high energy particle collisions, which are provided by the {\it Large Hadron Colider}, \lhc.
The latter is the world's most powerful particle accelerating machine built so far.
The \lhc machine accelerates, stores and colides two beams of protons. It is also capable of handling
heavy ion beams, \eg led, thus serving the physics program of the \alice experiment.
In both cases storage is achieved by magnetically forcing the beams to follow an eliptical trajectory.
The two beam trajectories are designed such that they colide with each other at four specific {\it interaction points},
where each of the four experiments is located. The \lhc accelerator as well as the four previously mentioned
experiments are located located about $100\m$ udnerground, to protect human population from radiation.
Given the complexity and techinical dificulties, building, operating and maintaining all these machines
is a remarkable achievement of human claboration.

\begin{figure}[t]
  \centering
  \includegraphics[width=\textwidth]{Figures/Chapter2/detector_cross_cmyk}
  \caption{Schematic side view of the \lhcb detector.}
  \label{lhcb_detector_cross_section}
\end{figure}

As mentioned in the previous paragraph the \lhcb experiment is dedicateed for flavor physics.
The design of the detector is optimized for this kind of physics and particularly for the detection
of \B mesons, introduced in \secref{Phenomenology}. The latter type of meson contains a \bquark quark
which originates from a proton-proton colision. In \figref{bb_roduction_angles} it can be seen
that \bquark quarks produced in such a colision, cluster mainly along the proton beam axis. In other words
roughly $50\%$ of the \bquark quarks {\it fly} inside a region of $\sim 45^\circ$ solid angle along both
directions of the beam axis. This region is also called {\it forward} region. The geometry of the \lhcb detector,
shown in \figref{lhcb_detector_cross_section}, is such that it exploits this special production feature of \bquark quarks. Implying that
a $4\pi$ detector geometry like the rest of the \lhc experiments, where the center of colision is sourounded
by layers of detectors, is not necessary. Instead the the \lhcb detector covers the forward region along one
of the two directions of the beam axis, impliyg that roughly $25\%$ of the b quarks produced are detectable
by \lhcb. This forward design allows for increased purity of the recorded \B mesons at the expence of a limited
physics program compared to the rest of the \lhcb experiments.

The full \lhcb detector specifications, designs and performance are described in \cite{Aaij:2014jba}.
The aproach of the current chapter is to very briefly address the function and perfromance of the detector components
throughout \secref{det_tracking} - \secref{det_calo}. Emphasis, is given to these sub-detectors that are relevant
to the analysis performed in \chapref{Data_Analysis} as well as to the trigger system which is of crucial importance
to efficiently running the \lhcb experiment.

\begin{figure}[t]
  \centering
  \includegraphics[width=0.8\textwidth, trim=0cm 0cm 0cm 2.5cm, clip=true]{Figures/Chapter2/08_rad_acc_scheme_right}
  \caption{Decay angles distribution of \bquark quark pair produced in \lhcb. Angles are defined with
           respect to the beam azis $z$. Quarks fly within a small solid angle around the beam axis.}
  \label{bb_roduction_angles}
\end{figure}
